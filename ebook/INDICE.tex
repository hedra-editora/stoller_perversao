
\renewcommand{\indexname}{Índice Geral}

\newcommand\indice[2]{\index{#1#2}\xspace}

% aberrante, comportamento sexual
\newcommand\idxaberr[1][]{\indice{aberrante, comportamento sexual, \textsc{ver} perversões; variações}{#1}}
\newcommand\idxaberrauto[1][]{\idxaberr[!autoproduzido, autocriado#1]}
\newcommand\idxaberrdef[1][]{\idxaberr[!definidos#1]}
\newcommand\idxaberranim[1][]{\idxaberr[!em animais#1]}
\newcommand\idxaberrestat[1][]{\idxaberr[!explicação estatística#1]}
\newcommand\idxaberrfisic[1][]{\idxaberr[!explicações físicas#1]}
\newcommand\idxaberrbiol[1][]{\idxaberr[!fatores biológicos#1]}
\newcommand\idxaberrfixa[1][]{\idxaberr[!fixações em#1]}
\newcommand\idxaberrpesq[1][]{\idxaberr[!pesquisa sobre#1]}
\newcommand\idxaberrteor[1][]{\idxaberr[!teorias comportamentais#1]}

% adolescência
\indice{adolescência}{|see{infância}}

% agressão
\newcommand\idxagres[1][]{\indice{agressão}{#1}}
\newcommand\idxagresmasc[1][]{\idxagres[!e masculinidade#1]}
\newcommand\idxagrespais[1][]{\idxagres[!dirigida contra os pais#1]}

% fase anal
\newcommand\idxfanal[1][]{\indice{fase anal}{#1}}

% androgênio, no desenvolvimento pré-natal
\newcommand\idxandrn[1][]{\indice{androgênio, no desenvolvimento pré-natal}{#1}}

% androgênio, síndrome da insensibilidade ao
\newcommand\idxandrs[1][]{\indice{androgênio, síndrome da insensibilidade ao}{#1}}

% comportamento animal
\newcommand\idxanim[1][]{\indice{comportamento animal}{#1}}
\newcommand\idxanimvers[1][]{\idxanim[!\textit{versus} comportamento humano#1]}
\newcommand\idxanimdese[1][]{\idxanim[!desenvolvimento pré-natal#1]}

% antropologia
\newcommand\idxantro[1][]{\indice{antropologia, \textsc{ver} relativismo cultural}{#1}}

% angústia
\newcommand\idxangu[1][]{\indice{angústia, \textsc{ver} trauma e triunfo}{#1}}
\newcommand\idxanguinfa[1][]{\idxangu[!na infância#1]}
\newcommand\idxanguporn[1][]{\idxangu[!e pornografia#1]}
\newcommand\idxangusimb[1][]{\idxangu[!da simbiose#1]}

% Bak, R. C.
\newcommand\idxbak[1][]{\indice{Bak, R. C.}{#1}}

% behaviorismo
\indice{behaviorismo}{|see{condicionamento}}

% bestialidade
\newcommand\idxbesta[1][]{\indice{bestialidade}{#1}}

% Bieber, Irving
\newcommand\idxbieb[1][]{\indice{Bieber, Irving}{#1}}

% amarrar, perversões que envolvem
\newcommand\idxamarr[1][]{\indice{amarrar, perversões que envolvem}{#1}}

% mecanismos biopsíquicos
\newcommand\idxmecan[1][]{\indice{mecanismos biopsíquicos}{#1}}

% controle da natalidade
\newcommand\idxcontr[1][]{\indice{controle da natalidade}{#1}}

% bissexualidade
\newcommand\idxbisse[1][]{\indice{angústia}{#1}}
\newcommand\idxbisseteor[1][]{\idxbisse[!teoria freudiana da#1]}
\newcommand\idxbissetend[1][]{\idxbisse[!tendências herdadas#1]}
\newcommand\idxbissedese[1][]{\idxbisse[!no desenvolvimento pré-natal#1]}

% Blumer, D.
\newcommand\idxblumer[1][]{\indice{Blumer, D.}{#1}}

% Boehm, R.
\newcommand\idxboehm[1][]{\indice{Boehm, R.}{#1}}

% enfado
\newcommand\idxenfa[1][]{\indice{enfado}{#1}}
\newcommand\idxenfafamil[1][]{\idxenfa[!gerado pela familiaridade#1]}

% Boss, M.
\newcommand\idxboss[1][]{\indice{Boss, M.}{#1}}

% cérebro, anormalidades
\newcommand\idxcerea[1][]{\indice{cérebro, anormalidades}{#1}}

% cérebro, estudos
\newcommand\idxcereb[1][]{\indice{cérebro, estudos}{#1}}
\newcommand\idxcerebhipe[1][]{\idxcereb[!hipersexualidade#1]}
\newcommand\idxcerebmeca[1][]{\idxcereb[!mecanismos do cérebro médio#1]}
\newcommand\idxcerebpraz[1][]{\idxcereb[!centros de prazer#1]}
\newcommand\idxcerebepil[1][]{\idxcereb[!epilepsia do lobo temporal#1]}

% seios
\newcommand\idxseios[1][]{\indice{seios}{#1}}

% castração
\newcommand\idxcastr[1][]{\indice{castração}{#1}}

% castração, angústia da
\newcommand\idxcasta[1][]{\indice{castração, angústia da}{#1}}
\newcommand\idxcastaexib[1][]{\idxcasta[!no exibicionismo#1]}
\newcommand\idxcastaiden[1][]{\idxcasta[!e identidade de gênero#1]}
\newcommand\idxcastaconf[1][]{\idxcasta[!no conflito edípico#1]}
\newcommand\idxcastamulh[1][]{\idxcasta[!e mulheres fálicas#1]}
\newcommand\idxcastatrav[1][]{\idxcasta[!no travestismo#1]}

% infância (primeira infância, adolescência)
\newcommand\idxinfan[1][]{\indice{infância (primeira infância, adolescência), \textsc{ver} trauma e triunfo; desenvolvimento da libido; conflito edípico; mães}{#1}}
\newcommand\idxinfanangu[1][]{\idxinfan[!angústia#1]}
\newcommand\idxinfanperig[1][]{\idxinfan[!perigo à sexualidade na#1]}
\newcommand\idxinfanego[1][]{\idxinfan[!desenvolvimento do ego#1]}
\newcommand\idxinfanexib[1][]{\idxinfan[!exibicionismo#1]}
\newcommand\idxinfanfrust[1][]{\idxinfan[!frustração na#1]}
\newcommand\idxinfancurio[1][]{\idxinfan[!curiosidade genital#1]}
\newcommand\idxinfansexua[1][]{\idxinfan[!sexualidade infantil#1]}
\newcommand\idxinfanexcit[1][]{\idxinfan[!excitação sexual#1]}
\newcommand\idxinfanolhar[1][]{\idxinfan[!o olhar eivado de sexualidade#1]}
\newcommand\idxinfanvitim[1][]{\idxinfan[!vitimização#1]}

% classificação
\newcommand\idxclass[1][]{\indice{classificação}{#1}}

% clitóris
\newcommand\idxclit[1][]{\indice{clitóris}{#1}}
\newcommand\idxclitorga[1][]{\idxclit[!orgasmo clitoridiano#1]}
\newcommand\idxclitfemi[1][]{\idxclit[!e feminilidade#1]}
\newcommand\idxclitdese[1][]{\idxclit[!no desenvolvimento fetal#1]}
\newcommand\idxclitfixa[1][]{\idxclit[!fixação no#1]}

% roupas, sexualização das
\newcommand\idxroup[1][]{\indice{roupas, sexualização das}{#1}}

% acordo de conciliação
\newcommand\idxacor[1][]{\indice{acordo de conciliação}{#1}}

% condicionamento
\newcommand\idxcond[1][]{\indice{condicionamento}{#1}}

% conflito
\newcommand\idxconf[1][]{\indice{conflito, \textsc{ver} conflito edípico; trauma e triunfo}{#1}}
\newcommand\idxconfcliv[1][]{\idxconf[!e a clivagem do ego#1]}
\newcommand\idxconfintr[1][]{\idxconf[!intrapsíquico#1]}

% coprofilia
\newcommand\idxcopro[1][]{\indice{coprofilia}{#1}}

% criatividade
\newcommand\idxcria[1][]{\indice{criatividade}{#1}}
\newcommand\idxcriavs[1][]{\idxcria[!homens \textit{vs.} mulheres#1]}
\newcommand\idxcriatran[1][]{\idxcria[!de meninos transexuais#1]}

% crime, delito; atos criminosos
\newcommand\idxcrime[1][]{\indice{crime, delito; atos criminosos, \textsc{ver} aspectos sociais da questão; responsabilidade moral; pecado, pecaminosidade}{#1}}

% travestismo
\newcommand\idxtrave[1][]{\indice{travestismo}{#1}}
\newcommand\idxtravetrau[1][]{\idxtrave[!e trauma infantil#1]}
\newcommand\idxtraveporn[1][]{\idxtrave[!na pornografia#1]}

% relativismo cultural
\newcommand\idxrela[1][]{\indice{relativismo cultural}{#1}}
\newcommand\idxrelaconf[1][]{\idxrela[!e o conflito edípico#1]}

% desumanização
\newcommand\idxdesu[1][]{\indice{desumanização}{#1}}
\newcommand\idxdesufant[1][]{\idxdesu[!na fantasia#1]}
\newcommand\idxdesuexci[1][]{\idxdesu[!e excitação sexual#1]}

% mecanismos de negação
\newcommand\idxnega[1][]{\indice{mecanismos de negação}{#1}}

% Devereux, G.
\newcommand\idxdever[1][]{\indice{Devereux, G.}{#1}}

% desvios
\newcommand\idxdesv[1][]{\indice{desvios}{#1}}

% diagnóstico
\newcommand\idxdiag[1][]{\indice{diagnóstico}{#1}}
\newcommand\idxdiagcrit[1][]{\idxdiag[!critérios para#1]}
\newcommand\idxdiagetio[1][]{\idxdiag[!etiologia#1]}
\newcommand\idxdiagrotu[1][]{\idxdiag[!e rótulos, rotulagem#1]}
\newcommand\idxdiagnome[1][]{\idxdiag[!nomenclatura#1]}
\newcommand\idxdiagpato[1][]{\idxdiag[!patogenia#1]}
\newcommand\idxdiagmeto[1][]{\idxdiag[!método científico em#1]}
\newcommand\idxdiagforc[1][]{\idxdiag[!como força social#1]}
\newcommand\idxdiagsind[1][]{\idxdiag[!síndromes#1]}

% Don Juanismo
\newcommand\idxdonj[1][]{\indice{Don Juanismo}{#1}}

% desenvolvimento egóico
\newcommand\idxegoi[1][]{\indice{desenvolvimento egóico}{#1}}
\newcommand\idxegoiauto[1][]{\idxegoi[!auto-confiança, sentimento de#1]}

% clivagem do ego
\newcommand\idxcliv[1][]{\indice{clivagem do ego}{#1}}

% Ellis, Havelock
\newcommand\idxellis[1][]{\indice{Ellis, Havelock}{#1}}

% desenvolvimento embrionário
\indice{desenvolvimento embrionário}{|see{desenvolvimento pré-natal}}

% sistema endócrino
\indice{sistema endócrino}{|see{hormônios}}

% exibicionismo
\newcommand\idxexibi[1][]{\indice{exibicionismo}{#1}}
\newcommand\idxexibicond[1][]{\idxexibi[!condenações, reincidência#1]}
\newcommand\idxexibiinfa[1][]{\idxexibi[!na infância#1]}
\newcommand\idxexibiexpo[1][]{\idxexibi[!exposição ao perigo no#1]}
\newcommand\idxexibiatra[1][]{\idxexibi[!\textit{vs.} atração sexual#1]}
\newcommand\idxexibitrav[1][]{\idxexibi[!no travestismo#1]}

% influências familiares
\newcommand\idxinflu[1][]{\indice{influências familiares, \textsc{ver} relacionamento pais-criança; simbiose mãe-filho}{#1}}

% fantasia
\newcommand\idxfanta[1][]{\indice{fantasia}{#1}}
\newcommand\idxfantaangu[1][]{\idxfanta[!a angústia na#1]}
\newcommand\idxfantacons[1][]{\idxfanta[!a construção da#1]}
\newcommand\idxfantareal[1][]{\idxfanta[!a realidade histórica na#1]}
\newcommand\idxfantapraz[1][]{\idxfanta[!prazer na#1]}
\newcommand\idxfantaporn[1][]{\idxfanta[!na pornografia#1]}
\newcommand\idxfantanece[1][]{\idxfanta[!a necessidade de repetição#1]}
\newcommand\idxfantareve[1][]{\idxfanta[!a reversão na#1]}
\newcommand\idxfantaexpo[1][]{\idxfanta[!a exposição ao perigo na#1]}
\newcommand\idxfantasado[1][]{\idxfanta[!sadomasoquismo#1]}
\newcommand\idxfantafato[1][]{\idxfanta[!fatores de segurança#1]}
\newcommand\idxfantamulh[1][]{\idxfanta[!das mulheres#1]}

% pais
\newcommand\idxpais[1][]{\indice{pais}{#1}}
\newcommand\idxpaisiden[1][]{\idxpais[!identificação com#1]}
\newcommand\idxpaismasc[1][]{\idxpais[!e a masculinidade na criança#1]}
\newcommand\idxpaisauto[1][]{\idxpais[!autoridade física dos#1]}
\newcommand\idxpaisfilh[1][]{\idxpais[!de filhos transexuais#1]}

% faute de mieux
\newcommand\idxfaute[1][]{\indice{\textit{faute de mieux}}{#1}}

% falo feminino, a fantasia do
\newcommand\idxfalo[1][]{\indice{falo feminino, a fantasia do}{#1}}

% feminidade / antes: condição de fêmea%%%
\newcommand\idxcondf[1][]{\indice{feminidade}{#1}}
\newcommand\idxcondfsupr[1][]{\idxcondf[!supremacia biológica da#1]}
\newcommand\idxcondfcapa[1][]{\idxcondf[!capacidade procriativa#1]}
\newcommand\idxcondfinfe[1][]{\idxcondf[!inferioridade da#1]}

% feminilidade
\newcommand\idxfemin[1][]{\indice{feminilidade}{#1}}
\newcommand\idxfeminfixa[1][]{\idxfemin[!e a fixação clitoridiana#1]}
\newcommand\idxfemincari[1][]{\idxfemin[!caricaturas homossexuais da#1]}
\newcommand\idxfemininve[1][]{\idxfemin[!a inveja do homem#1]}
\newcommand\idxfeminsimb[1][]{\idxfemin[!na simbiose mãe-filho#1]}
\newcommand\idxfeminmaes[1][]{\idxfemin[!em mães de filhos transexuais#1]}
\newcommand\idxfeminconf[1][]{\idxfemin[!no conflito edípico#1]}

% complexo de feminilidade, em homens
\newcommand\idxcompl[1][]{\indice{complexo de feminilidade, em homens}{#1}}

% Fenichel, Otto
\newcommand\idxfenic[1][]{\indice{Fenichel, Otto}{#1}}

% desenvolvimento fetal
\indice{desenvolvimento fetal}{|see{desenvolvimento pré-natal}}

% fetichismo
\newcommand\idxfetic[1][]{\indice{fetichismo}{#1}}
\newcommand\idxfeticangu[1][]{\idxfetic[!e a angústia da castração#1]}
\newcommand\idxfeticfreu[1][]{\idxfetic[!Freud, acerca do#1]}
\newcommand\idxfetichost[1][]{\idxfetic[!a hostilidade no#1]}
\newcommand\idxfeticnude[1][]{\idxfetic[!e nudez#1]}
\newcommand\idxfeticorga[1][]{\idxfetic[!e orgasmo#1]}
\newcommand\idxfeticarti[1][]{\idxfetic[!artifícios de segurança#1]}
\newcommand\idxfeticlobo[1][]{\idxfetic[!e transtornos do lobo temporal#1]}
\newcommand\idxfetictrav[1][]{\idxfetic[!e travestismo#1]}

% Fliess, Wilhelm
\newcommand\idxflies[1][]{\indice{Fliess, Wilhelm}{#1}}

% simbiose focal
\newcommand\idxsimbf[1][]{\indice{simbiose focal}{#1}}

% preliminares
\newcommand\idxpreli[1][]{\indice{preliminares}{#1}}

% Frankl, V. E.
\newcommand\idxfrank[1][]{\indice{Frankl, V. E.}{#1}}

% livre arbítrio
\newcommand\idxlivre[1][]{\indice{livre arbítrio, \textsc{ver} responsabilidade moral; aspectos sociais da questão}{#1}}

% Freud, Anna
\newcommand\idxannaf[1][]{\indice{Freud, Anna}{#1}}

% Freud, Sigmund
\newcommand\idxfreud[1][]{\indice{Freud, Sigmund}{#1}}
\newcommand\idxfreudaberr[1][]{\idxfreud[!sobre o comportamento sexual aberrante#1]}
\newcommand\idxfreudanali[1][]{\idxfreud[!\textit{Análise Terminável e Interminável}#1]}
\newcommand\idxfreudbiolo[1][]{\idxfreud[!biologização, acerca da@``biologização'', acerca da#1]}
\newcommand\idxfreudbisse[1][]{\idxfreud[!sobre a bissexualidade#1]}
\newcommand\idxfreudclito[1][]{\idxfreud[!sobre a função clitoridiana#1]}
\newcommand\idxfreudegoid[1][]{\idxfreud[!\textit{Ego and the Id, The}#1]}
\newcommand\idxfreudestru[1][]{\idxfreud[!estrutura de caráter egossintônica#1]}
\newcommand\idxfreuddesen[1][]{\idxfreud[!sobre o desenvolvimento erótico#1]}
\newcommand\idxfreudfetic[1][]{\idxfreud[!sobre o fetichismo#1]}
\newcommand\idxfreudident[1][]{\idxfreud[!sobre a identidade de gênero#1]}
\newcommand\idxfreudhomos[1][]{\idxfreud[!sobre a homossexualidade#1]}
\newcommand\idxfreudinsti[1][]{\idxfreud[!sobre os instintos#1]}
\newcommand\idxfreudinter[1][]{\idxfreud[!teorias interpessoais#1]}
\newcommand\idxfreudsuper[1][]{\idxfreud[!sobre a superioridade masculina#1]}
\newcommand\idxfreudmascu[1][]{\idxfreud[!teoria da masculinidade#1]}
\newcommand\idxfreudmente[1][]{\idxfreud[!sobre a questão mente-corpo#1]}
\newcommand\idxfreudneuro[1][]{\idxfreud[!sobre as neuroses#1]}
\newcommand\idxfreudedipo[1][]{\idxfreud[!sobre o conflito edípico#1]}
\newcommand\idxfreudmulhe[1][]{\idxfreud[!sobre as mulheres fálicas#1]}
\newcommand\idxfreudperve[1][]{\idxfreud[!sobre a perversão#1]}
\newcommand\idxfreudsadom[1][]{\idxfreud[!sobre o sadomasoquismo#1]}
\newcommand\idxfreudsexua[1][]{\idxfreud[!sexualidade, teoria da#1]}
\newcommand\idxfreudessay[1][]{\idxfreud[!\textit{Three Essays on the Theory of Sexuality}#1]}

% identidade de gênero
\newcommand\idxiden[1][]{\indice{identidade de gênero}{#1}}
\newcommand\idxidencer[1][]{\idxiden[!cerne da identidade#1]}
\newcommand\idxidenefe[1][]{\idxiden[!em homens efeminados#1]}
\newcommand\idxidenhis[1][]{\idxiden[!conversão histérica#1]}
\newcommand\idxidengen[1][]{\idxiden[!simbiose de gênero#1]}
\newcommand\idxidenedi[1][]{\idxiden[!no conflito edípico#1]}
\newcommand\idxidenpre[1][]{\idxiden[!no desenvolvimento pré-natal#1]}
\newcommand\idxidenrev[1][]{\idxiden[!reversão#1]}

% genética
\newcommand\idxgenet[1][]{\indice{genética}{#1}}
\newcommand\idxgenethomo[1][]{\idxgenet[!e homossexualidade#1]}
\newcommand\idxgenetvari[1][]{\idxgenet[!nas variações#1]}
\newcommand\idxgenetfase[1][]{\idxgenet[!fase genital#1]}
\newcommand\idxgenetgeni[1][]{\idxgenet[!genitália  #1]}
\newcommand\idxgenetgenicuri[1][]{\idxgenetgeni[!curiosidade infantil#1]}
\newcommand\idxgenetgenimist[1][]{\idxgenetgeni[!mistério da#1]}
\newcommand\idxgenetgenitran[1][]{\idxgenetgeni[!dos transexuais#1]}

% Gillespie, W. H.
\newcommand\idxgille[1][]{\indice{Gillespie, W. H.}{#1}}

% Glover, E.
\newcommand\idxglove[1][]{\indice{Glover, E.}{#1}}

% Grécia, homossexualidade na
\newcommand\idxgreci[1][]{\indice{Grécia, homossexualidade na}{#1}}

% Green, R.
\newcommand\idxgreen[1][]{\indice{Green, R.}{#1}}

% Greenacre, P. 104
\newcommand\idxgrena[1][]{\indice{Greenacre, P. 104}{#1}}

% Greenson, R. R.
\newcommand\idxgrens[1][]{\indice{Greenson, R. R.}{#1}}

% culpa
\newcommand\idxculpa[1][]{\indice{culpa}{#1}}
\newcommand\idxculpah[1][]{\idxculpa[!e hostilidade#1]}
\newcommand\idxculpar[1][]{\idxculpa[!redução da, na pornografia#1]}

% estados alucinatórios
\newcommand\idxaluci[1][]{\indice{estados alucinatórios}{#1}}

% enforcamento, erotismo no
\newcommand\idxenfor[1][]{\indice{enforcamento, erotismo no}{#1}}

% hermafroditismo
\newcommand\idxherma[1][]{\indice{hermafroditismo}{#1}}

% heterossexualidade
\newcommand\idxhetero[1][]{\indice{heterossexualidade}{#1}}
\newcommand\idxheteroadqu[1][]{\idxhetero[!como estado adquirido#1]}
\newcommand\idxheterocrit[1][]{\idxhetero[!como critério#1]}
\newcommand\idxheteroinfl[1][]{\idxhetero[!influências familiares#1]}
\newcommand\idxheterofase[1][]{\idxhetero[!fase genital#1]}
\newcommand\idxheterohomo[1][]{\idxhetero[!\textit{vs.} homossexualidade#1]}
\newcommand\idxheteroconf[1][]{\idxhetero[!no conflito edípico#1]}
\newcommand\idxheteroprim[1][]{\idxhetero[!primária#1]}
\newcommand\idxheterotrav[1][]{\idxhetero[!dos travestis#1]}

% homossexualidade
\newcommand\idxhomos[1][]{\indice{homossexualidade}{#1}}
\newcommand\idxhomosbiol[1][]{\idxhomos[!biológica, causas hormonais#1]}
\newcommand\idxhomosdeso[1][]{\idxhomos[!e desordens cerebrais#1]}
\newcommand\idxhomosargu[1][]{\idxhomos[!argumentos culturais#1]}
\newcommand\idxhomosafem[1][]{\idxhomos[!afeminação na#1]}
\newcommand\idxhomosetio[1][]{\idxhomos[!etiologia#1]}
\newcommand\idxhomosexib[1][]{\idxhomos[!exibicionismo#1]}
\newcommand\idxhomosmedo[1][]{\idxhomos[!medo das mulheres#1]}
\newcommand\idxhomoshera[1][]{\idxhomos[!herança genética da#1]}
\newcommand\idxhomosgrec[1][]{\idxhomos[!na Grécia#1]}
\newcommand\idxhomoshete[1][]{\idxhomos[!\textit{vs.} heterossexualidade#1]}
\newcommand\idxhomoshost[1][]{\idxhomos[!hostilidade na#1]}
\newcommand\idxhomoslate[1][]{\idxhomos[!latente#1]}
\newcommand\idxhomoslesb[1][]{\idxhomos[!lésbicas#1]}
\newcommand\idxhomosregr[1][]{\idxhomos[!e regressões da libido#1]}
\newcommand\idxhomosexpe[1][]{\idxhomos[!e experiências de maternagem#1]}
\newcommand\idxhomostrat[1][]{\idxhomos[!tratamentos neurocirúrgicos#1]}
\newcommand\idxhomosneur[1][]{\idxhomos[!medo neurótico à#1]}
\newcommand\idxhomospros[1][]{\idxhomos[!prostitutas#1]}
\newcommand\idxhomosperi[1][]{\idxhomos[!perigo e vingança na#1]}
\newcommand\idxhomosaspe[1][]{\idxhomos[!aspectos sociais da questão#1]}
\newcommand\idxhomossimb[1][]{\idxhomos[!e a angústia da simbiose#1]}

% hormônios
\newcommand\idxhorm[1][]{\indice{hormônios}{#1}}
\newcommand\idxhormhomo[1][]{\idxhorm[!e homossexualidade#1]}
\newcommand\idxhormdese[1][]{\idxhorm[!no desenvolvimento pré-natal#1]}

% hostilidade
\newcommand\idxhost[1][]{\indice{hostilidade}{#1}}
\newcommand\idxhostdese[1][]{\idxhost[!no desenvolvimento do ego na infância#1]}
\newcommand\idxhostdesu[1][]{\idxhost[!e desumanização#1]}
\newcommand\idxhostculp[1][]{\idxhost[!e culpa#1]}
\newcommand\idxhosthomo[1][]{\idxhost[!nos homossexuais#1]}
\newcommand\idxhostmaes[1][]{\idxhost[!em mães de transexuais do sexo masculino#1]}
\newcommand\idxhostrela[1][]{\idxhost[!nos relacionamentos pais-criança#1]}
\newcommand\idxhostporn[1][]{\idxhost[!na pornografia#1]}
\newcommand\idxhostpros[1][]{\idxhost[!na prostituição#1]}
\newcommand\idxhosttria[1][]{\idxhost[!tríade raiva-medo-vingança#1]}
\newcommand\idxhostsadi[1][]{\idxhost[!no sadismo#1]}
\newcommand\idxhostexci[1][]{\idxhost[!e excitação sexual#1]}
\newcommand\idxhostviti[1][]{\idxhost[!vitimização na#1]}

% Hunter, R. A.
\newcommand\idxhunt[1][]{\indice{Hunter, R. A.}{#1}}

% hiperadrenalismo
\newcommand\idxadren[1][]{\indice{hiperadrenalismo}{#1}}

% hipersexualidade
\newcommand\idxhiper[1][]{\indice{hipersexualidade}{#1}}

% idealização, processo
\newcommand\idxideal[1][]{\indice{idealização, processo}{#1}}

% identificação
\newcommand\idxident[1][]{\indice{identificação}{#1}}
\newcommand\idxidentsimb[1][]{\idxident[!na simbiose mãe-filho#1]}
\newcommand\idxidentmult[1][]{\idxident[!múltipla#1]}
\newcommand\idxidentconf[1][]{\idxident[!no conflito edípico#1]}

% estampagem, imprinting
\newcommand\idxestam[1][]{\indice{estampagem, \textit{imprinting}}{#1}}

% incesto
\newcommand\idxinces[1][]{\indice{incesto}{#1}}

% lactação, fase de (infancy)
\indice{lactação, fase de (\textit{infancy})}{|see{infância}}

% sexualidade infantil
\newcommand\idxsexui[1][]{\indice{sexualidade infantil}{#1}}

% teoria das relações interpessoais
\newcommand\idxinter[1][]{\indice{teoria das relações interpessoais}{#1}}

% técnicas de intimidade
\newcommand\idxintim[1][]{\indice{técnicas de intimidade}{#1}}

% Jacobson, Edith
\newcommand\idxjacob[1][]{\indice{Jacobson, Edith}{#1}}

% Johnson, V. E.
\newcommand\idxjohns[1][]{\indice{Johnson, V. E.}{#1}}

% Jost, A.
\newcommand\idxjost[1][]{\indice{Jost, A.}{#1}}

% Khan. M. M. R.
\newcommand\idxkhan[1][]{\indice{Khan. M. M. R.}{#1}}

% Kinsey, A.
\newcommand\idxkins[1][]{\indice{Kinsey, A.}{#1}}

% Klein, Melanie
\newcommand\idxklein[1][]{\indice{Klein, Melanie}{#1}}

% cleptomania
\newcommand\idxclept[1][]{\indice{cleptomania}{#1}}

% Klinefelter, síndrome
\newcommand\idxkline[1][]{\indice{Klinefelter, síndrome}{#1}}

% Kraft-Ebing, Richard von
\newcommand\idxkraft[1][]{\indice{Kraft-Ebing, Richard von}{#1}}

% Kunz, H. 106 n.
\newcommand\idxkunz[1][]{\indice{Kunz, H. 106 n.}{#1}}

% La Barre, W.
\newcommand\idxbarre[1][]{\indice{La Barre, W.}{#1}}

% leis
\newcommand\idxleis[1][]{\indice{leis}{#1}}

% teoria da aprendizagem
\indice{teoria da aprendizagem}{|see{teoria social da aprendizagem}}

% Leites, Nathan
\newcommand\idxnatha[1][]{\indice{Leites, Nathan}{#1}}

% lesbianismo
\newcommand\idxlesb[1][]{\indice{lesbianismo}{#1}}

% desenvolvimento da libido
\newcommand\idxlibid[1][]{\indice{desenvolvimento da libido}{#1}}
\newcommand\idxlibidfixa[1][]{\idxlibid[!fixação, regressão#1]}

% MacAlpine, I.
\newcommand\idxmacal[1][]{\indice{MacAlpine, I.}{#1}}

% Mahler, M. S.
\newcommand\idxmahle[1][]{\indice{Mahler, M. S.}{#1}}

% masculinidade
\newcommand\idxmasc[1][]{\indice{masculinidade}{#1}}
\newcommand\idxmascag[1][]{\idxmasc[!e agressividade#1]}
\newcommand\idxmascsu[1][]{\idxmasc[!superioridade da, crença na#1]}

% Malinowski, B.
\newcommand\idxmalino[1][]{\indice{Malinowski, B.}{#1}}

% marijuana, e testosterona
\newcommand\idxmarij[1][]{\indice{marijuana, e testosterona}{#1}}

% masculinidade
\newcommand\idxmascu[1][]{\indice{masculinidade}{#1}}
\newcommand\idxmascupat[1][]{\idxmascu[!e o papel paterno#1]}
\newcommand\idxmascumul[1][]{\idxmascu[!em mulheres transexuais#1]}
\newcommand\idxmascufem[1][]{\idxmascu[!o feminino como fundação#1]}
\newcommand\idxmascumen[1][]{\idxmascu[!em meninas#1]}
\newcommand\idxmascuhom[1][]{\idxmascu[!a transmissão homossexual da#1]}
\newcommand\idxmascuhum[1][]{\idxmascu[!a humilhação da, e o travestismo#1]}
\newcommand\idxmascupro[1][]{\idxmascu[!o protesto masculino@``o protesto masculino''#1]}
\newcommand\idxmascupap[1][]{\idxmascu[!o papel da mãe no desenvolvimento da#1]}
\newcommand\idxmascucon[1][]{\idxmascu[!no conflito edípico#1]}
\newcommand\idxmascufas[1][]{\idxmascu[!fase protofeminina#1]}
\newcommand\idxmascumud[1][]{\idxmascu[!mudança de sexo, o medo da#1]}
\newcommand\idxmascutra[1][]{\idxmascu[!e travestismo#1]}

% masoquismo
\newcommand\idxmasoq[1][]{\indice{masoquismo}{#1}}
\newcommand\idxmasoqfan[1][]{\idxmasoq[!fantasia no#1]}
\newcommand\idxmasoqhos[1][]{\idxmasoq[!hostilidade no#1]}
\newcommand\idxmasoqpor[1][]{\idxmasoq[!na pornografia#1]}
\newcommand\idxmasoqpun[1][]{\idxmasoq[!punição no#1]}
\newcommand\idxmasoqsad[1][]{\idxmasoq[!sadismo no#1]}

% Masters, W. H.
\newcommand\idxmaste[1][]{\indice{Masters, W. H.}{#1}}

% masturbação
\newcommand\idxmastur[1][]{\indice{masturbação}{#1}}
\newcommand\idxmasturf[1][]{\idxmastur[!\textit{faute de mieux}, princípio#1]}
\newcommand\idxmasturt[1][]{\idxmastur[!travestismo, fantasias com#1]}

% May, Rollo
\newcommand\idxmay[1][]{\indice{May, Rollo}{#1}}

% McDougall, J.
\newcommand\idxmcdoug[1][]{\indice{McDougall, J.}{#1}}

% questão mente-corpo (Freud)
\newcommand\idxquest[1][]{\indice{questão mente-corpo (Freud)}{#1}}

% Money, John
\newcommand\idxmoney[1][]{\indice{Money, John}{#1}}

% responsabilidade moral
\newcommand\idxrespo[1][]{\indice{responsabilidade moral, \textsc{ver} pecado, pecaminosidade}{#1}}

% mães
\newcommand\idxmaes[1][]{\indice{mães}{#1}}
\newcommand\idxmaesbiss[1][]{\idxmaes[!bissexuais#1]}
\newcommand\idxmaesmald[1][]{\idxmaes[!maldade nas@``maldade'' nas#1]}
\newcommand\idxmaesconf[1][]{\idxmaes[!no conflito edípico das meninas#1]}
\newcommand\idxmaeshomo[1][]{\idxmaes[!de homossexuais#1]}
\newcommand\idxmaeshost[1][]{\idxmaes[!hostilidade nas#1]}
\newcommand\idxmaeshosr[1][]{\idxmaes[!hostilidade em relação às#1]}
\newcommand\idxmaesmasc[1][]{\idxmaes[!masculinidade dos filhos#1]}
\newcommand\idxmaesfilh[1][]{\idxmaes[!de filhos transexuais#1]}
\newcommand\idxmaessimb[1][]{\idxmaes[!simbiose mãe-filho#1]}
\newcommand\idxmaesdese[1][]{\idxmaes[!e o desenvolvimento da feminilidade#1]}
\newcommand\idxmaesfoca[1][]{\idxmaes[!simbiose focal#1]}
\newcommand\idxmaeshoms[1][]{\idxmaes[!e homossexualidade#1]}
\newcommand\idxmaesiden[1][]{\idxmaes[!identificação nas#1]}
\newcommand\idxmaesproc[1][]{\idxmaes[!processo de separação#1]}
\newcommand\idxmaesfalo[1][]{\idxmaes[!o filho como falo#1]}
\newcommand\idxmaestran[1][]{\idxmaes[!e transexualismo#1]}

% múltipla identificação
\newcommand\idxmulti[1][]{\indice{múltipla identificação}{#1}}

% múltipla personalidade
\newcommand\idxmultp[1][]{\indice{múltipla personalidade}{#1}}

% assassinato, sexual
\newcommand\idxassas[1][]{\indice{assassinato, sexual}{#1}}

% mistério, na perversão
\newcommand\idxmist[1][]{\indice{mistério, na perversão}{#1}}
\newcommand\idxmistabo[1][]{\idxmist[!abolição do#1]}
\newcommand\idxmistolh[1][]{\idxmist[!no olhar eivado de sexualidade#1]}

% "experiências naturais"
\newcommand\idxexpen[1][]{\indice{experiências naturais@``experiências naturais''}{#1}}

% necrofilia
\newcommand\idxnecro[1][]{\indice{necrofilia}{#1}}

% neuroses
\newcommand\idxneuro[1][]{\indice{neuroses}{#1}}

% nomenclatura, problemas de
\newcommand\idxnomen[1][]{\indice{nomenclatura, problemas de}{#1}}

% não-mental, "memória"
\newcommand\idxnaome[1][]{\indice{não-mental, memória@não-mental, ``memória''}{#1}}

% normalidade
\newcommand\idxnorma[1][]{\indice{normalidade}{#1}}

% ninfomania
\newcommand\idxninfo[1][]{\indice{ninfomania, \textsc{ver} promiscuidade, compulsiva}{#1}}

% objetais, teoria das relações
\indice{objetais, teoria das relações}{|see{teoria social da aprendizagem}}

% conflito edípico
\newcommand\idxconfe[1][]{\indice{conflito edípico, \textsc{ver} simbiose mãe-filho}{#1}}
\newcommand\idxconfemeni[1][]{\idxconfe[!meninos e meninas#1]}
\newcommand\idxconfeestu[1][]{\idxconfe[!estudos interculturais#1]}
\newcommand\idxconfematu[1][]{\idxconfe[!e maturidade genital#1]}
\newcommand\idxconfehete[1][]{\idxconfe[!heterossexualidade, desenvolvimento da#1]}
\newcommand\idxconfemasc[1][]{\idxconfe[!masculinidade, desenvolvimento da#1]}
\newcommand\idxconfedese[1][]{\idxconfe[!desejos pré-edípicos#1]}
\newcommand\idxconfeangu[1][]{\idxconfe[!e angústia da simbiose#1]}
\newcommand\idxconfetran[1][]{\idxconfe[!de transexuais do sexo masculino#1]}

% fase oral
\newcommand\idxforal[1][]{\indice{fase oral}{#1}}

% vulnerabilidade do órgão
\newcommand\idxvulne[1][]{\indice{vulnerabilidade do órgão}{#1}}

% orgasmo
\newcommand\idxorgas[1][]{\indice{orgasmo}{#1}}
\newcommand\idxorgascli[1][]{\idxorgas[!clitoridiano \textit{vs.} vaginal#1]}
\newcommand\idxorgasfet[1][]{\idxorgas[!e fetichismo#1]}

% paranoia e homossexualidade
\newcommand\idxparan[1][]{\indice{paranoia e homossexualidade}{#1}}

% relacionamentos pais-criança
\newcommand\idxrelpc[1][]{\indice{relacionamentos pais-criança, \textsc{ver} influência da família; simbiose mãe-filho; conflito edípico}{#1}}
\newcommand\idxrelpccast[1][]{\idxrelpc[!castração dos pais#1]}
\newcommand\idxrelpciden[1][]{\idxrelpc[!e identidade de gênero#1]}
\newcommand\idxrelpchomo[1][]{\idxrelpc[!e homossexualidade#1]}
\newcommand\idxrelpctran[1][]{\idxrelpc[!e transexualismo#1]}
\newcommand\idxrelpctrav[1][]{\idxrelpc[!e travestismo#1]}

% pedofilia
\newcommand\idxpedof[1][]{\indice{pedofilia}{#1}}

% pênis  
\newcommand\idxpenis[1][]{\indice{pênis}{#1}}
\newcommand\idxpenisclep[1][]{\idxpenis[!e cleptomania#1]}
\newcommand\idxpenisprim[1][]{\idxpenis[!primado, supremacia do#1]}
\newcommand\idxpenisfant[1][]{\idxpenis[!nas fantasias de travestismo#1]}
\newcommand\idxpenisimag[1][]{\idxpenis[!a imaginada posse de, por parte das mulheres#1]}
\newcommand\idxpenisporn[1][]{\idxpenis[!na pornografia feminina#1]}

% pênis, inveja do
\newcommand\idxinvej[1][]{\indice{pênis, inveja do}{#1}}

% perversão
\newcommand\idxperv[1][]{\indice{perversão, \textsc{ver} comportamento sexual aberrante; fantasia; hostilidade}{#1}}
\newcommand\idxpervacide[1][]{\idxperv[!fatores acidentais@fatores ``acidentais''#1]}
\newcommand\idxpervenfad[1][]{\idxperv[!e enfado#1]}
\newcommand\idxpervconot[1][]{\idxperv[!conotações da#1]}
\newcommand\idxpervdesej[1][]{\idxperv[!desejo na#1]}
\newcommand\idxpervdeses[1][]{\idxperv[!desespero na#1]}
\newcommand\idxpervdesvi[1][]{\idxperv[!\textit{vs.} desvios#1]}
\newcommand\idxpervdiagn[1][]{\idxperv[!diagnóstico das#1]}
\newcommand\idxpervneuro[1][]{\idxperv[!como neurose erótica#1]}
\newcommand\idxpervtrans[1][]{\idxperv[!como transtorno de gênero#1]}
\newcommand\idxpervincid[1][]{\idxperv[!incidência#1]}
\newcommand\idxpervdirei[1][]{\idxperv[!\textit{vs.} direitos individuais#1]}
\newcommand\idxpervtecni[1][]{\idxperv[!técnicas de intimidade#1]}
\newcommand\idxpervpredo[1][]{\idxperv[!predominância masculina na#1]}
\newcommand\idxpervrespo[1][]{\idxperv[!e responsabilidade moral#1]}
\newcommand\idxpervmiste[1][]{\idxperv[!mistério na#1]}
\newcommand\idxpervneces[1][]{\idxperv[!necessidade da#1]}
\newcommand\idxpervorgao[1][]{\idxperv[!e órgãos não-genitais#1]}
\newcommand\idxpervnorma[1][]{\idxperv[!\textit{vs.} normalidade#1]}
\newcommand\idxpervrelac[1][]{\idxperv[!relação objetal na#1]}
\newcommand\idxpervorgas[1][]{\idxperv[!orgasmo na#1]}
\newcommand\idxpervrepet[1][]{\idxperv[!repetição na#1]}
\newcommand\idxpervmotiv[1][]{\idxperv[!por motivos de vingança#1]}
\newcommand\idxpervexpos[1][]{\idxperv[!exposição ao perigo#1]}
\newcommand\idxpervproce[1][]{\idxperv[!processo de separação na#1]}
\newcommand\idxpervnegac[1][]{\idxperv[!negação das diferenças sexuais#1]}
\newcommand\idxpervpecad[1][]{\idxperv[!como pecado#1]}
\newcommand\idxpervaspec[1][]{\idxperv[!e aspectos sociais da questão#1]}
\newcommand\idxpervsobre[1][]{\idxperv[!e a sobrevivência das espécies#1]}
\newcommand\idxpervsimbo[1][]{\idxperv[!simbolismo na#1]}

% mecanismo da perversão, o
\newcommand\idxmecap[1][]{\indice{mecanismo da perversão, o}{#1}}

% fase fálica
\newcommand\idxfasef[1][]{\indice{fase fálica}{#1}}

% mulheres fálicas
\newcommand\idxmulhf[1][]{\indice{mulheres fálicas}{#1}}

% prazer, princípio do prazer, o
\newcommand\idxpraz[1][]{\indice{prazer, princípio do prazer, o}{#1}}

% Polanyi, M.
\newcommand\idxpolan[1][]{\indice{Polanyi, M.}{#1}}

% pornografia
\newcommand\idxporno[1][]{\indice{pornografia}{#1}}
\newcommand\idxpornoangu[1][]{\idxporno[!angústia, soluções para#1]}
\newcommand\idxpornobest[1][]{\idxporno[!bestialidade#1]}
\newcommand\idxpornoenfa[1][]{\idxporno[!enfado#1]}
\newcommand\idxpornodesu[1][]{\idxporno[!e desumanização#1]}
\newcommand\idxpornoinst[1][]{\idxporno[!como instrumento de diagnóstico#1]}
\newcommand\idxpornofant[1][]{\idxporno[!fantasia na#1]}
\newcommand\idxpornodisp[1][]{\idxporno[!dispositivos redutores da culpa#1]}
\newcommand\idxpornoreal[1][]{\idxporno[!realidade histórica na#1]}
\newcommand\idxpornohost[1][]{\idxporno[!hostilidade na#1]}
\newcommand\idxpornoidea[1][]{\idxporno[!idealização na#1]}
\newcommand\idxpornohete[1][]{\idxporno[!para heterossexuais masculinos#1]}
\newcommand\idxpornomast[1][]{\idxporno[!e masturbação#1]}
\newcommand\idxpornosadi[1][]{\idxporno[!sadismo na#1]}
\newcommand\idxpornoexci[1][]{\idxporno[!excitação sexual na#1]}
\newcommand\idxpornoplat[1][]{\idxporno[!plateias específicas#1]}
\newcommand\idxpornoviti[1][]{\idxporno[!a vítima na#1]}
\newcommand\idxpornomulh[1][]{\idxporno[!para mulheres#1]}

% desenvolvimento pré-natal
\newcommand\idxprenat[1][]{\indice{desenvolvimento pré-natal}{#1}}

% período pré-edípico
\newcommand\idxpreedi[1][]{\indice{período pré-edípico, \textsc{ver} simbiose mãe-filho}{#1}}

% promiscuidade 
\newcommand\idxpromis[1][]{\indice{promiscuidade }{#1}}
\newcommand\idxpromiscomp[1][]{\idxpromis[!compulsiva#1]}
\newcommand\idxpromishost[1][]{\idxpromis[!hostilidade na#1]}
\newcommand\idxpromismora[1][]{\idxpromis[!\textit{vs.} moralidade vitoriana#1]}

% prostituição
\newcommand\idxprost[1][]{\indice{prostituição}{#1}}
\newcommand\idxprostenf[1][]{\idxprost[!enfado na#1]}
\newcommand\idxprosthom[1][]{\idxprost[!homossexual#1]}
\newcommand\idxprosthos[1][]{\idxprost[!hostilidade#1]}
\newcommand\idxprostdin[1][]{\idxprost[!dinheiro, a função do#1]}
\newcommand\idxprostpas[1][]{\idxprost[!passividade na#1]}
\newcommand\idxprostpot[1][]{\idxprost[!potência, impotência#1]}

% psíquica, energia
\newcommand\idxenerg[1][]{\indice{psíquica, energia}{#1}}

% psicanalítica, teoria
\newcommand\idxpsica[1][]{\indice{psicanalítica, teoria}{#1}}
\newcommand\idxpsicap[1][]{\idxpsica[!da perversão#1]}
\newcommand\idxpsicas[1][]{\idxpsica[!e pesquisa sobre sexo#1]}

% psicoses
\newcommand\idxpsico[1][]{\indice{psicoses}{#1}}

% transtornos psicossomáticos
\newcommand\idxpsicss[1][]{\indice{transtornos psicossomáticos}{#1}}

% Racker, H.
\newcommand\idxracker[1][]{\indice{Racker, H.}{#1}}

% estupro
\newcommand\idxestup[1][]{\indice{estupro}{#1}}

% vingança, fantasia de
\newcommand\idxvinga[1][]{\indice{vingança, fantasia de}{#1}}
\newcommand\idxvingahom[1][]{\idxvinga[!na homossexualidade#1]}
\newcommand\idxvingapro[1][]{\idxvinga[!na prostituição#1]}
\newcommand\idxvingatra[1][]{\idxvinga[!no travestismo#1]}

% exposição ao perigo
\newcommand\idxperigo[1][]{\indice{exposição ao perigo}{#1}}
\newcommand\idxperigocon[1][]{\idxperigo[!consciente \textit{vs.} inconsciente#1]}
\newcommand\idxperigofor[1][]{\idxperigo[!formas de#1]}
\newcommand\idxperigohos[1][]{\idxperigo[!hostilidade na#1]}
\newcommand\idxperigocof[1][]{\idxperigo[!e conflito edípico#1]}

% Sachs, H.
\newcommand\idxsachs[1][]{\indice{Sachs, H.}{#1}}

% sadismo
\newcommand\idxsadi[1][]{\indice{sadismo}{#1}}
\newcommand\idxsadiangu[1][]{\idxsadi[!angústia no#1]}
\newcommand\idxsadifant[1][]{\idxsadi[!fantasia no#1]}
\newcommand\idxsadihost[1][]{\idxsadi[!hostilidade no#1]}
\newcommand\idxsadimaso[1][]{\idxsadi[!no masoquismo#1]}
\newcommand\idxsadipuni[1][]{\idxsadi[!punições físicas#1]}
\newcommand\idxsadiporn[1][]{\idxsadi[!na pornografia#1]}
\newcommand\idxsadiolha[1][]{\idxsadi[!e o olhar eivado de sexualidade#1]}
\newcommand\idxsadivoye[1][]{\idxsadi[!e voyeurismo#1]}

% Salomé, Lou Andreas-
\newcommand\idxsalome[1][]{\indice{Salomé, Lou Andreas-}{#1}}

% satiríase
\newcommand\idxsatiri[1][]{\indice{satiríase}{#1}}

% bode expiatório, o
\newcommand\idxbodex[1][]{\indice{bode expiatório, o}{#1}}

% esquizofrenia
\newcommand\idxesquiz[1][]{\indice{esquizofrenia}{#1}}

% Schmideberg, M.
\newcommand\idxschmid[1][]{\indice{Schmideberg, M.}{#1}}

% método científico
\newcommand\idxmetod[1][]{\indice{método científico}{#1}}

% Schreber, caso (Freud)
\newcommand\idxschreb[1][]{\indice{Schreber, caso (Freud)}{#1}}

% escopofilia
\indice{escopofilia}{|see{voyeurismo}}

% sedução
\newcommand\idxseduc[1][]{\indice{sedução}{#1}}

% sexo, diferenciação durante o desenvolvimento pré-natal
\newcommand\idxsexod[1][]{\indice{sexo, diferenciação durante o desenvolvimento pré-natal}{#1}}

% sexo, pesquisa
\newcommand\idxsexop[1][]{\indice{sexo, pesquisa}{#1}}
\newcommand\idxsexopmeto[1][]{\idxsexop[!método clínico em#1]}
\newcommand\idxsexopteor[1][]{\idxsexop[!\textit{vs.} teoria do conflito#1]}
\newcommand\idxsexopiden[1][]{\idxsexop[!identidade de gênero#1]}
\newcommand\idxsexophomo[1][]{\idxsexop[!homossexualidade#1]}
\newcommand\idxsexoppsic[1][]{\idxsexop[!e a teoria psicanalítica#1]}

% sexual, comportamento
\indice{sexual, comportamento}{|see{perversão; comportamento sexual aberrante; variações}}

% sexual, excitação
\newcommand\idxsexue[1][]{\indice{sexual, excitação}{#1}}
\newcommand\idxsexuedif[1][]{\idxsexue[!e diferenças anatômicas#1]}
\newcommand\idxsexueang[1][]{\idxsexue[!angústia, na#1]}
\newcommand\idxsexuecas[1][]{\idxsexue[!e castração#1]}
\newcommand\idxsexueinf[1][]{\idxsexue[!na infância#1]}
\newcommand\idxsexueamb[1][]{\idxsexue[!como ambiguidade controlada#1]}
\newcommand\idxsexuedes[1][]{\idxsexue[!desumanização na#1]}
\newcommand\idxsexuedin[1][]{\idxsexue[!dinâmicas da#1]}
\newcommand\idxsexuehos[1][]{\idxsexue[!hostilidade, na#1]}

% sexual, liberdade
\newcommand\idxsexul[1][]{\indice{sexual, liberdade}{#1}}
\newcommand\idxsexulrep[1][]{\idxsexul[!\textit{vs.} repressão#1]}
\newcommand\idxsexulres[1][]{\idxsexul[!\textit{vs.} restrições 211–14#1]}

% sexualidade, eivado de, o olhar
\newcommand\idxsexuo[1][]{\indice{sexualidade, eivado de, o olhar}{#1}}

% sexualidade
\newcommand\idxsexua[1][]{\indice{sexualidade}{#1}}
\newcommand\idxsexuateo[1][]{\idxsexua[!Teoria freudiana da#1]}
\newcommand\idxsexuamis[1][]{\idxsexua[!mistificação, da#1]}

% pecado, pecaminosidade
\newcommand\idxpecad[1][]{\indice{pecado, pecaminosidade}{#1}}
\newcommand\idxpecaddel[1][]{\idxpecad[!delimitações#1]}
\newcommand\idxpecadori[1][]{\idxpecad[!origens de#1]}

% Smirnoff, V. N.
\newcommand\idxsmirn[1][]{\indice{Smirnoff, V. N.}{#1}}

% Socarides, C. W.
\newcommand\idxsocar[1][]{\indice{Socarides, C. W.}{#1}}

% aspectos sociais da questão
\newcommand\idxaspec[1][]{\indice{aspectos sociais da questão, \textsc{ver} pecado, pecaminosidade; liberdade sexual}{#1}}
\newcommand\idxaspeccon[1][]{\idxaspec[!conservadorismo#1]}
\newcommand\idxaspecrep[1][]{\idxaspec[!repressão moral#1]}

% social, teoria da aprendizagem
\newcommand\idxsocia[1][]{\indice{social, teoria da aprendizagem}{#1}}
\newcommand\idxsociadese[1][]{\idxsocia[!e desenvolvimento da libido#1]}

% clivagem
\indice{clivagem}{|see{clivagem do ego}}

% estatística e comportamento sexual
\newcommand\idxestat[1][]{\indice{estatística e comportamento sexual}{#1}}

% Straus, E.
\newcommand\idxstraus[1][]{\indice{Straus, E.}{#1}}

% angústia da simbiose
\newcommand\idxangus[1][]{\indice{angústia da simbiose}{#1}}
\newcommand\idxangushom[1][]{\idxangus[!e homossexualidade#1]}
\newcommand\idxangushos[1][]{\idxangus[!hostilidade da mãe na#1]}
\newcommand\idxangusmul[1][]{\idxangus[!nas mulheres#1]}

% sintoma, formação
\newcommand\idxsinto[1][]{\indice{sintoma, formação}{#1}}

% síndromes, sistema de classificação
\newcommand\idxsindr[1][]{\indice{síndromes, sistema de classificação}{#1}}

% telefone, obscenidade
\newcommand\idxtelef[1][]{\indice{telefone, obscenidade}{#1}}

% transferência
\newcommand\idxtransf[1][]{\indice{transferência}{#1}}

% transexualismo (masculino)
\newcommand\idxtranse[1][]{\indice{transexualismo (masculino) \textit{passim}}{#1}}
\newcommand\idxtransepres[1][]{\idxtranse[!pressões que não causam trauma#1]}
\newcommand\idxtransecere[1][]{\idxtranse[!e transtornos cerebrais#1]}
\newcommand\idxtranseetio[1][]{\idxtranse[!etiologia#1]}
\newcommand\idxtransetran[1][]{\idxtranse[!\textit{vs.} transexuais femininos#1]}
\newcommand\idxtransegeni[1][]{\idxtranse[!genitália#1]}
\newcommand\idxtransesimb[1][]{\idxtranse[!simbiose mãe–filho, na#1]}

% travestismo (masculino)
\newcommand\idxtravem[1][]{\indice{travestismo (masculino)}{#1}}
\newcommand\idxtravemangu[1][]{\idxtravem[!e angústia da castração#1]}
\newcommand\idxtravemhete[1][]{\idxtravem[!heterossexualidade, na#1]}
\newcommand\idxtravemhumi[1][]{\idxtravem[!humilhação, no#1]}
\newcommand\idxtravemperi[1][]{\idxtravem[!período de latência#1]}
\newcommand\idxtravemmasc[1][]{\idxtravem[!masculinidade, na#1]}
\newcommand\idxtravemfant[1][]{\idxtravem[!fantasias masturbatórias#1]}
\newcommand\idxtravempote[1][]{\idxtravem[!potência peniana, na#1]}
\newcommand\idxtravemporn[1][]{\idxtravem[!na pornografia#1]}
\newcommand\idxtravemving[1][]{\idxtravem[!e fantasias de vingança#1]}
\newcommand\idxtravemauto[1][]{\idxtravem[!como auto realização#1]}
\newcommand\idxtravemcliv[1][]{\idxtravem[!clivagem, identificação na#1]}
\newcommand\idxtravemsimb[1][]{\idxtravem[!simbolismo das roupas#1]}
\newcommand\idxtravempape[1][]{\idxtravem[!papel da esposa#1]}

% trauma e triunfo
\newcommand\idxtrauma[1][]{\indice{trauma e triunfo}{#1}}
\newcommand\idxtraumatrav[1][]{\idxtrauma[!e travestismo#1]}
\newcommand\idxtraumaexib[1][]{\idxtrauma[!no exibicionismo#1]}
\newcommand\idxtraumapros[1][]{\idxtrauma[!na prostituição#1]}
\newcommand\idxtraumanece[1][]{\idxtrauma[!necessidade de repetição#1]}
\newcommand\idxtraumareve[1][]{\idxtrauma[!reversão (do trauma em triunfo)#1]}
\newcommand\idxtraumaobje[1][]{\idxtrauma[!objetivo#1]}
\newcommand\idxtraumatrae[1][]{\idxtrauma[!no travestismo#1]}
\newcommand\idxtraumadesf[1][]{\idxtrauma[!desfazimento do trauma#1]}

% "self verdadeiro, self falso" (Winnicott)
\newcommand\idxself[1][]{\indice{self verdadeiro, self falso@``\textit{self} verdadeiro, \textit{self} falso'' (Winnicott)}{#1}}

% síndrome de Turner
\newcommand\idxturner[1][]{\indice{síndrome de Turner}{#1}}

% desfazimento, traumas precoces, de
\newcommand\idxdesfa[1][]{\indice{desfazimento, traumas precoces, de}{#1}}

% vagina
\newcommand\idxvagin[1][]{\indice{vagina}{#1}}
\newcommand\idxvaginorg[1][]{\idxvagin[!orgasmo vaginal#1]}

% Valenstein, E. S.
\newcommand\idxvalen[1][]{\indice{Valenstein, E. S.}{#1}}

% Vanggaard, T.
\newcommand\idxvangg[1][]{\indice{Vanggaard, T.}{#1}}

% variações
\newcommand\idxvaria[1][]{\indice{variações}{#1}}
\newcommand\idxvariacult[1][]{\idxvaria[!culturais#1]}
\newcommand\idxvariadefi[1][]{\idxvaria[!definições#1]}
\newcommand\idxvariagene[1][]{\idxvaria[!genéticas, constitucionais#1]}
\newcommand\idxvariafaut[1][]{\idxvaria[!\textit{faute de mieux}#1]}
\newcommand\idxvariaherm[1][]{\idxvaria[!hermafroditas#1]}

% voyeurismo
\newcommand\idxvoy[1][]{\indice{voyeurismo, \textsc{ver} o olhar eivado de sexualidade; exibicionismo}{#1}}
\newcommand\idxvoyrai[1][]{\idxvoy[!raízes infantis#1]}
\newcommand\idxvoysad[1][]{\idxvoy[!e sadismo#1]}

% Williams, A. H.
\newcommand\idxwill[1][]{\indice{Williams, A. H.}{#1}}

% Winnicott, D. W.
\newcommand\idxwinni[1][]{\indice{Winnicott, D. W.}{#1}}

