
\cleardoublepage

\pagestyle{empty}

\section*{GLOSSÁRIO:}  

[\textbf{ATENÇÃO: PARA USO EXCLUSIVO DO EDITOR E REVISOR TÉCNICO. DEVE SER DESCARTADO NO PDF FINAL}]

\bigskip

\noindent aberrant -> aberrante

\smallskip

\noindent alcohol addiction -> alcoolismo

\smallskip

\noindent anlagen -> formas incipientes

\smallskip

\noindent anxiety -> angústia (castration anxiety -> angústia de castração

\smallskip

\noindent anxiety of object loss -> angústia pela perda do objeto

\smallskip

\noindent anxiety neurosis -> neurose de angústia

\smallskip

\noindent awareness -> percepção, consciência

\smallskip

\noindent blissful-> arrebatado,  paradisíaco; sentimento de plenitude

\smallskip

\noindent blissfullness -> sensação de completude, de plenitude, arrebatamento

\smallskip

\noindent boredom -> enfado

\smallskip

\noindent cathexis -> catexia, investimento

\smallskip

\noindent compromise -> ceder, fazer concessões, recuar, abrir mão,
transigir, contemporizar

\smallskip

\noindent [``\textit{compromise
formation'}''-> (termo em
alemão: \textit{Kompromißbildung}) = solução intermediária,
meio-\ \ \ \ \ \ termo; \ \ acordo de conciliação [``Há,
no termo, algo da ordem de um
`conformar-se', de um
\ \ \ \ \ \ recuar para uma situação possível.'' (Luiz
Hanns)]]

\smallskip

\noindent craving -> desejo incontrolável (em francês, ``envie'')

\smallskip

\noindent cryptoperversions -> perversões crípticas 

\smallskip

\noindent daydream -> devaneio

\smallskip

\noindent delusion -> delírio

\smallskip

\noindent disavowal -> negação

\smallskip

\noindent dread -> medo

\smallskip

\noindent drive -> pulsão, impulso

\smallskip

\noindent driven -> compulsivo

\smallskip

\noindent driving (need) -> (necessidade)  instintiva

\smallskip

\noindent drug dependence -> toxicomania

\smallskip

\noindent drug dependence, heroin -> toxicomania por heroína

\smallskip

\noindent earliest stages (of infancy) -> primórdios (da infância)

\smallskip

\noindent enfold -> abrigar

\smallskip

\noindent experiment -> evidência, experimento, cobaia 

\smallskip

\noindent \textit{faute-de-mieux} -> não traduzi (em francês no original)

\smallskip

\noindent female (contexto cultural) -> menina, mulher, sexo feminino

\smallskip

\noindent female (contexto da biologia) -> fêmea

\smallskip

\noindent femaleness - condição de fêmea, de mulher

\smallskip

\noindent feminize -> feminizar

\smallskip

\noindent forcefulness -> uso da força

\smallskip

\noindent fuck, fucked -> (Quando usado pela paciente G., Capítulo 9:
comer, ser comida - na tradução \ \ \ \ \ \ francesa,
``baiser'', être baisée'')

\smallskip

\noindent foreclose -> barrar, forcluir

\smallskip

\noindent fullfillment -> gratificação

\smallskip

\noindent imprinting -> impregnação, estampagem, \textit{imprinting}

\smallskip

\noindent infancy -> primeira infância [(...) his emphasis on
infantile and childhood sexuality]]

\smallskip

\noindent \ \  [``sua ênfase na sexualidade da criança tanto enquanto
bebê (o \textit{infans}) como durante \ \ a infância'']

\smallskip

\noindent \textit{infans,} infant -> bebê; \textit{infans}

\smallskip

\noindent input -> estímulo

\smallskip

\noindent insight -> \textit{insight}

\smallskip

\noindent inspirational psychologists (such as May, Polanyi and Frankl)
-> psicólogos inspiracionistas (?) p. 211

\smallskip

\noindent intact -> íntegro

\smallskip

\noindent instinct -> instinto, pulsão (Conf.  \textit{Laplanche-Pontalis}: ``Deutsch:
Trieb; French: Pulsion; English: instinct ou \ \ \ \ drive; Espanhol: instinto
ou pulsão (Note-se que a Edição Standard inglesa preferiu traduzir ``Trieb''
por \ \ ``instinct'', afastando outras possibilidades como ``drive'' ou
``urge'').

\smallskip

\noindent joy -> prazer, gozo, felicidade transbordante

\smallskip

\noindent male (contexto da biologia) -> macho

\smallskip

\noindent male -> menino, homem, do sexo masculino

\smallskip

\noindent maleness (contexto cultural) -> virilidade

\smallskip

\noindent maleness (contexto da biologia) -> condição de macho

\smallskip

\noindent merging -> fusão, imersão, amálgama

\smallskip

\noindent mothering -> maternagem, cuidados maternos

\smallskip

\noindent OBS (nonpsychotic) -> síndrome cerebral orgânica
não-psicótica

\smallskip

\noindent parent -> progenitor

\smallskip

\noindent priming -> condicionamento

\smallskip

\noindent `psychophysiologic skin disorder' -> dermatite psicossomática

\smallskip

\noindent `psychophysiologic respiratory disorder (asthma) -> distúrbio respiratório psicossomático (asma)

\smallskip

\noindent repress -> recalcar

\smallskip

\noindent repression -> recalque

\smallskip

\noindent shizophreniform psychosis -> psicose esquizoide

\smallskip

\noindent shaping -> moldagem (\textit{`façonnement}' em francês)

\smallskip

\noindent shut -> confinar

\smallskip

\noindent social maladjustment -> desajuste social

\smallskip

\noindent splitting of the ego -> clivagem do ego

\smallskip

\noindent suppressed -> reprimida

\smallskip

\noindent supress -> reprimir, suprimir

\smallskip

\noindent tea-room  promiscuity -> promiscuidade de banheiro público
(Dalzell \& Victor 2007, p. 642) ``tearoom; t-room,
noun -> a public toilet. From an era when a great deal of
homosexual contact was in public toilets; probably an abbreviation of
`toilet room'.)

\smallskip

\noindent to act out -> atuar, levar ao ato

\smallskip

\noindent to harm -> fazer mal, causar dano, ferir, machucar

\smallskip

\noindent undoing -> anulação

\smallskip

\noindent upsurge -> surto, epidemia

\smallskip

\noindent variant -> variação

\smallskip

\noindent victimization -> opressão, represália, ação de vitimar

\cleardoublepage

