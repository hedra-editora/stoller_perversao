\chapter[Apresentação, por Flávio Carvalho Ferraz]{Apresentação}
\markboth{flávio c.~ferraz}{apresentação}
\bigskip

\hfill\textit{Flávio Carvalho Ferraz}\footnote{ Psicanalista, membro do
Departamento de Psicanálise do Instituto Sedes Sapientiae (São Paulo) e
livre-docente pelo Instituto de Psicologia da Universidade de São Paulo.}
\bigskip


\textsc{Robert J.~Stoller,} psiquiatra, psicanalista, professor e autor profícuo,
dedicou a maior parte de sua pesquisa, na Faculdade de Medicina da
Universidade da Califórnia, ao tema da sexualidade, especialmente às
perversões e à chamada problemática do \textit{gênero}, concernente à
formação da identidade sexual. Ainda que morto precocemente em um acidente,
deixou uma obra extensa, publicada, na maior parte, nos anos 1970. Seu
trabalho representou indubitavelmente um avanço significativo no estado em
que se encontrava a teoria da sexualidade na psicanálise, tendo
influenciado autores de destaque, como Joyce McDougall, na França. Além
disso, ele manteve uma fértil interlocução com autores que com ele
compartilhavam de duas qualidades raras e louváveis: a criatividade e a
liberdade de pensamento. Entre estes, inclui-se Masud Khan, na Inglaterra,
e Otto Kernberg, nos Estados Unidos.

A palavra \textit{perversão}, contida já no próprio título deste livro,
merece algumas considerações. Foi utilizada por Freud e se manteve no
vocabulário da maior parte dos psicanalistas das gerações subsequentes,
inclusive Lacan. Mas num certo momento, passou a ser contestada e evitada
por alguns autores de diferentes origens, não só da psicanálise, mas também
da psiquiatria e das ciências humanas em geral. Usando uma linguagem clara
e sem volteios e evitando até mesmo o emprego de termos próprios ao
vocabulário psicanalítico, Stoller discordava dos que propunham o abandono
do termo. Como, para ele, a presença da \textit{hostilidade} em relação ao
objeto era a característica central do ato perverso, o uso dessa palavra
nada tinha de impróprio. As tentativas de abolição do termo
\textit{perversão}, para Stoller, derivaram de um certo tipo de postura
intelectual --- que hoje poderíamos chamar de “politicamente correta” ---
preocupada com as conotações morais que ele pode carregar. O argumento
empregado correntemente era o de que essa palavra era usada com intuito
repressivo e moralizante. Uma outra tendência, favorável à manutenção do
termo \textit{perversão}, teria como motivação exatamente a conservação de
uma palavra que carrega o sentido de \textit{pecado}, a fim de preservar a
antiga moralidade.

Stoller discordava de ambas as tendências. Sua investigação tem como escopo
privilegiado a definição \textit{clínica} do termo. Todas as suas
formulações decorrem desta disposição. Para ele, o fator-chave na definição
da perversão é a \textit{hostilidade}. Assim, ao insistir no emprego do
termo, ele mantinha uma honestidade intelectual que resultava numa
liberdade de expressão.\footnote{ O leitor perceberá, na leitura deste
livro, que Stoller não se rendia aos imperativos do “politicamente correto”
professado por um certo tipo de “esquerda”. O que não significa,
evidentemente, que ele seja complacente com os esforços moralizadores da
“direita”. Aliás, muitíssimo ao contrário! Na parte final do livro, o
leitor testemunhará a forma contundente com que nosso autor se opõe ao
conservadorismo repressor na esfera do comportamento sexual.} Vejamos três
pontos essenciais de sua concepção de \textit{perversão} --- que o leitor
encontrará desenvolvidos em detalhes neste livro --- a partir dos quais
pretendo destacar outros desdobramentos:


\begin{enumerate}
\item Perversão é o resultado de um interjogo essencial entre hostilidade e
desejo sexual. Ora, tal definição, \textit{grosso modo}, afirma uma certa
proximidade entre a noção psicanalítica e a noção corrente de perversão.
Sem fazê-las necessariamente coincidentes, Stoller, no entanto, demonstra
que a perversão, na acepção psicanalítica, comporta elementos hostis, tal
como o uso comum do termo tem por suposto.

\item O perverso é acometido por uma infindável sensação de que é sujo,
pecaminoso e anormal.

\item O uso da palavra \textit{perversão} serve para que os indivíduos
“normais” projetem sobre outros suas próprias tendências perversas,
elegendo bodes expiatórios. A propósito desta ideia, Stoller vai
desenvolver uma hipótese --- bastante particular e controvertida --- sobre a
necessidade social da perversão. Para ele, a fim de se manter a moral
sexual social, é preciso que se crie uma categoria na qual se enquadrem os
desviantes. Tal separação estaria a serviço da própria “normalidade”
heterossexual e do imperativo da procriação.
\end{enumerate}

Para Stoller, a perversão é um produto da ansiedade, sendo que o
comportamento perverso molda-se a partir de remanescentes e de ruínas da
história do desenvolvimento libidinal, particularmente da dinâmica
familiar. Ele acredita que, se pudéssemos, de modo utópico, saber
\textit{tudo} o que aconteceu na história do sujeito que investigamos,
então encontraríamos certamente os eventos históricos que se fazem
representar em detalhes no ato sexual manifesto do perverso. Poderíamos,
assim, saber como e por quê uma determinada pessoa elevou suas experiências
sexuais precoces --- aquilo que mais prazer lhe causou --- à condição de parte
do cenário perverso.

A hipótese de Stoller é a de que a perversão é uma fantasia posta em ato por
meio de uma estrutura defensiva construída gradualmente através dos anos,
com a finalidade de preservar o prazer erótico. O desejo de preservar tal
gratificação seria proveniente de duas fontes: um extremo prazer físico
que, pela sua própria natureza, demanda uma repetição, e a necessidade de
manutenção da identidade.

Aproximamo-nos, assim, de mais um ponto fundamental da teoria de Stoller
sobre a perversão, que é também uma de suas contribuições mais originais: a
ideia de que a montagem da cena perversa não visa somente à negação da
castração, mas visa, sobretudo à manutenção da identidade sexual
ameaçada.\footnote{ Tal ideia de Stoller foi definitiva para a teorização
que Joyce McDougall viria a fazer acerca da perversão. Ver \textsc{mcdougall}, J. (1978)  
\textit{Em defesa de uma certa anormalidade: teoria e clínica psicanalítica.} 
Porto Alegre: Artes Médicas, 1989.}

Pesquisador e teórico freudiano \textit{sui generis}, Stoller levou às
últimas consequências aquele Freud que postulava a realidade do trauma na
determinação da psicopatologia e, por conseguinte, nas peculiaridades da
formação do sintoma, coisa que o próprio criador da psicanálise descartou
precocemente. É assim que, para Stoller, a perversão é o resultado de uma
determinada dinâmica familiar que, induzindo medo, força a criança imersa
na situação edípica a evitá-la. O desfecho do conflito edípico não seria,
portanto, a dissolução do mesmo pela via do recalcamento, mas sim a sua
evitação, o que adiaria \textit{ad infinitum} seu desfecho, mantendo-o
suspenso. É verdade que tal ideia, ainda que não coincida exatamente com o
conceito freudiano de \textit{recusa} (\textit{Verleugnung}), guarda certa
familiaridade com o mesmo.

Stoller, neste sentido, leva extremamente a sério as ideias do Freud
não-organicista, que debitava na conta da \textit{experiência} a formação
de toda e qualquer identidade sexual. Assim, não haveria uma sexualidade
\textit{natural}, dada pelos imperativos biológicos, mas toda forma
assumida pela sexualidade seria uma \textit{construção} baseada na história
das relações objetais, ou seja, seria \textit{contingente}. A
heterossexualidade também seria uma aquisição. De acordo com Stoller,
jamais chegaríamos a compreender a perversão se a tomássemos como um desvio
patológico, acreditando ser a heterossexualidade algo dado e natural.

Tal ponto de vista, entretanto, não impede que nosso autor encare a
perversão como uma \textit{aberração} na qual o ódio está presente na
qualidade de elemento estruturante primordial. Este seria um outro
ponto-chave de sua teoria da perversão: como já diz o próprio título desta
obra, ela é a \textit{forma erótica do ódio}, pois aquilo que preside o ato
perverso é o desejo de ferir ou danificar o outro: na prática, trata-se de
uma fantasia posta em ato.

Coincidindo com aquilo que Freud afirmava nos \textit{Três ensaios sobre a
teoria da sexualidade},\footnote{ \textsc{freud}, S. (1905) “Três ensaios sobre a
teoria da sexualidade”. \textit{E.S.B.}. Rio de Janeiro, Imago, 1980,
v.7.} Stoller observa que o modo de obtenção de prazer do perverso é
rígido e invariável; mas acrescenta que tal prática é necessária e
primariamente motivada pela \textit{hostilidade}. É sobre este ponto que
ele vai insistir, recuperando, sem contudo dizê-lo, a afirmação de Freud ---
feita nos mesmos \textit{Três ensaios} --- de que, na sexualidade perversa,
os componentes pré-genitais (orais e sádico-anais) constituíam o eixo
central, enquanto que, na sexualidade dita “normal”, era a genitalidade que
tinha a primazia.

Mas aqui cabe fazer uma distinção importante no vocabulário de Stoller:
\textit{hostilidade} não coincide com \textit{agressividade}. Enquanto a
primeira é um estado no qual se deseja ferir e causar dano e dor a um
objeto, a segunda não estaria impregnada de um sentido como tal, ou seja,
implicaria apenas a presença de uma força ou potência que precisa ser
escoada.

A hostilidade, na perversão, assume a forma de uma fantasia de
\textit{vingança} --- escondida em ações que a dissimulam --- que tem a função
precípua de converter um \textit{trauma infantil} em um \textit{triunfo
adulto}. Por ora, cabe acrescentar que esta operação de conversão da cena
traumática infantil (portanto, vivida passivamente) em triunfo vingativo
adulto (imaginado ativamente) é responsável pela produção da excitação
sexual. E, para incrementá-la, maximizando-a, há também que se montar uma
cena sexual que assuma o caráter de um \textit{ato arriscado}.

Assim caracterizando a formação perversa, Stoller dispensa a necessidade de
defini-la de acordo com a anatomia usada pelo sujeito em seu ato sexual, do
objeto escolhido, dos parâmetros da moralidade social estabelecida ou do
número de pessoas que fazem uso dela: o que importa verdadeiramente, na
definição do que é uma montagem perversa, \textit{é o significado que ela
assume para a pessoa que a pratica}.

Tal montagem perversa reproduz a situação traumática efetivamente vivenciada
na infância, que deve ser revivida e “corrigida” em ato nos detalhes da
cena perversa. A perversão, assim, é a revivescência de um trauma
\textit{sexual} --- e não de um outro tipo qualquer --- ocorrido quer sobre a
\textit{área sexual} (anatômica), quer sobre a \textit{identidade de
gênero} (como, por exemplo, no caso da criança que é tratada como se
pertencesse ao outro sexo biológico). No ato perverso, o passado é evocado
inconscientemente: neste momento, o trauma é transformado em prazer,
vitória e orgasmo. É como se a história fosse relembrada em ato, mas
contada com um desfecho oposto ao que teve na cena traumática real, agora
de modo favorável à vítima.\footnote{ Woody Allen, no filme
\textit{Desconstruindo Harry} (1997), propõe algo semelhante: um escritor,
ao escrever suas memórias, “corrige” a realidade passada dando um desfecho
diferente do que realmente ocorrera, narrando, portanto, uma história
distorcida e favorável a si próprio.} A passividade transforma-se em
atividade e a vingança se efetua sobre um objeto escolhido para representar
a criança vitimada. Mas a necessidade que o perverso tem de repetir sempre
(compulsivamente) e da mesma forma (repetitivamente) sua cena sexual atesta
a impossibilidade de tal ato pôr o sujeito, efetivamente, a salvo do
perigo. A memória do trauma é inconsciente e não cessa de manifestar-se e
de exigir uma defesa.

Para Freud, a excitação sexual vivida precocemente por uma criança, pelas
mãos de um adulto, representava um trauma e contribuía para a consolidação
de uma perversão. Stoller concorda com este ponto de vista, mas apenas nos
casos em que tenha havido \textit{muita estimulação} e \textit{pouca
descarga} ou então \textit{um severo sentimento de culpa} como decorrência.
Este tipo de experiência, sentida como traumática, é que deve ser
transformada imaginariamente, por meio do \textit{ritual perverso}, em uma
aventura bem-sucedida.

O ritual perverso adequado à conjuração do trauma é construído, através de
ensaio e erro, na história de vida do sujeito. Se porventura ocorrer uma
falha na construção deste ritual, então a vida sexual será marcada pela
falta de interesse sexual e pela ansiedade, manifestas sob a forma de um
distúrbio de potência. A partir desta constatação, Stoller vai concluir que
uma das finalidades da estruturação de uma perversão é a manutenção da
possibilidade de se obter prazer sexual.

A introdução do sentido de \textit{risco} no ritual visa ao propósito do
sujeito de lutar contra o desinteresse sexual que poderia resultar de sua
história traumática. A função do risco é exatamente incrementar a
excitação e manter a gratificação sexual. Mas o risco não pode ser extremo:
a situação deve estar, em alguma medida, sob controle. O risco é baixo, ou,
melhor dizendo, o que efetivamente importa é a impressão de que se está
correndo risco.

A função do risco como fator de excitação e de prazer sexuais é inerente à
dinâmica da vingança. A falta de interesse sexual seria o resultado de uma
ausência de risco. A excitação é o produto de uma oscilação entre a
possibilidade de falhar (que é pequena) e a antecipação do triunfo (que é
grande). Assim, a perversão poderia ser também descrita, em uma de suas
facetas, como um complicado atalho que passa pelo perigo em direção à
gratificação sexual triunfante. Lembremos que, na perversão, o prazer
sexual é salvo pela erotização do risco, quando há uma revivescência
inconsciente do trauma mas seu desfecho é alterado em fantasia.

Stoller supõe que, quando o trauma é completo, talvez não resulte dele uma
perversão, e sim que a função sexual seja simplesmente apagada, isto é, dela
resulte a abolição da função sexual ou a impossibilidade de exercê-la.
Assim, a perversão resulta do \textit{estrago} da função, mas não de sua
\textit{destruição}. Alguma esperança subsiste.

Um outro ponto relevante da obra de Stoller concerne ao estatuto do objeto
na perversão. Todo o trabalho de construção da fantasia a ser encenada pelo
perverso tem por corolário a \textit{desumanização} do objeto
sexual.\footnote{ Joyce McDougall, Janine Chasseguet-Smirgel e Masud Khan
fazem, cada um à sua moda, formulações semelhantes a esta quando tratam do
estatuto do objeto na perversão. Para Masud Khan, por exemplo, o objeto não
chega a ser inteiramente o outro --- diferente, separado e independente ---,
mas algo intermediário entre o próprio eu e o mundo externo, ou seja,
aquilo que Winnicott denominou \textit{objeto transicional}.} Este não é e
nem pode ser --- sob pena de colocar em risco a montagem perversa --- encarado
como pessoa ou alteridade. Muito embora, na prática, o objeto seja uma
pessoa real com sua personalidade, o perverso nele procura vislumbrar uma
criatura sem humanidade ou simplesmente um fragmento anatômico ou de
personalidade. Isto explica porque o objeto é sempre descartável (um
“palito de fósforo que se queima”, como costumava dizer um paciente
meu)\footnote{ Este caso clínico encontra-se relatado no capítulo 3 de
\textsc{ferraz}, F.C., \textit{Perversão} (São Paulo, Casa do Psicólogo, 2000), e
também no capítulo “A possível clínica da perversão” de \textsc{ferraz}, F.C. \&
\textsc{fuks}, L.B. (orgs.), \textit{A clínica conta histórias} (São Paulo, Escuta,
2000).} e também nos mostra a razão pela qual a promiscuidade faz parte
quase necessária da vida sexual do perverso. Em relação ao fragmento
anatômico, Stoller parece referir-se ao que Freud chamava de \textit{objeto
parcial}: um órgão sexual ou qualquer outra parte do corpo do parceiro.

Poderíamos nos perguntar agora, caracterizada a necessidade que o perverso
tem de montar sua cena sexual, como é que o prazer é aí introduzido. De
acordo com Stoller, ele se consuma por obra e graça da \textit{fantasia}: é
por meio desta que o trauma pode ser desfeito ou anulado. Na reconstrução
modificada de sua história, que se processa na montagem da cena sexual, os
devaneios (sonhos diurnos) têm o papel de contribuir para a consecução do
prazer por meio de uma série de características que assumem a fim de
“corrigir” o passado que é “rememorado inconscientemente”. Vejamos o que
rege a formação de tais devaneios:

\begin{itemize}
\item o perigo de o trauma repetir-se é retirado;
\item elementos que estimulam o risco são neles incluídos, introduzindo uma
excitação pela tensão;
\item um “final feliz” é garantido, provando ao sujeito que ele evitou o
trauma, ou até mesmo que foi ele quem traumatizou os que originalmente o
atacaram;
\item por fim, quando tais devaneios ligam-se à excitação sexual e ao
orgasmo, instala-se um círculo vicioso que motiva o perverso a repetir seu
ato indefinidamente.
\end{itemize}


Uma outra questão que vale a pena examinar, se quisermos ter uma visão mais
ampla da teoria geral da perversão de Stoller, é a maneira particular como
ele compreende o complexo de Édipo freudiano, produzindo uma verdadeira
inversão das afirmações de Freud sobre os avatares da constituição da
identidade sexual do menino quando comparados aos da menina.

Enfatizando a qualidade da presença dos pais como modelo identificatório
para os filhos, Stoller subverte a tese freudiana de que a feminilidade da
menina é um destino identificatório cujo caminho é mais longo e tortuoso do
que aquele verificado no caso da masculinidade do menino.\footnote{ Apenas
para relembrar, Freud, nos textos dos anos 1920 sobre o complexo de Édipo,
afirma que o menino se encontra, de partida, na posição heterossexual,
visto ser a mãe o primeiro objeto investido; já a menina, partindo de uma
posição homossexual, teria, para atingir a feminilidade, de levar a cabo
uma dupla mudança: a primeira seria a \textit{objetal} (abandonando o
investimento libidinal sobre a mãe para dirigi-lo a uma figura masculina) e
a segunda seria a da \textit{zona erógena}, quando o clitóris (visto como
órgão correlato ao pênis e, portanto, pertencente à sexualidade masculina
original da menina) deve ceder lugar à vagina como órgão proeminente na
obtenção do prazer.} Para Freud, a primeira relação do menino, por ter a
mãe como objeto, teria um caráter heterossexual, enquanto que, para a
menina, a primeira relação seria homossexual. \textit{Stoller não
privilegia, como Freud, o investimento sexual primário como determinante da
posição sexual primária. Para ele, o que importa é a posição
identificatória inicial}. Segundo este ponto de vista, então, tanto o
menino quanto a menina estariam originariamente identificados com a mãe.
Assim, para o menino, atingir a masculinidade implica separar-se dela,
rompendo a unidade mãe-filho. As condições para que tal processo ocorra de
maneira equilibrada são dadas pela atitude materna: se a mãe força uma
intimidade exagerada com seu filho, ela estará colocando um obstáculo à
formação de sua identidade masculina.

Como fica patente, este modo particular de compreender a formação da
identidade sexual contrasta com postulações centrais da teoria sexual de
Freud. Em Stoller, não há primazia do pênis, mas do seio e da capacidade
procriativa da mulher. Ele discorda textualmente da importância que Freud
atribuía ao pênis, entrevendo nos caracteres femininos os verdadeiros
atributos que uma criança deseja possuir. Uma consequência deste modo de
conceber a identificação sexual é a conclusão de que os homens, quando em
fantasia atribuem um falo à mulher, não o fazem para negar a inferioridade
feminina, mas sim a superioridade da mulher! Portanto, a descrição
freudiana da formação do fetiche e de sua função precisaria ser reescrita
se quiséssemos reformulá-la nos termos da teoria de Stoller.

Para a menina, originalmente identificada à mãe, não haveria necessidade de
mudanças tão intensas para a aquisição da feminilidade. Já para o menino,
existe a necessidade de uma desidentificação que é altamente ansiógena. Se
ele permanece unido à mãe, sua masculinidade não é atingida. E como este
destino identificatório é intensamente cobrado pela sociedade, ele se vê na
obrigação de obtê-lo e, consequentemente, na angústia diante da
possibilidade de fracassar em tal empreitada.

É na atitude da mãe frente à necessidade de separação que tem o seu filho
que vamos localizar, segundo Stoller, os fatores etiológicos da perversão.
O trauma sexual necessário à consolidação de uma perversão, repito, deve
ocorrer sobre a anatomia sexual propriamente dita ou sobre a identidade
sexual. Portanto, uma atitude da mãe contrária à separação do filho
(desidentificação) constituiria um trauma da segunda espécie. No extremo de
tal atitude encontraríamos uma mãe que nem sequer permite a seu filho a
entrada no conflito edípico, quando então ambos protagonizam uma relação
idílica da qual o pai se exclui completamente. Stoller vê aí a gênese do
transexualismo,\footnote{ Não entrarei aqui nos detalhes da teoria de
Stoller sobre o transexualismo, tema que o ocupou em muitos de seus
trabalhos e sobre o qual ele deixou uma contribuição original. Sobre isto,
ver \textsc{garcia}, J.C., \textit{Problemáticas da identidade sexual} (São Paulo,
Casa do Psicólogo, 2001), obra que se ocupa de tal assunto e que inclui o
ponto de vista de Stoller, trazendo, inclusive, um apanhado das críticas
que ele recebeu de Moustapha Safouan.} que estaria localizada, assim, em
um estágio praticamente pré-sexual. Não se trataria de um fracasso na
elaboração do conflito edípico, mas de algo mais regredido, que seria a
“não entrada” no mesmo.

Neste ponto, o trabalho de Stoller coloca uma franca objeção ao pensamento
freudiano, que pode ser assim resumida: se a formação da masculinidade,
como quer Freud, é mais simples e linear do que a formação da feminilidade,
por que a perversão incide com maior frequência nos homens?

Para Stoller, a perversão masculina é, no fundo, um \textit{transtorno de
gênero} construído sobre uma tríade da hostilidade: \textit{raiva},
\textit{medo} e \textit{vingança}. O menino tem raiva da identificação
inicial com a mãe, tem medo de não conseguir escapar de sua órbita e almeja
vingar-se dela porque ela o colocou nesta condição.

Na perversão, se examinarmos detalhadamente a fantasia que a subjaz,
encontraremos em seu âmago os elementos remanescentes das experiências
individuais infantis, no contato com pessoas do mundo real, que provocaram
tal configuração psíquica. E, no centro de tal formação, encontra-se a
\textit{hostilidade}. Este é o ponto fundante da concepção stolleriana da
perversão.

A hostilidade tem o propósito de fazer com que o sujeito se sinta superior e
triunfante sobre o outro. Se nas práticas sexuais sádicas isto é evidente
por si só, em outras variantes da perversão, no entanto, tal asserção não é
facilmente visível. Stoller explica sob este prisma, por exemplo, a
promiscuidade comum na dinâmica da perversão: ela seria uma resultante da
hostilidade, pois o interesse do perverso encontra-se na sedução e não no
amor. É assim que, em Don Juan, a gratificação não provém do prazer do ato
sexual ou da intimidade estabelecida com outra pessoa, mas exclusivamente
do ato de sedução.

No caso do masoquista --- que nos desafia na manutenção dessa mesma lógica ---,
temos que ele nunca é uma verdadeira vítima, pois sabemos que o perverso
não perde o controle na cena sexual que monta. O cenário é, assim, montado
com o propósito de forjar um \textit{sofrimento fraudulento}. Ademais, não
podemos esquecer que nas cenas sadomasoquistas há uma situação
identificatória complexa que não nos permite tomar ingenuamente cada
participante como jogando com a própria identidade. \textit{Os lugares
identificatórios são flexíveis e intercambiáveis, permitindo que se goze o
gozo atribuído em fantasia ao outro}. Portanto, é perfeitamente possível
pensarmos em uma identificação com o agressor ou, ainda, em um sentimento
de superioridade da vítima sobre o algoz, quando a hostilidade é mantida
secreta na fantasia.

No fetichismo, a hostilidade pode parecer ausente. Mas ela também existe e
está dirigida ao adulto responsável pelo trauma, isto é, àquele que foi
percebido como o veiculador da ameaça de castração ou que atacou a
identidade sexual da criança (que Stoller exemplifica através da mulher que
veste o menino com trajes femininos). O fetiche, salvando a potência
sexual, faz com que o fetichista se sinta triunfante em sua potência
masculina \textit{exatamente no ponto em que percebeu que queriam que ele
falhasse}.

O ápice do prazer, na perversão, coincide com o momento em que a parte
central do trauma está sendo encenada no ato sexual. Este é um momento de
grande suspense, pois é quando o máximo risco parece estar presente e,
portanto, antecede o triunfo colossal que está por vir. Stoller lembra que
este mesmo tipo de sensação ocorre em outras atividades que não as
estritamente sexuais, como no caso dos triunfos contrafóbicos verificados
na prática de esportes perigosos. O perverso, no entanto, mantém a noção de
que seu triunfo acontece na fantasia e, neste sentido, ele difere do
psicótico. Assim, a realidade do trauma não é efetivamente removida e, por
isso, ele deve recomeçar tudo novamente. Daí o caráter compulsivo da
prática perversa. O orgasmo é vivido como excepcional porque é revestido
deste caráter de triunfo, isto é, da sensação de estar a salvo da situação
traumática e do risco corrido.

Além da hostilidade, Stoller aponta um outro componente que toma parte na
montagem da cena perversa: trata-se do \textit{mistério}, que remonta, em
última instância, ao mistério que reveste a sexualidade, especialmente para
uma criança. Afinal, se pensarmos sob o ponto de vista cultural, é inegável
que existe uma mistificação social e cultural da anatomia, das funções e
dos prazeres sexuais que aguça a curiosidade infantil e atira a criança ao
afã de produzir fantasias e teorias sexuais, como demonstrou Freud no
artigo \textit{Sobre as teorias sexuais das crianças}.\footnote{ \textsc{freud}, S.
(1908) “Sobre as teorias sexuais das crianças”. \textit{E.S.B.}. Rio de
Janeiro, Imago, 1980, v.9.}

Na perversão, o papel do mistério e do perigo é aumentado porque a criança
foi traumatizada ou superestimulada explicitamente no exato ponto
misterioso: os genitais ou o seu desejo de investigá-los. Se o trauma
incide sobre outras partes do corpo ou sobre outras funções, então o seu
resultado é uma neurose. Para demonstrá-lo, Stoller recorre a
Fenichel,\footnote{ \textsc{fenichel}, O. (1945) \textit{Teoria psicanalítica das
neuroses}. Rio de Janeiro, Atheneu, 1981.} para quem os indivíduos nos
quais a ansiedade de castração foi provocada de forma abrupta e intensa são
candidatos potenciais à perversão.

\asterisc

Ao procurar estabelecer uma “genealogia” teórica do pensamento de Stoller,
deparamo-nos com uma primeira dificuldade, decorrente de seu estilo
peculiar: esta genealogia deve ser primordialmente inferida, à medida que
ele não a explicita expressamente de modo constante. Além do mais, Stoller
é um autor que emite conceitos genuinamente próprios, entremeados aos de
outros autores ou afirmados por meio da oposição a formulações clássicas.

Stoller recusa o uso excessivo de termos próprios do vocabulário
psicanalítico, preferindo colocar suas ideias sobre o funcionamento mental
em uma linguagem ordinária e não-hermética ou, segundo ele mesmo, de modo
com que todos possam compreender o que se quer dizer (em outros trabalhos,
ele chega a ser irônico em relação ao modo empolado como os psicanalistas
usam um vocabulário “técnico” para dizer coisas que poderiam perfeitamente
ser ditas com palavras comuns).\footnote{ Ver, por exemplo, seu livro
\textit{Excitação sexual: dinâmica da vida erótica} (São Paulo: Ibrasa,
1981) ou o prefácio ao livro \textit{The new ego}, de Nathan Leites (New
York: Science House, 1971).} Considerando-se a força que tem a tradição
vocabular de uma disciplina ou de uma ciência, trata-se de um
empreendimento difícil, mas que o autor consegue levar a cabo de maneira
exemplar.

A obra de Freud é a referência maior de Stoller: é em torno dela, e não da
de algum outro autor da segunda geração dos grandes teóricos da psicanálise
que ele organiza seu pensamento. É neste sentido que seu trabalho aparenta
não possuir muitos intermediários entre si e o texto freudiano, muito
embora sejam visíveis alguns pontos de apoio laterais para suas ideias
sobre a perversão, particularmente em Masud Khan\footnote{ \textsc{khan}, M.M.R.
\textit{Alienation in  perversions.} London, The Hogarth Press, 1979.}
(ambos, Stoller e Khan, utilizam em comum três autores que os precedem na
investigação da perversão, que são W.H. Gillespie, E. Glover e P.
Greenacre). É possível perceber também que, em alguns pontos em que se
afasta de Freud, Stoller busca mostrar-se coincidente com Winnicott, o que
não significa que sua teoria seja \textit{baseada} na obra deste autor. Do
mesmo modo, é visível que algumas de suas contestações de conceitos-chave
de Freud, especialmente no que concerne ao complexo de Édipo e à primazia
do falo, assemelham-se a postulações kleinianas; contudo, não são
referências explícitas a elas.

De qualquer maneira, ao afirmar que a referência a Freud é constante em
Stoller, tomei a precaução de dizer que é “em torno” da obra freudiana que
este autor se organiza. Usei propositalmente a expressão “em torno” porque
tal referência não vai significar uma filiação direta e total. Em sua obra,
a referência pode ter por objetivo a afirmação do oposto, como nos casos já
citados do trauma e da primazia do falo.

A abordagem que Stoller faz da perversão tem aspectos realmente originais,
quando vistas dentro de um quadro comparativo dos principais ramos de
desenvolvimento da psicanálise a partir de Freud. Stoller é um freudiano à
própria moda, que valoriza determinados pontos da teoria de Freud (como a
do trauma) e contesta frontalmente outras noções muito caras ao fundador da
psicanálise. A compreensão da perversão através do eixo da hostilidade é a
marca maior de sua teoria e também seu aspecto mais original. Não que a
ligação entre perversão e hostilidade seja necessariamente uma novidade em
psicanálise, mas a extensão explicativa que isto toma em sua obra é que
delineia o caráter original de sua contribuição.

Partindo do pressuposto de que é o ódio que sustenta a formação perversa,
Stoller coloca-se uma pergunta de profunda implicação epistemológica para a
metapsicologia: \textit{como o ódio é erotizado?} Ou, de outra forma: como
a vida erótica é investida pelo ódio e pela angústia? Esta é uma pergunta
que incide sobre a lógica do funcionamento psíquico. Trata-se de um tipo de
problema que se coloca para a psicologia, sobre o qual a psicanálise teve
de debruçar-se inúmeras vezes a fim de conferir uma consistência
epistemológica a seus postulados. É o que Freud fez, por exemplo, no artigo
``Uma criança é espancada'',\footnote{ \textsc{freud}, S. (1919) ``\,`Uma criança
é espancada': uma contribuição ao estudo da origem das perversões sexuais''.
\textit{E.S.B.}. Rio de Janeiro, Imago, 1980, v.17.} quando se pôs a
enfrentar o enigma do masoquismo, que desafiava uma lógica mais superficial
--- a da consciência --- ao ligar o prazer à dor.

Para chegar a uma explicação suficiente sobre a ligação do ódio e da
hostilidade com a formação perversa em Stoller, é necessário passar, então,
pela noção de \textit{trauma}. O ódio, para ele, não é um sentimento “dado”
ou primário que vem influenciar a vida psíquica e determinar o aparecimento
de mecanismos defensivos. Não se trata, em absoluto, deste ódio quase
metafísico, mas sim de um ódio reativo e \textit{causado}, com raízes na
experiência real. Ora, aqui fica evidente que Stoller dialoga com
Winnicott, cuja obra tem como marca central a valorização do ambiente na
constituição da subjetividade.

A cena sexual perversa para Stoller é, em essência, uma \textit{fantasia
atuada}. Cada uma de suas partes e cada um de seus detalhes guardam
relações com a cena infantil traumática da qual o sujeito deseja livrar-se.
Tal ideia, que norteia toda a teorização que uma autora como Joyce
McDougall\footnote{ \textsc{mcdougall}, J. (1978) “A cena sexual e o espectador
anônimo”. In: \textit{Em defesa de uma certa anormalidade: teoria e clínica
psicanalítica}. Porto Alegre, Artes Médicas, 1989.} fará da perversão,
encontra um correlato na postulação de Masud Khan de que o ato sexual
perverso é um sonho que, não podendo ser sonhado, é colocado em ato. A
fórmula matricial desta construção, evidentemente, é clássica: encontra-se
em Freud, para quem o sintoma, como qualquer outra formação do
inconsciente, reconta uma história e tem uma função defensiva.

Se Freud abandonou a teoria da sedução --- do trauma sexual real --- para
justificar a gênese da neurose, dando ênfase ao papel da fantasia, Stoller,
ao contrário, vai afirmar que, se fosse possível investigar toda a vida
pregressa do paciente perverso, certamente localizaríamos o trauma, que
incidiu sobre o sexo anatômico ou sobre a identidade de gênero. Freud
pensou inicialmente em uma cena de sedução, da responsabilidade de um dos
pais, na etiologia da histeria. Mais tarde, ele compreendeu esta “cena”
como fantasia decorrente da chegada disruptiva da sexualidade
biológica. A origem da neurose, entretanto, permanece atrelada a este
momento, ainda que mítico. Mas o que Stoller postula, no caso da perversão,
é diferente: haveria, aí, um trauma real, que poderia ser encarado como o
próprio fator diferencial entre perversão e neurose.

A partir da realidade do trauma, quando o sexo anatômico ou a identidade de
gênero foram atingidos, o sujeito vai organizar, defensivamente, uma vida
sexual sintomática, cujo escopo é negar ilusoriamente o resultado do trauma
infantil, anulando seus resultados e “recontando”, por meio da montagem de
uma cena sexual perversa, a história da experiência traumática. Tal cena
deve ter um “argumento”, à moda da cena teatral, que, invariavelmente, é o
de afirmar a vitória sobre o agressor e a reversão de papéis. Em uma
palavra, o argumento deve ser o da \textit{vingança}, como define Stoller.

Assim, a vingança contida na cena perversa, além de ser em si mesma uma
descarga do ódio --- é uma espécie de tentativa de elaboração em que a
passividade é trocada pela atividade, exatamente como se dá no modelo que
Freud enuncia em \textit{Além do princípio do prazer}\footnote{ \textsc{freud}, S.
(1920) “Além do princípio do prazer”. \textit{E.S.B.}. Rio de Janeiro,
Imago, 1980, v.18.} ainda que, em momento algum, Stoller se refira a esta
ligação. O triunfo na perversão obedeceria, assim, ao mesmo princípio que
rege o aparecimento do brincar, ou seja, ao imperativo da elaboração de uma
experiência traumática através da transformação da passividade em
atividade. Mas ocorre que, na perversão, esta operação não pode ser contida
nos limites do plano psíquico, necessitando de um objeto do mundo externo
para ser atuada, isto é, encenada e transformada em um ritual (tal como o
“sonho atuado” de Masud Khan).\footnote{ \textsc{khan}, M.M.R. \textit{Op. cit.}}

Este objeto do mundo externo que se utiliza será alguém que, embora tenha
uma personalidade, é desinvestido de tal qualidade e reduzido a coisa
(objeto parcial). Stoller já vê, nesta operação, um indicador da atuação do
ódio, manifesto na \textit{desumanização} do objeto. Este ponto de visa
guarda alguma semelhança com a ideia de Masud Khan\footnote{ \textit{Idem.}} de que o
objeto do perverso é um \textit{objeto transicional}, intermediário entre o
eu o objeto externo, alvo da destruição e, ao mesmo tempo, fruto de uma
construção. Mas Stoller parece ter uma visão mais pessimista, pois afirma a
\textit{coisificação} do objeto e a atuação da hostilidade sobre o mesmo,
não fazendo menção aos aspectos “construtivos” supostamente contidos nesta
modalidade perversa de relação objetal, como faz Masud Khan.

Neste ponto talvez seja bom lembrar que a noção de \textit{dissociação} em
Stoller, no caso da perversão, aplica-se particularmente ao objeto, ainda
que seja, é certo, correlata à dissociação do ego. No caso do fetichismo ou
da escolha de uma parte do corpo do parceiro como objeto sexual, a
dissociação incidirá, sobretudo, entre a parte e a totalidade, sendo,
evidentemente, correlata à negação da plenitude da alteridade. Assim, o
“uso” que se faz do objeto só se torna possível em virtude desta
dissociação.

De todo modo, o fato de a cena perversa ser encarada por Stoller como uma
retrospectiva corrigida de uma cena traumática, não deixa de atestar o
fundo freudiano que preside sua teorização sobre a gênese e o sentido do
sintoma, pois o sintoma neurótico em Freud também obedecia à mesma lógica
interna. Senão, lembremos da cena montada pela mulher obsessiva,\footnote{ \textsc{freud}, S. 
(1913) “A disposição à neurose obsessiva”. \textit{E.S.B.}. Rio
de Janeiro, Imago, 1980, v.12.} que chamava a criada ao quarto nupcial
após ter derramado tinta vermelha no lençol: assim procedendo, ela
recontava a cena de fracasso do marido na noite de núpcias, de modo a mudar
os resultados reais, isto é, de modo a provar que houvera, de fato, a
defloração. Mas há uma diferença da maior importância entre o sintoma
neurótico e o sintoma perverso: como já afirmava Freud, enquanto que o
retorno do sexual recalcado na neurose provoca desprazer ao ego, na
perversão a fantasia sexual consciente provoca prazer (e aqui reside o
sentido do postulado freudiano da perversão como \textit{negativo da
neurose}).

Outra diferença entre as modalidades de sintoma, que merece a nossa atenção,
é a necessidade de atuação (\textit{acting-out}) que se verifica no sintoma
perverso; ou, melhor dizendo, o sintoma é, ele próprio, uma atuação, visto
que não se limita ao domínio do psíquico, mas envolve um outro em sua
consecução. Esta constatação nos alerta para a situação intermediária da
perversão entre a neurose e psicose.

Este fato não escapou a Christopher Bollas, que, concordando
com o postulado central de Stoller de que a perversão é uma \textit{forma
erótica do ódio}, lembra que a prática perversa consiste em uma

\begin{quote}
externalização por parte do perverso do processo instintivo, por meio do
qual sua compulsão interna deve estar vinculada a um outro real em uma
atuação em grande escala da preocupação social.\footnote{ \textsc{bollas}, C.
\textit{Hysteria}. São Paulo, Escuta, 2000, pp. 256--257.} 
\end{quote}

Ou seja: as bordas do ego são tênues, e a experiência conflitiva não pode ser 
contida dentro dos seus limites.

Finalmente, para concluir este voo panorâmico sobre o trabalho de Stoller a
respeito da perversão, cabe um comentário a respeito de sua visão do
complexo de Édipo, ponto sobre o qual emite opiniões arrojadas, se
comparadas às de Freud e, principalmente, se comparadas ao modo cauteloso
como os principais discípulos do fundador da psicanálise manifestavam suas
discordâncias em relação a ele. Stoller, como vimos, simplesmente inverte o
primado do falo, sem fazer disso bandeira de luta e sem tampouco atacar
Freud.\footnote{ É necessário lembrar, todavia, que Freud, depois de ter
circunscrito o conflito edípico em torno do medo da castração fálica,
viu-se na necessidade de estender seu olhar para as relações mais precoces
entre a criança sua mãe, relações que ele chamou de pré-edípicas; ele passa
a suspeitar que elas tinham influência decisiva na determinação da forma
assumida pelo complexo de Édipo e, portanto, na formação da identidade
sexual, especialmente para a menina.}

Stoller enfatiza a relação primária com a mãe e não a escolha objetal
primária. Assim, a feminilidade seria uma condição inicial, e a
masculinidade implicaria os esforços de separação e de desidentificação em
relação à mãe. Diz ele: 

\begin{quote}
É, então, possível, que o menino não comece heterossexual,
como Freud presumiu,  e sim que ele precise se separar do corpo de
mulher e da feminilidade da mãe, vivenciando um processo de
individuação em direção à masculinidade. A heterossexualidade, nos
homens, é uma conquista e, portanto, não, como disse Freud, um
pressuposto; se esta hipótese puder ser confirmada, então a
masculinidade não é a condição de ocorrência natural que Freud dizia
ser. Existe uma forma embrionária de feminilidade.\footnote{ Cf. p.~\pageref{cita1} desta edição.}
\end{quote}

Kernberg, tecendo comentários sobre a obra de Stoller, faz uma
interessante observação: iniciando seu raciocínio, lembra que Janine
Chasseguet-Smirgel\footnote{ \textsc{chasseguet-smirgel}, J. (1984) \textit{Éthique
et estéthique de la perversion}. Paris, Éditions du Champ Vallon, 1984.} é
uma autora que se preocupou com 

\begin{quote}
as consequências patológicas de grave
agressão precoce, no processo de separação-individuação da mãe,
particularmente nas vicissitudes da identificação da menina com funções
femininas e a tolerância do menino pela rivalidade edípica com o pai.\footnote{ \textsc{kernberg}, 
O F. \textit{Agressão nos transtornos de personalidade e nas perversões}. Porto Alegre, Artes Médicas, 1995,
pp. 272--273.}
\end{quote}


Prosseguindo, Kernberg vai render tributo à originalidade de Stoller que,
como representante de uma certa tendência psicanalítica norte-americana,
foi quem enfatizou a importância do processo de separação-individuação na
gênese da perversão, ao descrever 

\begin{quote}
o medo da feminilidade, nos homens, como
sendo uma expressão de suas ansiedades no que diz respeito à perda da
identidade sexual, derivada da identificação primária com a mãe, a qual
necessita ser desfeita, a fim de uma identidade masculina poder ser
desenvolvida.\footnote{ \textit{Idem.}}
\end{quote}

\asterisc

Estamos, como o leitor pode pressentir, diante de um autor que investigou
como poucos os mistérios da dinâmica da perversão e da sexualidade em
geral, tendo sido referência para autores tão diversos como Otto
Kernberg,\footnote{ \textsc{kernberg}, O. \textit{Op. cit.}} Lionel Ovesey \& Ethel
Person,\footnote{ \textsc{ovesey}, L. \& \textsc{person}, E. “Gender identity and sexual
psychopatology in men: a psychoanalytic analysis of homosexuality,
transsexualism and transvestism”. \textit{J. Am. Acad. Psychoanal.},
1:53--72, 1973.} Masud Khan,\footnote{ \textsc{khan}, M.M.R.  \textit{Op. cit.}}
Christopher Bollas,\footnote{ \textsc{bollas}, C.  \textit{Op. cit.}} Joyce
McDougall,\footnote{ \textsc{mcdougall} J. \textit{The many faces of Eros}. London,
Free Association Books, 1995.} Janine Chasseguet-Smirgel\footnote{ \textsc{chasseguet-smirgel}, 
J.  \textit{Op. cit.}} e Jacques André,\footnote{ \textsc{andré},
J. “Introduction: le masochisme immanent”. In: \textsc{andré}, J. (org.) 
\textit{L’énigme du masochisme}. Paris, \textsc{puf}, 2000.} entre outros. Sua obra
veio para ficar ao lado dos clássicos sobre a sexualidade humana. Sua
ausência precoce privou a psicanálise do muito que ele ainda traria e que
já se anunciava em sua obra. Já era mais do que hora deste livro capital
ser publicado em português.



