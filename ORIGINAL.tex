
\begin{comment}
KARNAC --- London/New York {}- PERVERSÃO, A forma Erótica do Ódio -
Robert J. Stoller, M.D.

[Publicado pela primeira vez em 1975 --- Reimpresso em 1986 com
autorização da Pantheon Books, New York, por H. Karnac (Books) Ltd. 6
Pembroke Buildings London NW10 6RE Publicado pela terceira vez em 2003
©1986

\{Todos os direitos reservados. Nenhuma parte desta publicação pode ser
reproduzida, armazenada em sistema de recuperação de dados nem
transmitida, sob qualquer forma ou meio, eletrônico, mecânico,
fotocópia, gravação, ou seja de que modo for, sem o prévio
consentimento, por escrito, do editor.\}

British Library Cataloguing in Publication Data

Uma C.I.P. para este livro pode ser obtida da British Library

ISBN: 0 946439 20 6

www.karnacbooks.com]
\end{comment}

\newcommand\imagem[2][0]{%
  \begin{figure}[p]
    \vspace*{-20.4mm}\hspace*{\dimexpr-28mm + #1mm}%
    \noindent\includegraphics[height=225mm]{img/#2}
  \end{figure}}

\cleardoublepage

%\openany


\chapter*{\ }

\section*{Agradecimentos}
\thispagestyle{empty}

Gostaria de agradecer a Nathan Leites, Ph.D., cujos esforços em prol de
uma linguagem clara e precisa desempenharam, mais uma vez, a função de
um crítico, de quem só eventualmente consegui fugir; e à minha
secretária, Thelma Guffan, pela paciência e competência com que há anos
vem tornando minha tarefa mais fácil.



\begin{comment}
Robert J. Stoller, M.D., é psicanalista e Professor de Psiquiatria
na Universidade da Califórnia na Los Angeles School of Medicine, onde
leciona para estudantes de medicina e alunos residentes da Psiquiatria.
É membro da American Psychoanalytic Association e recebeu seu
treinamento no Los Angeles Psychoanalitic Institute. Stoller é autor de
vários trabalhos focados no desenvolvimento da identidade de gênero
(masculino-feminino). Tem três publicações anteriores sobre o assunto:
\textit{Sex and Gender: On the Development of Masculinity} \textit{and
Femininity} (1968), \textit{Splitting} (1973) e \textit{Sex and Gender
Volume II: The Transsexual} \textit{Experiment} (1975). Adicionalmente
às pesquisas concentradas na identidade de gênero, Stoller tem
demonstrado um inesgotável interesse no campo da pesquisa na área médica.
\end{comment}

\part{Perversão}
\chapter{Prefácio}
\markboth{robert j.~stoller}{prefácio}

Por que, em tempos esclarecidos como os nossos, alguém escolheria dar
a uma obra o título de ``Perversão'', termo
que vem caindo em desuso? A grande pesquisa publicada nas últimas
duas décadas vem nos ensinando que o comportamento sexual aberrante\idxaberr[|(] é
encontrado também em outras espécies --- e que é onipresente entre os
homens; que ele é produto de componentes cerebrais e hormonais, cujo
funcionamento independe de algum tipo de instância a que pudéssemos
chamar de psique. Tais descobertas, além disso, fazem com que os
pesquisadores lastimem não\idxaspec[|(] só a postura\idxpervrespo{} moral\idxrespo{} da sociedade,\idxpervaspec[|(] que encara
a aberração sexual como algo fora do natural e pecaminoso,\idxpecad[|(] assim como a
ação social repressiva que essa perspectiva desencadeia. É desse modo
que, ao se livrar do conceito de perversão, uma combinação tentadora é
obtida: a de uma boa pesquisa posta a serviço de uma causa humanitária.
Entretanto, eu sustento que a perversão existe --- e esta será a tese
a ser explorada no decorrer deste livro.

As conotações da palavra\idxpervconot{} são desagradáveis e contêm uma aura de
moralismo, de livre escolha\idxlivre{} --- coisas consideradas antiquadas em nossos
dias de ciência e determinismo.\idxpervincid{} Foi justamente para evitar tais
conotações que termos mais brandos foram introduzidos, como
``variações'',\idxvaria{} ``desvios'' ou ``aberrações''. Atualmente, as pessoas
decentes --- dentre as quais muitas são cientistas --- estão cada vez mais
preocupadas com o preço pago por seus semelhantes --- e, mais ainda, por
sociedades inteiras --- ao não poupar esforços para reprimir
comportamentos sexuais aberrantes sem vítimas. E assim, em nome da
decência, hoje em dia é considerado de bom tom fazer uso de todas as
armadilhas da Ciência para tentar nos livrar do conceito de perversão.
Isto é feito não somente alterando o termo para outros, cujas
implicações são menos graves, mas também tentando demonstrar a
inexistência total (ou quase total) de estados que verdadeiramente se
encaixem nas torpes conotações do termo
``perversão''. Esses pesquisadores tiram suas
conclusões lançando mão de meios objetivos que, segundo eles, evitam
os embaraços do material introspectivo. Entre esses meios, está o
estudo dos mecanismos cerebrais em animais e humanos, que revelam um
potencial herdado para comportamentos aberrantes, presumivelmente
inserido nas estruturas hormonais do sistema nervoso central; outro
meio utilizado são as estatísticas, que revelam quão generalizados são
esses atos supostamente odiosos; temos, ainda, os estudos
antropológicos, que demonstram que o comportamento sexual aberrante tem
sido o comum --- e não o excepcional --- através da história e das
culturas; e, finalmente, a observação e a experimentação, usando
animais íntegros. Em todos esses casos, os dados reunidos revelaram que
práticas sexuais aberrantes\idxaberr[|)] se encontram em todas as espécies animais,
e que elas são onipresentes no comportamento humano. É fácil, então,
concluir que a aberração, disseminada entre os homens, não deve ser
considerada um comportamento que a pessoa desejou --- ou seja, um ato
livremente escolhido, envolto em pecaminosidade, uma atitude de
desobediência em relação à moral aceita ---\idxpervincid{} mas, antes, uma tendência
natural do impulso da sexualidade no reino animal.

E existem os que se opõem à extinção do termo --- filósofos e
ensaístas, mais do que pesquisadores formais --- porque se colocam a
serviço de um outro tipo de decência, também admirável: este tenta nos
desviar do abismo da licenciosidade, dando destaque aos aspectos
desumanizados,\idxdesu{} não amorosos, do comportamento sexual, dando mais ênfase
à gratificação anatômica do que à afetiva. Precisar reduzir uma pessoa
a um seio,\idxseios{} ou a um pênis ou a um pedaço de pano para ser capaz de
conseguir concentrar seu desejo é algo muito triste --- e perigoso; estas
graves falhas da potência, este aviltamento da condição amorosa, só
fazem fomentar os outros processos que, hoje em dia, degradam nossa humanidade.

O primeiro grupo quer se ver livre do conceito de perversão por ele
ter conotações morais,\idxpervconot{} inadmissíveis no estudo científico do
comportamento, e porque ele pode ser usado pelas forças repressivas da
sociedade. O outro grupo quer preservar o termo
``perversão'' porque precisa de uma palavra
totalmente carregada da ideia de pecado,\idxpervpecad{} para assim preservar a velha
moralidade que, após ter sido por tanto tempo posta à prova, continua a\idxpervaspec[|)]
constituir o pilar da sociedade.

Existe alguma verdade em cada uma dessas posições --- mas ambas estão
equivocadas.

A proposta deste livro não é descrever ou discutir as perversões,
nem oferecer uma teoria compreensiva de suas origens e de sua dinâmica;
é, antes, encontrar o seu significado, definir, clinicamente, o termo,
de modo a podermos reconhecer o fator comum, se ele aparecer,
independentemente dos comportamentos específicos que fazem com que um
tipo de perversão seja diferente do outro. Para tanto eu examinarei,
primeiramente, a hostilidade\idxhost{} --- que acredito ser o fator capaz de dar
definição ao termo.

O primeiro ponto é que a perversão resulta de uma interação
primordial entre hostilidade e desejo sexual, hostilidade que se
evidencia nas conotações do termo\idxpervconot{} (o dicionário, ao lidar com
``perverso'', ``perversão'' e ``perverter'' se utiliza de palavras como
``que tem péssimas qualidades morais: traiçoeiro, malvado''; ``mudança do bem em mal;
corrupção; desmoralização, depravação''; ``mudar para mal, tornar mau: depravar, corromper,
desmoralizar; transtornar, desnaturar, desvirtuar''.
Em segundo lugar, as pessoas que têm perversões sentem (são
obrigadas a sentir) uma perpétua sensação de serem sujas, pecadoras,\idxpervpecad{} 
 dissimuladas, anormais, e de constituírem uma ameaça àqueles cidadãos
melhores, os não perversos, que supostamente constituem a maioria da
sociedade. Em terceiro lugar, a própria palavra reflete a necessidade
de que os indivíduos, na sociedade, continuem impossibilitados de
reconhecerem suas próprias tendências perversas, providenciando bodes
expiatórios\idxbodex{} que isentam todos os outros humanos, na medida em que
servem como objetos de nossas próprias tendências, inaceitáveis e
projetadas. Todo esse sentido repugnante, de pecado,\idxpecad[|)] se perde na
brandura de um termo como ``variação''\idxvaria{} com
seu conspícuo anseio por respeitabilidade e candidez estatística (por
mais que eu possa digerir o termo
``perversão'', as implicações deploráveis de\idxaspec[|)]
se chamar alguém de ``pervertido'' chegam a
quase me impedir de usar a palavra.)

Este livro é o quarto de uma série sobre o desenvolvimento da
masculinidade e da feminilidade. Quando meu trabalho começou, em 1958,
eu não esperava que ele me conduzisse a um estudo do significado das
aberrações sexuais. Porém, recentemente, eu me percebi acreditando que
a perversão surge como uma maneira de lidar com ameaças à identidade de
gênero\idxiden{} da pessoa, ou seja, com a própria noção de masculinidade e
feminilidade; tenho constatado isto nos pacientes de que trato. Agora,
e nos próximos anos, a tarefa será verificar se é verdadeira a
hipótese de que as perversões, via de regra, são transtornos da
masculinidade e da feminilidade, e não somente uma coincidência que
encontrei nos poucos pacientes estudados. Nestas observações
introdutórias, é meu desejo sublinhar também que, pensar em termos de
masculinidade e feminilidade, nos dá uma perspectiva diferente (embora
não contraditória) daquela da metapsicologia psicanalítica\idxpsica{} clássica,
com sua coloração neutra --- ego, id, superego, neutralização, catexia, e
assim por diante.

É conveniente que eu diga algumas palavras sobre meu estilo de
escrita, já de início. O leitor logo perceberá que, embora as hipóteses
sejam extraídas de material clínico, as generalizações são
frequentemente afirmadas como se não tivessem que ser ainda comprovadas
e como se eu tivesse estudado casos suficientes, que me permitissem
declarações taxativas. Mas, à medida que a leitura progredir, se
puderem conservar em mente que essas ideias são provisórias, vamos
nos poupar do uso excessivo dos termos ``parece'' e
``talvez''. Tratem o livro como uma argumentação a ser ponderada, refutada, posta à prova.

Além disso, e de modo geral, não acredito que a linguagem técnica da
psicanálise seja necessária; isto fará, talvez, com que o leitor sinta
falta de uma qualidade mais pesada, costumeiramente presente nas
discussões analíticas sobre sexualidade. Não creio, contudo, que
qualquer autor diga algo a mais por fazer uso habitual de palavras tais
como ``catexia'', ``narcisismo'' ou ``neutralização''. Temas psicológicos
importantes podem, na maioria das vezes, ser expressos com precisão na
linguagem comum, com a vantagem adicional de que qualquer fragilidade,
em termos de argumento ou de dados, ficará mais visível para ambos,
autor e leitor, se o tom for informal.

Por acreditar nisso, acredito também que os psicanalistas, ao lidar
com assuntos clínicos, deveriam escrever --- simultaneamente --- para
analistas \textit{e} leitores comuns (Freud\idxfreud{} é o modelo). Isso força a
pessoa a ser mais clara.\idxsexop[|(] Problemas da teoria analítica que as pessoas
mais competentes e motivadas não compreendem, provavelmente também não
devem estar claras para o nosso colega analista. (A desvantagem deste
método é que, por vezes, eu precisei repisar, para os leigos, assuntos
que todo analista conhece bem; em alguns poucos lugares, fiz
referência a trabalhos de analistas com os quais os leigos não têm
familiaridade. Tentei conservar esses parágrafos pouco numerosos, e
curtos.)

Outro motivo para escrever sobre perversão, e não sobre \textit{as}
perversões, é a minha crença de que seria prematuro aventurar-me no
segundo tema. Embora existam compêndios de casos bizarros, a ênfase,
nesses trabalhos, tem sido o relato clínico superficial, revestido das
teorias etiológicas que a pseudociência tanto venera. As discussões
teóricas dão respostas em que há um excesso de verbosidade; e até mesmo
o material clínico, à primeira vista tão detalhado, é superficial,
incompleto, impreciso. Conservando isto em mente, eu não abrangerei
todas as aberrações; considerarei em detalhe somente uma condição
específica, apenas para ilustrar algumas hipóteses.

Espero, portanto, que este livro induza alguns leitores a verificar
se não existe, de fato, uma tal lacuna na informação --- esperando que,
aqueles que a reconhecerem, se encorajem e deem prosseguimento a este
estudo. Pois parece haver uma situação estranha, no que concerne à
pesquisa\idxaberrpesq{} do comportamento sexual aberrante: a relativa ausência de
discussão, nos anos recentes, a nos dar a impressão de que não existe
muito mais a ser feito. Raramente se publica material clínico
abrangente na literatura psiquiátrica ou psicanalítica, como se os
critérios para diagnósticos estivessem muito nítidos, e a natureza das
síndromes de tal modo evidente que dispensasse descrições detalhadas ---
consideradas exóticas, desnecessárias. (Talvez haja um número excessivo
de profissionais que consideram que os estudos do início do século,
levados a cabo por nomes como Kraft-Ebing\idxkraft{} e Havelock Ellis,\idxellis{} acabaram
por nos causar uma certa indigestão descritiva.) A literatura analítica
raramente relata outros tipos de síndrome, além da homossexualidade\idxhomos{} e
do fetichismo.\idxfetic{} Os dois ou três trabalhos que aparecem a cada ano, com
demasiada frequência (embora com algumas honrosas exceções) apenas dão
nova roupagem a alguma posição teórica anterior. Não seria melhor
tornarmos pública nossa ignorância, de modo a podermos avançar em nossa
compreensão sobre o comportamento sexual?\idxsexop[|)]

Começaremos a remediar esta situação:\idxlibid{} vamos separar as aberrações
sexuais que são produzidas essencialmente como tentativa perene de
``curar'' uma certa tensão psíquica, daquelas
em que essa dinâmica não está na raiz do comportamento. Acredito que a
perversão, diferentemente da aberração, seja produto da angústia,\idxangu{} e que
o comportamento sexual perverso esteja permeado de reminiscências, de
destroços e de outros indicadores da história pregressa do
desenvolvimento sexual do indivíduo --- especialmente no âmbito de suas
dinâmicas familiares.\idxinflu{} Caso o observador soubesse \textit{tudo} o que
aconteceu na vida da pessoa que está estudando, ele encontraria esses
eventos históricos representados nos detalhes do ato sexual manifesto.
O observador saberia, então, quando e por que essa pessoa desistiu
daquilo que mais poderia apreciar, em termos de erotismo, para escolher
as alternativas que constituem o cenário da perversão. Esta hipótese,
então, é que a perversão é uma \textit{fantasia}\idxfanta[|(] posta em ação --- uma
estrutura de defesa gradualmente erguida com o passar dos anos, com a
finalidade de preservar o prazer erótico. O desejo de preservar aquela
gratificação advém de duas fontes principais: (1) um extremo prazer
físico que, por sua própria natureza, exige repetição; (2) da
necessidade de conservar sua identidade.

Eu não vejo como a fantasia possa ser excluída de nossas
especulações sobre o comportamento sexual humano; não é segredo que a
fantasia, sob a forma de devaneios, está conscientemente presente em
grande parte da atividade sexual. De fato, quando ouvimos falar de uma
pessoa que não tem fantasia sexual, suspeitamos da existência de uma
inibição agindo ali. Porém, vasculhem através dos nomes dos grandes
pesquisadores da sexualidade, da geração passada ou da anterior.
Notarão que, não importando sua área de estudo, as técnicas utilizadas
ou o tipo de descoberta que relatem, os dados obtidos eram referentes a
aberrações sexuais não motivadas por fantasias --- ou seja, não envolviam
relatos que moldassem uma ``realidade'' nova,
melhor. Esses pesquisadores enfatizam as origens não inventadas, não
conflitantes, \mbox{extrapsíquicas} da excitação sexual, perversa ou não.
Eles tratam as manifestações intrapsíquicas como se não existissem. Por
exemplo: um punhado de casos foi descoberto, na imensidão do universo
humano, em que um comportamento sexual aberrante foi deflagrado por um
problema do sistema nervoso central; conclusão: a perversão resulta da
epilepsia. Exemplo: animais em liberdade que algumas vezes adotam
alguma característica do comportamento reprodutivo do sexo oposto (como
quando uma vaca, por alguns momentos, monta em outra vaca); conclusão:
a homossexualidade faz parte da atividade animal e o ser humano, sendo
parte do mundo animal, está simplesmente expressando sua herança
natural, ao abraçar a homossexualidade. Exemplo: um chimpanzé macho,\idxanim{} em
Nova Orleans, se masturbava enquanto acariciava uma bota; conclusão: o
fetichismo é resultado de simples condicionamento.\idxcond{} Certas sociedades
não consideram como perversão algumas atividades que definimos como
perversas; conclusão: o ato, praticado em nossa sociedade, com a mesma
anatomia, tem o mesmo significado para o indivíduo, e brota das mesmas
fontes psíquicas, como nas culturas\idxrela{} de outros povos.

E assim por diante. Tais estudos se unem, num esforço de contestar a
motivação psíquica e substituí-la por forças primitivas, tais como
evolução, herança cromossômica e genética,\idxgenet{} neurofisiologia,
condicionamento e impregnação --- forças que agem sobre uma
psicobiologia desprotegida; ou proclamando que o que é a norma é o
normal. Eu discordo; porém, acredito que esses fatores são (ou, em
alguns casos ainda sem comprovação, possam ser) contribuições
essenciais à sexualidade humana. Tudo quanto peço, é que consideremos
também os efeitos intrapsíquicos do passado das pessoas, especialmente
quando expressos nas sutilezas dos relacionamentos interpessoais.
Aqueles que consideram a si próprios cientistas, podem estar cometendo
um erro histórico ao desprezar esse fator. Eles não sabem que aquilo
que rotulam como sendo apenas o comportamento de alguém, pode ser
também uma explicação do homem. Por sua complexidade, hoje em dia o
espírito humano escapa às técnicas experimentais; os métodos de
pesquisa do atual estágio do desenvolvimento científico ainda não têm
competência para revelar ou explorar a fantasia. Mas, se a fantasia
existe, ela pode ser estudada. E, enquanto aguardamos pela evolução da
ciência, talvez devamos nos voltar para aquela técnica, de descoberta
incerta, porém competente, que é o método psicanalítico,\idxpsica[|(] com seu
confuso rebento, a teoria analítica.

O objetivo de minha pesquisa é encontrar as origens psicológicas
daquilo que rotulamos como sendo a ``identidade de
gênero''\idxiden{} ou seja, a masculinidade e a feminilidade. Para
tanto, parece haver três métodos nos quais os analistas são
especialmente habilidosos: (1) a análise de adultos e crianças, para
entrever as raízes de seus comportamentos. Este procedimento tem sido a
fonte dos grandes \emph{insights} dos psicanalistas, ao longo de gerações; a
perspectiva é essencialmente intrapsíquica (ego, superego, id;
consciente, pré-consciente, inconsciente; fixação e regressão;
mecanismos de defesa; fantasia;\idxfanta[|)] e por aí vai). Esperemos, contudo, que
a analogia feita por Freud,\idxfreud{} do analista a vasculhar o passado como um
arqueólogo, não encoraje ninguém a querer descobrir tudo sobre seu
passado baseando-se apenas nas descobertas feitas pela psicanálise. Se
você fosse um historiador, e se existisse essa possibilidade,
preferiria ficar coletando restos de ruínas ou ir visitar as cidades
quando elas ainda estavam vivas? Por isso, é muito aconselhável também:
 (2) observar as mães\idxmaes{} com suas crianças, assim como os pais entre si, o
modo como a família\idxinflu[|(] interage; tais estudos, na geração passada, puseram
à prova e expandiram as descobertas do item (1). (3) Analisar os pais ---
especialmente as mães --- das pessoas cujo comportamento é objeto de
nossa pesquisa; nos últimos doze anos, depois de algum tempo contando
somente com os dois primeiros, eu me concentrei neste último método,
pelo \emph{insight} que ele propicia sobre as pressões que a criança precisou
suportar, e em virtude das quais acabou por adotar um comportamento
aberrante (meus colegas tratam a criança e um dos progenitores,
enquanto eu analiso o outro progenitor.)

Trabalhar desse terceiro modo, entretanto, não significa acreditar
que as aberrações sexuais possam ser adequadamente compreendidas por
esse tipo de estudo; o meu trabalho é, no melhor dos casos,
complementar ao corpo principal das descobertas da psicanálise.
Precisamos considerar todos os fatos, reunidos através dos três\idxpsica[|)]
métodos.

A perversão é, portanto, o resultado de dinâmicas familiares que, ao
provocarem medo, forçam a criança, que anseia por uma completa imersão
na situação edípica\idxconfe{} (desejo de possuir o progenitor de sexo oposto ao
seu, e de se identificar com o progenitor do mesmo sexo) a evitá-la.\idxinflu[|)] A
heterossexualidade\idxhetero[|(] é um estado complexo porque,\idxangu[|(] embora ela precise de
frustração e de dor para se produzir, ainda assim, um outro tipo de
frustração e de dor podem vir a alterá-la.\idxheteroadqu{} Se quisermos compreender os
indivíduos que estudamos, precisamos apreender a natureza exata de suas
frustrações e de suas dores, para que isto nos ajude a compreender por
que os diferentes resultados se produzem. Compreenderemos melhor as
origens da perversão se a encararmos como uma heterossexualidade
corroída; como veremos, isto não é verdadeiro para todas as aberrações.

Naturalmente, agora eu me encontro em maus lençóis porque, como os
demais analistas, eu também acredito que a maioria dos comportamentos
sexuais --- e não apenas aqueles rotulados como perversos --- resulta de
experiências da vida, ou de conflitos a que se sobreviveu, e de coisas
das quais se foi obrigado a abrir mão; de tal maneira que, se pudermos
olhar bem de perto, fazendo uso do nosso microscópio, a ideia de
normalidade\idxnorma{} (e não a de normatividade)\idxpervnorma{} se esfacela. Nós precisamos, por
exemplo, admitir quão comum é a patologia sexual nos heterossexuais,
que pretensamente constituiriam o padrão da normalidade, e que usamos
como referência para designar os anormais. Até que possamos
compreender a heterossexualidade --- e não considerá-la como um
pressuposto, como algo que dispensa explicações --- também não
compreenderemos a perversão. E, para diminuir esse estado de
ignorância, precisamos começar por nos lembrar que a heterossexualidade
é uma aquisição; não podemos varrer o assunto para o lado, alegando que
a heterossexualidade\idxhetero[|)] é uma predisposição, por ser necessária à
sobrevivência das espécies; e que, portanto, ela está \mbox{biologicamente}
garantida. Nós simplesmente não temos o direito de aceitar este
postulado biológico --- que, embora sensato, não está comprovado --- e
tomá-lo como verdadeiro --- seja nos humanos, nas abelhas ou nos ratos.

Por outro lado, existe uma quantidade de aberrações que não se
originam de concessões feitas pelo indivíduo, e que lhe foram impostas
pela angústia.\idxangu[|)] De acordo com minha terminologia, nem todos os atos
sexuais aberrantes são perversões.

Agora, nós precisamos de definições.


\part[Parte \textsc{i}: Definição]{Parte \textsc{i}:\\ Definição}


\chapter[\textbf{1}\quad Definições]{{\large\textit{Capítulo 1}}\\ Definições}
\markboth{Definição}{Definições}

Descreverei, primeiramente, a maneira como eu concebo uma aberração, uma
variação e uma perversão; nos capítulos subsequentes, revisarei os
dados e os conceitos que me conduziram às definições. Acredito que a
perversão existe; que suas conotações ultrajantes refletem uma
consciência turva de que, no cerne do ato perverso, existe o desejo de
fazer mal a alguém; e que o conceito deveria ser preservado --- não por
ser uma arma de propaganda, útil à preservação da sociedade, mas pelo
fato de a condição poder ser demonstrada.

Por \textit{aberração},\idxaberrdef{} aqui, eu me refiro a uma técnica, ou a uma
constelação de técnicas, eróticas, que alguém usa como seu ato sexual
completo, e que difere da definição de normalidade tradicional,
admitida por sua cultura. As aberrações sexuais podem ser divididas em
duas classes: variações (desvios) e perversões.

Por \textit{variação}\idxvariadefi{} refiro-me a uma aberração que não é, em
essência, a encenação de fantasias proibidas --- especialmente fantasias
que envolvam causar dano a alguém.\idxhost{} Exemplos disto seriam comportamentos
que só se desencadeariam sob a influência de alguma atividade anormal
do cérebro,\idxcerea{} como, por exemplo, quando se constata a presença de um
tumor e se faz uso de alguma droga experimental, ou de impulsos
elétricos emitidos por eletrodos implantados; ou um ato aberrante a que
se é conduzido por \textit{faute de mieux};\idxfaute{} ou tentativas sexuais que
a pessoa faça por curiosidade, e acabe não as considerando
suficientemente excitantes para que as repita.

\textit{Perversão}, a forma erótica do ódio, é uma fantasia, em\idxpsicap[|(]
geral atuada, mas que ocasionalmente fica restrita ao devaneio
(elaborado pela própria pessoa ou produzido por terceiros, ou seja,
material pornográfico). É um tipo de aberração habitual, preferencial,
necessária à satisfação plena e, principalmente, motivada pela
hostilidade.\idxhost{} Por ``hostilidade'' eu me refiro
a um estado em que alguém deseja danificar um objeto --- o que a torna
diferente da agressividade que, em geral, implica apenas o uso da
força.\idxlibid{} A hostilidade, na perversão, toma a forma de uma fantasia\idxfanta[|(] de
vingança\idxvinga{} que se oculta nas ações que a consumam, e que serve para
transformar um trauma\idxtrauma{} de infância\idxinfan{} em um triunfo de adulto. Para que se
produza o máximo de excitação, a perversão também precisa ser
representada como um ato que envolve algum\idxperigo{} tipo\idxfantaexpo{} de perigo.\idxpervexpos{}

À medida que essas definições removem incongruências anteriores,
elas nos impõem um novo ônus: que a pessoa nos revele o que é que a
motiva. Mas, ao agir assim, livramo-nos de um processo de
caracterização que negligenciou a personalidade e a motivação do
sujeito. Não precisamos mais definir uma perversão em termos da
anatomia utilizada, do objeto escolhido, da moralidade professada pela
sociedade ou do número de pessoas que a praticam. Tudo quanto
precisamos saber é o que ela significa para a pessoa que dela se
utiliza; embora, possivelmente, nos seja difícil o deslindar deste
significado, continua não existindo qualquer razão apriorística para
que rejeitemos esta técnica de caracterização.

Alguns analistas (e outros profissionais também) usaram os termos
``aberração'', ``desvio'' ou ``perversão'' como
sinônimos, classificando determinado ato como aberrante não
necessariamente com base nos critérios de quem os pratica, e sim nos de
quem o observa. Então, quando uma teoria sobre a perversão tiver sido
estabelecida --- conflito pré-edípico e edípico;\idxconfe{} fixação e regressão
oral, anal ou fálica; fantasias com objetos ameaçadores introjetados,
tais como o mau pênis do pai ou o ainda pior seio da mãe;\idxmaes{} a clivagem
do ego; a condenação, ou a permissividade, por parte de um superego
falho; ou o que quer que seja --- a explicação estará completa antes
mesmo de se ter visto o paciente. Para evitar tamanha eficiência, estou
tentando nos obrigar a um regresso, em busca de informações sobre a
própria pessoa que pratica o ato. Hoje é notório que os sistemas que
acabamos de citar foram oferecidos como explicação para todos os tipos
de comportamento, o que, naturalmente, significa que eles explicam
pouco; com toda sua força colocada na descrição das dinâmicas, tais
conceitos são incapazes de responder de que forma a perversão difere,
digamos, de um tique, de uma alucinação, da mania, ou de um forte
desejo por picles.

Aliás, é sugerido que o ato perverso proporciona gratificação
porque, entre outros ``indicadores específicos de
perversão'' listados, através dele (do ato) ele (o
paciente"-prototípico) atuou uma confrontação entre imagens \mbox{idealizadas}
e degradadas de sua\idxmaes{} mãe\ldots{} ele (o ato) gratificou desejos sádicos e
masoquistas que, de outra maneira, seriam inaceitáveis\ldots{} a angústia da
 castração\idxcasta{} e a culpa --- que ele normalmente experimentaria na relação
sexual --- foram eficientemente evitadas pelo sistema defensivo
perverso\ldots{} a perversão levou ao ato um desejo proibido, de maneira
disfarçada --- especificamente, tanto o desejo edípico\idxconfe{} quanto a
transferência homossexual\ldots{}\idxhomos{} ela reencenou a cena primal\ldots{} ela
reencenou também a sedução infantil e a gratificação por parte dos
pais\ldots{} e permitiu uma verdadeira gratificação através de um verdadeiro
objeto substituto, de tal modo que a angústia pela perda do objeto foi
dissipada\ldots{} (114, p.\,47)\footnote{ As citações completas das
publicações mencionadas ou a que se faz referência neste livro poderão ser
encontradas na lista de Referências. Os
números entre parênteses referem-se aos itens enumerados nessa lista.}

Podemos nos permitir uma sensação de \textit{déjà vu.}

Dentro dos parâmetros comuns da psicanálise, presumimos que todas as
aberrações sexuais são energizadas por dinâmicas patológicas; assim,
tratamos nossos pacientes --- e vamos criando nossas teorias --- em
correspondência a isto. Então, em virtude de utilizarmos estes
conceitos psicanalíticos, para deixar implícita e existência de um
distúrbio, percebemo-nos usando aquele protótipo ``o
heterossexual''\idxhetero{} ou, ainda mais vago, ``o
normal''\idxnorma{} como linha de base para mensuração da patologia
daqueles menos favorecidos. Isto feito, entretanto, caímos na armadilha
de ignorar nosso conhecimento do contumaz mau funcionamento do
``heterossexual'' ou de fracassar ao tentar\idxpsicap[|)]
encontrar ou descrever ``o normal''.\idxpervnorma{}

Eu prefiro definições que se ancorem em dados clínicos, não em
teorias; portanto, a partir de agora, apresentarei várias sugestões e
hipóteses baseadas na observação, e que ajudarão a elaborar essas
definições.

Em primeiro lugar,\idxaberrdef{} quando usamos o termo ``sexual'' em ``aberração
sexual'' deveríamos restringi-lo à excitação\idxsexue{} erótica
consciente, evitando seus significados psicanalíticos mais amplos, que
consideram como ``sexuais'' todos os
prazeres; esta conotação tem seu valor em determinados contextos, mas,
aqui, só iria nos atrapalhar. Por exemplo, um transtorno de identidade
de gênero,\idxiden{} como a efeminação,\idxhomosafem{} está presente, na maioria das vezes, sem
implicar excitação sexual; tanto pode ser encontrado em homens
perversos quanto em homens não perversos.

Em segundo lugar, saber se uma aberração é uma variação\idxvariadefi{} ou uma
perversão, isto se determina pela atitude da pessoa em relação ao
objeto que causa a excitação. Se a pessoa escolheu esse objeto --- homem,
mulher, cão, parte do corpo, objeto inanimado, seja o que for ---
motivada pelo desejo de danificá-lo, e se o ato tem para ela o
significado de uma vingança,\idxvinga{} então ele é perverso.

Em terceiro lugar: toda vez que o ato perverso for realizado, quer
ele envolva outras pessoas ou seja solitário, como na masturbação, seu
autor se sente como que celebrando um triunfo.

Quarto, o trauma\idxtrauma[|(] de infância,\idxinfan[|(] que mencionamos na definição, ocorreu
efetivamente, e é rememorado nos detalhes da perversão. Minha hipótese
é que uma perversão é uma revivescência de um trauma sexual verdadeiro,
que visa precipuamente o sexo (em termos anatômicos) ou a identidade de
gênero (masculinidade ou feminilidade),\footnote{ Valenstein\idxvalen[|nn] diz que o
trauma pode não ter sido verdadeiramente vivenciado, que pode ter sido
a ``concepção vivencial falsa'' de um
acontecimento real.} e que, no ato perverso, o passado se dissipa.
Desta vez, o trauma é transformado em prazer,\idxfantapraz[|(] em orgasmo, em vitória.
Mas a\idxtraumanece[|(] necessidade de repeti-lo\idxpervrepet[|(] --- \textit{ad infinitum}, sempre da mesma
maneira --- resulta da incapacidade de a pessoa se sentir
completamente livre do perigo, do trauma. Ele aconteceu; e, exceto por
aquele momento em que o ato de perversão é repetido, e se processa sem
problemas, o indivíduo não pode fingir para si mesmo que o trauma não
aconteceu --- embora sua lembrança seja inconsciente. Não precisamos de
uma construção nebulosa, como a ``compulsão por
repetição'' para explicar por que o ato se repete: na
perversão, a repetição acontece porque a pessoa quer escapar ao velho
trauma, e porque a vingança\idxvinga{} e o orgasmo são suficientes como\idxtrauma[|)]
recompensa.\idxpervrepet[|)] Estes motivos\idxtraumanece[|)] bastam.

A partir de Freud, tem se afirmado que a\idxsexueinf{} excitação precoce\idxinfanexcit{}
 contribui para a perversão. Eu concordaria, mas apenas --- o que
geralmente é o caso --- quando tiver havido um excesso de estimulação e
uma escassez de descarga, ou grave culpa. Estas estimulações serão,
então, percebidas como traumáticas, e precisarão ser transformadas,
através do ritual perverso, em um acontecimento vitorioso. Por outro
lado, quando há muita gratificação e pouca culpa, em idade precoce,
acredito que o resultado será não a perversão, mas a aberração;\idxinfan[|)] um
apegar-se, já na idade adulta, àquela forma desviante de obter prazer,
que não é impelida, como o é na perversão, pela necessidade de
danificar um objeto.

Em quinto lugar, é preciso assegurar, no decurso de anos de
tentativa e erro na construção de fantasias,\idxfantacons[|(] que finalmente se
encontrou uma fórmula --- a perversão adulta --- que funciona com
regularidade. O fracasso desta construção está marcado, por um lado,
pela falta de interesse sexual --- pelo\idxpervenfad{} enfado\idxenfa{} --- e, por outro, pela\idxfantaangu[|(]
angústia.\idxangu[|(] Ambos se manifestam por perturbações da potência. Caso o
devaneio funcione, é preciso que o enredo da história não suscite
demasiada angústia, pois esta é, em sua forma não adulterada, inimiga
do prazer. Mas é necessário, também, reduzir a angústia sem aniquilar a
excitação. O que se consegue ao se introduzir uma sensação\idxperigo[|(] de perigo\idxfantaexpo[|(] na
história. Uma \textit{sensação} de perigo;\idxpervexpos[|(] em realidade, o perigo não
pode ser grande, ou a angústia virá à tona. Portanto, o que deverá
existir aí é tão somente uma impressão de perigo.

Alguns comentários são bem-vindos aqui, e eu os tecerei desde já,
embora vá retomá-los, em relação ao material clínico. Existem atos
sexuais em que é essencial o enfrentamento de perigos reais: por
exemplo, enforcar-se\idxenfor{} para conseguir um orgasmo. O que precisamos
distinguir, entretanto, é que o perigo que se imiscui entre as
fantasias do ato sexual não é o mesmo que o que \mbox{ameaça} o mundo real; o
perigo do enforcamento --- a morte --- não é o perigo --- angústia\idxfantaangu[|)] ou culpa ---
que a fantasia\idxangu[|)] deve evitar.\idxperigo[|)]

Em sexto lugar, o resultado de todo este trabalho de construção da
fantasia,\idxfanta[|)] que conhecemos como perversão, é que nossos objetos sexuais
são desumanizados.\idxdesufant[|(] Isto é óbvio, por exemplo, no fetichismo\idxfetic{} e na
necrofilia.\idxnecro{} Mas, ao olhar mais de perto para as perversões crípticas,
tais como o estupro,\idxestup{} ou uma preferência por prostitutas,\idxprost{} ou a
promiscuidade\idxpromiscomp{} compulsiva (síndrome de Don Juan,\idxdonj{} ou ninfomania), e que o
observador ingênuo poderá encarar simplesmente como entusiasmos
heterossexuais, percebemos que o objeto é uma pessoa, dotada de uma
personalidade, ao passo que o indivíduo perverso enxerga a criatura
como desprovida de humanidade --- apenas como uma anatomia, ou fragmentos
estereotipados da personalidade (o que se percebe em expressões como
``que bela bunda'' ou ``todos os homens são brutos''). E isto dificilmente é uma ideia
nova. Em 1930, E. Straus\idxstraus{} observou: ``O prazer\idxpraz[|(] nas
perversões é causado pela\ldots{} destruição, pela humilhação, pela
profanação, pela \textit{deformação} do próprio indivíduo perverso e de
seu parceiro'' (os itálicos são de Straus).\footnote{ E.
Straus, \textit{Geschehnis und Erlebnis} (Berlim, 1930), p.\,113. Citado
em 7, p.\,20}

Mas essas proposições não nos dizem como o prazer é introduzido. Se
a perversão é o resultado da ameaça, e do ódio que ela provoca, de onde
vem o prazer? Um trauma ou uma frustração, quando não são mitigados,
não contêm em si desejo --- como também acontece em relação à raiva. O
prazer é liberado somente quando a fantasia --- o que torna a perversão
exclusivamente humana --- tiver operado. Com a fantasia, o trauma se
desfaz\idxtraumadesf[|(] e, no devaneio --- o conteúdo manifesto, o fio condutor da
história consciente, construída,\idxfantacons[|)] da fantasia --- ele pode ser
desfeito \textit{ad infinitum}, de acordo com a necessidade.

Ao refazer o mundo, os devaneios contribuem para o prazer,\idxfantapraz[|)]
primeiramente livrando a pessoa do medo de que o trauma se repita. Em
segundo lugar, o devaneio traz em si elementos que simulam o\idxpervexpos[|)] perigo,\idxfantaexpo[|)] de
tal modo que a excitação --- a tensão --- se introduz por aí. Em terceiro
lugar, o devaneio garante um final feliz --- o que quer dizer que,
desta vez, a pessoa não apenas superou o trauma, mas até mesmo
ludibriou, se é que não traumatizou, aqueles que, originariamente, eram
os ofensores. Finalmente, quando o devaneio se vincula à excitação
genital e, especialmente, ao orgasmo, a
``justificação'' do devaneio é reforçada, e a\idxpraz[|)]
pessoa se sente\idxtraumanece{} motivada a repetir\idxpervrepet{} a experiência, sob circunstâncias
semelhantes.\idxtraumadesf[|)]

Outra questão: se a perversão é o ódio erotizado, por que (exceto na
homossexualidade) ela é mais encontrada entre os homens?\idxpervpredo{} Se ela
é a erotização do ódio, então teremos que encontrar mais ódio --- ou uma
forma diferente de ódio --- nos homens do que nas mulheres. Pode ser que
seja assim. Daremos prosseguimento a este estudo adiante (Capítulo 8).

A fim de começar a avaliar estas ideias, analise sua própria
experiência.\idxinfan{} Pense nas perversões com as quais você está familiarizado:
necrofilia,\idxnecro{} fetichismo,\idxfetic{} estupro,\idxestup{} assassinato sexual,\idxassas{} sadismo,\idxsadi{} masoquismo,\idxmasoq{} voyeurismo,\idxvoy{} pedofilia\idxpedof{} --- e muitas mais. Em cada uma delas
encontramos --- de forma explícita ou oculta, mas desempenhando papel
essencial na fantasia --- hostilidade,\idxhost{} vingança, triunfo, e um objeto
desumanizado.\idxdesufant[|)] Antes mesmo de arranhar a superfície, podemos ver que a
característica de causar dano a alguém é essencial, na maioria dessas
condições. Mais tarde, testaremos essas ideias com maior rigor,
considerando as condições sob as quais este mecanismo fica menos
manifesto; descobriremos que se precisa de pouco aparato clínico --- ou
produção teórica --- para descobrir o mecanismo hostil. Descobriremos,
também, que o ato perverso tece seu caminho entre a angústia e o\idxpervenfad{} enfado,\idxenfa{} em busca do tipo certo de perigo, para criar a excitação.

Que pena que minha tentativa de definição fracasse em nos livrar de
um velho problema: não posso afirmar a quantidade de perversão que é
necessária para poder\idxpervdiagn[|(] diagnosticar\idxdiag[|(] a condição como perversa (assim como
não existe uma forma exata de medir, digamos, a neurose de angústia
pela quantidade de angústia que tem que estar presente nela; ou um
ponto definido em que a estrutura de caráter se transforma em desvio de
caráter). A exigência, entretanto, é artificial; o diagnóstico, em
medicina, não passa de uma conveniência, de um esforço de transmitir o
maior número de informação possível, com o menor número de palavras,
sobre as condições clínicas, sublinhando as dinâmicas patológicas e sua
etiologia.

Aqueles, entre nós, que são psiquiatras, todos frequentamos a
faculdade de medicina, de tal modo que ansiamos por um sistema de
diagnóstico que seja tão eficaz quanto os que abrangem a maioria das
patologias das outras especialidades. E se o psiquiatra for como os que
pesquisam a sexualidade, ou seja, se não acreditarem que existam
estados psíquicos que tenham sido originados por um conflito, e depois
mantidos por mecanismos mentais, tais como a fantasia, a repressão, a
anulação e a clivagem, ele continuará pressionando, exigindo
diagnósticos psiquiátricos que sejam tão detalhados como, digamos, o
de uma ``fratura exposta cominutiva do fêmur'' de uma ``apendicite''
ou uma ``hidrofobia''.

Não funciona. Nossa classificação\idxclass[|(] se utiliza de uma multiplicidade
de métodos diferentes para chegar a um diagnóstico. Em outras palavras,
os princípios para fazer um diagnóstico escorregam, deslizam de
categoria para categoria, revelando que uma conveniência improvisada,
mais do que a lógica ou os fatos, controla a disposição dos fatores.
Sem esforço, posso pensar numa quantidade de maneiras que não têm
qualquer afinidade entre si e que, hoje em dia, são usadas para fazer
diagnósticos --- e você, com certeza, será capaz de adicionar outras.


\begin{enumerate}

\item \textit{O diagnóstico tal como é empregado nas demais áreas da
medicina}; são exemplos: ``trissomia autossômica do grupo \versal{G}'' ou ``psicose com trauma cerebral''.

\item \textit{Uma síndrome}; um exemplo é o rótulo ``esquizofrenia'',\idxesquiz{} que a maioria de nós considera ser, em verdade, um grupo de condições com diferentes
etiologias, evoluções e prognósticos.

\item \textit{Um sintoma proeminente} (sem levar em consideração a
estrutura de caráter subjacente e outros sintomas neuróticos também
presentes); alguns exemplos são ``neurose de angústia'' ou ``neurose
fóbica''.

\item \textit{Um sinal proeminente} (idem 3); são exemplos:
``homossexualidade'' ou ``fetichismo''.

\item \textit{Um único sintoma}; são exemplos:
``tique'' ou ``distúrbios da fala''.

\item \textit{Um único sinal}; são exemplos
``enurese'' ou ``encoprese''.

\item \textit{Uma maneira crônica de vida}; são exemplos:
``personalidade paranoide'' ou ``personalidade inadequada''.

\item \textit{Patologia de um órgão do corpo devida, em parte, a
estados mentais}; um exemplo seria ``dermatite psicossomática''.

\item \textit{Toxicomania}; seriam exemplos ``alcoolismo'' ou ``toxicomania
por heroína''.

\item \textit{Miscelânea}: são exemplos: ``desajuste
social'' ou ``desajuste conjugal''.

\end{enumerate}

Isso é um sistema?

Se o diagnóstico, na maioria dos casos, fornece apenas a ilusão de
exatidão, será mais garantido, para nós, se nos servirmos de simples
descrições que resumam o que nos foi possível observar; podemos adotar
este modo de agir de imediato, e assim não seremos forçados a fazer o
impossível --- tentar medir se alguém é suficientemente perverso para que
possamos afirmar a presença de uma perversão. Nosso trabalho, portanto,
não ficará prejudicado;\idxclass[|)] ainda seremos capazes de tomar nossas decisões,
sábias ou tolas, a respeito de, digamos,\idxdiag[|)] julgar se um caso é para
tratamento ou para punição penal.

As definições aqui delineadas precisam\idxpervdiagn[|)] ser discutidas em
profundidade; porém, antes de considerar mais de perto as diferenças
entre ``variação'' e ``perversão'' eu gostaria de comentar uma
recente pesquisa sexual contra o pano de fundo da teoria e das
descobertas psicanalíticas, que têm dominado as ideias a respeito do
comportamento sexual por várias gerações.




\chapter[\textbf{2}\quad O impacto dos novos avanços da pesquisa sexual\\\hspace*{6mm}
na teoria psicanalítica]{{\large\textit{Capítulo 2}}\\ O impacto dos novos avanços da pesquisa sexual na teoria psicanalítica}
\markboth{Definição}{O impacto dos novos avanços da pesquisa sexual\ldots{}}

A teoria psicanalítica\idxsexop[|(] é uma criação de\idxfreudsexua[|(] Freud;\idxfreud[|(] a \mbox{maioria} das
modificações\idxsexuateo[|(] que lhe foram acrescentadas não apenas são secundárias,
como também são elaborações de posições teóricas que ele já havia
introduzido, de forma explícita. É por esta razão que o presente
capítulo examinará exclusivamente as teorias sexuais freudianas.
Embora, em termos gerais, não se possa discutir qualquer segmento de
sua obra sem conservar em mente o quanto ela se alterou com o passar
dos anos, isto é menos verdadeiro no que concerne às suas teorias sexuais.

 Ao examinar essas teorias, reconhecemos que Freud não esclareceu o que
ele queria dizer com ``sexualidade''; assim,
algumas vezes, ele a discutiu de maneira obscura. Se procuramos por
explicações universais, então a precisão sistemática poderá ser um
estorvo; porém, para nossas necessidades imediatas, será de ajuda
observar as diferentes áreas de observação ou de raciocínio abrangidas
pelos termos ``sexo'', ``sexual'' e ``sexualidade.'' É verdade que pesquisas
recentes reconheceram que são diversos seus significados.

 Em primeiro lugar, para Freud ``sexual'' significava toda e qualquer característica dos tecidos vivos que
expresse uma entropia negativa; a isto ele chamou de \textit{libido}:\idxlibid{} uma tendência mística cuja direção é o estar vivo, permanecer vivo e reproduzir a vida.

 Depois, existem características biológicas que definem um organismo
como masculino ou feminino; elas podem ser genéticas, anatômicas ou
fisiológicas. Em geral, em si mesmas elas não têm conotações
psicológicas --- apesar de a predileção de Freud por biologizar o tenha
levado a investigar motivações psicológicas ancestrais (por exemplo,
instintos\idxfreudinsti{} de vida \textit{versus} instintos de morte) em processos tão
mecânicos quanto funções celulares, e até mesmo na química molecular.

 Além do que, ``sexual'' descrevia as mesmas experiências que outros definiam como ``sensual''; se uma atividade trouxesse prazer\idxpraz{} corporal, aquele prazer deveria ser rotulado de ``sexual'' pois Freud descobriu, nas primeiras \mbox{experiências} de prazer que os bebês sentem, desde o nascimento, as origens das atividades que, posteriormente, todos reconhecem como eróticas.

 E mais: ``sexual'' se relaciona ao masculino
e ao feminino --- além de se referir ao comportamento reprodutivo.

 E, finalmente, ``sexual'' remete ao erótico,
ou seja: a sensações intensas em várias partes do corpo, especialmente
nos genitais; e que são acompanhadas por fantasias (conscientes ou
inconscientes) de intimidade com outros objetos, criando um desejo por
satisfação genital.

 Em vista de conotações tão amplas, que cobrem todas as atividades e
tendências dos tecidos vivos, vamos nos complicar caso nossa discussão
não fique confinada aos significados habitualmente mais aceitos de
``sexual''. Isto tem valor prático, mantendo
minha apresentação dentro de seus limites; e somos forçados a esta
estratégia também porque, quanto mais grandiosa e mística for uma parte
da obra de Freud, menos é provável que algum procedimento de pesquisa
possa ser desenvolvido para testá-la.

 Sendo assim, focaremos esta discussão quase que completamente nas duas
áreas de comportamento em que o termo
``sexual'' é comumente usado: a busca por
prazer erótico que se origina no impulso reprodutivo, e o
desenvolvimento e a manutenção da masculinidade e da feminilidade.

 Algumas outras observações, à guisa de orientação. Criado numa tradição
altamente neurofisiológica, embora por natureza especulativa, Freud
sempre se sentiu atraído pelo dilema\idxquest[|(] corpo--mente.\idxfreudmente[|(] Um excelente
observador, quiçá o maior naturalista do comportamento humano de todos
os tempos, ele se deixava enfeitiçar também, pelo menos em igual
proporção, pela especulação biológica.\idxfreudbiolo[|(] Ele queria construir a ponte
sobre o abismo entre as descobertas da biologia, tanto a experimental
quanto a natural, e aquele misterioso produto da neurofisiologia, a
mente. Suas intermináveis ruminações acerca da pulsão --- um termo que
ele usava para preencher aquela lacuna --- são evidência disso.
Portanto, se fosse para solucionar este problema, sua busca bem
poderia colocá-lo cara a cara com as questões da sexualidade, em que
corpo e mente parecem interagir de modo tão flagrante (``O
conceito de pulsão é, portanto, um dos que traça os limites entre o
anímico e o físico'' [24, p.\,168]. Essa mesma busca fez
com que ele fosse estendendo mais e mais o significado de
``sexualidade'' à medida que os anos foram
passando, até que ele tornou o termo sinônimo de
``vida''. Ele deixou que uma palavra ---
``pulsão'' --- fizesse o trabalho que a
metodologia científica deveria ter tentado fazer melhor, através da
observação, seguida por tentativas controladas para confirmá-la. (Se o
dilema mente-corpo poderá ou não ser resolvido um dia, ou até se essa
questão de fato existe, é ainda duvidoso.)\idxfreudmente[|)] Certamente, ele cedo
percebeu que os conhecimentos neurofisiológicos eram ainda muito
rudimentares, e remeteu para o futuro suas esperanças de um dia poder
obter os dados que faltavam. Em nossa fantasia, podemos nos comprazer
imaginando o quanto os avanços na pesquisa sexual teriam agradado a
Freud; ele nunca demorava a recuar diante de novas\idxquest[|)] descobertas, ou a
abrir mão de posições antigas.

 Minha apresentação se baseia nos cinco conceitos sobre a sexualidade,
que podem ser encontrados em meio aos escritos de Freud praticamente
desde o início de sua grande obra --- imediatamente antes e a partir de
1900 --- até a sua morte; examinarei criticamente cada um deles,
confrontando-os com os recentes avanços feitos pela pesquisa sexual. Os
cinco conceitos são: a bissexualidade,\idxbisse{} a\idxsexui{} sexualidade infantil\idxinfansexua{} e o
complexo de Édipo,\idxconfe{} a teoria da libido,\idxlibid{} o primado\idxpenisprim{} do pênis e o conflito.\idxconf{} Embora cada um deles se entrelace aos demais, a fim de dar coerência à
teoria freudiana da sexualidade, eu os separarei, para facilitar sua
análise. Os ``novos avanços'' não serão
examinados em profundidade; serão comentados mais em função do impacto
que causaram.


\section{Bissexualidade constitucional}


  Freud acreditava\idxfreudanali[|(] existir um substrato biológico --- a ``pedra
fundamental'' (34) --- da bissexualidade,\footnote{ Ele usa
``bissexualidade''\idxbisse[|nn] de muitos modos, ignorando
distinções, para chegar assim ao mais alto nível de abstração.
Portanto, frequentemente não temos certeza se ele está se referindo a
um princípio primordial de todas as células vivas, a algum estado
anatômico do embrião, ao prazer anal\idxfanal[|nn] da criança, à amizade entre
pessoas do mesmo sexo, à homossexualidade\idxhomos[|nn] declarada ou a algum atributo
universal da sexualidade humana. Segundo ele, todos esses eram aspectos
do mesmo fenômeno; eu discordo.} sobre a qual todo o desenvolvimento
psicológico posterior se apoiava. Sua base para essa tese veio a partir
de fontes tão bizarras quanto a teoria de Fliess,\idxflies{} dos números
periódicos que controlam o destino humano (28 para as mulheres e 23
para os homens), até os sólidos estudos \mbox{embriológicos} que demonstram
rudimentos de um sexo dentro do outro. Em sua monumental obra
\textit{Três ensaios sobre a teoria da sexualidade}\idxfreudessay{} (1905) ele
estabeleceu sua norma fundamental, da qual nunca se desviou:
``Desde que me familiarizei com a noção de bissexualidade,\idxbisse{} eu a considero como sendo o fator decisivo; se não a levarmos em conta,
 acredito que dificilmente poderemos chegar a uma compreensão das
manifestações sexuais que devemos observar, de fato, em homens e
mulheres'' (24, p.\,208). Dali por diante, isto passou a
ser para ele uma certeza biológica, para a qual ele nunca necessitou de
evidências adicionais.

 A bissexualidade biológica foi, assim, considerada uma forma incipiente
da bissexualidade psicológica, que Freud acreditava estar presente em
todos os seres humanos. Ele (bem como os analistas clássicos que o
seguiram mais de perto) encarava a bissexualidade --- ou, mais
exatamente, o medo dela --- fator etiológico nas psicoses, nas neuroses,
nas perversões, nos vícios, em todas as formas de psicopatologia e,
finalmente, em todos os desenvolvimentos normais. Ela estava presente
como raiz de todos os sintomas, de todos os comportamentos. Em seu
último grande trabalho, \textit{Análise terminável e interminável}
(1938), ele ainda a considerava essencial. Em seu final, como que para
resumir todo o resto, ele disse: ``Frequentemente temos a
impressão de que, com o desejo de um\idxinvej{} pênis [nas mulheres] e o protesto
masculino [nos homens] nós penetramos todas as camadas psicológicas,
chegando à base; e que, assim, nossas atividades chegaram ao fim. Isto
é provavelmente verdade, dado que, para o campo psíquico, o campo
biológico desempenha, realmente, o papel de pedra angular, aquela que
dá sustentação a todas as outras\idxfreudanali[|)] camadas.''\idxfreud[|)] (34, p.\,252).

 Em que pé estamos hoje, em relação a esse conceito que envolve a\idxprenat[|(]
existência de uma bissexualidade\idxbisse[|(] biológica?\idxsexopiden[|(] E em que pé nos encontramos
com referência a uma ideia correlata, a de que tal
``força'' é um fator essencial do\idxsexopiden{} comportamento humano, normal e patológico? Acredito que a maioria dos\idxsexod[|(]
analistas, hoje em dia, acredita em alguma coisa que se assemelha à
bissexualidade\idxbissedese[|(] freudiana (agora chamada, de forma mais estilosa, de ``bipotencialidade sexual'' ou
``dimorfismo sexual''): sabemos que células,\idxhormdese[|(]
tecidos e órgãos, em ambos os sexos,\idxidenpre[|(] podem ser modificados em direção
ao sexo oposto. A maioria de nós, entretanto, não encararia tais
descobertas como indicativas de bissexualidade --- não exatamente da
mesma maneira como Freud compreendia o termo.

 Já foi amplamente demonstrado, sobretudo depois\idxcondfsupr[|(] dos trabalhos sobre
embriologia de\idxjost{} Jost (recentemente publicado [71]), que, nos mamíferos,
é impossível a existência de masculinidade, independentemente do sexo
cromossômico (\textsc{xx} nas fêmeas, \textsc{xy} nos machos),
a menos que o feto secrete hormônios masculinos (o que, aparentemente,
se desencadeia por ação do cromossomo \textsc{y}). Especialmente o cérebro necessita
desta masculinização, nos mamíferos --- caso contrário, a feminilidade se instalará.

 Que os humanos, via de regra, compartilham essa regra geral, do caráter
feminino do tecido mamífero, isto parece se corroborar quando
examinamos as \mbox{``experiências} naturais''\idxexpen{} causadoras dos transtornos endócrinos.\idxanimdese{} Vemos ali, em cada caso, que o
feto que é privado de androgênio,\idxandrn{} durante alguns períodos críticos de
seu desenvolvimento, acaba não desenvolvendo as qualidades anatômicas
masculinas. Por exemplo, o bebê menina \textsc{xo} (síndrome de Turner),\idxturner{} sejam
quais forem suas anomalias, não tem tecidos masculinos, pois ela não
tem gônada para produzir androgênio; na síndrome da insensibilidade
androgênica,\idxandrs{} a incapacidade dos tecidos-alvo em responder ao androgênio
circulante limita o desenvolvimento da feminilidade no feto. Por outro
lado, o feto feminino que é exposto a uma dose aumentada de androgênio
se masculiniza e,\idxadren{} em casos extremos, o clitóris\idxclitdese{} é anatomicamente
indiferenciável de um pênis.

 Mas estes exemplos só ensinam a respeito de anatomia; nenhum deles, por
si só, resvala nas teorias psicanalíticas do comportamento, exceto para
indicar, como Freud,\idxfreud[|(] aliás, já havia notado, que os tecidos do
organismo masculino podem ter aparência feminina, e vice-versa para as
fêmeas. O interesse de Freud, como psicólogo, não estava nessas
questões anatômicas, mas no dilema\idxquest{} corpo--mente.\idxfreudmente{} De que forma esses
estados fisiológicos afetam o comportamento? Aqui, estamos
especialmente em dívida com John Money,\idxmoney{} cujos estudos sobre pessoas com
esse tipo de transtornos endócrinos, como os referidos acima, sugerem
que o cérebro do feto humano também necessita ser
``excitado'' pelo androgênio, para um
desenvolvimento masculino normal; e que, se o cérebro de um feto
feminino for exposto a andrógenos, um aumento discreto, porém
mensurável, no comportamento masculino da menina em crescimento poderá
ser esperado, em comparação às meninas do grupo de controle (108).
Outros estudos sugerem que um número incomum de homens com
hipogonadismo congênito --- do que se infere, presumivelmente, uma
quantidade de androgênio\idxandrn{} insuficiente no período fetal (por exemplo, a\idxsexod[|)]
síndrome de Klinefelter),\idxkline{} são femininos no comportamento desde os\idxsexopiden[|)]
primeiros dias da infância,\idxbissedese[|)] independentemente da\idxcondfsupr[|)] \mbox{criação} que tenham recebido (110, 137).\idxprenat[|)]

 Artigos recentes sugeriram que a homossexualidade\idxgenethomo{} masculina é causada\idxsexophomo[|(]
essencialmente por forças biológicas. Os geneticistas lançaram a
hipótese de que a homossexualidade\idxhomosbiol[|(] fosse herdada\idxhomoshera{} (72, 128). Um
pesquisador descreveu a cura instantânea da homossexualidade por meios
neurocirúrgicos\idxhomostrat{} (coagulação do núcleo de Cajal no hipotálamo
ventromedial), o que implicaria a existência de um centro específico
para esse tipo de comportamento (122, 9).\idxidenpre[|)] Outros estudos demonstraram
existir uma relação direta entre o grau de homossexualidade e alguns
fatores, tais como uma diminuição dos níveis de testosterona no plasma,
uma espermatogênese comprometida (83) ou uma proporção anormal entre
androsterona e \mbox{etio}cholanalone (98).\idxhormhomo{} Estes estudos não se contradizem
entre si; eles poderiam estar medindo diferentes aspectos do mesmo
processo: genes, neuroanatomia ou processos químicos. (Pode ser também
que eles estejam redondamente enganados em suas conclusões.) E nenhum
deles contradiz, necessariamente, a teoria psicanalítica, se cada um
deles for uma peça da ``pedra angular'',
basilar, a que Freud se referiu. Entretanto, se se pretender que esses
mecanismos biológicos sejam \textit{a} causa da \mbox{homossexua}lidade,\idxhomosbiol[|)] então
certamente a teoria freudiana dos distúrbios nos\idxhormdese[|)] relacionamentos
interpessoais (conflito edípico e pré-edípico)\idxconfe{} é posta em cheque.\idxsexophomo[|)]

 Mas esses estudos fisiológicos, animais ou humanos, não elucidam o
comportamento sexual humano; eles nos falam apenas a respeito de
potencialidades biológicas subjacentes, como tantos outros estudos o
fazem, em relação a um aspecto do comportamento humano. (Um ataque
epilético nos diz alguma coisa a respeito de agressão e violência ---
assim como o faz um gato decorticado --- mas isso não nos diz tudo.)
Como sempre, o que foi feito dessas potencialidades é, em geral,
encontrado na área das influências ambientais. Para obter auxílio no
que diz respeito a isto, voltemos para as teorias interpessoais, para
as relações objetais e para o aprendizado social.

\section{A sexualidade infantil \mbox{e o Complexo de Édipo}}

Freud acreditava\idxbisseteor[|(] que a\idxconfe{} bissexualidade\idxsexui[|(] 
(assim como toda sexualidade)\idxinfansexua[|(]
deriva de duas fontes. A primeira, como já vimos, é biológica. Ela
determina um fator que é inalterável na psicologia humana e que conduz,
no homem, a um medo de não ser viril e, nas mulheres, a um anseio por
masculinidade. A segunda fonte é o meio-ambiente.

``Seria um erro acreditar que, a partir da instalação
dos diversos componentes da constituição sexual de um indivíduo, já se
estabeleça com precisão a forma que sua vida sexual tomará para o resto
da vida. Pelo contrário, ela continuará a ser determinada pelo meio
externo [24, p.\,237]\ldots{} O fator constitucional precisará vivenciar
experiências, antes de se revelar; o fator acidental deve ter uma base
constitucional, para que possa entrar em operação. Para cobrir a
maioria dos casos, podemos imaginar o que tem sido descrito como uma
`série complementar', em que a intensidade decrescente de algum fator
é compensada pela intensidade crescente de outro; contudo, não há razão
para negar a existência de casos extremos, em ambas as pontas da série.'' [p.~239]

Por ``fator acidental'' Freud\idxfreudinsti{} se refere à
experiência ambiental, afirmando aquilo em que a maioria de nós
acredita: a importância do biológico, ou do ambiental, na determinação
do comportamento sexual, varia de pessoa para pessoa, e de época para
época. Ao sentir que todas as suas teorias sexuais eram, em última
análise, biológicas, ele não fez uma diferenciação adequada entre um
filosofar biológico e os rigores da pesquisa biológica. A primeira
dessas vias deveria ser ignorada (embora, infelizmente, ela tenha
servido como base das mais ferozes batalhas dentro da psicanálise,
sendo um exemplo o conceito de energia psíquica,\idxenerg{} a pulsão de vida
contra a pulsão de morte, a teoria da libido\idxlibid{} e a herança lamarckiana de
experiências passadas da raça humana), a menos que os analistas estejam
dispostos a abraçar também o trabalho científico necessário para a
defesa dessas\idxfreudbiolo[|)] teorias.

 De fato, pondo de lado a ``biologização'' de Freud, consideramos suas contribuições, agora esclarecidas, talvez até mesmo mais formidáveis. Dentre elas, uma das \mbox{maiores} foi o modo como ele enfatizou a sexualidade da \mbox{criança}, tanto em sua fase de bebê (o \textit{infans}) quanto durante a infância. Ele destacou a importância crucial dos relacionamentos progenitor--criança\idxrelpc[|(] (examinaremos isso em breve, ao discutir o complexo de Édipo) como pedra angular de seu trabalho, quase que desde o início. O que Freud empreendeu, já a partir de 1900 (23), foi a mais formidável teoria de aprendizagem social de que se tem notícia\idxbisse[|)] para explicar o desenvolvimento humano.

 Durante décadas, suas especulações biológicas não foram confirmadas.
Entretanto, suas teorias interpessoais\idxfreudinter{} e suas observações sobre as
interações progenitor--criança, que em muitos de seus pontos
fundamentais não foram refutadas, constituíram um rico manancial para
inúmeros pesquisadores --- e continuam a sê-lo, até os dias de hoje
(inclusive para muitos daqueles que, vergonhosamente, recusaram-se a
reconhecer o presente que lhes foi dado).

 Freud nos disse, como ninguém jamais antes, que os pais exercem a maior influência possível sobre o desenvolvimento de seus filhos; que, em resposta a isso, as crianças criam sua estrutura psíquica, que pode ser \mbox{rastreada} na vida sexual adulta. A fonte dessa influência é a primeira infância, onde se originam também o desejo e a satisfação sexual, muito antes de seu óbvio irromper na puberdade. De que forma se dá,
exatamente \textit{como} os pais transmitem essas influências a seus
filhos, este é um tema que tem sido objeto de crescente estudo através
dos anos, por parte de analistas e de não analistas. As ideias de Freud
estimularam as pesquisas de fisiologistas, behavioristas, etólogos,
especialistas da teoria geral de sistemas\idxbisseteor[|)] --- enfim,\idxfreud[|)] são inúmeros os
pesquisadores que hoje acreditam que tanto a primeira infância, como a\idxrelpc[|)]
infância em si, são fases cruciais do desenvolvimento.

 Os teóricos da aprendizagem social, como são chamados nos círculos
acadêmicos, ou teóricos das relações objetais, como são conhecidos
entre os psicanalistas, têm, naturalmente, grandes diferenças entre si,
em termos de teorias. Essas diferenças, contudo, podem mascarar uma
similaridade que é preponderante: a crença de que o comportamento pode
ser radicalmente modificado única e exclusivamente pela influência de
uma pessoa sobre a outra. Esses pesquisadores compartilham também a
crença de que a personalidade do bebê e da criança pequena, mais do que
a do adulto, é especialmente vulnerável a uma modificação
comportamental permanente. A área em que acontece o seu maior
desencontro talvez seja a que trata de saber se o trauma,\idxtrauma{} o conflito, a
defesa e o acordo de conciliação, como formas de resolução de
conflitos, contribuem de maneira fundamental para a formação da
personalidade.

 Devemos, aqui, fazer distinção entre os dois diferentes aspectos da
sexualidade que foram apresentados anteriormente. O primeiro é aquele
que se preocupa com o prazer genital, mais ou menos ligado ao
comportamento de reprodução --- ou ao desejo de evitá-la; o segundo é o
comportamento de gênero,\idxiden{} aquele relacionado à masculinidade e à
feminilidade. Aqui, mais uma vez, quero enfatizar o perigo da
extrapolação do comportamento animal\idxanimvers[|(] para o comportamento humano ---
ainda que tal precaução esteja, hoje em dia, fora de moda. De todas as
áreas de comportamento em que existe descontinuidade entre os animais ---
 mesmo os primatas --- e o homem, é no comportamento motivado que essa
descontinuidade é mais flagrante. A lei da evolução propõe não só que
certos comportamentos fundamentais persistem, por estarem vinculados a
estruturas anatômicas e a circuitos que se mantêm constantes através
das espécies, mas também que, quanto mais alto se sobe na escala
evolucionária, maior a quantidade de escolhas à disposição do
organismo.

 Os substratos cerebrais\idxcereb[|(] daquilo a que chamamos de
``escolha'' ou ``liberdade'' simplesmente inexistem em
qualquer outra criatura: são exclusivamente humanos. Ninguém
questionará com seriedade o fato de o potencial humano para
variabilidade de comportamento ser maior do que o de qualquer outro
animal, ou que o comportamento humano, mesmo em suas raízes
neurofisiológicas, tem mais necessidade de ser marcado (organizado)
pelo meio ambiente. Por exemplo, podemos procurar \textit{ad infinitum}
na escala evolucionária para encontrar as raízes biológicas da
agressão,\idxagres{} mas não podemos determinar, a partir daquela origem, por que
o homem assassina seu semelhante com tanta facilidade. Talvez, algum
dia, encontremos um foco talâmico para a ereção peniana no homem, do
mesmo modo como acontece com os macacos (89); esse tipo de
comportamento --- fundamental --- obedece a regras evolutivas. Porém,
existe todo um outro nível de comportamento que é muito mais
complicado, embora a ação final seja simples e fisiológica --- a saber, a
ereção. O estímulo que conduz a esse último comportamento não é
meramente talâmico ou hipotalâmico, mas percorre uma neurofisiologia
desconhecida, resultante da experiência prévia que está fixada na
memória, e que se modifica pela fantasia\idxfanta{} (especialmente pela fantasia
inconsciente). Tal processo é único do ser humano. A relação que o
homem tem com suas lembranças é diferente da de todos os outros
animais: \textit{ele simboliza e fantasia} e, desta forma, não apenas
refaz o passado, como também inventa, antecipando, o futuro.

 Retornando a nosso assunto, a perspectiva evolucionária não conseguiu
nos ensinar muita coisa em relação aos desejos sexuais humanos, às suas
escolhas objetais ou à sua patologia. Por exemplo, as perversões, com
sua necessidade habitual, instintiva, por satisfação genital aberrante,
não são encontradas em nenhum outro animal das espécies inferiores
vivendo em liberdade --- mas são onipresentes no homem. As débeis
tentativas de demonstrar perversão em animais inferiores (por exemplo,
a afirmação de que as vacas que montam em outras fêmeas são
homossexuais) não são nada convincentes.

 De forma similar, estudos reveladores de distorções de comportamento,
que certas experimentações são capazes de induzir permanentemente em
animais (como é o caso dos condicionamentos,\idxcond{} da técnica de
\textit{imprinting}\idxestam{} ou por intermédio da
implantação de eletrodos em determinadas estruturas do cérebro),
remetem-nos a potencialidades, mas não nos dão respostas acerca do
comportamento livre do homem; os dados não vão além de nos fornecer
outras questões que deveríamos nos colocar. Em outras palavras, essas
descobertas em relação aos animais não confirmam nada sobre os humanos:
apenas fornecem indícios.\idxanimvers[|)] É importante para os analistas, entretanto,
reconhecer que, ao construírem suas teorias,\idxcereb[|)] eles ignoram\idxsexui[|)] esses
indícios por sua própria conta e risco.

 Para Freud,\idxfreud[|(] o desenvolvimento sexual\idxinfansexua[|)] depende poderosamente dos
relacionamentos entre pais e filhos, ou seja, do complexo de\idxconfe[|(] Édipo.\idxmascucon{} Revisemos rapidamente esta hipótese.

 Primeiramente, a masculinidade. O bebê menino,\idxconfemeni[|(] abençoado por sua
condição biológica inerentemente superior, começa a vida como\idxconfehete{} heterossexual,\idxheteroconf[|(] diz Freud.\idxfreudedipo[|(] A partir do momento\idxidenedi[|(] em que se torna um ser
separado --- no momento do nascimento --- o primeiro objeto de sua
consciência, intimidade, necessidade e amor é uma pessoa do sexo
oposto, uma mulher: sua mãe. À medida que se conscientiza do mundo à
sua volta e de seu próprio corpo, ele percebe que seu pênis é uma fonte
de intensas sensações e que lhe serve como prova principal de sua
masculinidade --- portanto, de sua superioridade. Isto pode, inclusive,
ser reforçado por um sentimento herdado, inconsciente, de supremacia.
Ele logo aprende, através da observação (e quem sabe, mais uma vez de
modo atávico, até mesmo como parte da sabedoria do inconsciente
coletivo) que ele é diferente de um outro grupo, as mulheres. A essa
altura da infância, ele já dá grande valor à sua masculinidade.
À medida que seu corpo se desenvolve, ele atinge uma fase em que seu pênis
é foco de intensas sensações eróticas. Esta excitação, com sua demanda
por gratificação, fica inextricavelmente vinculada a seu primeiro e
contínuo objeto amoroso --- a mãe --- de tal modo que ele deseja
tomar o lugar do pai na vida dela. Entretanto, por ser pequeno e
vulnerável, ele não pode fazê-lo, pois seu poderoso pai\idxmascupat{} lhe bloqueia o
caminho. Toda e qualquer esperança que ele possa ter de possuir a mãe
lhe é arrancada pela ameaça de castração,\idxcastaconf[|(] com a inevitável angústia que
ela implica. E assim, durante um período de vários anos, ele luta para
controlar seus desejos edípicos; terá êxito nessa empreitada, sem
sérios prejuízos à sua masculinidade,\idxpaismasc{} se conseguir compreender que
seus desejos sexuais pela mãe podem ser postergados para um momento
futuro, quando serão transferidos para outra mulher. O pai, embora
seja um rival em sua vida, torna-se também seu aliado, servindo como
modelo de masculinidade e encorajando o menino a imitar seu
comportamento masculino --- desde que isso não implique a posse\idxconfemeni{} da mãe.

 O curso do desenvolvimento sexual da menina é, de acordo com Freud,
mais tortuoso. Ele diz que ela começa com um relacionamento
homossexual, pois seu primeiro amor é uma mulher. O segundo obstáculo
que deverá ultrapassar, em seu desenvolvimento, será a descoberta de
que existem pessoas --- homens --- equipados com uma aparelhagem sexual
superior; assim, nos primórdios da infância, a menina sente inveja\idxinvej[|(] por
não ser menino, e culpa sua mãe\idxmaesconf{} por esta privação. Na verdade, algumas
meninas acreditam não terem sido simplesmente privadas desse atributo:
acreditam tê-lo possuído, em alguma época pregressa, e que ele lhes foi
posteriormente extirpado. A partir daí, se quiser se tornar feminina, a
menina precisará desistir de suas esperanças de poder ser homem (ou de
sua fantasia de ter sido um, anteriormente) e, resignada face a essa
derrota, enveredar por uma nova trilha: a da feminilidade. À medida que
for capaz de agir assim, ela transferirá o amor sentido por seu
primeiro objeto, homossexual, para o pai. Este processo, em boa parte,
só alcançará êxito se a menina abrir mão da fixação em seu clitóris,\idxfeminfixa{} que ela encara como sendo um pequeno pênis. Isso será possível caso ela
se volte para o pai, na esperança de que ele possa lhe dar o substituto
ideal do pênis, qual seja, um bebê. Se ela puder fantasiar que as
coisas se passarão assim, acabará transferindo a sensibilidade erógena
do clitóris\idxclitfemi{} para a vagina --- mas este é um processo incerto, o que se
faz comprovar pelo grande número de mulheres que dependem do\idxvaginorg{} orgasmo\idxorgascli{} clitoridiano. A sexualidade da mulher madura, parte da qual é a
feminilidade, é, portanto, marcada, segundo Freud, pela capacidade do
orgasmo vaginal; e as mulheres incapazes de senti-lo são, por
definição, não femininas, não importando sua aparência, seus interesses
ou as fantasias que tenham.

 A situação edípica é, portanto, o ponto crucial do desenvolvimento
sexual.\idxrelpciden[|(] No menino, a masculinidade só vingará se ele obtiver sucesso em
ultrapassar os perigos da castração imaginária, que seria feita pelo
pai, enquanto na menina, o desenvolvimento da feminilidade antecede o
conflito edípico,\idxfeminconf[|(] e é essencial para que ele ocorra;\idxcastaconf[|)] só quando a menina
se torna feminina é que ela abre mão de seu apego à mãe e tenta
alcançar o pai. Nisto, ela será frustrada pela mãe --- que, mais uma
vez, se torna a vilã (a primeira vez foi quando a mãe a privou de seu
pênis). Portanto a menininha, assim como o garotinho, se tiver sucesso,
adiará sua completa heterossexualidade\idxconfehete[|(] para quando ficar mais velha e
for capaz de exercê-la, ao encontrar outro homem que não seja seu pai.
Assim, a masculinidade, nos meninos, requer o sucesso na resolução da\idxinvej[|)]
situação edípica, ao passo que a feminilidade, nas meninas, requer\idxconfemeni{} apenas que ela chegue às portas desse conflito.

 Como Freud encarou esse processo como mais carregado de conflito, e
mais tortuoso, para as meninas, ele acreditou ter uma explicação para
sua convicção: a de que a sexualidade da mulher adulta é mais incerta,
menos gratificante e mais misteriosa que a dos homens. Um acréscimo
curioso a essa teoria era a sua crença em que o desenvolvimento dos
garotinhos e das garotinhas fosse mais ou menos o mesmo, até a
instalação do desenvolvimento completo da situação edípica, por volta
dos cinco ou seis anos; e que, consequentemente, nenhuma feminilidade
verdadeira, significativa, estaria presente nas meninas antes dessa
idade. ``Ambos os sexos parecem passar pelas primeiras
fases do desenvolvimento libidinal da mesma maneira\ldots{} Ao entrar na
fase fálica,\idxfasef{} as diferenças entre os sexos são completamente eclipsadas
por sua semelhança. Somos, agora, obrigados a reconhecer que a\idxconfemeni[|)]
garotinha é um garotinho''\idxfreudedipo[|)] (33, pp.~117--118).

 Alguns autores (por exemplo\idxmalino{} Malinowski) [97] alegaram que o complexo de Édipo é imaginário, por suas formas serem diferentes em algumas sociedades, como naquelas em que o lugar genético do pai, como pai psicológico, é tomado por um dos parentes da mãe. Porém, até o momento, inexistem descrições de culturas\idxrelaconf{} em que a criança,\idxconfeestu{} ao crescer, não erga seu olhar em direção a um homem, grande e poderoso, que servirá como modelo de masculinidade\idxpaismasc{} para os meninos e como modelo de objeto heterossexual,\idxconfehete[|)] para as meninas; ou em direção a uma mulher, que é a mãe.\idxfeminsimb[|(] O que varia, de família para família e de cultura para cultura, é
a quantidade de conflito existente no âmago do complexo,
independentemente de as famílias serem constituídas ou não por uma mãe,
 um pai e crianças cujos atributos, em termos de poder e de sexo,\idxheteroconf[|)]
equivalham, aproximadamente, aos observados por Freud em Viena.

 Nos dias de hoje, em que pé está essa teoria, do desenvolvimento da
masculinidade e da feminilidade?\idxmaesdese[|(] Um novo fator foi lançado à discussão
pelos estudiosos dos primeiros estágios do desenvolvimento sexual, e
que contradizem Freud em várias colocações. Descobriu-se que alguns
meninos, em virtude de uma singularidade no modo como são criados, são
notavelmente femininos já desde os primeiros dias de vida. Eles
passaram um tempo excessivo numa intimidade intensa, paradisíaca, com
suas mães; e que as mães que têm maior propensão para criar essa
proximidade com seus filhos, tendem a se casar com homens distantes e
passivos. No geral, quanto mais pura a forma tomada por essa
constelação numa família, mais cedo e mais arraigada e irreversível é a
feminilidade que se desenvolve no menino (142). Por outro lado,
descobriu-se que os meninos que têm um relacionamento próximo com o
pai, não têm mães como essas, e que esses meninos são masculinos (4).

 Uma menina que tem uma mãe distante, pouco afetuosa, mas cujo pai lhe é
próximo, desenvolverá masculinidade\idxpaismasc{} caso\idxmascumen{} seu pai a encoraje a adotar os
mesmos interesses que os seus (141). Uma menina cuja mãe aprecia o fato
de ter uma filha, que não se envergonha por sua filha ter um corpo
feminino, e cujo pai a encoraje à feminilidade, crescerá feminina (80).

 Ainda resta a esclarecer se a comunicação entre cada um dos pais e o
recém-nascido molda o comportamento do bebê por impregnação, por
condicionamento (clássico ou operante), por identificação, ou por uma
combinação de todos esses fatores. O que fica demonstrado pelo conjunto
dos estudos, todavia, é que o tipo de interação que ocorre entre os
progenitores e o bebê exerce um papel fundamental na geração de
masculinidade e da feminilidade, em ambos os sexos. Isto reduz o
aspecto conflitante (angústia de castração)\idxcastaconf{} do desenvolvimento da
identidade de gênero; em contraste à teoria de Freud, essa descrição,
em que o desenvolvimento ocorre sem que exista conflito, tem um papel
proeminente.

 Neste ponto, surge um desacordo maior com a teoria analítica clássica.
Ao mesmo tempo em que é suficientemente óbvio que o primeiro objeto de
amor do menino é uma mulher (a mãe), a grande intimidade física e
\mbox{emocional} (a fusão) que acontece entre ele e o corpo e a psique da mãe,
 em seus primeiros estágios da vida, introduz o perigo de um sentimento
de unidade, de constituir um único corpo com uma mulher. E, assim, uma
das primeiras tarefas do menino, em seu caminho em direção à
masculinidade, é a de se separar da mãe (cap.\,8); esse processo pode
ser subvertido por uma mãe que propicie uma intimidade excessiva. Essa
mesma intimidade não coloca a menina em perigo, pois a proximidade com
a mãe apenas encorajará sua feminilidade.

 Novos fatos estão surgindo, e que revelariam uma diferença na forma
como as mães lidam com bebês do sexo masculino e com os do sexo
feminino (13); o mais comum (que tende a produzir feminilidade nas
meninas e masculinidade nos garotos), é que as meninas tenham um maior
contato, físico e visual, com as mães, do que os meninos, durante os
primeiros meses (48). Em geral, as mães têm mais facilidade em
estabelecer intimidade com seus bebês meninas do que com os bebês
meninos. Portanto, o menino não tem o desenvolvimento sexual retilíneo que Freud alegava. Ao invés disso, ele tem um impedimento da maior grandeza em seu caminho rumo à heterossexualidade: ele terá que se livrar de qualquer vestígio de feminilidade que possa vir a ter desenvolvido durante a fase de simbiose mãe--bebê. Somente então, num estágio posterior, ele poderá encarar a mãe como o objeto separado e
desejável da situação edípica clássica\idxmaesdese[|)] (26).

 Assim, em vez de as meninas serem garotinhos, esses dados indicariam
que as garotinhas são moldadas para a feminilidade desde o início. E
isso é o que a simples observação revela: as meninas, em geral,
simplesmente não são masculinizadas nos primórdios da infância. Uma
feminilidade nítida, via de regra, poderá ser vista por volta de um ano
de idade; e nada indica que essa feminilidade seja uma fachada, ou uma
imitação de feminilidade.\idxfeminsimb[|)] É por isso que não posso concordar com a
afirmação de Freud: ``Como todos sabemos, uma distinção\idxrelpciden[|)]
nítida entre o caráter masculino e o caráter feminino só se estabelece\idxidenedi[|)]
na puberdade'' (24, p.\,219).\idxconfe[|)] Estas questões serão\idxfeminconf[|)]
examinadas em detalhe mais adiante (capítulo 8).


\section{O primado do pênis}

Freud aceitou,\idxfreudsuper[|(] sem\idxpenisprim[|(] jamais questionar,\idxmascsu[|(] a crença na superioridade\idxcondfinfe[|(] do sexo
masculino. Ele supunha que isso fosse um fato estabelecido em relação a
todos os mamíferos, pela superioridade física do macho, em termos de
força: os machos mais capazes são selecionados vencendo combates
mortais, em que a força física é decisiva. Este fato, que conta com o
pênis como sua representação simbólica mais impactante, refletia-se
então na mitologia, nos contos folclóricos, nas instituições da
sociedade, nas produções artísticas, nos cultos religiosos, nos sonhos
--- enfim, em toda parte.

 Na família, esse poder era garantido ao pai não apenas porque os
costumes assim o exigiam, mas também porque, desde a antiguidade, era
sua responsabilidade tácita proteger a família do perigo físico,
garantir o alimento e, pelo menos, um mínimo de conforto; também
porque, como o membro fisicamente mais forte da família, ele exercia um
controle de vida e morte sobre cada um de seus membros. Tal poder,
derivado, em última instância, da evidência da força física,\idxpaisauto{} estava
institucionalizado na sociedade, desde os mais altos escalões do
governo até a família.

 Freud,\idxfreudident{} que provinha de uma cultura em que essa autoridade estava ainda
claramente depositada no pai, não precisou questionar esse princípio,
segundo o qual ``anatomia é destino''.
Contudo, qualquer teoria que abrace essa ideia como seu elemento
constitutivo essencial ficará enfraquecida se ela não for correta.

 O entusiasmo de Freud pela superioridade masculina se encaixava bem naquilo que ele considerava como um fato observável: que as mulheres são enigmáticas e pouco sinceras (24, p.\,151), mais masoquistas (33, p.\,116), menos autossuficientes (p.~117), mais dependentes e \mbox{influenciáveis} (p.~117), mais invejosas e ciumentas (p.~125); têm superegos imperfeitos (p.~129) e escasso senso de justiça (p.~134); são mais bissexuais (p.~113), mais narcisistas (p.~132), têm menos interesses sociais (p.~134), menos capacidade para sublimar os instintos (p.~134) e se tornam mais rígidas, incapazes de mudar, mais precocemente (pp.~134--135). Elas são intelectualmente inferiores por serem biologicamente criadas para a tarefa, não intelectual, da maternidade; moralmente, elas são inferiores porque, por já serem desprovidas de pênis, não podem ser facilmente ameaçadas e, estando mais concretamente ligadas ao mundo real, elas, consequentemente, preocupam-se menos com questões estéticas, como a moralidade (ou seja, elas obedecem ao comando da biologia mais do que a qualquer outro tipo de apelo etéreo).

 É corolário da tese da superioridade masculina que o atributo principal
da virilidade, o pênis, é um órgão superior, tanto física quanto
simbolicamente --- e Freud podia apontar para o culto ao falo, em sua
miríade de formas, como prova a ser fornecida àqueles que não
escutavam, como ele o fazia, os sonhos dos homens e das mulheres. No
conceito de angústia de castração,\idxcasta{} ele encontrou razões para acreditar
que os homens consideravam o pênis o principal órgão da raça e, na
inveja\idxinvej{} do pênis, ele encontrou a prova de que as mulheres também
estavam de acordo, em relação a seu primado. O fato de o pênis ser um
órgão visível, de poder mudar de forma, de poder assumir o formato de
uma arma e poder penetrar e atemorizar as mulheres, por ser fonte de
sensações tão intensas já desde a infância, tudo isso também demonstra
sua superioridade. Quando contrastado à genitália feminina, a tese se
comprova uma vez mais. O falo da mulher, o clitóris,\idxclit{} é muito menor, em
geral não é visível, não pode penetrar, não se apoderou da imaginação
da humanidade, jamais foi simbolizado ou exaltado e --- acreditava Freud
--- não é uma fonte qualificada de prazer. Seu significado fica ainda
mais enfraquecido por precisar compartilhar seu destino com um outro
órgão, a vagina,\idxvagin{} que Freud julgava como sendo universalmente
considerada um órgão inferior: oculto, escuro, misterioso, incerto,
sujo --- e em que não se podia confiar, em termos de propiciar\idxcondfinfe[|)] prazer.

 Tudo isso é um bocado de evidência;\idxmascsu[|)] para onde quer que ele olhasse,
fosse para o mundo exterior ou para a vida mental, o primado\idxfreudsuper[|)] do pênis
lhe parecia um fato comprovado.

 Contra o argumento de Freud, o que temos a oferecer, como excitante
novidade, é a pesquisa a que fizemos alusão anteriormente. Nas espécies
mamíferas, a função das células é feminina, em ambos os sexos, até que
o androgênio seja adicionado durante a vida fetal. Na realidade, exceto
pelos cromossomos, não é possível se falar em dois sexos até que o
androgênio tenha sido acrescentado; até ali, só existe a condição de
fêmea.\idxcondfsupr{} Freud, que sempre teve faro para o mistério, e que descobriu
tantos mistérios nos mais significativos níveis --- funções celulares,
ou mistérios mais primitivos ainda --- teria ficado embaraçado com essa
descoberta. E ficaria ainda mais perplexo, em sua argumentação, ao
tomar conhecimento de que essa matiz feminina, em termos de tecido, se
estende até o sistema nervoso central --- local em que, como tem sido
demonstrado agora em outros mamíferos, e não só no homem, o futuro
comportamento masculino, no macho, requer a organização que somente o
androgênio produz; ao passo que, na fêmea, nada há a ser acrescentado à
sua feminilidade. Portanto, a nova pesquisa pareceria colocar toda a
argumentação de Freud na mais precária das condições; e, tendo ele
optado por estender suas crenças do reino das dinâmicas psíquicas para
a moralidade e outras questões cósmicas, nestes últimos tempos sua
teoria tem sofrido duros golpes.

 Em todo caso, ainda não deveríamos considerar os novos dados como
prova de que a humanidade não acredita no primado do pênis; podemos,
ainda, nos perguntar onde é que, na psique da criança, reside esse
conhecimento embriológico, ou da capacidade dos tecidos. É impossível
encontrá-lo. Mas podemos, com facilidade, detectar as atitudes de
meninos e meninas em relação ao pênis: eles ainda o consideram bastante
impressionante. Isso é primado? Alguns já não pensam assim, em vez disso,
consideram uma pena que Freud não tenha enfatizado com
maior vigor que as crianças, de ambos os sexos, também ficam bastante
perturbadas pelo significado dos seios e do útero; o poder reprodutivo\idxcondfcapa{} é
mais difícil de se representar visualmente, mas, se ele for mensurado pela
quantidade de mistério que gera, torna-se até mais importante do que o
pênis.

 Além disso, as observações de Masters\idxmaste{} e Johnson\idxjohns{} (102) causaram um
grande abalo nas ideias de Freud sobre a feminilidade. Freud dizia que
uma menina é masculina até desistir de sua esperança de ter um pênis;
enquanto essa esperança dura, ela retém uma fixação em seu clitóris,\idxfeminfixa{} como se ele fosse um pênis. Só ao deslocar seu erotismo para o espaço
procriador interno, a vagina e os órgãos pélvicos, ela se tornará
feminina. Porém, Masters e Johnson descobriram que todos os orgasmos
femininos têm sua origem no\idxclitfemi{} clitóris;\footnote{ Parece que ninguém se
lembra de Freud também ter dito\idxfreudclito[|nn] o seguinte: ``Quando
finalmente o ato sexual é permitido (pela primeira vez) e o próprio
clitóris\idxclit[|nn] se excita, ele ainda acumula uma função: a saber, a tarefa de
transmitir excitação às partes sexuais femininas adjacentes, exatamente
como --- mal comparando --- é possível acender algumas aparas de
pinho para fazer com que uma tora de madeira mais dura se
incendeie'' (24, p.\,221).} eles não tiveram notícia de
nenhum orgasmo\idxvaginorg{} vaginal. Assim, o argumento de Freud parecia ter sido
posto por terra.

 E acredito que o tenha sido --- mas de maneira alguma pelo trabalho de Masters e Johnson. São incontáveis as mulheres que tiveram a sensação de que têm dois tipos de orgasmos, um clitoridiano\idxclitorga{} e outro vaginal; e elas não têm qualquer dificuldade em distinguir um do outro. Só porque as grandes alterações vaginais, no momento do orgasmo, não são visíveis
a esses observadores, isso não prova a inexistência do orgasmo vaginal. Talvez por sua fisiologia --- como pode acontecer com uma dor ou uma coceira intensa em um músculo, ou na pele --- sua visualização não é \textit{flagrante;} ou, quem sabe, o orgasmo vaginal requeira relações sexuais com um homem que signifique alguma coisa para aquela mulher e, por isso, não pode ser reproduzido em laboratório; ou talvez tanta ação esteja acontecendo na vagina, no momento de um orgasmo produzido pela penetração, que o fenômeno fique obscurecido. O trabalho deles não refutou a existência de um orgasmo que pode ser sentido mais no âmago
do corpo do que o orgasmo clitoridiano. Portanto, o argumento deles não desbanca Freud. Mas outro tipo de argumento o faz: existe um número imenso de mulheres que sabemos nunca terem experimentado orgasmos vaginais (111), e existem outras --- e seu número é igualmente grande --- que o experimentam e que, pela definição freudiana de maturidade, não podem ser consideradas femininas --- trata-se de mulheres esquizofrênicas, mulheres neuróticas de todos os tipos e graus e, até mesmo, de mulheres francamente masculinas.

 Mas trata-se, aqui, de uma discussão não científica, mascarada por uma racionalização pseudocientífica (106, 127). Como alguém poderá provar que um sexo é superior ao outro, se primeiramente não forem fixadas as
categorias a serem mensuradas? Se a superioridade for medida pelo
tamanho do corpo, pelas dimensões fálicas, pelo talento no futebol,
pela paternidade ou pela produção de esperma, as mulheres são, de modo
inequívoco, inferiores; as diferenças podem ser medidas. Da mesma
forma, se a superioridade for medida pelo tamanho do seio, da
capacidade gestacional, ou pela capacidade de ovular, as mulheres levam
uma grande vantagem. No ponto intermediário, existem inúmeros outros
campos em que nem homens nem mulheres levam vantagem, tais como a
tecelagem, o cultivo de arroz, a resolução de problemas na pesquisa
psicanalítica, a administração de uma agência de propaganda, ou a
capacidade de dar muita importância a ninharias. Além disso, existem
imponderáveis, tais como: será que uma mulher é superior, por poder ter
um número ilimitado de orgasmos? Um homem é superior por ficar
totalmente satisfeito após ter tido um só orgasmo, ou após ter tido
cinco? Todos esses tipos de bobagem, a respeito de que tanto se discute
em nossos dias, não provam nada. Em vez de pontificar acerca de
pretensas superioridades, poderíamos simplesmente tentar observar o
desenvolvimento dos homens e das mulheres, e da masculinidade e da
feminilidade. Livremo-nos do ônus de ter que decidir\idxpenisprim[|)] qual dos sexos é o
melhor.


\section{A teoria da libido}

 A teoria\idxsociadese[|(] da libido\idxlibid[|(] é parte da teoria geral dos instintos de Freud.
Não me deterei aqui nas questões científicas e epistemológicas, que
foram sendo levantadas com o passar dos anos, sobre o conceito de
instinto (ou pulsão); em vez disso, meu desejo é discutir apenas a
teoria da libido: aquela descrição que define a maturação da
sexualidade como movimento e elaboração através de estágios, cada um
dos quais tendo seu foco em uma parte diferente do corpo.
Concomitantemente a essa sua teoria das relações objetais,
corporificada em sua descrição do complexo edípico, Freud viu um
desenvolvimento governado por um mecanismo de tempo herdado, no qual ---
em todos os humanos --- a ``energia psíquica''\idxenerg{} converge para uma parte do corpo, ``catexizando-a'' com a ``libido''. (Não tenho tempo --- setenta anos não foram suficientes --- para discutir os prós e os contras do conceito de ``energia psíquica'' nem precisamos nos preocupar aqui com ``catexia'' nem com
``libido'', termos que jamais foram definidos cientificamente.)

 A inexorável progressão do desenvolvimento libidinal começa com a fase oral,\idxforal{} em que as pulsões de sobrevivência, de afeto e de sensualidade do bebê se concentram na boca e em suas funções. Vem, em seguida, a fase anal,\idxfanal{} com seus prazeres em expelir e em reter as fezes (e a urina); em seguida vem a fase fálica,\idxfasef{} em que o menino e a menina se concentram nas intensas sensações que lhes são transmitidas pelo pênis e pelo clitóris --- fase em que notam suas diferenças anatômicas. A fase\idxgenetfase{} libidinal final é a maturidade genital,\idxconfematu{} que consiste em relações heterossexuais\idxheterofase{} gratificantes, tanto sob uma perspectiva amorosa quanto no que concerne à satisfação genital; essa fase só é alcançada pelos poucos eleitos que conseguem superar o conflito edípico. Um importante corolário foi adicionado à teoria da libido, um alicerce conceitual para toda a psicologia humana: vários tipos de transtornos emocionais têm como origem dois tipos de perturbação, que ocorrem sempre numa dessas fases
libidinais: fixação,\idxlibidfixa{} devido à gratificação excessiva durante determinada fase, ou regressão, causada pela angústia, forçando ao retorno de uma fase mais avançada para uma fase anterior. Se estou mencionando essa teoria especial agora, é só porque Freud baseou suas teorias relativas ao surgimento das perversões especificamente na descrição da progressão sexual, de zona para zona, que está contida na
teoria da libido.

 As observações feitas por Freud das fases, estabelecidas por zonas,
como descritivas do desenvolvimento da infância humana, têm sido
confirmadas e continuam valendo, a qualquer tempo, para as crianças
biologicamente normais. Entretanto, não foi publicado nenhum estudo que
confirme as implicações extraídas dessas observações. Ainda não foi
demonstrado que algum tipo de neurose\idxneuro{} --- inclusive a perversão --- ou que
alguma psicose tenha sido causada por uma perturbação das experiências
sensuais, seja das vias orais, do sistema enterológico ou urinário, ou
ainda do falo (vejam, por exemplo, 99, 11). (Há, contudo, muita
evidência de que \textit{relações} \textit{objetais} perturbadas,
durante essas fases, causam psicopatologia.) A explicação do surgimento
das neuroses pela \mbox{teoria} da libido parece algo tão disparatado que
ninguém, até hoje, pensou com seriedade em colocá-la cientificamente à
prova.

 Um aspecto \textit{sui generis} da teoria da libido é a noção de que
ela é uma energia quantitativa que flui, ou que pode ser represada; e
que a função do ``aparelho mental'' é reduzir
a ``tensão pulsional'' --- o desprazer --- que
resulta desse represamento. É verdade que as pessoas geralmente sentem
prazer ao reduzir a tensão, como acontece quando dormem, comem, mantêm
relações sexuais, esvaziam seus intestinos, dão vazão aos afetos, coçam
a pele e assim por diante. Mas será isso um efeito da suposta libido?
Sendo um construto neurofisiológico, a libido pode ser desafiada por um
modelo neurofisiológico. Como fica patente, o conceito de libido é tão
difícil de apreender quanto o eram os ``humores'' das eras passadas. Aliás, os
mamíferos (113), inclusive o homem (66), têm um centro cerebral muito
definido que produz a experiência subjetiva a que chamamos de prazer.\idxcerebpraz{} Experimentalmente, ele não se esgota, como se tivesse sido drenado. Ele
pode ser ligado e desligado \textit{ad infinitum}, de tal modo que um
animal pode vivenciar o mesmo grau de prazer intenso até milhares de
vezes por hora (113). Até mesmo a mais parcimoniosa das explicações a
respeito do prazer, atualmente, não apresenta semelhanças com esse
modelo hidráulico, de uma torrente de substância ou energia; os\idxlibid[|)]
requisitos energéticos de um mecanismo de comutação do sistema nervoso\idxsociadese[|)]
central são infinitesimais.

\section{O conflito}

 Dois tipos de situações dolorosas\idxconf[|(] podem influenciar o
desenvolvimento da personalidade e, portanto, o desenvolvimento sexual.
Primeiramente existem os traumas\idxtrauma[|(] (agudos, crônicos ou cumulativos),
duras imposições, exercidas sobre o bebê, por acontecimentos percebidos
como não originados de sua própria psique. Tais acontecimentos podem
ser sensações internas desagradáveis, como fome, dores físicas ou
dificuldade respiratória, ou podem ser lutas externas, contra objetos
separados de seu corpo, que frustram ou traumatizam --- objetos
inanimados de forma intermitente, e pessoas importantes de modo
persistente, especialmente a mãe, no início. Nem todos os traumas
produzem conflito; a segunda categoria de situações dolorosas, de
conflito, implica uma luta \textit{intra}psíquica que envolve ter que
\textit{escolher} entre possibilidades. Assim, se uma criança pequena
tem um impulso sensual poderoso, e que é proibido por seu progenitor,
isso não causa um conflito intrapsíquico\idxconfintr{} --- embora a criança possa
alterar seu comportamento exterior ao ser punida por esse progenitor.
Em uma etapa posterior da infância, contudo, se o sistema de valores
desse progenitor tiver sido aprendido e aceito (internalizado) pela
criança, estará presente, dentro de sua psique, um conjunto de
posições morais recebido dos pais, somado a uma técnica interna de
autopunição baseada em culpa.\idxculpa{} Isto exemplifica, então, o conflito
intrapsíquico: uma parte da própria pessoa ameaça ou pune a outra
parte, frustrando o mais recente impulso em direção à gratificação.

 É mais provável que um trauma ou uma frustração provoque reação
(mudança) do que conflito. Por exemplo, nas primeiras etapas do
desenvolvimento infantil, um estímulo pode produzir mudança sem
conflito, por um processo etológico como a impregnação,\idxestam{} por
condicionamento\idxcond{} clássico ou por condicionamento operante. (Tais
processos participam da criação desses comportamentos não conflituosos,
como a maneira de falar ou as preferências em matéria de brinquedos,
roupas ou alimentos.) Freud\idxfreudperve[|(] não acreditava que o trauma causasse
perversão do desenvolvimento sexual, a menos que ele viesse a causar
conflito; o conflito é a percepção da necessidade de escolher entre
alternativas; e isso requer um desenvolvimento suficientemente
adiantado para que a memória, o julgamento e talvez a fantasia já
tenham começado a influenciar o comportamento.\idxaberr[|(] Ele parece excluir a
ideia de uma aberração sexual que possa não ser, ao mesmo tempo, uma
perversão; ou seja, ele nega a existência de um ato erótico habitual,
aberrante, que não tenha sido produzido pelo conflito. ``Assim, cada
desvio da vida sexual, a partir do momento em que se fixava,\idxaberrfixa[|(] era
considerada como resultante de uma inibição do desenvolvimento, e como
infantilismo'' (24, p.\,231). Para ele, todas as aberrações
sexuais resultavam de fixações\idxlibidfixa[|(] e de traumas ocorridos nos vários
estágios do desenvolvimento libidinal da infância, com ameaças e
punições rondando o desejo pelos pais --- o conflito edípico --- como\idxtrauma[|)]
fatores decisivos.

 A teoria freudiana das causas da perversão (assim como de toda a
sexualidade) é uma combinação das cinco categorias que estivemos
considerando: a bissexualidade, o complexo de Édipo,\idxconfe{} o primado do
pênis, a teoria da libido e o conflito. Levando em conta algumas
influências constitucionais mais ou menos poderosas, tais como
tendências bissexuais herdadas,\idxbissetend{} ou capacidade constitucional incomum
para sensações agradáveis em partes não genitais do corpo, ele sentia
que, acima de tudo, era um conflito infantil\idxconfemeni{} --- angústia de castração,\idxcasta{}
conflito edípico e pré-edípico, medo da homossexualidade --- o que fazia
que a sexualidade normal se transformasse em perversão. Num sumário
resumidíssimo, ele acreditava que a perversão, nos homens, era devida
ao medo, sentido pelo menino, de que seu desejo pela mãe pudesse fazer
com que o pai acabasse por lhe cortar seu valorizado pênis, o que teria
o poder de transformá-lo numa menina,\idxvaginorg{} anatômica e psiquicamente
inferior, em virtude da castração. Na menina, ele dizia que a perversão
resultava da inabilidade em aceitar o fato de já ter sido castrada; ela
precisa negar o fato por uma supervalorização de seu clitóris,\idxclitfemi{} o que
evita que ela faça a opção pela vagina,\idxvagin{} mais feminina, ou causando nela
uma má vontade em se voltar para o pai --- em direção à
heterossexualidade. Sem acolher o pai como seu novo objeto amoroso
(renunciando à mãe), ela fracassa em adentrar o conflito edípico como
pessoa do sexo feminino, que quer ser tornada completa pelo advento de
um bebê. A perversão pode caracterizar fracasso em qualquer dos
estágios do processo do desenvolvimento edípico, nos meninos e nas
meninas. (Somada a essa teoria [interpessoal] edípica, Freud sentia que
existiam elementos específicos nos atos perversos, que resultavam de
uma fixação\idxfeminfixa{} libidinal.\idxclitfixa{} Com isso ele queria dizer que, quando o
desenvolvimento posterior da criança era bloqueado pela angústia da
castração, o menino ou a menina poderiam regredir\idxhomosregr{} para um estágio de
gratificações libidinais anteriores.\idxaberrfixa[|)] Se, por qualquer razão, a boca, o
ânus ou os intestinos, a uretra, a pele ou qualquer outra parte do
corpo tivesse sido o foco de intensa excitação libidinal anterior, a
criança poderia regressar para essa ``posição'' mais segura e mais gratificante,
em virtude de grave angústia.\idxlibidfixa[|)] Isso explicaria,\idxfreudperve[|)] por exemplo, o
intercurso anal entre homossexuais homens, e o sexo oral em homens e
mulheres (homossexuais).

 Nos animais,\idxaberranim[|(] quase não se veem aberrações sexuais levando a orgasmo (a
menos que eles tenham sido manipulados por forças não naturais, como
experimentos ou cativeiro).\idxanim[|(] Por outro lado, tais necessidades são
onipresentes no homem; os desvios eróticos são tão tipicamente humanos
como o assassinato, o humor, a fantasia, os esportes competitivos, a
arte e o cozimento dos alimentos. Esta observação é tão evidente, que é
justo perguntar por que ela não encontra força nas especulações dos
pesquisadores sexuais modernos. \textit{Quase todos os estudos de que
se tem notícia acerca do comportamento sexual humano, depois de Freud,
tentaram provar que a pessoa não cria seu próprio desvio}, mas que este
lhe é imposto\idxaberrauto{} --- pelos genes, pelos hormônios, pelo circuito elétrico
cerebral, por impregnação, por condicionamento, pelas estatísticas.
Como Freud nos perturbou! Ainda não conseguimos lidar com sua
``acusação'' de que somos humanos.

 Alguém poderia fazer o favor de explicar a pedofilia\idxpedof{} em termos
genéticos? Ou o fetichismo\idxfetic{} por sapatos como produto de um mecanismo
cerebral constante, através do desenvolvimento evolutivo? Ou o
exibicionismo\idxexibi{} peniano como um defeito hormonal? Ou a necessidade de
estuprar\idxestup{} mulheres velhas como efeito de condicionamento? Ou a
necrofilia\idxnecro{} como meramente uma exceção estatística, encontrada além dos
limites de uma curva de sino?

 A nova pesquisa que acontece nos laboratórios de fisiologia e de
química --- no animal íntegro e nas cobaias humanas --- e na
observação naturalista, parece ter como objetivo específico lançar por
terra a teoria do conflito;\idxsexopteor[|(] nenhum outro aspecto do sistema freudiano
criou tamanha resistência --- talvez porque Freud acreditasse que a
perversão fosse motivada, ou seja, que uma pessoa, de alguma maneira,
em seu âmago, sente-se parcialmente responsável por sua perversão.
Freud sentia o ato sexual como sendo o produto da grande capacidade
humana de escolha e, portanto, em última instância, como uma qualidade
moral (ainda que nossa responsabilidade seja mitigada, em virtude de a
escolha ser inconsciente, e causada involuntariamente por
circunstâncias ameaçadoras ocorridas na infância). Os pesquisadores
modernos, entretanto, negam que o conflito intrapsíquico desempenhe
algum papel, ou que a fantasia impulsione e perpetue a atividade
aberrante. Em outras palavras, esses críticos alegam que a aberração
não é psiquicamente motivada. Estranhamente, sua pesquisa pode se
aplicar também ao comportamento não desviante, pois sua lógica diz que
\textit{todos} os comportamento sexuais \textit{não} têm motivação
psíquica. Eles acreditam nisso, em seus laboratórios e em seus
escritórios. Será que pensam do mesmo modo quando estão na cama?

 O ataque à teoria do conflito tomou quatro formas. A primeira diz que
as aberrações, nos humanos, devem-se simplesmente a mecanismos
fisiológicos\idxaberrfisic[|(] --- quer sejam disfunções orgânicas ou uma fisiologia normal
herdada que, simplesmente, produz variação comportamental, semelhante à
que se vê nos animais inferiores, e resultante dos mesmos mecanismos
cerebrais e hormonais. Na segunda explicação --- teoria da aprendizagem\idxsocia{} ---
o desvio é impingido por uma força exterior, um tipo de
condicionamento, de tal forma que não se trata de escolha, e não tem
origem na fantasia. A terceira explicação é\idxestat{} estatística:\idxaberrestat{} existe uma
curva de sino para o comportamento sexual, e as variações não são
anormais, apenas não constituem a norma. E a última diz que, embora
algumas culturas\idxrela{} julguem uma aberração como patológica, é bem possível\idxsexopteor[|)]
que o único fator patológico seja a própria condenação social,\idxconf[|)] ou seja,\idxsexuateo[|)]
é a sociedade,\idxfreudsexua[|)] e não o indivíduo,\idxfreud[|)] que é doente.

\section{Genética e constituição}

 O que vem\idxgenet{} a seguir representa os tipos de estudos que tentam demonstrar
que as aberrações sexuais são induzidas por forças físicas, e não
mentais.

 Inúmeros trabalhos interessantes foram feitos sobre os animais
(resumidos às páginas 37 e 140), pois existem agora técnicas para
influenciar amplas áreas do cérebro por estímulos elétricos e químicos, ou por ablação cirúrgica, ou ainda por privação da fase \textsc{rem} do sono. Essas experiências criam perturbações nos padrões sexuais normais dos animais,\idxhiper{} hipersexualidade\idxcerebhipe[|(] ou comportamento sexual indeterminado; enquanto elas acontecem, pode ser que o animal deixe de se importar com o sexo de seu parceiro, ou até mesmo com a espécie de objeto pelo qual se interessa. Existem, também, experiências nas quais diminutos estímulos elétricos ou hormonais em áreas ínfimas, circunscritas, do cérebro, podem alterar a função sexual (104). Mas nosso velho problema permanece: de que maneira \textit{esses} substratos neurais em animais\idxaberranim[|)] se relacionam \textit{àquele} homem que compra fotografias de mulheres acorrentadas?\idxanim[|)] Por que \textit{aquela} mulher, que era uma garota feminina até os seis anos de idade, acabou se tornando tão masculina?

 Entretanto, obviamente, estudos similares não foram aplicados a humanos
--- embora se saiba que o comportamento sexual pode ser alterado por
manipulação cerebral\idxcerea{} (66). Já tive ocasião de mencionar o relato de um
caso de cura neurocirúrgica da homossexualidade masculina (122).\idxhomosbiol{} Já
foram relatadas uma cura da homossexualidade masculina (85) e o
tratamento da hipersexualidade\idxcerebhipe[|)] (12) por agentes antiandrogênicos.
Vários relatos tentaram implicar os lobos temporais\idxfeticlobo{} no comportamento fetichista, especialmente no travestismo --- porém, eles eram ou casos isolados, ou envolviam uma amostra pequena demais (resumidos em Blumer,\idxblumer{} 5), ou lhes faltavam controles adequados (151); ou, simplesmente, se tratava de pura especulação, e não de pesquisa propriamente dita (16).
Eles indicam que um caso raro pôde ser associado a um distúrbio do lobo temporal;\idxcerebepil{} mas isto não encoraja a acreditar que um distúrbio cerebral como este seja a causa subjacente a todos os comportamentos desse tipo (145). (Por que não existem relatos de fêmeas igualmente afetadas?)

 Quanto às reivindicações de que uma predisposição constitucional possa
tornar certas pessoas suscetíveis a certos tipos particulares de
desvio, até agora simplesmente não existe evidência aceitável disso,
exceto em casos raríssimos (108, 137). Nenhum estudo fidedigno
demonstrou tendências familiares para qualquer das perversões, exceto
talvez para a homossexualidade. A visão que em geral se compartilhava,
bem no início do século 20, de que a perversão era o resultado de uma
``degeneração'' significando alguma
inferioridade física difusa, jamais encontrou apoio na evidência. Mas
todos os estudos --- e eles são pouquíssimos --- que tentaram demonstrar
fatores genéticos\idxhomoshera{} na\idxhomosbiol{} homossexualidade\idxgenethomo{} (72, 128) foram incapazes de
fazer frente ao ataque das críticas, até o momento (considerados em 73,
101, 11). Até mesmo as observações de Money,\idxmoney{} envolvendo mulheres que
sofreram processo de androgenização\idxhormhomo{} quando ainda eram fetos,
demonstraram que as meninas se tornam apenas levemente masculinas no
comportamento, sendo, contudo, heterossexuais (108).

 É cedo demais para dizer se esse trabalho, envolvendo a genética e a
função cerebral, colocarão a teoria psicanalítica à prova (se esse
fosse o caso, poderíamos contar com tratamentos\idxhomostrat{} que tornariam a
terapia psicanalítica obsoleta para alguns distúrbios sexuais).
Precisamos ser sempre cuidadosos (como, aliás, não fui, quando
especulei acima sobre a significância de um centro de prazer no sistema
nervoso central), para não igualar a descoberta dos mecanismos do
mesencéfalo\idxcerebmeca{} com a descoberta das causas do comportamento humano
integrado, motivado. Os substratos límbicos do comportamento oral e
genital ficam muito próximos um do outro; quantas pessoas considerariam
seriamente a hipótese de ter sido por esse motivo que o homem inventou
o sexo oral? Será que o centro do pé fica próximo ao centro genital,
num fetichista? Será que a masturbação\idxmastur{} se deve à ativação de um centro
masturbatório no hipotálamo? Chego a pensar que algumas das teorias
modernas foram inventadas por um computador. Pelo menos, levará algumas
gerações de pesquisas corticais, antes de podermos descobrir alguma
coisa a respeito do pensamento, do desejo e do comportamento; esse
córtex não existe nas outras espécies.

 Talvez o maior desafio recente sejam os relatos (comentados
anteriormente, na discussão sobre a bissexualidade) que demonstram que,
quanto mais exclusivamente homossexual\idxhomosbiol{} um homem for, menores serão seus
níveis de testosterona,\idxhormhomo{} e mais imperfeita sua espermatogênese. Tais
descobertas, caso confirmadas, reduziriam grandemente a importância
de uma teoria que afirma que a condição é causada por perturbações no
relacionamento de um menino com sua mãe e seu pai.\idxaberrfisic[|)] Teremos que aguardar
por mais estudos, que efetuem um controle sobre influências
inespecíficas como a tensão, variações normais de nível durante um
ciclo diurno, variações normais na quantidade e no conteúdo do sêmen,
ingestão de drogas (a maconha\idxmarij{} pode diminuir os níveis de testosterona
no plasma) e assim por diante (20, 84). Mas os relatos que desmentem
essas teses estão começando a ultrapassar, em número, esses dados (10,
15, 117, 149).

\section{A teoria da aprendizagem}

Os teóricos da\idxsocia[|(] aprendizagem\idxaberrteor[|(] constituem o segundo grupo cuja pesquisa
contradiz a ideia de que existem aberrações sexuais que funcionam
como acordos de conciliação, destinadas a salvaguardar o prazer sexual
dentro de uma situação repleta de angústia e conflito. Também parece
não existir aí uma teoria do aprendizado que tente, como a psicanálise
o faz, explicar o desenvolvimento do comportamento erótico no ser
humano --- comportamento esse que conduzirá ou à heterossexualidade
``normal'' ou às aberrações. Entretanto,
foram feitos estudos que atestam a existência de comportamentos
sexuais em cuja origem inexistem conflitos intrapsíquicos.

 A impregnação\idxestam{} de criaturas quando bebês, num período crítico, pode
causar o apego a um objeto inanimado, a um animal\idxanim[|(] ou a um homem, ao
invés de, como normalmente aconteceria, apegarem-se à mãe. Na vida
adulta, a escolha de parceiros sexuais poderia corresponder ao tipo de
objeto com que foram impregnados (87). Embora as conceitualizações que
estão por trás dessas observações tenham sido estendidas para o bebê
humano (8, 50, 133), ainda não dispomos de dados que as confirmem.

 Também foi investigado como o condicionamento clássico contribui para o
desenvolvimento do comportamento sexual, o que pode ser resumido pela
afirmação de que, com animais, quase tudo é possível no laboratório:
através de técnicas de condicionamento, um animal pode ser treinado
para ficar sexualmente excitado por objetos que, na ordem natural, não
teriam essa capacidade; até mesmo estilos de gratificação, tais como a
masturbação,\idxmastur{} podem ser produzidos artificialmente. Mas continuamos sem
saber se essas descobertas estão relacionadas, e como, ao
desenvolvimento da sexualidade em condições naturais de vida.

Também se investigou a possibilidade de relações ``interpessoais''\idxinter[|(]
contribuírem para o desenvolvimento da sexualidade, nos animais. Por
exemplo, entre eles, a superpopulação produziu efeitos excepcionais,
alterando a escolha do objeto e o estilo de coito, bem como sua
frequência e sua capacidade (40). Experiências em que macacos foram
criados por outro tipo de mães, inclusive mães inanimadas, causaram
profundas perturbações na socialização e na capacidade para o
comportamento reprodutivo (64). Falhas na habilidade sexual também
ocorreram quando macacos foram privados do relacionamento com seus
companheiros, quando jovens (63). Estes estudos parecem justificar a
ênfase que os analistas dão à influência dos relacionamentos
interpessoais, no início da vida, sobre a sexualidade que se instalará\idxinter[|)]
depois.

 Meu entusiasmo por muitos estudos interessantes sobre animais se reduz
quando seu autor não consegue resistir à tentação e, depois de realizar
uma experiência que produz anormalidade, extrapola seu resultado
diretamente para o comportamento humano --- sugerindo até mesmo, com
base nesse resultado, modificações na educação das crianças, ou no
funcionamento de sociedades inteiras --- sendo que a descoberta foi
feita, digamos, em ratos.\idxanim[|)]

 Experiências de condicionamento humano capazes de produzir
comportamento sexual\idxaberr[|)] são raras. Numa delas, um fetichismo\idxfetic{} leve, que
rapidamente se extinguiu, foi artificialmente produzido, intercalando
figuras eroticamente estimulantes com sapatos (118). Reforços positivos
e negativos, numa determinada cultura, podem ser, em parte,
responsáveis por alterar os estilos sexuais de geração para geração
(por exemplo, a moda feminina).

 Os teóricos da aprendizagem social sublinham os efeitos da moldagem ---
através de recompensa e da punição --- na criação da personalidade (2,
126), assim como os efeitos da imitação e da identificação.\idxident{} Freud e
outros analistas o fizeram também, como nas discussões sobre o
desenvolvimento do ego e do superego (21, 31, 65, 94) ou do cerne da
identidade de gênero\idxidencer{} (137). Existem estudos, a respeito de
identificação, em que alguns teóricos da aprendizagem e psicanalistas
encontram um campo em comum. Esses estudos mostram que o bebê, assim
como a criança que passa um tempo considerável numa relação próxima e
amorosa com um dos pais, pode vir a escolher as qualidades de gênero
daquele progenitor. Assim, um garoto demasiadamente próximo da mãe pode
se feminizar, mas ser masculino quando estiver com o pai, e vice-versa
para as meninas (137, 4). Os analistas estão em desacordo com a
maioria dos teóricos da aprendizagem,\idxaberrteor[|)] por acreditarem que o
desenvolvimento da personalidade não ocorre sem a influência do
conflito intrapsíquico no que tange as questões vinculadas ao afeto, e\idxsocia[|)]
também ao sexo.




\section{<<Taxinomia>>}

O terceiro argumento é\idxestat[|(] o argumento estatístico\idxaberrestat[|(] ou, como alegou
Kinsey,\idxkins[|(] ``taxinômico''. (Esta palavra foi escolhida por suas implicações de objetividade, naturalidade e ausência de julgamentos morais; ``normativo'' soa melhor do que ``normal''.) Ele pressupõe o comportamento de um naturalista, que observa o ser humano da mesma forma como o faria
com qualquer outra espécie animal. Nas mãos de Kinsey, ele se tornou
uma poderosa ferramenta de pesquisa --- embora, também, um
extraordinário modo de impor uma moralidade social, uma vez que, com
sua astúcia, Kinsey acabou por fazer um julgamento, ao dizer que
deveríamos nos abster de julgar.

 Seus dados (77, 78) e o bom trabalho sociológico subsequente
(especialmente o de sua equipe [41]) não abalaram a teoria analítica,
pois eles demonstraram aquilo que os analistas já sabiam de longa data
--- que o comportamento sexual humano é muito mais variável do que jamais
se admitiu. O desafio kinseyano não se encontra em seus dados, mas na
posição que ele adotou antes mesmo de haver coletado quaisquer dados: a
de que a vida interior não era significativa para essa pesquisa
psicológica: o observador já saberá o suficiente quando tiver terminado
a contagem. Neste aspecto,\idxestat[|)] Kinsey\idxkins[|)] faz aliança com os behavioristas.




\section{Relatividade cultural}

O quarto argumento é uma extensão do terceiro,\idxrela[|(] e é uma exortação.
Aqui, o autor usa a pesquisa de outros cientistas para dar suporte à
sua posição em prol da liberdade sexual. Este é especialmente o caso em
relação aos ativistas que se reúnem para aliviar sua culpa\idxculpa{} e a
degradação social que tradicionalmente lhes é atribuída. Os
homossexuais\idxhomosargu{} exemplificam essa abordagem, apoiando-se nas três
categorias citadas acima para negar alegações de anormalidade ou de
doença: primeiramente porque sua condição é amplamente disseminada nos
animais inferiores, portanto ela é herdada, fisiologicamente induzida
por hormônios ou de alguma outra forma; ou ela é o resultado de
condicionamento, na infância ou na adolescência; ou ela é aberrante só
em termos estatísticos. O que é crucial, para cada uma dessas defesas,
é o alívio da culpa:\idxculpa{} uma vez que o próprio \textit{self} da pessoa não
escolheu a condição, não se tem responsabilidade sobre ela --- e, além
disso, ela deixa de ser vergonhosa.

 Hoje em dia, há uma grande controvérsia (3, 51, 68, 100, 131) para
decidir se o comportamento aberrante é ou não perverso (ou seja,
nojento, doente) ou apenas\idxpervdesvi{} desviante\idxdesv[|(] (uma variância estatística);\idxaberrestat[|)] as palavras-chave têm sido ``normal'' e ``saudável''. Esses argumentos revelam-se menos do que inspirados, porque cada uma das posições, ambas tendo a Ciência a seu lado, ignoram o que seus oponentes querem dizer com ``normal'' ou com ``saudável''. Um dos grupos afirma que a pessoa perversa é anormal porque o comportamento aberrante pode ser \mbox{rastreado} até se chegar a traumas e conflitos da infância, e que presentemente mascara (ou talvez nem mesmo seja capaz de mascarar) uma grave psicopatologia. O outro diz que o desviante\idxdesv[|)] não é anormal porque, exceto por suas diferenças sexuais, ele é capaz de administrar sua vida tão bem quanto os heterossexuais --- que, aliás, não se destacam por sua alegria e criatividade.

 Eu concordo com ambos --- e com nenhum dos dois. Muitas aberrações e
perversões, por emergirem como soluções para conflitos, produzem assim,
em seu cerne, uma carga de culpa\idxculpa{} e um sentimento de estar se colocando
em risco. Por outro lado, não acredito que, para a maioria dos que se
envolvem com elas, essas dinâmicas causem mais mutilações do que as
geradas pela resolução de conflito, e que produz o comportamento
(heterossexual) normativo. (Isso vai se tornando menos verdadeiro
quanto mais bizarra for a aberração.) É possível que essas questões
morais, sob um disfarce científico, não sejam tão fáceis de se
resolver como ambos os lados esperam.

 Para resumir (e minha reflexão foi sobre o impacto dos novos avanços,
sem detalhá-los, de tal forma que quaisquer conclusões não passarão de
opiniões), eu creio que o impacto mensurável\idxsexoppsic[|(] desta pesquisa sobre a
teoria psicanalítica\idxpsicas{} tenha sido tênue. Em primeiro lugar porque não
existe nenhum procedimento científico (já inventado) capaz de testar a
maior parte de suas proposições. Em segundo lugar, os psicanalistas se
preocupam com o homem, mas a nova pesquisa ainda não dispõe das
técnicas para atingir sua própria meta principal --- embora
geralmente não declarada: a de demonstrar de que maneira os achados de
qualquer experiência, seja com animais, seja sobre uma parte isolada da
fisiologia humana, ou sobre a função psíquica que se estiver
investigando, pode ter relevância para o comportamento sexual de um
humano em sua vida como pessoa --- não como objeto a ser investigado num
laboratório. Ainda assim, embora o impacto mensurável sobre a teoria
tenha sido leve, seu impacto sobre os analistas talvez tenha sido
considerável. Muitos estão ouvindo com atenção os pesquisadores e, nos
escritos e nas conversas entre os analistas, manifesta-se uma
impaciência, pois eles se sentem confinados em posições teóricas que se
mantêm coesas mais pela tradição do que por fatos.

 Como\idxsexopmeto[|(] em muitas outras áreas da pesquisa psiquiátrica,\idxcereb{} as novas técnicas
deflagraram um surto de interesse pelas descobertas sobre os mecanismos
neurofisiológicos (inclusive químicos) envolvidos em muitos aspectos do
comportamento sexual. Junto a esta excitação causada pelo maravilhoso
mundo das experiências, existe cada vez menos entusiasmo pela
investigação da sexualidade no ser humano pelos métodos clínicos,
especialmente através da coleta de dados no ambiente protegido,
adequado e condizente, do tratamento. Acredito que temos muito a lucrar
com a intensificação do trabalho nos laboratórios; entretanto, temos
muito a perder, caso passemos a ignorar o caso isolado estudado em\idxsexoppsic[|)]
profundidade.

 Será bom se formos capazes de reverter a crença de que o método clínico
é ou muito fraco (como alguns cientistas de laboratório estão
convencidos) ou que ele já tenha nos fornecido tudo quanto tinha a
fornecer (como alguns psicanalistas tendem a sentir). Agora, anos
depois do trabalho de observadores como Freud,\idxfreud{} Kraft-Ebing\idxkraft{} ou Havelock
Ellis,\idxellis{} ainda precisamos de observações de caráter naturalístico sobre o
comportamento sexual, normal e anormal. O trabalho de Masters\idxmaste{} e Johnson\idxjohns{}
nos convenceu disso. Mas, agora, eu não estou me referindo apenas à
observação da resposta fisiológica bruta --- coisa que eles estão
fazendo --- e sim à coleta das descrições subjetivas, precisas, da
experiência sexual, das fantasias que a acompanham, dos indicadores dos
processos inconscientes e das influências da infância, enfim, tudo
quanto o método psicanalítico for capaz de reunir.

 E podem haver bônus adicionais, se conseguirmos ressuscitar a pesquisa
clínica no comportamento sexual. Por exemplo, grandes \textit{insights}
nos podem ser fornecidos por essa área negligenciada, \textit{insights}
a respeito das causas da violência,\idxagres{} cuja pesquisa, atualmente, merece
prêmios mais por seu volume do que por suas descobertas. Não é
coincidência que pessoas violentamente agressivas geralmente tenham
impulsos sexuais estranhos e graves conflitos no que concerne à
masculinidade e a feminilidade. Há, ainda, uma segunda área que a
pesquisa sexual pode esclarecer: alguns pontos controversos, legais e
morais, que vinculam questões de responsabilidade e de normalidade,
correm o risco de serem relegados ao esquecimento, sob o forte calor
da narrativa dos fatos.

 Talvez o grupo que mais possa contribuir conosco, em virtude de até o
momento ser (com poucas exceções notáveis, como La Barre)\idxbarre{} o que menos o
fez, seja o dos antropólogos.\idxantro{} Sua aceitação de material superficial,
anedótico, e seu fracasso em desenvolver um método para a obtenção
material detalhado, preciso, num assunto tão privado como é o desejo
sexual, têm mutilado nossa compreensão da sexualidade humana. Nós,
simplesmente, não podemos prescindir de estudos comparativos entre as
diferentes culturas; mas, também, não podemos nos conformar com os
relatos superficiais e preconceituosos que eles, com frequência, nos
fornecem. É evidente que os analistas precisam de suas descobertas,
para que nos conservemos honestos, ou seja, para nos impedir de
generalizar, com rapidez excessiva, do paciente específico que\idxsexopmeto[|)]
analisamos --- um membro de nossa cultura --- para a humanidade em geral.

 Finalmente, é crucial que nos lembremos de quão pouco sabemos sobre os
mecanismos, ou sobre as causas, do comportamento sexual humano --- do
normativo e do variante, do normal e do anormal. Ainda sabemos
demasiado pouco sobre o que afeta a evolução da sexualidade além dos
estímulos da hereditariedade, da constituição e do ambiente primitivo
em que se foi criado. Sabemos pouco demais até mesmo sobre o que as
pessoas fazem, sobre o que pensam enquanto o estão fazendo, e o que
pensam sobre o que estão fazendo. Entretanto, talvez não seja um
excesso de otimismo dizer que, agora, estamos desenvolvendo, pela\idxrela[|)]
primeira vez, ferramentas e ideias que nos capacitarão a estudar bem\idxsexop[|)]
essas questões.




\chapter[\textbf{3}\quad Variações: aberrações que não são perversões]{{\large\textit{Capítulo 3}}\\ Variações: aberrações que não são perversões}
\markboth{Definição}{Variações: aberrações que não são perversões}


Após ter analisado as recentes\idxvaria[|(] pesquisas sobre sexualidade sob um
enfoque psicanalítico, examinemos agora se existe algum tipo de
aberração que não seja uma perversão. Talvez alguns poucos exemplos
possam validar o conceito. A qualidade que define todas as categorias
deste grupo é que, embora seja comum encontrar comportamentos sexuais
que fogem às normas sociais, eles não têm como característica
predominante a troca de hostilidades.




\section{Fatores genéticos e outros fatores constitucionais}

Vimos (cap.\,2) que, em ambos os sexos,\idxanimdese[|(] em termos de anatomia,\idxidenpre[|(] a
masculinidade\idxhormdese[|(] ocorre somente quando ocorre a adição de andrógenos,\idxandrn[|(] no
período pré-natal.\idxprenat[|(] Além disto, os hormônios pré-natais organizam o
cérebro de todos os mamíferos, inclusive dos humanos, de maneiras que
predeterminam o comportamento sexual (tanto o comportamento que conduz
a experiências eróticas, ou seja, que usa o aparelho reprodutivo, como
aquele outro tipo de comportamento, relacionado ao comportamento não
erótico e não reprodutivo, a que chamamos, nos seres humanos, de
masculinidade e feminilidade). Independentemente de o organismo ser
masculino ou feminino sob um ponto de vista genético, se os andrógenos
não estiverem presentes nos períodos pré-natais apropriados, o
comportamento masculino não ocorrerá. Esta regra foi invariável em
inúmeros experimentos com animais inferiores, e se confirma também,
embora de maneira menos definitiva, nas ``experiências naturais''\idxexpen{} com
humanos.

Em tais casos, o comportamento aberrante, em termos de erotismo ou
de gênero é, portanto, resultante de uma função cerebral determinada,
em primeiro lugar, antes do nascimento. Por exemplo, mulheres com a
síndrome da insensibilidade ao androgênio\idxandrs{} são, em termos de
cromossomos, homens, cujos testículos produzem testosterona em
quantidades normais. Contudo, seus tecidos são incapazes de responder à
testosterona. A aparência exterior de seus corpos é feminina (sendo os
testes criptorquídicos) e elas são, invariavelmente, mulheres
heterossexuais, femininas (109). Homens de aparência anatomicamente
normal que têm a síndrome de Klinefelter\idxkline{} (\textsc{xxy}) apresentam uma
frequência inesperadamente alta de distúrbios de identidade de gênero,
indo desde a homossexualidade\idxhomos{} e passando pelo travestismo\idxtrave{} até a
reversão genérica total, em que a pessoa anseia por uma cirurgia para
``mudar de sexo''. Mulheres normais em termos genéticos e anatômicos, mas
que foram masculinizadas \textit{in utero} --- quer por quantidades
excessivas de androgênio,\idxandrn[|)] produzido pelas suprarrenais, quer pela
progesterona que foi fornecida às suas mães, para prevenir o aborto ---
são mais masculinas, em seus interesses e em seu comportamento, do que\idxprenat[|)]
um grupo de controle constituído por meninas.\idxanimdese[|)] Estes, portanto, são
exemplos de aberrações no comportamento sexual em que as dinâmicas da
perversão, como definidas anteriormente,\idxhormdese[|)] não estão presentes como\idxidenpre[|)]
condições causantes.




\section{Transtornos pós-natais da função cerebral}

Existe um punhado de relatos de casos em que o comportamento sexual
aberrante resulta de doença cerebral.\idxgenetvari[|(] Blumer\idxblumer{} (5) examinou essa parca
literatura ---\idxvariagene[|(] foram encontrados alguns poucos casos de\idxfeticlobo[|(] fetichismo\idxfetic{} e de
travestismo em que havia focos de anormalidade\idxcerea[|(] no lobo temporal,\idxcerebepil{}
algumas vezes acompanhados por flagrantes manifestações de epilepsia;
um caso no qual um ``objeto fetiche (um alfinete de gancho) tornou-se
invariavelmente o gatilho de ataques epilépticos de lobo temporal\ldots{}
Durante a confusão pós-ictal, o paciente passava a se vestir com as
roupas da mulher. Uma lobotomia da têmpora esquerda, estando o paciente
com trinta e oito anos de idade, aliviou-lhe tanto a epilepsia quanto o
fetichismo''; uma série de sessenta epiléticos de lobo temporal continha
dois homossexuais\idxhomosdeso{} e um ``cujo interesse por garotas era mínimo\ldots{} e que
apresentava ereção parcial ao colocar as fraldas de sua irmã, ainda
bebê.''\footnote{ Uma vez que, comparando com a predominância da
homossexualidade na população em geral, tão poucos são homossexuais,
poderíamos anunciar, a partir dessa estatística, que a doença do lobo
temporal \textit{protege} contra a aberração sexual.} Walinder (151)
examina outras evidências de lesões cerebrais (especialmente no tocante
a ``travestismo/transexualismo''),\idxtrave{} inclusive\idxtransecere{}
um relatório que registrava a primeira ocorrência de
``travestismo/homossexualismo'' em homens com alterações
cerebrais causadas pela senilidade.

Se uma aberração ocorre apenas em presença de distúrbio cerebral,\idxcerea[|)]
desaparecendo com o tratamento da lesão, não existe razão para se
chamar a isto de perversão --- embora alguns o façam, ao tentarem\idxfeticlobo[|)]
demonstrar que as perversões se devem a disfunções do\idxgenetvari[|)] cérebro,\idxvariagene[|)] e não da
psique (16).




\section{Identidade hermafrodita}

Em geral, quando um\idxherma{} bebê nasce\idxvariaherm{} já apresentando uma genitália mista,
que não é explicitamente masculina nem feminina, os pais são incapazes
de aceitar essa criança como sendo completamente homem ou mulher; ao
invés disso, os pais têm um sentimento difuso de que a criança é uma
mistura dos dois sexos, ou que não é nem de um sexo, nem do outro. Uma
identidade hermafrodita se desenvolve, sendo que a criança passa a
acreditar que pertence a um sexo diferente dos dois aos quais todos os
demais seres pertencem (134). Tais pessoas, a partir daí, poderão se
achar no direito de ter relações sexuais com ambos os sexos. Porém, uma
vez mais, a força que motiva a aberração não é aquela anteriormente
definida para a perversão, e sim o resultado de impulsos originados da
identidade bissexual: uma identidade que teve sua origem nos pais, que
ensinaram à criança que ela era uma hermafrodita; essa origem não
corresponde a uma defesa da criança contra os perigos edípicos e
pré-edípicos.


\section{Transexualismo masculino}

A forma mais extrema de feminilidade em homens anatomicamente
perfeitos é o transexualismo,\idxtranse{} uma condição que se manifesta muito
precocemente na infância; o menino alimenta o desejo de ser uma menina,
chegando a querer mudar de sexo (mudança que poderá tentar concretizar
mais tarde). Essa condição rara --- e só ela, e ela não inclui a
maioria das pessoas que, hoje em dia, reivindicam
``mudança de sexo'' (139) --- não resulta de
uma fantasia de reparação e vingança, alimentada por toda a vida e que
sobreviveu, como acontece nas perversões; ela é, antes, o resultado de
um modo de conduta dos pais,\idxrelpctran{} que implantaram uma atmosfera
excessivamente isenta de conflitos e traumas, propícia a fazer com que
a feminilidade se desenvolvesse. (Examinaremos o transexualismo mais
detalhadamente no capítulo 8).

Este, portanto, é mais um caso em que a força que produz o
comportamento aberrante não advém de uma fantasia criada para retificar
um passado traumático, mas é simplesmente outro caso de moldagem
indolor da personalidade do sujeito, em que os perigos são eliminados
por dinâmicas constantemente presentes na família.

(Obs.: Só estou considerando aqui os transexuais masculinos, pois
acredito que a etiologia, no transexualismo feminino, tenha elementos
bem mais traumáticos, não sendo simplesmente uma variação.)

\section{Variações culturais}

As formas\idxvariacult[|(] que a masculinidade e a feminilidade assumem numa cultura
variam de acordo com as eras (ou, dependendo do tempo considerado, de
ano para ano) e podem diferir de cultura para cultura. Seria um tanto
tolo, para um observador que não fizesse parte de uma determinada
cultura (separado seja por gerações --- ou até mesmo por séculos --- seja
pela distância) decidir o que é uma aberração; muito menos ainda, se o
que ele observou é ou não uma perversão. O que é aberrante entre
culturas poderá não o ser dentro de uma determinada cultura. Da mesma
forma, estilos de desempenho erótico podem se alterar devido a
circunstâncias históricas e sociais, e toda cautela é pouca ao se
julgar a existência de uma perversão (ou a sua ausência) com base numa
perspectiva superficial.

Um bom exemplo --- especialmente por ser tão frequentemente usado, em
termos polêmicos, hoje em dia --- é a questão da homossexualidade\idxhomosgrec{} na
Grécia Antiga.\idxgreci{} Diferentes analistas leram os fatos diferentemente.
Alguns acreditaram que essa homossexualidade não era uma perversão,
pois ela era aceita por toda a cultura; já outros acreditaram que a
cultura, como um todo, era perversa. A versão que prefiro é a de
Vanggaard\idxvangg{} (150). Ele descreve a homossexualidade, a que era aceitável
entre esses gregos, como estando restrita à classe mais alta, e
exclusivamente aos homens, sendo por eles aceita como a mais honrosa
forma de relacionamento. E ela não substituía a heterossexualidade. Sua
função era a de transmitir os mais apurados padrões éticos da
sociedade. Um homem deveria se ligar a um rapaz para ensinar a ele a
honradez, a força, a fidelidade e o desprendimento; cada ato sexual era
uma oferenda --- no próprio comportamento e literalmente, na doação do
sêmen --- dos preceitos e da substância da masculinidade.\idxmascuhom[|(] A efeminação
não desempenhava nenhum papel; somente o homem que amasse
exclusivamente outros homens era considerado homossexual, com o sentido
depreciativo que o termo carrega hoje. Quando o jovem, depois de ser
honrado pelo comportamento erótico de seu parceiro mais velho, entrava
na adolescência, o relacionamento terminava; do jovem, se esperava que
não persistisse na homossexualidade --- exceto quando chegasse a hora de
conceder a outro jovem o mesmo tratamento que havia recebido. A
homossexualidade ritualizada não substituía a heterossexualidade, mas
funcionava como uma espécie de forma substituta do relacionamento com
mulheres de sua própria condição.

Em outras palavras, as dinâmicas desse tipo de homossexualidade eram
diferentes do padrão que se vê hoje em dia. A hostilidade entre os
parceiros não era uma motivação dominante (como hoje, eventualmente
também não o é, nos relacionamentos homossexuais que envolvem respeito
e cuidado).

Vanggaard\idxvangg{} descreve outras culturas em que uma forma similar de
transmissão da masculinidade é o principal propósito da
homossexualidade institucionalizada, e que é conscientemente implantada
na mística da cultura --- o que não é o caso da homossexualidade
masculina dos dias atuais, em nossa cultura (apesar de algumas dessas
dinâmicas, como por exemplo, a ideia da força que se recebe através do
esperma, possa estar presente,\idxmascuhom[|)] entre muitas outras).

Isto não quer dizer que a perversão não poderia se ocultar dentro
das normas culturais; caso existisse em algum lugar o costume de todos
os homens terem que trajar roupas femininas para alguma cerimônia
religiosa, existiriam sempre alguns poucos que, secretamente,
praticariam a cerimônia em razão de ela ser sexualmente excitante, e
não porque fosse religiosamente sublime. Mas a regra seria igual à que
descrevemos acima: a definição de perversão seria feita com base no\idxvariacult[|)]
significado que o ato tem para a pessoa.


\section{<<Faute de Mieux>>}

Quando, porque o objeto de\idxfaute[|(] preferência da pessoa\idxvariafaut[|(] não está
disponível, ela apela para substitutos, o ato pode ser aberrante, mas
não será perverso. A\idxpornobest{} bestialidade\idxbesta{} é um exemplo. Embora seja sempre
listada como perversão nos compêndios, pode ser que, frequentemente,
não seja assim. A menos que a pessoa que pratica o coito com o animal o
faça como forma de preferência, é possível que sua motivação não tenha
sido gerada pela fantasia,\idxfanta[|(] como costuma acontecer na perversão. Com os
pastores de ovelha, por exemplo: o coito com o rebanho, em geral,
ocorre porque há falta de coisa melhor, e não porque as ovelhas sejam
seu objeto de preferência. (A única exceção de que tenho conhecimento
aparece num filme de Woody Allen.) Se for este o caso, então procure
por uma dinâmica mais complexa. No caso do tipo de pornografia em que
mulheres são vistas mantendo relações sexuais com animais, são grandes
as chances de que a perversão (que não é bestialidade) esteja no homem
que compra as fotos; a mulher que posa para elas pode estar sendo
motivada por necessidades mais simples (dinheiro), ou pode ser uma
perturbada mental, ou uma vítima psicótica do pornógrafo.

Não se pode julgar pela aparência externa do ato, e sim pelo que se
passa na mente de quem o executa. A masturbação\idxmastur[|(] é um exemplo disso.
Ela, certamente, é normativa. Sempre que os objetos de preferência
estão ausentes, recorre-se mais a ela. Mas seria inexato generalizar e
dizer que, sendo assim, a masturbação\idxpornomast{} não é perversa. Muito
frequentemente, mesmo quando ela ocorre por \textit{faute de mieux}, a
pornografia de que se faz uso (seja ela comercial ou de cunho
particular, ou seja, através dos próprios devaneios) conterá elementos
perversos. Nessa situação, a fantasia é usada porque ela satisfaz
alguma coisa que não pode ser satisfeita através de um ato sexual real,
com um parceiro. Neste caso, a masturbação não é apenas um substituto,
mas um ato sexual verdadeiramente único, com seus motivos e energias
próprios e específicos. O mesmo ocorre quando se recorre a prostitutas.
Pode acontecer de um homem apelar para prostitutas por não dispor de
alternativas: ele é um minerador de ouro no Yukon e as únicas
mulheres disponíveis são as prostitutas. Mas e se ele for um corretor
de ações de Nova Iorque, cuja potência sexual só se manifesta em
presença de mulheres oficialmente\idxmastur[|)] degradadas?

A questão da \textit{faute de mieux}, portanto, nem sempre tem seus
contornos nítidos. Frequentemente, mesmo sob circunstâncias de
privação, a fantasia usada revela que o ato é uma mistura de ausência
de alguma coisa melhor unida à oportunidade de fazer uso de fantasias\idxfanta[|)]
perversas, por meio de devaneios ou de\idxfaute[|)] material\idxvariafaut[|)] pornográfico.


\section{O argumento animal: uma pretensa desculpa}

Ao alegar estarem falando de variações ou de desvios, e não de
perversões,\idxanimvers[|(] os especialistas em sexo da atualidade, aliados a outros
profissionais que lutam pelos direitos sexuais civis, fazem uso de
observações extraídas do comportamento animal.\idxaberranim[|(] Como vimos no Capítulo
2, o argumento usado é o de que, embora encontremos neles inúmeros
exemplos de aberrações, os animais não podem ser acusados de um
comportamento deliberado, voluntário, maldoso. Os humanos têm
afinidades com alguns animais por estruturas cerebrais, mecanismos
fisiológicos e comportamentos semelhantes. Portanto, a perversão não
existe em nós, pois as raízes para a aberração são filogenéticas.
Porém, quando fazemos com que um rato macho tente cruzar com um gato
macho, fornecendo a ele uma substância que prejudica sua função
cerebral, com isto não se comprova nem que os ratos, como espécie,
tenham um desejo inconsciente de violentar gatos, nem que, nos estágios
primitivos da evolução, os ancestrais dos ratos fossem quer
homossexuais, quer inclinados à prática da bestialidade (desejo por
gatos).

Não obstante, concordo com os que classificam como variações esses
comportamentos nos animais. Porém, discordo deles numa perspectiva mais
ampla: o erro, em termos de lógica, só se faz presente se dissermos
que, porque os animais não são perversos; porque seus cérebros podem
ser manipulados experimentalmente, conduzindo-os a todo tipo de
comportamento; porque eles podem ser forçados por manipulações
experimentais em seus cérebros, seja por meio de hormônios, drogas,
eletricidade ou cirurgia, ou que, pelo fato de os animais poderem ser
experimentalmente condicionados, então esses mesmos estímulos, por
funcionarem algumas vezes também nos humanos, são \textit{as} causas
de nossa aberração, como espécie.\idxaberranim[|)] Este tipo de argumentação não tem
nenhum valor.

A perversão\idxvaria[|)] é exclusivamente\idxanimvers[|)] humana.




\chapter[\textbf{4}\quad Perversões: aberrações que não são variações]{{\large\textit{Capítulo 4}}\\ Perversões: aberrações que não são variações}
\markboth{Definição}{Perversões: aberrações que não são variações}


O capítulo 2 examinou alguns argumentos que são usados pelos
pesquisadores da sexualidade para negar que a aberração possa
resultar de uma escolha voluntária,\idxpervrespo[|(] moral.\idxrespo[|(] Prossigamos com essa
discussão, mas deixando de lado o estudo dos animais, do cérebro, da
evolução, das estatísticas, dos exemplos --- todos tópicos importantes,
mas que seria perigoso vasculhar aqui, neste ponto. Daqui por diante,
enfatizarei um elemento --- conhecido, mas não diretamente observável ---
que desafia a\idxsexop[|(] pesquisa em questão: o \textit{desejo},\idxpervdesej[|(] como motivação
principal do comportamento. Para o fisiologista, o reconhecimento da
disposição de espírito ``eu quero'' pode ser uma miragem, não mais que
uma emanação do cérebro; para o behaviorista ortodoxo, ela é uma
manifestação, um correlato ou um resultado, mas não uma causa; para o
estatístico, ela é um efeito supérfluo, num mundo cuja inevitabilidade
está predeterminada por aquele fator essencial, a curva de sino: sem a
indução do ``eu quero'', ações e pulsões continuariam, supostamente, a se
distribuir entre o provável e o improvável. Mas não há dúvidas de que
nós, que consideramos essas atitudes simplistas, não nos sentimos
totalmente em casa quando reconhecemos ser o desejo a verdadeira causa
do comportamento: ninguém ainda aprendeu a lidar com o desejo no
laboratório. Até mesmo dispondo da linguagem para dar expressão aos
sentimentos, não conseguimos mensurar com exatidão alguma coisa tão
complexa, tão paradoxal, tão variável e contraditória como o são o
nosso desejo, a nossa raiva,\idxhost[|(] a nossa inveja, o prazer ou amor que
sentimos em relação a outra pessoa. Quanto menos então seremos capazes
de conhecer o psiquismo --- ou, mais corretamente dizendo, o
pré-psiquismo --- dos bebês, em sua fase pré-verbal, um dos objetos cuja
compreensão é necessária ao estudo da perversão. Cientistas bem
criados, privados de tudo, exceto de fragmentos do método experimental,
face à vida mental, evitam o estudo do efeito do desejo sobre a função\idxpervrespo[|)]
sexual --- especialmente sobre as origens do desejo nas águas
turbulentas da primeira infância.\idxrespo[|)] Alguns chegam ao extremo\idxpervdesej[|)] de negar-lhe
a existência.

A evidência adicional, repito, que faz com que nossa opinião difira
da conclusão a que chegaram os modernos pesquisadores da sexualidade ---
 a de que os humanos não escolhem seus estilos sexuais, mas que estes
lhes são impostos pelo comportamento --- encontra-se no estudo da
\textit{fantasia}, este veículo que nos traz esperança, cura-nos os
traumas, nos protege da realidade, oculta-nos a verdade, fixa nossa
identidade; veículo restaurador da tranquilidade, inimigo do medo e da
tristeza, purificador da alma. E criador da perversão. Desde que Freud,
pela primeira vez, a observou, ficamos sabendo que, nos humanos, a
fantasia é tão parte da etiologia das perversões\idxfetic{} --- e, mais ainda, de
toda a excitação sexual --- como o são os fatores fisiológicos e
ambientais que os sexólogos estão nos ajudando a compreender. Os
detalhes da perversão --- o enredo da história --- ficam incompreensíveis,
em sua origem e significado, caso ignoremos o processo e a função da
fantasia. Podemos estudar cada célula do cérebro e cada animal do
reino, e continuar sem saber por que um homem se excita ao usar um
sapato de mulher, ou diante de um cadáver,\idxnecro{} ou de uma pessoa amputada,
ou de uma criança,\idxpedof{} ou de um pênis. Mais ainda,\idxfantareal{} se examinarmos a
fantasia sem ignorar nenhum detalhe, acredito que encontraremos,
permeando todo este conteúdo, reminiscências das experiências
individuais com as outras pessoas que, no mundo real e durante a
infância, provocaram a reação a que denominamos perversão. E, em seu\idxsexop[|)]
cerne, está a hostilidade.\idxsadihost[|(]

Quando expandimos nossa definição, usando a hostilidade como medida,
incluímos agora todo um conjunto de comportamentos sexuais, muitos dos
quais, certamente, são onipresentes; portanto, num sentido estatístico,
eles nem ao menos constituem uma aberração. A hostilidade é,
frequentemente, fácil de ser encontrada. Em várias perversões, ela é
condição central do conteúdo que se manifesta e denota, até mesmo para
o observador pouco treinado, a qualidade bizarra da condição. Quanto
mais grosseira a hostilidade,\idxhostsadi[|(] menos dúvidas temos de estar lidando com
uma perversão. O assassinato\idxassas{} que causa excitação sexual, a mutilação
com o fito de provocá-la, o estupro,\idxestup{} o sadismo, com suas punições\idxsadipuni{}
físicas específicas, tais como chicotear ou cortar, os jogos de
acorrentar e amarrar,\idxamarr{} de defecar\idxcopro{} ou urinar em seu objeto --- todas essas
modalidades expressam, numa escala diminutiva, uma raiva consciente
direcionada para seu objeto sexual, e cujo propósito essencial, para a
pessoa, é o de se sentir superior, de machucar, de triunfar sobre o
outro. E isto também é assim no sadismo não físico, como o
exibicionismo,\idxexibi{} o voyeurismo,\idxvoy{} as cartas ou os telefonemas\idxtelef{} obscenos, o
uso de prostitutas e a maioria das formas de promiscuidade.\idxpromismora[|(]
Estatísticas, observação de animais e manipulações cerebrais não nos ajudarão
em nada a compreender por que, e como, esse tipo de
excitação funciona; mas penetrar a mente do sujeito e nela procurar
pela natureza, pela origem dessa necessidade de causar dano a seu
parceiro, esta é uma atitude possível, e muito reveladora.

Pegue o mais comum desses comportamentos --- a promiscuidade, em que
existe a menor quantidade visível de hostilidade, e o mais utilizado
hoje em dia --- e leve-o para ser discutido por aqueles que, em sua
tentativa de eximir a sociedade, argumentam que, se um comportamento é
onipresente, ele é normal. A lógica funciona assim:\idxanim{}


\begin{enumerate}
\item A maioria dos animais não é monogâmica; o homem é um animal.

\item Desejos promíscuos são encontrados em quase todos os humanos;
portanto, eles estão longe de serem estatisticamente aberrantes;

\item Aqueles que negam ter tais desejos, e os que são incapazes de
colocá-los em ação, não são seres superiores e sem pecado, como
apregoam: são limitados e inibidos; os ideais da era vitoriana foram,
há muito tempo, desmascarados.

\item Portanto, deixemos as pessoas gozarem de seus corpos livremente,
caso elas assim o desejem --- contanto que elas não vitimem seus
parceiros.

\item Isto feito, ver-se-á que o termo ``perversão'' era apenas mais uma
das técnicas usada por uma sociedade amedrontada, inibida, para
proteger sua neurose maciça.
\end{enumerate}

Eu chego quase a concordar plenamente com este argumento. Como
técnica de ação social ele é bastante bom, pois acredito que esteja
quase correto; é uma boa mistura de observações e conclusões sensatas.
Faz com que cheguemos à conclusão de que a promiscuidade\idxpromishost{} é divertida,
inofensiva, revigorante; de que ela propicia a expansão da mente,
libera a sociedade. Mas será que, assim, a pessoa consegue evitar seu
sentimento pessoal --- embora neurótico --- de culpa? O erro reside no fato
de o argumento deixar a hostilidade de fora.\idxhostsadi[|)] Pensem em Don Juan,\idxdonj{} aquele
paradigma da promiscuidade; ele revela seu ódio às mulheres de maneira
muito inocente, até inconsciente, àquela plateia que ele precisa reunir
para poder atestar seu desempenho: seus interesses estão na sedução,
não no amor; estão em contar para os amigos quantas mulheres ele teve,
e o quanto elas se degradaram, em meio à volúpia da paixão que ele
despertou nelas. A excitação e o júbilo que ele sentia não vinham da
sensualidade do ato sexual, ou da intimidade que ele pudesse ter
estabelecido com a mulher; na verdade, ele demonstrava pouco interesse
pelo coito, concentrando seus esforços em superar a resistência da
parceira, aparentemente relutante. Mulheres fáceis não o atraíam. Sua
necessidade interminável, frenética, de se autoafirmar, tendo como
única recompensa a quantidade de suas conquistas, revela que o seu
corpo estava mais a serviço do poder do que do erotismo.

Assim, não deveríamos generalizar e, ao nos deparar com uma pessoa
promíscua, dali já concluir se tratar, simplesmente, de uma alma livre
que está dando vazão à exuberância sexual natural, tão inerente às
espécies --- como, aliás, todos nós deveríamos fazer, não fosse o fato de
estarmos escravizados pela sociedade. Tal poderia ser o caso --- e,
talvez, até mesmo passe a ser, caso a sociedade evolua; porém, é a
diferença daquilo que o ato significa para cada um de nós,
individualmente, que estabelece se sua natureza é ou não perversa --- e
não as partes anatômicas envolvidas, dessa ou daquela pessoa, dessa ou
daquela maneira. (Há um tipo de promiscuidade não hostil, alegre, que\idxpromismora[|)]
costuma ser retratada em livros, em filmes e em peças teatrais com
regularidade, criando a ilusão de ser uma coisa comum.)

Além disso, existem aqueles estilos de sexo em que o agente parece
ser a vítima, em vez de ser quem perpetra a hostilidade:\idxmasoq[|(] é o caso das\idxsadihost[|)]
pessoas que buscam a asfixia, ou que fazem uso de anestésicos, para\idxsadimaso[|(]
assim obter um orgasmo; há os que precisam ser amarrados\idxamarr{} com cordas ou
com correntes, ou ser imobilizados por vestimentas excessivamente
justas; os que desejam ser espancados, ou cortados; os que se excitam
ao receber sobre si urina ou excrementos; os que, invariavelmente,
escolhem para si parceiros que os humilham e os abandonam. Nesses
casos, a hostilidade\idxmasoqhos{} fica disfarçada no ato perverso, mas é
secretamente mantida nas fantasias\idxmasoqfan{} daquilo que se está causando ao
parceiro, ao se fazer de ``vítima''. Essas pessoas sentem a volúpia da
gratificação dos mártires, que se resume a fórmulas do tipo: ``eles vão
lamentar quando eu tiver partido'', ou ``Deus, pelo menos, me ama'' ou
``comparem"-minha"-santidade"-com"-a"-dos"-que"-me"-ferem'', convertendo a vítima
física em vencedor psicológico, que triunfa sobre seu algoz; o ato é
executado diante de uma audiência imaginária, cuja função é a de
reconhecer que o parceiro, sádico, é um bruto. Além disso, como criador
do ato, o masoquista nunca é efetivamente uma vítima, pois ele, na
verdade, jamais abre mão do controle; e, nesse sentido, todo cenário
que ele constrói (de maneira pré-consciente, se não consciente)
reproduz, sabidamente, nada mais do que um sofrimento fraudado. Eu
tenho minhas dúvidas de que seja algo frequente que masoquistas --- no
estrito sentido da perversão sexual --- escolham indivíduos sádicos\idxmasoqsad[|(] --- no
estrito sentido da perversão sexual --- para seus parceiros sexuais.
Estou mais inclinado a acreditar que cada uma das partes intuitivamente
sabe, ao observar a excitação do outro, se as fantasias do parceiro
coadunam com as suas. O sádico sabe que, se seu parceiro é voluptuoso,
então ele não poderá ser aquele sofredor humilhado que sua fantasia
exige, não importa quantas chicotadas forem dadas, ou quantos gritos de
dor ele puder arrancar. Este é um exemplo do contrato masoquista que
Smirnoff\idxsmirn{} descreveu (129).

Há muito tempo (24) Freud\idxfreudsadom{} já havia percebido que sadismo e
masoquismo andam de mãos dadas. E inúmeras pessoas têm sido, no
masoquismo --- desde os analistas com seus pacientes, até as esposas
dos masoquistas --- focos de sadismo\idxfantasado{} (retribuição e restituição).\idxsadifant{} Uma
paciente me diz com uma tristeza silenciosa, como que se desculpando, e
contando com minha grandeza de alma: ``Eu não o culpo por não suportar
meu suor em seu sofá'' (que, sendo de couro, suporta o teste de seu
sofrimento). Certamente que o que ela está dizendo é: ``Seu idiota; você
alega ser um analista --- um médico, um curador, alguém solidário,
compreensivo e disposto a perdoar a dor natural da humanidade --- mas, na
verdade, você não consegue fugir de seu passado: você é \textit{homem}
e tem nojo das secreções sujas de meu corpo de mulher.'' Sua fantasia
sexual, que durou da adolescência até seu masoquismo ser analisado, era
a de que um diretor, um homem extremamente frio e sádico,\idxmasoqsad[|)] a forçava a
ser estuprada, num frenesi de excitação, por um garanhão sexualmente
enlouquecido, numa exibição pública, sobre um palco; e o ato era
testemunhado por um círculo de homens, silenciosos e com os membros em\idxmasoq[|)]
ereção.

Finalmente, existem tipos de perversão em que toda forma de
hostilidade\idxfetichost{} parece estar ausente: são os fetichismos. Eles vão desde a
necrofilia\idxnecro{} (em que o agente escolhe um cadáver, um corpo que não foi
ele quem assassinou), passando pelo uso de objetos inanimados (em geral
peças de vestuário, cuja conexão com um objeto humano foi reduzida ao
simbólico), até uma forma de fetichismo muito disseminada, e que
consiste em tratar as pessoas como se elas se reduzissem a um órgão
apenas (por exemplo, seios, ou um pênis) ou funções (bater, trepar,
vitimar, se fazer de robô, se fazer de escravo.) Uma vez que a
hostilidade frequentemente parece ausente --- especialmente nas formas
clássicas de fetichismo, que usa objetos inanimados, como vestimentas ---
 essas modalidades deveriam ser testadas, quanto à presença hipotética
de hostilidade, com maior rigor do que se faz em relação às perversões
do sadismo e do masoquismo, em que a hostilidade é absolutamente
flagrante.

 Um exame mais acurado do fetichismo demonstra que, nele, o desejo de
fazer mal é apenas silencioso, oculto. Ao sermos desafiados com a
pergunta: ``Onde está a hostilidade, quando se fica excitado com um
pedaço de pano?'', existe uma resposta possível. No próximo capítulo,
examinaremos o mais desumanizador dos artifícios --- a pornografia --- e
analisaremos um caso que torna explícita a minha proposição acerca da
hostilidade. Essa paciente nos mostra a raiva que está oculta no
fetiche e, além disso (como eu creio ser verdadeiro para todas as
perversões), a fonte de sua raiva, que reside em sua vitimização\idxinfanvitim{}
durante a infância, em geral pelos próprios pais, ou por seus
substitutos. Através da perversão, a raiva\idxhost[|)] é transformada em vitória\idxsadimaso[|)]
sobre seus abusadores; na perversão, o trauma\idxtrauma{} se transforma em triunfo.



\part[Parte \textsc{ii}: Dinâmicas: trauma, hostilidade, perigo e vingança]{Parte \textsc{ii}:\\ Dinâmicas: trauma, hostilidade, perigo e vingança}


\chapter[\textbf{5}\quad Pornografia e perversão]{{\large\textit{Capítulo 5}}\\ Pornografia e perversão}
\markboth{Dinâmicas: trauma, hostilidade, perigo e vingança}{Pornografia e perversão}


Se a fantasia\idxfantaporn{} é fator\idxporno[|(] determinante\idxvinga[|(] para estabelecer se um ato sexual é
perverso ou não, deveríamos então examinar com maior atenção aquilo em
que pensa determinada pessoa, e o que ela sente, para podermos
compreender sua perversão. A pornografia permite que o façamos com
facilidade.

A pornografia\idxpornofant{} é um devaneio complexo em que certas atividades, em
geral (mas não necessariamente) claramente sexuais, são reproduzidas
sob a forma de texto, de imagens e/ou de sons, para assim provocar
excitação genital. Nenhuma representação é pornográfica até que
adicionemos as fantasias de quem a observa;\idxpornoplat[|(] nada é pornográfico por si
mesmo.

Vemos,\idxtravemporn[|(] a seguir, a capa de um livreto pornográfico, ou seja, uma pequena
brochura que foi produzida por alguém que acreditou existirem leitores
em número suficiente para tornar sua impressão lucrativa. O livreto foi
adquirido por um homem que sabia que iria lhe causar excitação sexual.
Os que olham para a ilustração podem ser divididos em dois grupos: os
que se sentem excitados por ela, e os que não. Este último grupo, em
minha opinião, é consideravelmente maior. A maioria dos leitores serão
incapazes de compreender por que a imagem, e sua história, são
excitantes; eles nem ao menos acreditarão seriamente que o livreto
possa ter esse poder.

\imagemgrande{``O assalto às calcinhas'' (imagem de capa do folhetim)}{./img/ilustracao.png}

O que --- caso você não seja um travesti\footnote{ Optou-se por traduzir
o termo inglês ``\textit{transvestite}'' por “travesti”.
Faz-se necessário alguns esclarecimentos a respeito da dificuldade de uma
tradução exata, decorrente de mudanças linguísticas ocorridas nos Estados Unidos
e no Brasil, desde a publicação deste livro. O sentido que o autor dá a esta
palavra circunscreve-se aos sujeitos que se vestem com roupas características do
outro sexo, acompanhados ou não de traços de comportamento, gestuais e
locutórios, por exemplo, do sexo oposto ao seu. Hoje em dia emprega-se o termo
\textit{crossdresser}, inclusive em português, para representar o tipo a que
se refere Stoller. Trata-se do sujeito que não se veste o tempo todo com roupas
do outro sexo, mas que o faz em momentos específicos com o claro propósito de
conjurar uma angústia muitas vezes avassaladora. Tais indivíduos não são
necessariamente homossexuais. Podem ter uma vida exclusivamente heterossexual,
como certos homens casados que contam com a ajuda da parceira para se produzirem
com vestes femininas. O termo “travesti”, em seu sentido clássico, consignado em
nossos dicionários, não difere de “\textit{crossdresser}”. Ocorre que a
compreensão mais comum que se tem do travesti hoje é a do homem que assume as
vestes femininas de modo mais ou menos definitivo, promovendo mudanças no
próprio corpo, como o implante de próteses de silicone, e uso de hormônios
femininos que trazem mudanças corporais significativas. Note-se, no entanto, que
``travesti'', no imaginário popular, está francamente associado à prostituição.
Poder-se-ia ter traduzido a palavra “\textit{transvestite}” por
“\textit{crossdresser}”, de uso recorrente entre nós hoje em dia. Não o fizemos
porque o autor não a empregou no original, embora pudesse tê-lo feito. Portanto,
em razão de uma maior fidelidade à escolha vocabular do autor, utilizou-se aqui
a palavra “travesti”, mas na acepção clássica, não distintiva, que consta da
maioria de nossos dicionários, isto é, designando o sujeito no momento em que
está portando vestes culturalmente peculiares ao sexo biológico oposto ao seu.
Veja também a nota a seguir. [Nota da Edição]}
 --- você vê na figura?
Provavelmente não muito: apenas algumas mulheres, retratadas de modo a
passar a ideia de uma beleza poderosa, perigosa, feminina, e que estão
agindo de maneira intimidante, opressora, sobre um homem indefeso,
encolhido, humilhado,\idxtravemhumi{} vestindo trajes de mulher.

Existem muitos gêneros de pornografia, e cada um deles é criado
tendo em vista uma necessidade específica de perversão, dando uma
atenção muito precisa aos detalhes; cada um deles define uma área de
excitação específica, que não terá o mesmo efeito em outras pessoas.\idxpornoplat[|)]
Assim, por exemplo, um sádico escolherá representações de atos sádicos,
e um travesti fetichista escolherá imagens de travestismo.\idxtravemporn[|)] Como em
todas as perversões, a pornografia é uma questão de estética: a delícia
de alguns corresponderá ao enfado de outros. Também como acontece com
todas as perversões, em seu cerne encontra-se um ato de vingança\idxpervmotiv{}
fantasiado,\idxfantaporn{} condensando em si a história da vida sexual do sujeito ---
suas memórias e fantasias,\idxpornofant{} seus traumas, suas frustrações, suas
alegrias. Existe sempre uma vítima,\idxpornoviti{} não importa quão disfarçada ela
esteja: se não houver vítima, não é pornografia. O uso desse tipo de
material é um ato de perversão, com vários componentes. O mais aparente
é o voyeurismo.\idxvoy{} O segundo, oculto (a menos que a pessoa seja um sádico
sexual declarado) é o sadismo;\idxsadiporn{} o sadismo, entretanto, é bastante fácil
de ser demonstrado. O terceiro, mais oculto (a menos que a pessoa seja
um masoquista sexual declarado)\idxmasoqpor{} é o masoquismo; o masoquismo é difícil
de ser demonstrado, uma vez que ele está oculto numa identificação\idxvinga[|)]
inconsciente com a vítima retratada.

Esses três componentes são universais para os consumidores de
pornografia. Existe ainda um quarto componente sobre o qual nos
debruçaremos neste capítulo, e que é específico a cada consumidor: seu
próprio estilo de perversão.

A pornografia é um tipo de reparação; sua criação e seu uso são atos
ritualizados e todo desvio que possa ocorrer, e que fuja de uma
estreita e determinada trilha, produzirá um decréscimo na excitação
sexual. A perversão é necessária como uma espécie de preservador da
potência. A verdadeira história de vida sexual --- a memória
inconsciente de eventos históricos reais --- subsiste, nas fantasias
conscientes que se expressam na pornografia.

O desenvolvimento do devaneio, complexo, explícito, e que se
exterioriza através da pornografia, é uma espécie de crônica que
abrange um período de anos de fantasias, sendo que cada uma das
elaborações ocorre no momento em que um fragmento de dor (ou de prazer
incompleto) é convertido em (maior) prazer, até que todas essas
fantasias, como blocos de construção que foram sendo montados para
criar a perversão adulta, se apresenta abertamente. Porém, existe um
grão de realidade histórica\idxpornoreal{} embutido em cada fantasia, e as diferenças
entre o que realmente aconteceu nas vidas de diferentes pessoas
explicam em boa parte (embora não completamente) as pequenas variações
encontradas, até mesmo num grupo homogêneo de pessoas que compartilham
o mesmo tipo de perversão.\idxtravemporn[|(] Examinemos a pornografia da perversão\idxtrave[|(] do
travestismo\footnote{ Usarei o termo
``travestismo''\idxtrave[|nn] somente para aquelas pessoas
em quem o uso de roupas do sexo oposto causa excitação erótica. Existem
outras condições em que o travestismo ocorre (139), mas elas são
diferentes do travestismo fetichista, e não devem ser confundidas com
ele.} (travestismo fetichista) para encontrar esses trechos de
realidade histórica. Deveria ser vantajoso para nós o uso de uma
condição tão bizarra como nosso exemplo, pois ele é relativamente raro,
e a pornografia que ele contém só é considerada excitante pelos
travestis. (Poder-se-ia sugerir, não muito seriamente, que um teste
	para estabelecer o diagnóstico\idxpornoinst{} de travestismo, ou de qualquer outra
perversão, em homens, seria mostrar esse tipo de pornografia a vários
indivíduos: somente aqueles cujo fluxo sanguíneo peniano aumentasse se
encaixariam no diagnóstico. Não se poderia cogitar um procedimento
diagnóstico mais rápido e preciso. Este tipo de teste demonstraria
também, da maneira mais concreta possível, que as psicodinâmicas dos
travestis são diferentes daquelas das demais pessoas.)

No cardápio da literatura pornográfica destinada aos travestis,
existem repetidas histórias com esse mesmo tema: um menino --- ou um
rapaz --- assustado, patético, indefeso, é apanhado, contra sua
vontade, por mulheres poderosas, perigosamente atraentes, que o
intimidam e o humilham.\idxtravemhumi[|(] A pobre vítima\idxpornoviti{} --- sendo que o ápice dessa
vitimização é alcançado quando as mulheres o forçam, fisicamente, a
vestir roupas de mulher --- dificilmente parece ser um tema criado
para induzir excitação sexual. E, no entanto, os homens que precisam
desse tipo de material satisfazem sua maior expectativa justamente
nesse ponto da história, quando o homem, humilhado, é mostrado
justamente no momento em que é exposto à sua maior angústia. A imagem
típica, e o texto que a acompanha, mostram-no sentado, encolhido,
enquanto de pé, a seu redor, encontram-se mulheres muito fálicas,\idxmulhf{} com
seus gestos e feições ameaçadores. (O termo
``fálico'', aqui, não é simplesmente a
aplicação de um conceito: os desenhos mostram repetidamente, como tema,
 objetos que têm a forma de falo --- os saltos em forma de estiletes,
as pernas da mesa e da cadeira, chicotes, canetas.)

Aqui estão alguns excertos da pornografia. A Fraternidade
determinara que Bruce King, como parte de sua iniciação, deveria
assaltar o varal de uma das \textit{Sororities}; em meio à ação,
``gritinhos e risos borbulhantes'' subitamente o envolveram.
Ele é apanhado e amarrado\idxamarr[|(] pelas garotas da
irmandade, em meio a uma estridente excitação de júbilo.

\begin{quote}

Ele tentou protestar, mas sua mordaça estava apertada
demais; ele se contorceu mas, com isso, tudo quanto conseguiu foi
possibilitar que a força das unhas afiadas das garotas penetrassem no
tecido muscular de seus quadris e coxas. O que provocou muitas
gargalhadas por parte de suas raptoras vitoriosas, que simplesmente
exultavam mediante os vãos esforços de seu macho cativo\ldots{}

A garota de nome Lori, aparentemente a líder do grupo, era uma
amazona loiro-platinada. De pé, ela parecia uma estátua, com mais de um
metro e oitenta, orgulhosamente ereta, seu peito deliberadamente
empinado para frente, o que lhe dava um ar estranhamente arrogante
que exigia obediência e respeito. Lori usava um maravilhoso vestido de
puro cetim com um cinto afivelado, muito justo, à sua cintura; o traje
era composto por uma saia de plissado permanente, que tremeluzia a cada
vez que ela se movimentava, como se fosse composta por inúmeras tiras
de couro. A blusa, turquesa, ostentava motivos de flores e frutas. Sua
cintura estava capturada (\textit{sic}) num enorme e reluzente cinto de
verniz preto, cujo contrastante fecho em prata tinha o formato de um
cadeado, com um buraquinho minúsculo para a chave, como a desafiar sua
entrada e saída. Seus quadris estavam enfiados num corpete, de tal
forma que ela caminhava com alguma dificuldade, mas com muito orgulho.
E os sapatos de Lori: eles eram o prêmio ideal para qualquer invasor de
dormitório feminino. Os saltos agulha, inacreditavelmente finos,
deveriam atingir a perfeição dos 18 cm! Feitos de um couro envernizado
branco --- acreditem ou não --- os sapatos tinham tiras, feitas de
correntes prateadas, que lhe prendiam os calcanhares, e um debrum de
\textit{pou-de-soie} (\textit{sic}) na frente cortada, o dedo para
fora, quase agradecido por estar liberado de seu confinamento. A gaspa
estava charmosamente decorada por um par de reluzentes olhos de vidro!
Sim, e os olhos até mesmo piscavam, com malícia, enquanto Lori
movimentava suas pernas esguias. Um couro como esse, em verniz branco,
polido a uma perfeição láctea, merecia todo o respeito, era para ser
mantido na mais alta estima e admiração! À medida que Lori ia calcando seus
pés, delicados e encantadores, mas poderosamente calçados, pequenas
faíscas escapavam de seus calcanhares, que pareciam estiletes de 18 cm!

Bruce estremeceu, lutando contra a amarra dos cintos do roupão.
``Lori'' sua voz tentou ser forte e
confiante, ``você vai me libertar agora? Tudo bem, eu
fracassei no meu assalto às calcinhas. Eu perdi! Os irmãos da
fraternidade vão me escorraçar a raquetadas --- ele estremeceu com
esse pensamento --- e isso será tudo. Vamos então, simplesmente,
esquecer o assunto?''

``Oh, mas nós não queremos que você seja escorraçado,
de jeito nenhum!'' disse outra garota.
``Lori, o que você acha de lhe darmos\ldots{} como é mesmo o
nome dele?''

``Bruce\ldots{} Bruce King.''

``[\ldots{}] vamos dar ao Bruce um visual feminino completo,
antes de levá-lo de volta aos seus irmãos da fraternidade. Isso sim,
será algo de que eles vão se lembrar por muito tempo!''
Lori sorriu. O modo como ela cruzou seus longos braços de cisne sobre seu
peito, permitiu a Bruce avistar suas unhas compridas, pintadas em um tom de
vermelho-sangue, longas feito as garras de algum abutre perverso.
``Muito bem, Sandra. Vamos encher o Bruce de artigos
frufru, farfalhantes\ldots{} calcinha, anágua, sutiã, vestido, meias de seda
presas por cintas-ligas --- que ele vai ganhar também --- e um belo
par de sapatos de salto alto\ldots{}''

Antes que Bruce pudesse protestar, ele foi atacado pelas garotas,
que lhe arrancaram sua camisa branca, sua calça cáqui de algodão (ele
deu graças a Deus por estar vestindo seus calções de treino); seus
sapatos lhe foram arrancados também, assim como as meias de lã.
``Está frio\ldots{}'' ele reclamou com uma voz
trêmula, mais de vergonha e humilhação do que de frio, já que era
início da primavera. Ser desnudado, atado, cair cativo de quatro tipos
dominadores de mulheres, era certamente uma experiência que aniquilava
sua masculinidade. Não dava para prever o que elas poderiam fazer a
seguir, ao ouvir a ameaça que Lori anunciava agora:
``Vamos lhe ensinar qual dos gêneros da espécie é
agressivo \textit{de fato} na raça humana!\ldots{}''

``Nós vamos vestir você, Bruce'', ela
ronronou, seus olhos verdes reluzindo com um estranho fascínio, diante
do espetáculo de ter um homem ali, amarrado, em cativeiro.
``Agora, meninas --- fora com esse calção que ele está
usando\ldots{} meninos comportados não deveriam usar trapos como esses.
Vamos ensinar a nosso amigo Bruce como se vestir.''

``Não, não!, ele protestou; porém, quatro pares de mãos
femininas já puxavam seu calção para baixo. Com um suspiro de alívio,
ele se lembrou de estar usando seu minúsculo protetor atlético sob o
calção, detalhe que as garotas ridicularizaram com gargalhadas,
`Olhem --- ele está de tapa-sexo!'\,''

Lori então disse: ``Está bem, garotas, desamarrem-no.\idxamarr[|)]
Será mais fácil para vesti-lo. Mas Bruce, amiguinho'', ela
disse em tom de falsete, ``você não vai muito longe nesse
tapa-sexo. Então comporte-se, ou ele também será tirado de
você.''

Bruce enrubesceu e, assim que percebeu que tinha braços e pernas
libertados, tentou se cobrir com as mãos; mas sua estranha posição,
que denunciava um profundo embaraço --- com os joelhos muito juntos,
os pés afastados e os ombros recurvados --- só provocaram mais risos.
``Muito engraçado! Muito engraçado!'' ele
arfou. ``Vamos lá, garotas'', riu-se Lori.
``Eu mal posso esperar para ver como ele ficará, usando
roupas realmente elegantes. Primeiro, essa calcinha\ldots{}''

Lori ergueu então alguns sutiãs, e finalmente escolheu um item
charmoso. ``Veja, Bruce'', ela o balançava
diante dele, como que ameaçando sua masculinidade, ``este
sutiã tem enchimento, para aumentar o volume dos seios e erguê-los. Sua
frente, cavadíssima, é o que há de melhor para dar protuberância aos
seios; em uma garota, ele é delirantemente pecaminoso. Em
você'', e deu uma gargalhada profunda,
``ficará bem obsceno\ldots{}''

Ele não protestaria. Isso só iria enfurecê-las --- e, talvez, elas
intensificassem seu ódio contra ele. E, agora\ldots{} sim\ldots{} aqui estava: o
vestido que Bruce deveria usar.

``Você gosta?'' Lori perguntou, já às
gargalhadas, junto com as demais garotas, prelibando a visão de como
ele ficaria dentro do vestido. ``É importado da França. É
um modelo exclusivo.'' O vestido ostentava um encaixe em
``V'' de seda transparente, com um forro cor
da pele, ornamentado com tachinhas que produziam um efeito
\textit{sexy}; e um debrum de passamanaria. As costas, eram
decotadíssimas. As mangas eram feitas de uma seda muito macia, como um
\textit{voile}, num tom esfumaçado de vermelho. Na cintura, um cinto de
camurça, justíssimo; sua fivela era uma enorme réplica da figura de
Satanás, com seus dois dentes fazendo as vezes de pinos para o fecho.
Um minúsculo tridente despontava da fivela de prata polida. A saia
desse vestido incomum cintilava, com suas três fileiras de franjas de
couro, de uns 15 cm de comprimento. Cada franja era delicada como um
cadarço de sapato, porém forte como rédeas de montaria capazes de
obrigar um grupo de cavalos a obedecer ao comando do condutor. A cada
movimento dos quadris, as três camadas de franjas dançavam em todas as
direções, como um grupo de adoradores primitivos em transe diante de
um estranho Deus-fetiche.

À medida que a roupa foi sendo vestida em Bruce, ele notou que seu
coração batia, enlouquecido; suas emoções transbordavam, ele estava sem
fôlego, num frenesi de antecipação. Ele não ousava admitir seus reais
sentimentos para quem quer que fosse; nem mesmo para si mesmo! Afinal,
ele havia sido \textsc{forçado} a tudo isso\ldots{} pelos irmãos da fraternidade; em
seguida, ele tinha sido \textsc{capturado} e \textsc{amarrado pelas mulheres}, e obrigado
a cumprir as suas ordens\ldots{}
\end{quote}

Como é possível que a humilhação, produzida por ser forçado por
mulheres hostis a vestir roupas femininas, cause\idxsexue{} excitação sexual?\idxpornoexci{}
Existem alguns elementos que conseguem (quase) explicar a causa dessa
excitação.\footnote{ Na verdade, não sabemos de que maneira a
excitação\idxsexue[|nn] sexual\idxpornoexci[|nn] se produz nas pessoas em geral, não apenas nos
perversos. Como uma mulher (ou um corpo de mulher) excita um homem
heterossexual? O que foi que ele aprendeu, desde seus tempos de
lactente? E como as respostas, não genitais, da infância e da
primeira infância, se convertem na resposta genital do adulto? Será que
a explicação é meramente fisiológica? (Não me parece.) Será que a
ansiedade desempenha algum papel, tanto nas pessoas normais quanto no
perverso? Assim como Masters\idxmaste[|nn] e Johnson\idxjohns[|nn] levaram a cabo a tarefa
naturalista de revelar a aparência física bruta da excitação sexual,
assim também os mecanismos da experiência psicológica da excitação
sexual deveriam ser descobertos --- o que faz com que se instale? O
que a mantém e protege? O que faz com que diminua, transformando-se,
finalmente, em enfado?}

Em primeiro lugar, embora o homem, na ilustração, seja humilhado, o
homem que lê o livro é humilhado só em efígie; embora se identifique
com o personagem, ao mesmo tempo também ele está claramente em posição
segura: a identificação não é tão grande assim. Ele sabe que essa
experiência, que lhe acontece via pornografia, é só uma fantasia.\idxtravemhumi[|)]

Em segundo lugar, a excitação é acompanhada por um estratagema de
eliminação de culpa\idxculpar{} que é inerente à história: uma vez que o
garoto-homem está sendo forçado a vestir aquelas roupas pelas mulheres
malvadas, ele não pode ser acusado de querer, ele mesmo, fazer isso.
(Na pornografia, como no humor, existe sempre um estratagema para
diminuir a culpa.\idxpornodisp{} Isto poderia ser verdadeiro também para muitas outras
atividades sublimadas com componentes hostis, como, por exemplo, o
teatro, as artes visuais, as relações sexuais
``normais''\ldots{} Imaginem considerar a relação
heterossexual uma ``atividade sublimada''!)

Entretanto, as duas razões acima são apenas estratagemas secundários
para \textit{proteger} a excitação --- não são causas em si mesmas.
Chegaremos mais perto se estudarmos a história\idxpornoreal{} de vida que estiver
presente na pornografia, de maneira tão condensada.

O homem que primeiramente me mostrou essas dinâmicas --- e que
também trouxe o livreto que acabo de citar --- havia sido forçado, por
mulheres, a se vestir de menina na infância. Eu contei sua história em
outras obras (137, 142).

Felizmente para a pesquisa (e desgraçadamente para ele), ele posou
para fotografias, que foram abertamente colocadas no álbum de família,
traçando o desenvolvimento de seu travestismo.\idxmascuhum[|(] Ademais, as mulheres que
fizeram isso com ele estão vivas; embora eu não tenha podido
entrevistá-las, elas deram informações a ele e à sua mulher,\idxtravemporn[|)]
completando a história que as fotos assinalavam.

O paciente é um homem biologicamente normal, na casa de seus trinta
e poucos anos, casado e pai de filhos. O interesse dominante em sua
vida é a excitação sexual produzida pelas roupas de mulher; ele é
masculino\idxmascu{} no comportamento, na escolha de roupas --- sempre que não
está expressando sua perversão --- e na profissão.

Pelos primeiros quase três anos de sua vida, ele foi tratado pela
mãe e pelo pai como sendo o que era, um menino normal, que eles
esperavam que viesse a se tornar um homem. Ele foi batizado com um nome
inequivocamente masculino, e não foram emitidas mensagens encobertas a
contradizer o reconhecimento de que sua atribuição ao sexo masculino
era correta. Como resultado, ele desenvolveu, como quase todos os
menininhos o fazem, a convicção de pertencer ao sexo masculino --- um
primeiro estágio, necessário, no desenvolvimento da masculinidade em
todos os homens. Foi então que sua mãe foi vitimada por uma doença
crônica que a afastou de casa, e que culminou com sua morte pouco menos
de dois anos depois. Durante o período em que a mãe esteve
hospitalizada, o pai mobilizou a tia do garoto, com sua filha
adolescente, para cuidar da criança. Essas duas mulheres, infelizmente,
nutriam um imenso ódio pelos homens e pela masculinidade.\idxtraumatrae{} Dada a
liberdade com que lhes foi possível atuar sobre o garoto, elas
triunfaram\idxtraumatrav[|(] em atacar, sem serem incomodadas, sua masculinidade em
desenvolvimento. E fizeram-no alterando sua aparência. É fácil: as
mulheres podem simplesmente vestir os meninos com roupas pouco
masculinas, ou até mesmo com roupas de mulher. O que as incitou a
fazê-lo, eu sublinho, foi sua já presente masculinidade; era isso o que
elas odiavam, e que pôde ser melhor atacado --- elas tinham
consciência disso --- causando-lhe danos, e não tentando destruí-la. O
que essas mulheres queriam não era que o menino deixasse de ser homem;
antes, elas queriam aliviar sua inveja, afirmando que a masculinidade é
coisa sem importância e inferior. Para fazê-lo, elas deixaram claro,
para si mesmas e para o garoto, que elas queriam humilhá-lo\idxtravemhumi{} --- o que
requeria que ele conservasse para sempre seu desejo de ser homem, e a
consciência de que podia ser humilhado.

Em seu quarto aniversário, poucas semanas antes de morrer, sua mãe
esteve em casa para visitá-lo. Nessa ocasião, a tia e a prima
apresentaram à mãe ``uma nova menina da
vizinhança'' que era, de fato, o filho da moribunda; e
tiraram fotos, para registrar a brincadeira. O homem que esse garoto
foi um dia não guarda memória do evento traumático\idxtravetrau[|(] --- que só foi
descoberto pela mulher dele, num álbum da família, durante o período em
que se trataram comigo. A história foi, então, corroborada pela tia.

Até onde sabemos, a excitação sexual teve início dois\idxtravemperi{} ou três anos
mais tarde. Só a essa altura a memória do paciente, no que diz respeito
ao travestismo, se inicia. Naquele tempo, como punição, ele era forçado
--- era uma forma de castigo --- por outra mulher, a vestir suas
meias. Ele foi instantaneamente tomado por um sentimento voluptuoso,
que ele tem certeza não haver jamais ter experimentado antes. Por mais
prazeroso que fosse, ele sentia também uma aura de culpa;\idxculpa{} e assim, por
muitos anos, repetiu a experiência somente algumas vezes. Na puberdade,
entretanto, ela se tornou vinculada ao orgasmo; desde então, e até os
dias de hoje, essa tem sido sua forma dominante de prazer. Mesmo
durante o coito, ele só consegue ficar completamente potente quando
está travestido. (Talvez não coincidentemente, a mulher que o
travestia, como castigo, teve um filho, a quem dispensou o mesmo
tratamento; eu tenho comigo um retrato dele, vestido como Shirley
Temple.)

No decorrer dos anos da infância, e a partir dali até a
adolescência e a idade adulta, e até os dias de hoje, sua
masculinidade não foi destruída, só avariada. É exatamente como as
mulheres que o atacaram teriam desejado: caso ele tivesse assumido
completamente uma aparência normal de
``mulher'' elas teriam perdido sua vítima.
Ao invés disso, ele lutou secretamente contra elas, de modo a proteger
essa essência de seu \textit{self}.\idxmascuhum[|)]

Discuti em outra obra (137) a evidência de que o cerne da identidade
de gênero de cada pessoa --- o sentimento de ser homem ou mulher ---
se instala durante os primeiros três anos de vida, sendo, a partir daí,
 virtualmente inalterável, como aconteceu com esse menino.\idxcond[|(] Se alguém já
tiver desenvolvido esse sentimento de maneira inequívoca, experiências
posteriores poderão ameaçá-lo, forçando modificações sobre a pessoa
enquanto ela tenta proteger este cerne, que permanecerá.

Examinamos, até o momento, o esforço que a criança traumatizada faz
para salvar a si mesma. O caso acima exemplifica essa luta; mas também,
ao introduzir a questão da ameaça à masculinidade ou à feminilidade da
pessoa, expande a nossa compreensão a respeito da natureza exata
daquela vitimização: o medo de que a noção de pertencimento a um
determinado sexo, já constituída,\idxcastatrav{} possa ser destruída. Nos círculos
analíticos, chamamos a isto\idxtravemangu{} ``angústia\idxcastaiden{} da
castração'';\footnote{ Fenichel\idxfenic[|nn] resumiu, em linguagem
clássica: ``O pervertido é uma pessoa cujo\idxtravemangu[|nn] prazer sexual
está bloqueado pela ideia de castração.\idxcastaiden[|nn] Através da perversão, ele tenta
provar a inexistência da castração.\idxcastatrav[|nn] Desde que ele acredite nessa prova,
o prazer sexual e o orgasmo tornam-se novamente
possíveis'' (18, p.\,327).} mas o termo é muito estreito,
pois a pessoa teme mais do que a perda de seus genitais. Significa,
antes, que se a pessoa perder seus genitais, isto poderá significar
uma perda mais profunda: a pessoa não pertencerá mais à classe
masculina, convicção que está enraizada no âmago de seu ser. Homens
adultos, cujos genitais tenham sofrido algum dano, ou mesmo no caso de
os terem perdido, não perdem seu senso de masculinidade, muito menos
sua certeza de que ela existe; embora a experiência seja traumática,\idxtravetrau[|)]
ela não cria perversão, nem --- nas pessoas com identidade de gênero\idxtraumatrav[|)]
ilesa --- psicose.

Contudo, eu tendo a discordar da explicação behaviorista, de que o
ato perverso está vinculado apenas de forma fortuita --- condicionado
--- ao devaneio, ou à representação do devaneio (no travesti, por
exemplo, quando ele se veste com roupas de mulher pela primeira vez). A
explicação behaviorista tenta remover a infância e as dinâmicas
psíquicas da pessoa; ela parece atestar que \textit{qualquer} objeto ou
evento que ocorra por ocasião da primeira vez em que se obtém o máximo
de sensação prazerosa marcaria o ponto de partida daquela forma de
excitação sexual.\idxfantaporn{} Os analistas, ao contrário, acreditam tratar-se do
ponto de chegada, e que o agente da excitação, digamos, roupas
femininas, não foi acidental mas consequente, previsível e até mesmo
escolhido.\idxcond[|)] Procedendo-se a uma anamnese cuidadosa do paciente, essa
posição analítica se confirmará.

Voltando a nosso caso: o paciente foi capaz de manter um senso de
virilidade e de masculinidade através dos anos, a despeito da ameaça
feita por seus carrascos. Os travestis\idxtravemporn[|(] são famosos de longa data por
terem --- assim como esse homem --- uma aparência masculina, exceto
quando sexualmente excitados. Ele não costumam, como os homossexuais
afeminados,\idxhomosafem{} caricaturar constantemente as mulheres. Quase sempre eles
são abertamente heterossexuais,\idxtravemhete{} geralmente casados e com filhos, e
capazes de se comportar de maneira masculina sem esforço.

Porém, em nosso paciente, onde é que está a hostilidade,\idxvinga{} a vingança\idxpervmotiv{}
e o triunfo que são de se esperar, por minhas observações anteriores?
Se a tese está correta, esses componentes aparecerão na fantasia sexual
do travesti. Presumimos que, quando forçado a um papel não masculino,
por ter vestido roupas de mulher, o garoto tenha se sentido ameaçado em
seu âmago; e é de se esperar que ele tenha tentado se proteger, como
todas as crianças o fazem, criando um devaneio reconfortante. Ficamos a
par desses devaneios porque os travestis os contam, reagem a ele quando
a pornografia\idxpornofant{} de que se utilizam assim o sugerem, e os atuam, ao se
vestirem de mulher. Recapitulemos.

Todos os acontecimentos da história\idxpornoreal[|(] que passaremos a contar
encontram-se integrados na trama da pornografia.




\textit{História.} Da infância até a idade dos três anos, o garoto
teve um desenvolvimento masculino.

\textit{Pornografia}. A história tem início com um homem, masculino,
heterossexual, que até o momento não havia manifestado nenhum interesse
em roupas de mulher ou em quaisquer outros maneirismos femininos ou
afeminados.




\textit{História}. Quando o menino tinha três anos, a mãe deixou a
família e sua ``maternagem'' foi transferida
para uma tia e sua filha, prima mais velha do menino, sendo que ambas
nutriam desprezo pelos homens.

\textit{Pornografia.} O homem cai cativo de um grupo de mulheres,
que desdenham sua virilidade e imediatamente o subjugam.




\textit{História}. Seu pai quase nunca se encontrava em casa, de dia
ou de noite, por muitos anos; efetivamente, ele abandonou o garoto,
deixando-o nas mãos das duas mulheres.

\textit{Pornografia}. Ele também é o único homem na história.




\textit{História.} Essas duas mulheres providenciaram roupas novas
para o garoto, roupas com pregas, roupas afeminadas. Mais tarde
chegaram a vesti-lo não somente de forma afeminada mas com verdadeiras
roupas de menina, ``como diversão''.

\textit{Pornografia.} As perigosas mulheres forçam o homem,
envergonhado e humilhado ao extremo, a vestir roupas de mulher. Além do
mais, a pornografia mostra o quanto elas se divertem, como riem.




\textit{História.} As mulheres, por serem mais velhas e maiores,
psicologicamente eram extremamente poderosas, e podiam exercer seu
domínio sobre ele sem ter que lutar.

\textit{Pornografia.} O homem não tem forças para lutar, muito menos
para escapar.




\textit{História.} Não obstante, o garotinho necessitava e queria,
até mesmo amava, essas mulheres. Que escolha ele tinha aos três,
quatro, até os cinco anos de idade? Elas não só lhe serviram como
modelo para identificação mas como objeto heterossexuais de desejo,
pois agora elas eram a sua ``mãe''.

\textit{Pornografia.} As mulheres estão desenhadas como fálicas\idxmulhf{} e
perigosas, mas também como belas e femininas.




\textit{História.} Apesar de o\idxtravempape[|(] vestirem ocasionalmente em roupas de
menina, essas mulheres sempre passaram a ele a noção de que ele era
menino e tinha masculinidade.\idxmascuhum{} Exceto pelas raras\idxtraumatrae[|(] ocasiões em que o\idxtraumatrav[|(]
travestiam,\idxtravetrau[|(] ele usava roupas masculinas. Suas brincadeiras e jogos
sempre foram masculinos. Hoje em dia, ele lidera outros homens num
negócio tipicamente masculino. A tia e a prima, para tornarem completo
seu prazer, deveriam provar que a masculinidade era algo desprovido de
valor, muito abaixo de seu desejo. Para conseguir seu intento, elas
deveriam assegurar que essa masculinidade não fosse destruída, mas
apenas debochada. Assim, o garoto não foi feminizado ao ponto de
desejar que seu corpo se transformasse num corpo de mulher, nem perdeu
o prazer que podia extrair de seu pênis.

\textit{Pornografia}. O homem está claramente identificado como
homem; isto nunca é negado. Seu nome é definitivamente masculino, e não
é alterado no decorrer da história. As mulheres expressam
reconhecimento em relação à sua masculinidade. O ataque é
cuidadosamente endereçado, não visando prejudicar sua masculinidade,
mas sua identidade, seus atributos masculinos, dos quais os mais
visíveis são as roupas. Na pornografia dos travestis, um homem não se
transforma numa mulher.


\textit{História.} O desastre se transforma em triunfo. Com a idade
de seis anos, o menino ficava sexualmente excitado ao vestir roupas
femininas.

\textit{Pornografia.} O homem, ao final da história, encontra-se
completamente vestido de mulher.


\textit{História.} Ele encontrou uma mulher aparentemente afável,
gentil, que se casou com ele apesar de ter conhecimento de (na verdade,
fiquei sabendo alguns anos depois, movida por) seu travestismo. Ela
gostava de ajudá-lo a comprar roupas e perucas de mulher, e o ensinou a
se vestir com estilo, assim como a maneira correta de se maquiar e de
se comportar como mulher. (Esse tipo de mulher e seu relacionamento com
o travestismo foi discutido em outra obra [137].)

\textit{Pornografia.} As megeras agora são gentis, amistosas,
benevolentes, totalmente femininas; se comportam de maneira bastante
jovial.


\textit{História.} Ele, atualmente, vive normalmente sua vida,
passando, de vez em quando, por mulher.

\textit{Pornografia.} Eles saem juntos, e o rapaz aparenta ser uma
mulher normal; as moças lhe prometem que, em breve, tornariam a repetir\idxtraumatrav[|)]
a brincadeira, dessa vez como amigas.\idxpornoreal[|)]

O que falta no enredo pornográfico, mas que se verifica em meio aos
travestis,\idxcastatrav[|(] é um período de latência\idxtravemperi{} após o trauma, questão de meses ou
anos, durante os quais inexiste evidência de um travestismo aberto;
após este período, a primeira manifestação superficial da perversão se
faz notar (qual seja, excitação sexual ao vestir roupas de mulher).
Este período de latência, sendo silencioso, jamais foi estudado.
Portanto, podemos apenas conjecturar que, durante ele, o menino está
desenvolvendo um sistema de fantasias, para assim preservar sua
masculinidade contra a investida à sua identidade, que lhe foi
perpetrada por aquelas mulheres, carregadas de ódio e que realmente\idxtravempape[|)]
comprometeram seu senso de virilidade e sua masculinidade.

Não se trata de coincidência ele ter criado seu sucesso exatamente
na iminência do desastre. Ou seja, ele se utiliza de agentes do trauma
--- roupas de mulher e a aparência de feminilidade --- para preservar
sua masculinidade e sua potência. O que não significa que isso é tudo
de que se necessita para criar a perversão, pois, embora o pavor de ser
emasculado seja crucial, também o é a construção (defensiva) de que as
mulheres todo- poderosas são dotadas de pênis, e de um poder de
super-homens (18, 32, 35). Como observamos, isto também está indicado
no enredo pornográfico.

Tentei precisar --- tanto aqui como em outras obras (137, 147) ---
a natureza desse trauma específico (tentativa de feminização por parte
de mulheres mais velhas, poderosas) narrando casos que demonstram o
papel de mães (e suas substitutas) e de pais\idxrelpctrav{} na criação do travesti.
Esses dados sugerem que \textit{no travestismo fetichista a}
\textit{negação da angústia da castração\idxtravemangu[|(] e as mulheres fálicas\idxmulhf[|(]
imaginadas baseiam-se numa realidade histórica}. Nesses pacientes,
realmente aconteceu de o garoto ter sido ameaçado com a perda da
masculinidade, e humilhado por mulheres mais poderosas do que ele ---
não somente de alguma forma vaga, mas de maneira muito precisa, já que
ele foi vestido com roupas de mulheres. (Embora sem evidência para
comprová-lo, suspeito que o travestismo de um garotinho só seja
profundamente traumático caso ele tenha sido ameaçado em seus anos de
desenvolvimento, antes da primeira humilhação patente. Devem existir
também garotinhos que, depois de terem sido travestidos por alguma
menina ou mulher, simplesmente não são suscetíveis o suficiente para
que levem a sério essa forma de vitimização.)\idxpornoviti[|(]

Em que ponto nos encontramos para encontrar o suposto triunfo\idxtraumatrav{} que
preserva a potência do travesti?\idxtravempote[|(] Ele não pode vir simplesmente do
alívio de um trauma. Se o trauma é recapitulado na perversão, como o
prazer vem a substituir a angústia? Eu presumo, como ocorre em outros
episódios de dominação, que ele venha das seguintes fontes: seja por
descobrir que, em verdade, vez após vez, sobreviveu-se ao trauma, ou
das infinitas vezes em que se pôde fazer uso da repressão e da negação.
Mais especificamente, entretanto, sugere-se o que segue: (1) Conversão
de um sentimento de ser inferior e desvalorizado em fantasias
exibicionistas\idxexibitrav{} (``vejam em que linda mulher eu me
transformo!''). (2) ``Auto
realização''\idxtravemauto{} a criação gradual e consciente de um papel
``feminino'' desenvolvido em grau máximo:
alguns travestis aprendem a atuar tão bem como mulheres que podem fazer
aparições públicas como se o fossem, sem serem detectados.

Mais importante: (3) Fantasias (conscientes, pré-conscientes e
inconscientes)\idxtravemving[|(] de vingança\idxvingatra[|(] contra mulheres, sendo que tais fantasias
criam um sentimento exultante de ter revertido a questão. (4)
Identificando-se, nos enredos da pornografia\idxpornofant[|(] e também em outras formas
fantasias, não somente com os homens humilhados,\idxtravetrau[|)] mas também com as
agressoras dominadoras, as mulheres fálicas.

A vítima se transforma em vencedor. O garotinho foi humilhado, mas
agora quem manda é o adulto perverso, vestido em roupas de mulher.
Essas vestimentas, antes agentes do trauma, agora fazem sua delícia
--- ele está forte, sequioso, plenamente potente, e estão inteiros,
tanto seu pênis quanto seu \textit{self}, cheios de vigor e prontos
para o orgasmo. Que melhor maneira pode haver de comprovar seu triunfo,
do que comprovar sua potência na presença daquilo que, originalmente, o
traumatizou? Assim, ele se vinga. As mulheres, tão misteriosamente
poderosas na infância --- embora não percam nada de sua força ---
agora não são capazes de sobrepujá-lo: ele o comprova a cada vez que se
veste com suas roupas. Em cada uma dessas ocasiões, seu pênis demonstra
que elas fracassaram: ele conseguiu triunfar em defesa própria e,\idxpornoviti[|)]
portanto, frustrou o intento delas.

Mas infelizmente ele precisa\idxpervrepet{} repetir\idxfantanece{} a ação \textit{ad infinitum},
pois de alguma forma ele sabe que a perversão é somente um construto,
uma fantasia; ela não pode jamais provar que ele venceu. Ela só o faz
momentaneamente; e a cada vez, no decorrer de sua vida, que emergem as
circunstâncias que fazem eco à situação traumática original, ele pode
aplacar sua ansiedade somente através da repetição do ato perverso,
cuja função é assegurar-lhe, mais uma vez: ele é um vencedor e está
ileso.

Uma qualidade essencial na pornografia (e na perversão) é o\idxsadiporn{} sadismo\idxpornosadi{}
--- vingança por um trauma que foi passivamente vivenciado. Não estou
me referindo às bem conhecidas fantasias de vingança e de atos sexuais,
que se encontram também nos homens que não são travestis, como aquelas
de envenenar ou de humilhar seu parceiro com a ejaculação, ou de causar
dano físico a alguém com uma investida fálica. Suponho que algumas
vezes tais fantasias estejam presentes entre os travestis; mas, somado
a isto, e mais importante, o travesti consegue sua vingança pelo
simples fato de ser capaz de uma ereção. Ou seja, ele consegue a
vitória sobre a mulher justamente quando acreditava que falharia. Ainda
de forma mais triunfal, ele consegue essa vitória exatamente naquele
que deveria ser o momento de seu maior fracasso, quando ele está
vestido de mulher e deveria ser humilhado. Naturalmente, um fato
crucial o sustenta, ao estar vestido assim: sua consciência constante
de ter um pênis\idxpenisfant[|(] sob aquelas roupas de mulher --- o que o torna,
também, uma mulher fálica.\idxcastamulh{} Freud\idxfreudmulhe{} e a maioria dos analistas que o
seguiram acreditam que a fantasia da mulher dotada de falo é sempre a
invenção que um menino (homem) acredita ser necessária para negar que o
fato terrível da castração\idxcastatrav[|)] poderia lhe acontecer. Nessa teoria, as
mulheres são fundamentalmente --- anatomicamente --- inferiores, a
menos que ganhem uma prótese.\idxtravemangu[|)] Acredito que nem sempre seja o caso, nem
que o seja exclusivamente. Quando os homens fantasiam\idxtravemfant[|(] dar às mulheres
um falo, pode ser que o façam para negar não a inferioridade feminina,
e sim a sua superioridade;\idxtravemmasc[|(] o falo substitui, para os homens, um medo do
mistério da capacidade feminina de gerar --- um poder interno, oculto,\idxtravemving[|)]
como o que reside na procriação,\idxcondfcapa{} ou na onipotência sobre a vida e a
morte de seu bebê --- com o que lhes é familiar, um pênis. Adiante\idxvingatra[|)]
(caps.~6 e 8) voltaremos a esta questão.

Na pornografia, a expectativa de prazer atinge seu ápice no momento em que a vítima\idxpornoviti[|(] fica
sabendo, pelas poderosas mulheres, que seu castigo será ser vestido em roupas de mulher (ou após justamente ter sido vestido com elas). Não se trata, portanto, de coincidência, que a fantasia escolha o momento do
maior trauma para aquilo que, agora, é o momento da maior excitação.
Não existe triunfo mais perfeito do que obter a vitória após ter
justamente passado pelos\idxperigo{} mesmos perigos\idxpervexpos{} que o teriam vencido quando
criança. (Existem similaridades entre este triunfo contrafóbico e
outros, tais como corridas de automóvel, atuar num palco, saltar de
paraquedas, competições desportivas e tantas outras situações que
promovem, de forma aguçada, a ansiedade, pela perspectiva da vitória em
potencial.)

Quem é a vítima nessa fantasia de travestismo?\idxfetictrav[|(] No devaneio
manifesto, é o travesti em transformação retratado, com que o
travesti observador conscientemente se identifica. Mas de modo
adicional e inconsciente, a vítima é a cruel mulher fálica\idxcastamulh{} retratada,
pois o travesti, na realidade de sua masturbação,\idxmasturt{} está obtendo sua
vitória final sobre esta mulher. Apesar de tudo o que ela lhe fez, em sua
infância, para acabar com sua virilidade, ele conseguiu escapar dela
--- embora por um triz, e ao preço de uma potência gravemente
comprometida, que só consegue triunfar por meio da perversão.

Mesmo assim, ele vence:\idxtravemcliv[|(] ele sobreviveu. Seu pênis não apenas foi
preservado; agora, quando ele celebra seu sacramento, ele não se
sente mais cindido, mas concentradamente unificado em sua excitação
sexual.

Ele se identifica com o agressor, e então (como pode ser
frequentemente o caso com o uso desse mecanismo) ele acredita (ou tenta
acreditar) ser melhor do que o agressor: uma mulher melhor do que
qualquer outra mulher, pois ele possui o melhor dos dois sexos. Ele
está sempre consciente de sua masculinidade\idxmascutra{} (parte essencial no
travestismo), e também está ciente de sua feminilidade. Ele sente que,
tendo sido homem, e tendo vivido intermitentemente como homem, ele tem
um olho clínico para aquilo que é o mais apreciado pelas mulheres, e
ser uma ``mulher'' permite-lhe colocar essa
sabedoria em ação. Num nível mais profundo, ele se acredita (trabalha
constantemente para fazer com que ele mesmo acredite) ser uma mulher
melhor do que qualquer outra; porque, definitivamente, ele é a única
mulher dotada de um pênis. E agora, identificado com as poderosas
mulheres, ele não é mais o menininho humilhado; ele não vivencia mais,
conscientemente, aquela sua parte durante o ato perverso. Abertamente,
ele apenas existe no \textit{script}. Ele encontrou um modo de ser
sádico, expressando essa satisfação ao se convencer de que não é o
rapaz assustado retratado na história.\idxtravemmasc[|)] Cindindo sua identificação\idxtravemfant[|)]
entre o vitorioso e a vítima, ele é capaz de satisfazer, como se\idxpornoviti[|)]
existissem, a duas pessoas diferentes,\idxtraumatrae[|)] dentro dele mesmo.\idxpornofant[|)]

Apesar de os travestis serem, em sua grande maioria, abertamente\idxtravemhete{}
heterossexuais,\idxheterotrav[|(] e ansiarem por heterossexualidade, eles precisam
trabalhar contra um impulso inconsciente que os conduz em direção à
identificação com as mulheres. Considerando que a intimidade com uma
mulher de carne e osso é desejável, mas perigosa, eles substituem as
roupas femininas,\idxtravemsimb[|(] inertes, por sua pele viva. Note essas descrições,
tiradas do livreto, sobre essas roupas: ``As alças eram de
um branco leitoso; a transparência do tecido
enfeitiçava''; ``pura seda'';
``puro cetim''; ``calças de um
branco virginal''; ``colada à
pele''; ``verde transparente, como espuma do
mar''; ``fresca, com a suavidade da seda,
sensualmente íntima''; ``leve como uma
película''; ``de conformação
suave''; ``de um rosa corado'';
``delicadamente moldado''; ``uma
seda suave, macia, transparente'' e assim por diante, por
muitas páginas.

A experiência é bissexual: não apenas o travesti está estabelecendo
contato, através do fetiche (de maneira segura, de forma indireta) com
a pele da mulher (tomando a mulher como objeto heterossexual),\idxheterotrav[|)] mas ele
também está se vestindo com ela (identificando-se com a mulher).

Esta descrição deixa de fora muita coisa importante --- e
especulativa --- tal como uma investigação mais completa da crença do
travesti\idxfetictrav[|)] em mulheres fálicas,\idxcastamulh{} tanto do tipo poderoso, que originalmente
o atacou, quanto do tipo que ele representa, com seu pênis ereto por
debaixo das roupas de mulher; ou os significados simbólicos que as
roupas têm para ele (por exemplo, um pênis intacto);\idxpenisfant[|)] ou a angústia da
castração; ou os adornos como objetos transicionais entre sua mãe e o
estar separado dela; e uma legião de formulações psicanalíticas
adicionais.\idxtravemsimb[|)] Elas não serão detalhadas\idxtravemsimb[|nn] aqui, pois a tarefa de que\idxtravempote[|)]
estamos nos ocupando é, simplesmente, a de definir o conceito de\idxmulhf[|)]
perversão.\footnote{ Existem explicações em demasia. A teoria
psicanalítica\idxpsica[|nn] é o sistema mais sincrético desde o Panteão romano; um
novo Logos pode ser adicionado sem deslocar nenhum dos antigos
elementos: os trajes do outro sexo presumivelmente\idxtravempote[|nn] simbolizam também o
pênis do pai;\idxpenisfant[|nn] ou entrar dentro da pele materna, estando, assim,
protegido dentro de seu útero ou (caso você pertença a uma escola
diferente) seu pênis; ou, sendo ela mesma mãe, quer uma mãe provida de
falo, ou desprovida, ou ambas simultaneamente; ou os trajes servindo
para proteger a mãe da destruição; ou sendo o pênis do pai dentro da
vagina materna; ou a necessidade de proteger o pênis introjetado do pai
no útero materno (do qual existe um conhecimento atávico no
inconsciente coletivo) de ataques orais e anais. Conseguir,
dentro disso tudo, fazer a distinção entre metapsicologia, especulação,
fantasias, pomposidades, fraudes, absurdos, opiniões ultrajantes embora
incontestes, sugestões maravilhosas, \emph{insights} brilhantes, e descobertas
originais e demonstráveis, isto exigiria um trabalho de dedicação
obsessivo-compulsiva que não tenho o desejo, nem a paciência, de
empreender. Poucas palavras-chave de nossa linguagem são definíveis
--- a menos que seja por intermédio de outras palavras-chave,
indefiníveis elas mesmas (por exemplo, ``narcisismo é a
catexia do \textit{self}''); pouco é afirmado como uma
proposição que pode ser testada no mundo observável, mas apenas
recorrendo à autoridade, ou manipulando mais teoria. E, mesmo após ter
corrido todos esses riscos desnecessários, somos frequentemente
deixados com apenas uma exposição extremamente complicada do óbvio. O
leitor que assim o desejar, encontrará essa evidência na devastadora\idxtravemcliv[|)]
e muito pouco conhecida obra de Leites\idxnatha[|nn] (86).}

O leitor poderá perguntar se este é um estudo da pornografia ou do
travestismo,\idxpornoinst{} pois a matéria aqui abordada salta de uma perspectiva para
a outra.\idxpornoplat[|(] Isto, já por si, salienta um ponto óbvio: que a pornografia,
assim como o assunto padrão do devaneio do sujeito, em termos de
psicodinâmica, gira em torno da mesma coisa que a sua perversão. Ela é
a história altamente condensada de sua perversão: suas origens
históricas na realidade, suas elaborações na fantasia, seu conteúdo
manifesto que disfarça e revela o conteúdo latente. Sem a pornografia
pode-se ainda, obviamente, estudar as dinâmicas das perversões; porém,
com a pornografia, obtemos uma ferramenta especial, que eventualmente
irá nos fornecer pistas que, de outra maneira, poderíamos deixar
escapar. Especialmente de ajuda é o fato --- uma vez que a
pornografia, para seu criador, é produzida para fazer dinheiro --- de
que esse criador estará motivado ao máximo para desenvolver uma
fantasia que não seja idiossincrática. Já que sua pornografia deverá
ser paga, ele precisará, intuitivamente, extrair daquilo que sabe sobre
sua audiência aqueles elementos que todos compartilham, de modo geral.
Se ele não agir dessa forma, correrá o risco de vender uma cópia
apenas. Assim, ele precisará criar um trabalho suficientemente preciso
para excitar, e geral o bastante para excitar a muitos. Portanto, a
pornografia é, para o pesquisador, um tipo de estudo estatístico das
psicodinâmicas --- um método mais colorido e mais poderoso do que a
sondagem de opinião, que algumas vezes nos é impingida como rigorosa
pesquisa.

Com o relaxamento das leis que restringiam a produção da
pornografia, o mercado cresceu; tem sido financeiramente possível, para
os produtores, fornecer, com maior precisão, em conformidade ao gosto
de seus seletos leitores. E assim, onde, antigamente, todos os
travestis, independentemente de seus múltiplos interesses dentro do
gênero, tinham que se contentar com a mesma história, cada um agora
poderá encontrar formas variadas, projetadas mais precisamente para o
\textit{seu} gosto em particular. Portanto, nem todos os homens que se
trasvestem de maneira intermitente, e que se excitam sexualmente com
roupas femininas, vão escolher o enredo pornográfico que citamos aqui
como sendo sua primeira escolha. Eles dizem que, no passado, eles se
contentavam com ele, compravam cada livro que saísse com essa
descrição, mas que ela não se encaixava exatamente dentro da sua
predileção. Portanto, para aqueles travestis que consideram o flagrante
sadomasoquismo\idxpornosadi{} daquela história\idxsadiporn{} demasiadamente intenso, hoje em dia
existem histórias mais charmosas à disposição, sobre o homem feliz,
tímido, e a mulher, feliz e competente, alegremente comprando roupas de
mulher; e, então, a feliz mulher vestindo as adoráveis roupas no não
menos feliz homem.

O que vem a seguir é uma história de uma revista para travestis.\idxtraveporn{} Um
homem masculino, sem prévios interesses no travestismo, foi aconselhado
por uma amiga a ir se vestir:

\begin{quote}
Havia chegado o momento de preparar-se para o churrasco e Lynn [o
homem que vai se transformar num travesti] escolheu uma túnica florida
e um par de sapatos cujo saltinho era mínimo, para a ocasião. Dispensou
mais tempo do que o habitual para a aplicação da maquiagem, dos olhos e
da boca. Como ela se deliciava ao fazer o contorno aos lábios, aquela
maravilhosa curvatura com que a natureza a contemplara. Atenção
especial foi dada ao penteado, até atingir a perfeição, e na seleção
dos colares, no comprimento exato, para combinar com sua indumentária
colorida. Millie [a mulher que o está encorajando a travestir-se] se
vestiu de modo parecido, mas adicionou duas flores artificiais sobre as
orelhas.

``Você está simplesmente linda'', elogiou
Millie, ``garota mais linda simplesmente não existe.
Entretanto, tente não falar demais esta noite; prefira observar o que
os outros fazem e dizem, está bem?''

Logo, as duas moças misturaram-se com os outros hóspedes, em volta
da piscina, e a estreia de Lynn teve, assim, o seu início. Millie pôde
notar, enquanto observava Lynn movendo-se para lá e para cá, o quanto
a sua amiga parecia graciosa e feminina.
\end{quote}

Mais tarde:

\begin{quote}
``Que noite maravilhosa'', Millie exclamou
com entusiasmo. ``Bill é mesmo encantador e sabe cativar
as mulheres! Você também se divertiu?''

``Sim e não'', Lynn respondeu.
``Para ser honesta, eu me senti por fora dos
acontecimentos, mas não quis me envolver demais, o que poderia fazer
com que eu me denunciasse.''

``Não seja tonta, fique segura e seja você mesma da
próxima vez em que sairmos. Ainda assim posso entender por que você
não se sentiu totalmente à vontade'', Millie respondeu,
``embora não houvesse como alguém suspeitar que você não é
o que aparenta ser.''

``Para você é fácil me dizer que seja eu mesma, mas
lembre-se de que o ``eu'' que existia
até dois meses atrás era completamente homem. Negócios e esportes não
seria a conversa que se poderia esperar que eu mantivesse com os
homens, não é mesmo?'' Lynn redarguiu.
``Posso me sentir perfeitamente bem quando estou cercada
apenas por mulheres. Deus sabe que a quantidade de matérias versando
sobre assuntos femininos que li nos últimos meses deve corresponder ao
material de leitura delas por toda uma década\ldots{} e as conversas que nós
duas temos tido deram-me a confiança necessária para estar em meio a
elas, mas o mesmo não se dá em relação aos homens.''

``Não preocupe a sua linda cabecinha com isso
agora'', Millie diz. ``Resolveremos esse
problema também, em seu devido tempo. Descanse um pouco, pois amanhã
devemos ter um dia cheio.'' Beijou Lynn na testa e saiu.
\end{quote}

Mais tarde:

\begin{quote}
Millie estava completamente vestida, e logo elas estavam conversando
sobre todas aquelas pequenas coisas que a maioria das mulheres tanto
apreciam. Ao terminarem, Millie insistiu que se ocuparia dos pratos,
para que Lynn pudesse ir logo se vestir. ``Use o \textit{tailleur}
bege e aquela adorável blusa coral de que você tanto gosta'' ordenou.
``Não quero que você tenha uma aparência muito chamativa hoje, pois
passaremos a maior parte do dia fora.''
\end{quote}

Mais tarde:

\begin{quote}
As senhoras foram logo atendidas, e Lynn estava encantada com o modo
como o garçom as ajudou a se sentarem. Durante sua frugal refeição,
Millie contou a Lynn os planos que fizera para ambas, no decorrer do dia:
``Vamos ambas nos inscrever na Escola de Charme de
John Robert Powers, onde elas não somente receberão instruções sobre
maquiagem e roupas, quesitos em que você tem se saído muito bem, mas
também na arte da conversação e no desenvolvimento de sua personalidade
feminina. A maioria das mulheres que frequentam esses cursos também
deixam a desejar neste quesito, e da próxima vez em que formos estar em
grupos mistos, entre homens e mulheres, como aconteceu naquele
churrasco, quero que você se sinta à vontade; o curso vai ajudá-la a
aperfeiçoar-se com relação a isto''.
\end{quote}

Na extremidade oposta do espectro estão as histórias em que o
sadomasoquismo\idxpornosadi{} é intenso,\idxsadiporn{} e em que ele se coloca mais em primeiro plano
do que o travestismo.\idxtrave[|)] Nesta modalidade,\idxtravemporn[|)] a história é tão violentamente
instantânea que é frequentemente representada simplesmente por
fotografias, sem texto. Mostram uma
``mulher'' amarrada em cordas e correntes, em
posições desconfortáveis --- na realidade, um homem em roupas de
mulher; mas o que é excitante, neste tipo de pornografia, não é
simplesmente o homem estar em roupas de mulher, mas o fato de
``ela'' estar acorrentada. Com a pornografia
se tornando suficientemente específica para cada tipo de homem, há
menos necessidade de adquirir a pornografia do passado --- que era
aceitável, mas não ideal.

Eu tenho a impressão (os casos não são suficientes para que se possa
ter certeza) de que aqueles que, em sua infância, foram tratados de
maneira menos cruel por uma mulher (ou mulheres), preferem um tipo de
pornografia mais alegre, em que a humilhação sem rodeios, ou mesmo um
flagrante sadismo físico, não faz parte evidente da história.

Contudo, essas pornografias diversificadas\idxpornoangu[|(] têm em comum a evocação
de um perigo que é superado (humilhação, angústia,\idxanguporn[|(] medo, frustração).
Neste sentido, toda pornografia provavelmente contenha a psicodinâmica
das perversões. Eu defendo a tese de que não existe pornografia que não
seja perversa, ou seja, conteúdo sexualmente excitante em que a
hostilidade\idxhostporn{} não\idxpornohost{} seja empregada como finalidade.\idxpornohete[|(] A maior parte das
produções pornográficas é endereçada a um público heterossexual,
masculino, e como existe uma clientela tão grande, e uma quantidade tão
grande dessa pornografia de segundo escalão, este tipo de literatura é
``normal'' no sentido estatístico de agradar
a muitos homens. Portanto, para a maioria dos homens, em nossa
sociedade, a pornografia consiste de fotografias de mulheres nuas e de
atos sexuais retratando a penetração entre homens e mulheres. O fato de
essas formas serem comuns, não significa que elas não surjam como
soluções para conflito, angústia, frustração e raiva. Caso elas fossem
``normais'' no sentido de serem universais,
uma expressão biológica da busca por prazer, sem nenhum conflito, então
a nudez\idxfeticnude{} deveria ser sexualmente fetichista em todas as sociedades (o
que não acontece), e não apenas nas sociedades como as nossas, em que\idxpornoplat[|)]
ela se torna tantalizante, em função da frustração.

A pornografia poupa o indivíduo da ansiedade de depender de outra
pessoa para poder ter sexo; as pessoas, na página impressa, sabem seu
lugar, e agem conforme as instruções.

Apesar de ser popular, a pornografia pode, contudo, não ser
simplesmente (embora possa, especialmente na adolescência, sê-lo
parcialmente) um substituto, em função da falta de objetos sexuais mais
adequados. Ela existe porque satisfaz a necessidades voyeurísticas e
sadomasoquistas, que, em algumas pessoas, não podem ser satisfeitas,
não importando a quantidade de parceiros sexuais à disposição e que, de
boa vontade, se candidatariam. Apesar de o orgasmo genital ser a
desembocadura comum para o prazer e para o alívio da tensão da
necessidade perversa, as perversões frequentemente se utilizam de atos
que são executados sobre objetos, ou sobre partes do corpo que,
simplesmente, não são passíveis de serem completamente aliviados pelo
orgasmo (cf.\,152,~p.\,316). Os órgãos não genitais\idxpervorgao{} --- por exemplo
olhos, pele, ânus --- e outros afetos, que não o amor --- por
exemplo a raiva, a angústia,\idxanguporn[|)] a depressão --- podem, sabemos, ser
erotizados; mas a tensão não pode ser facilmente liberada, como
acontece na genitália. Isto concede à perversão, a meu ver, uma
intensidade, uma compulsividade --- uma desesperança. A teoria
analítica, que vincula a perversão à neurose e aos transtornos
psicossomáticos,\idxpsicss{} há muito tempo vem sugerindo que, caso seja criada uma
tensão erótica sobre um órgão incapaz de descarregar-se de forma
adequada, um processo de alteração celular crônica passa a ocorrer.\idxpornoangu[|)]

Se, na pornografia, as atividades sexuais estão retratadas de uma
forma em que sempre há uma vítima,\idxpornoviti[|(] quem é a vítima nos filmes e nas
descrições de penetração heterossexual? Quem é a vítima, e o que é a
atividade sexual, numa fotografia que exibe nudez?


Ao passo que muito da excitação, na pornografia da
heterossexualidade, pode advir simplesmente da identificação com os
participantes retratados, que estão ali a demonstrar sua agilidade (e
que não sentem angústia, ou medo de disfunção erétil, como o espectador
poderia sentir) é possível, também, que um quê de picante seja
acrescentado, pela fantasia da cena primal de uma criança que consegue
levar a melhor e assistir ao que não deveria; e, talvez, um sentimento
de superioridade, por fazer parte da audiência e, portanto, não estar
exposta a nenhum perigo. As vítimas então são os
``adultos'' cuja ausência de onipotência se
comprova, uma vez que não sabem que estão sendo observados.

São muito populares as descrições de uma mulher que, no início, é
fria, superior, sofisticada e desinteressada; mas é transportada, pelas
ações precisamente descritas do homem, a um estado de luxúria, com uma
monumental perda de controle. Vê-se facilmente, aí, uma luta de poder
disfarçada como sexualidade: a mulher perigosa que é reduzida a vítima,
e o garoto que, através da pornografia, por um momento, na ilusão do
poder, torna-se homem (106).\idxpornoviti[|)]

Eu disse que uma dinâmica essencial na\idxsexuehos[|(] pornografia\idxpornohost[|(] é a hostilidade.\idxhostporn[|(]
Talvez a diferença mais importante entre a pornografia mais perversa e
a menos perversa (``normal''), assim como
entre perversão e ``normalidade'' seja o
grau de hostilidade (fantasias de ódio e vingança) que está contido, ou
que é liberado, na atividade sexual. Podemos levantar a talvez
controversa\idxhostexci[|(] questão de saber se nos humanos (e especialmente nos
homens) uma poderosa excitação sexual poderá jamais existir sem que a
brutalidade também esteja presente (seja ela mínima, reprimida,
distorcida por formação reativa, atenuada, ou flagrante, nos casos mais
patológicos). Isto pode ser comparável a indagar se uma peça de humor
pode existir sem hostilidade (25). No humor, a hostilidade não é
simplesmente anexada, ela é condição \textit{sine qua non} (apesar de não ser a
única). Será possível que, na excitação sexual não perversa, a\idxpornohete[|)]
hostilidade inconsciente seja também essencial, e não simplesmente
anaclítica?

Será que alguém poderia fornecer exemplos de comportamentos, na
excitação sexual, em que, pelos menos em humanos do sexo masculino, a
hostilidade,\idxhostdese{} disfarçada na fantasia, não seja parte da potência? Já nos
habituamos com uma situação familiar em que a hostilidade sobrepujada é
essencial para funções normais, pois sabemos que o desenvolvimento\idxegoi{}
normal exige que os bebês sejam cada vez mais frustrados,\idxinfanfrust[|(] a fim de
possibilitar a separação,\idxinfanego{} que resultará nas funções egoicas e na
identidade do ego, necessárias para se lidar com o mundo externo. Este
processo, que usa a frustração como ferramenta essencial, cria um
reservatório de raiva inconsciente; saber lidar com esta raiva ajuda a
determinar o desenvolvimento da personalidade, se bem sucedida ou se
mal ajustada. A dominação, a mais gratificante das experiências, e que
frequentemente acontece como um acerto de contas, uma compensação pela
frustração que foi passivamente suportada, atua através da criação de
fantasias, de estruturas de caráter ou de modos de atividade que, em
sua forma mais primitiva, são brutais; mas que, filtradas por um
processo de sublimação, podem terminar consideravelmente destituídas de
seu ódio original.

Se a hostilidade pudesse ser totalmente excluída da excitação
sexual, não existiriam perversões; mas que quantidade de sexualidade
amorosa poderia ser possível? As diferenças entre cada uma das
perversões, e entre os diferentes tipos de perversão e o comportamento
sexual mais comum, pode ser que se encontre nas diferenças específicas,
em termos de frustrações\idxinfanfrust[|)] e gratificações (frequentemente determinadas
pela sociedade, mas aplicadas pelos pais, e especialmente pela mãe) que
foram vivenciadas pelo bebê,\idxhostexci[|)] na primeira\idxsexuehos[|)] infância,\idxpornohost[|)] e pela criança, na\idxhostporn[|)]
infância.

Algumas poucas palavras podem vir a calhar, em relação ao fato
enigmático de que, se alguém fosse tentar vender pornografia às
mulheres, acabaria morrendo de fome. Por quê? Existe informação no
próprio questionamento, e não, simplesmente, na resposta. A pergunta é
como ``o mistério ancestral'' ---
colocando-o em sua forma mais madura --- ``O que é a
Mulher''? Só os homens se preocupam com o mistério da
mulher; as mulheres não, porque para elas não existe mistificação. Isto
não significa necessariamente que elas compreendam as dinâmicas de sua
própria sexualidade, mas simplesmente que, por experimentá-la, não a
consideram como sendo misteriosa. Se as mulheres quisessem, poderiam\idxpornomulh[|(]
colocar questões sobre o mistério da sexualidade dos homens, assunto
que poderia não ser tão claro quanto alguns nos levariam a acreditar.
Poderíamos tomar como pressuposto nossa compreensão sobre a sexualidade
masculina, pelo fato de a maioria dos trabalhos que trataram do assunto
ter sido feita por homens que, por vivenciá-la, não necessitam ser tão
curiosos, nem mistificar a questão.

Com relação à questão de saber por que as mulheres não respondem à
pornografia tão intensamente quanto os homens, talvez a pergunta esteja
incorreta. Os homens tendem a igualar a pornografia em geral com aquilo
que é pornográfico para eles, em particular; mas, por exemplo, uma
representação precisa do ato sexual, do coito, apesar de excitante, é
menos irresistível para as mulheres. São poucas as mulheres que
compram revistas que exibem nus masculinos, o que não significa que
elas não tenham sua própria forma de pornografia --- elas a têm. Em
virtude de suas experiência da infância, em nossa sociedade, serem
diferentes, as mulheres precisam que alguns aspectos de sua pornografia
sejam diferentes da dos homens. Hoje em dia, sabemos que as mulheres
estão repletas de suas próprias fantasias sexuais,\idxfantamulh[|(] e que são atiçadas
pela pornografia (ver, por exemplo, 36, 69).\footnote{ Minha impressão
neste ponto, antes de estudar cuidadosamente a questão, é a de que
menos \textit{tipos} de perversão são representados na pornografia para
mulheres, que, além dos delicados romances de sadomasoquismo, parece
consistir mais frequentemente em infinitas variações do conto
edípico"-masoquista a"-garota"-predileta"-do"-harém"-do"-sultão ou a fantasia
sádico"-reparadora da ``supermulher que frustra hordas de
garanhões fogosos.''} Os homens, ao julgar a pornografia
das mulheres, cometem o mesmo erro que ao julgar a pornografia de
qualquer indivíduo com dinâmicas diferentes; dado que aquilo não os
excita, não conseguem perceber que o material poderia ser excitante
para outrem. Ao ler as histórias românticas, de fundo masoquista, que
em anos recentes têm sido produzidas para excitar as mulheres, os
homens podem considerar aquilo sem valor, já que lhes parece tão pouco
sensual. É o tipo de literatura que pode aparecer \textit{ad libitum},
não sendo reconhecida --- nem mesmo pela legislação --- como obscena.

Além disso, a maioria de nós acredita (cf.\,78) que, embora elas
possam ser atiçadas pela pornografia, as mulheres são menos
\emph{voyeurísticas},\idxvoy{} e o voyeurismo é uma qualidade essencial da pornografia.
Embora algumas mulheres, hoje em dia, admitam que apreciam admirar as
calças masculinas, as mulheres não são verdadeiramente\textit{
\emph{voyeuses}}. Pode ser que isto não reflita uma diferença biológica entre
os sexos, e sim as inibições que são impostas por nossa sociedade ao
direito que um garotinho possa ter de lançar olhares sensuais, e o
treinamento, imposto às meninas, para que não permitam que esse tipo de
olhar lhes seja dirigido --- o que implica que, em relação a si mesma,
não faz muita diferença que ela olhe ou não. Pode também decorrer daí
que, à medida que a pornografia heterossexual rotineira se torne mais
acessível também para ela, numa sociedade mais condescendente, mais
mulheres passem a descobrir um gosto por esse tipo de produtos. Se
acontecer de a visão de órgãos sexuais masculinos\idxpenisporn{} se tornar uma coisa
proibida, mas que as garotas apreciam --- assim como a visão dos
seios, para os meninos --- as mulheres se sentirão atraídas pela
pornografia que os exiba.

Eu sublinhei o óbvio: que o que é pornografia para uma pessoa pode
não o ser para outra, que tenha uma história de vida diferente, um
outro psicodinamismo. Ao ler as histórias repetitivas, sempre iguais,
dos travestis, os que não são travestis rapidamente se distraem, a
mente vagueia, e eles rapidamente perdem o interesse, não conseguem ler
mais. Um dia eu pedi a um travesti que me trouxesse um exemplo de
pornografia que fosse adequada a seu tipo de travestismo; ele me
respondeu que as histórias que ele já havia me mostrado, e que eu havia
considerado muito enfadonhas para que as pudesse ler eram, de fato, o
seu tipo de pornografia. Da mesma maneira, o que as mulheres consideram
excitante, nos livros e nos filmes, causará inquietação nos homens da
plateia, que ficarão aguardando que a história evolua para novamente
conquistar o seu interesse.\idxpornomulh[|)]

É óbvio, também, que os políticos de hoje, ao legislarem sobre a
pornografia, tenderão a definir como sendo pornográficas somente
aquelas coisas que os excitam e, como obscenas, apenas as produções que
lhes causam repugnância.

As sociedades temem a pornografia em virtude de temerem a
sexualidade; mas, talvez, exista também uma razão menos doentia: elas
respondem, intuitivamente,\idxpornohost{} às fantasias hostis,\idxhostporn{} disfarçadas, mas ainda
ativas, na pornografia. E assim, a pornografia será asquerosa para a
pessoa que responde a ela (é, na verdade, a reação da pessoa que
transforma uma lengalenga em pornografia); a palavra
``asquerosa'' assim como\idxfantamulh[|)]
``repugnante'' implica não apenas uma
sensualidade proibida, mas, também, o medo de que a hostilidade possa\idxporno[|)]
vir à tona.



\chapter[\textbf{6}\quad Hostilidade e mistério na perversão]{{\large\textit{Capítulo 6}}\\ 
Hostilidade e mistério na perversão}
\markboth{Dinâmicas: trauma, hostilidade, perigo e vingança}{Hostilidade e mistério 
na perversão}


Para que a perversão se constitua,\idxhost[|(] são necessários também outros
fatores, além da hostilidade, do perigo e da reversão do trauma em
triunfo --- em que este estudo se concentra. Seria, por exemplo, de
valor inestimável um livro inteiro a respeito do papel da culpa,\idxculpa{}
consciente e inconsciente. Mas não faz sentido repetir o que já foi
estabelecido, ou especular sobre a especulação alheia; todas essas
ideias terão que ficar como pano de fundo para a discussão a que este
livro se limita. Gostaria apenas de sublinhar que o estudo da perversão
é o estudo da hostilidade, mais do que o estudo da libido. Este livro
não revela --- e não pretende fazê-lo --- uma teoria completa da
perversão. Minha decisão, por exemplo, de não comentar em detalhe
algumas obras de outros autores sobre a importância do conflito\idxconfe{} edípico
na criação da perversão tem seu propósito; faz com que eu possa ganhar
tempo, permitindo concentrar-me em questões mais controversas do
período\idxpreedi[|(] pré-edípico.\idxconfedese{}

Ainda assim, poderá ser útil rever rapidamente a tese freudiana\idxfreudaberr[|(] a\idxfeticfreu[|(]
respeito da perversão e, então, continuar com nossa discussão anterior
(cap.\,2) acerca dos argumentos que foram aventados contra ela, nesses
novos tempos de avanço da pesquisa sexual. Freud\idxfreudfetic[|(] dizia que a aberração
sexual era produto de fatores constitucionais (biológicos)\idxaberrbiol[|(] e acidentais\idxpervacide{}
(interpessoais). Ele acreditava existirem alguns raros casos em que o
comportamento aberrante era quase que puramente produto de fatores
biológicos; e outros, em que seria um produto de influências não
biológicas, psíquicas;\idxaberrbiol[|)] mas que, em seu maior número, devia-se a uma
mescla desses dois elementos (24). Reconhecendo que a compreensão dos
fatores inatos se situava fora do escopo da psicanálise e além do
alcance do laboratório, seu principal interesse --- que constituiu,
também, sua maior contribuição --- era descobrir como o fator
``acidental'' conduzia à perversão.\idxhost[|)] Ele o
faz, Freud afirma, do mesmo modo como os sintomas neuróticos se
produzem: uma pulsão instintiva do id encontra um
``não'' imperativo, no superego ou na
realidade; isto faz com que o ego elabore um acordo de conciliação\idxacor{} que
gratificará (parcialmente) o desejo instintivo, ao mesmo tempo que
aplacará a exigência do superego, ou da realidade, de que o desejo se
extinga. O ``não'' se escora em ameaça: Freud
postulou também que a perversão era --- para usar meu conceito --- um
transtorno de gênero; ou seja, a perversão resultava seja da tentativa
de evitar a castração ou, nas mulheres, como compensação pelo
``fato'' da castração. E, assim, o conflito
edípico e sua resolução, tanto nos homens quanto nas mulheres, fazia
parte da explicação; com o tempo, alguma elaboração inicial sobre a
importância dos desejos pré-edípicos\idxconfedese{} também foi introduzida. Das muitas
adições e refinamentos que vieram a complementar esta explanação,
talvez a de maior importância histórica seja o conceito de
``clivagem''.\idxconfcliv[|(] Freud descreve como a criança,
em meio a um conflito entre ``um poderosa pulsão do
instinto'' e ``um perigo real quase
intolerável'', ``responde ao conflito com duas reações contrárias, ambas
válidas e efetivas. Por um lado, com a ajuda de certos mecanismos, ela
rejeita a realidade e se recusa a aceitar qualquer proibição; por outro
lado, e simultaneamente, ela reconhece o perigo da realidade, assume o
medo que sente daquele perigo como um sintoma patológico e tenta, em
seguida, se despojar desse medo. É preciso confessar que esta é uma
solução bastante engenhosa da dificuldade. Ambas as partes da disputa
obtêm o seu quinhão: o instinto ganha permissão para manter sua
satisfação, e um devido respeito é demonstrado pela realidade. Mas
existe um preço para tudo, de um modo ou de outro; e este êxito é
alcançado ao preço de uma rachadura, uma fenda no ego que nunca
cicatriza, mas que aumenta, com o passar do tempo. As duas reações\idxpreedi[|)]
contrárias ao conflito se mantêm, constituindo o cerne de uma clivagem
do ego''\idxcliv[|(] (35).

Embora, aqui, Freud esteja falando apenas sobre fetichismo,\idxfeticfreu[|)] existe
razão, tanto no que ele diz quanto no trabalho subsequente de outros
autores, para estender o conceito da clivagem\idxcliv[|)] para todas as
perversões.\footnote{ Isto se compreende quando, ao falar sobre o modo
como a vítima se torna o vencedor, eu mencionei as identificações\idxmulti[|nn]
múltiplas\idxidentmult[|nn] --- algumas vezes com o masoquista, outras com\idxsadi[|nn] o sádico\idxmasoq[|nn] ---
que estão simultaneamente presentes, em diferentes níveis de
consciência, na fantasia perversa. Outro exemplo é a demonstração de
Williams\idxwill[|nn] (154) sobre como a clivagem\idxconfcliv[|nn] é\idxcliv[|nn] usada naquilo que ele chama de
assassinatos\idxassas[|nn] sexuais (com o que ele não se refere às pessoas que matam
para satisfação genital, e sim aos homens que assassinam\idxfreudfetic[|)] mulheres,\idxfreudaberr[|)]
tornando, deste modo, nebuloso, para mim, o significado de
``sexual'').}




\section{A implicação da responsabilidade moral}

Antes de examinar\idxpervrespo[|(] melhor\idxrespo[|(] o papel da hostilidade, é preciso que eu
aborde uma questão moral que é inerente a cada aspecto da teoria
psicanalítica, mas que é especialmente visível nas discussões sobre
comportamentos flagrantemente sexuais e flagrantemente agressivos: a
questão do livre-arbítrio;\idxlivre[|(] ela se encontra na raiz das conotações do
termo ``perversão'' e, portanto, nos
acompanhará, à medida que criamos a teoria. Explicar o comportamento
patológico como originado pelo conflito, e dizer que ele emprega
mecanismos tais como a repressão, a negação, o repúdio ou a clivagem,\idxconfcliv[|)]
isto equivale a dizer, por intermédio do conceito de superego, que as
decisões voluntárias --- decisões informadas --- estão sendo
logicamente tomadas por ``instâncias''
psíquicas, que desempenham as funções para as quais foram construídas.
Muito dessa atividade ocorre de modo inconsciente, o que mitiga, mas
não elimina, a responsabilidade. Que, a princípio, os conflitos emerjam
dos perigos do mundo externo --- os costumes que a sociedade
transmitiu através das idiossincrasias neuróticas dos pais --- mais
uma vez apenas mitiga o senso de responsabilidade. Quando --- é o que
nossa teoria diz --- somos atacados por esses perigos externos na
primeira infância, nós escolhemos proteger nossos prazeres instintuais,
disfarçamos nossas intenções reais, ludibriamos as pessoas do ambiente
externo, que são a origem dos perigos, e distraímos a nós mesmos de tal
maneira a perder (recalcar) nosso conhecimento sobre o que aconteceu
conosco e o porquê de nosso comportamento. Desta forma, nossa teoria
da motivação mistura determinismo e livre-arbítrio na mesma poção.

Para complicar essa visão objetiva (cientificista) das origens da
motivação psíquica, temos nossa crença subjetiva, onipotente
(narcisística) de que nosso comportamento não é determinado (nossas
escolhas não são determinadas) e sim, quase sempre, resultado de nossa
livre escolha; esta é a mensagem do austero superego. Em resumo,
portanto, quer de forma consciente ou inconsciente, a pessoa acredita
que escolhe a sua perversão; é assim que ela sente. Em consequência, e
em que pese o observador objetivo poder não concordar, a pessoa
perversa está certa de que foi ela mesma quem criou, foi conivente,
encorajou, disfarçou, manipulou: ela considera sua perversão sua
própria produção magistral. A prática analítica baseia-se na tese de
que tal \textit{insight} pode ser conquistado --- que o paciente
acabará por compreender como acabou acreditando ter desejado sua
própria perversão --- desde que a técnica do analista seja
suficientemente boa.

A este respeito, os conceitos de\idxwinni{} Winnicott\idxself{} sobre o
``\textit{self} verdadeiro'' e o
``\textit{self} falso'' (155) são da maior
ajuda em termos clínicos, e um avanço necessário sobre
``ego'' ``id'' e ``superego'' em considerável parte de nosso
discurso. Mas seus conceitos fazem com que mergulhemos mais
profundamente na questão do livre-arbítrio e do determinismo, pois o
``\textit{self} verdadeiro'' é visto como
aquela parte de nós em que se pode confiar, que não falsifica nosso
conhecimento fundamental --- ele é nossa convicção subjetiva suprema
(a onisciência) --- e o ``falso \textit{self}'' é uma pessoa interna que se recusa a lidar
com a verdade. O ``\textit{self} verdadeiro''
é, portanto, a consciência do superego. Como diferenciei entre
``perversão'' e
``variação'' na primeira o nosso \textit{self} verdadeiro conhece
o seu próprio mal, enquanto na segunda isso não acontece.

Opositores da teoria psicanalítica da perversão tomam uma posição
oposta, qual seja, amoral (a posição é amoral, não os seus\idxlivre[|)]
proponentes). Embora possam discordar entre eles mesmos, eles se unem
ao concordar, como vimos, que o comportamento sexual aberrante não é\idxpervrespo[|)]
produto de conflito (moral, ou seja,\idxrespo[|)] do superego).


\section{O mistério e o papel da hostilidade na perversão}

Deixem-me\idxpervmiste[|(] adicionar\idxmist[|(] outro componente\idxsexuamis[|(] que, em nossa sociedade, é
também fator de frustração, com potencial para trauma na infância: a
mistificação da anatomia, das funções e dos prazeres da sexualidade.
Com suas punições, sua promessa de maravilhas na vida adulta, sua
transmissão dos tabus e das preocupações ligadas à sexualidade por
parte da sociedade (muito frequentemente pelas perturbações culposas,
dissimuladas, dos pais), essa mistificação pode contribuir para a
perversão, caso sejam muito intensas ou bizarras. Pois isso vitimiza as
crianças, atraindo fortemente sua atenção com a insinuação de prazeres
perigosos que, sendo misteriosos, clamam sempre por solução, embora não
exista solução possível para eles. Por exemplo, as diferenças
anatômicas entre os sexos pode promover perversão --- voyeurismo ---\idxvoy[|(]
em sociedades que sexualizam as roupas\idxroup{} e a nudez. A pornografia que
envolve a nudez será onipresente quando uma classe --- em nossa
sociedade, os homens --- é incessantemente informada, de forma aberta
e subliminar, desde a infância, que é proibido olhar, mas que, caso o
fizessem, a visão seria de atordoar os sentidos.

Freud\idxfreudinsti{} deixou bem claro o fato (24, 29, 30) de que a instabilidade
pulsional é resultado da hostilidade, que pode ser de dois
tipos:\footnote{ De dois tipos com os quais estou de acordo; quanto a um
terceiro, que em sua forma final é chamado de ``instinto
de morte'', este é demasiadamente religioso para o meu
gosto.} a saber, aquela que nos é infligida de fora e
a que é gerada de maneira intrapsíquica, como forma de reação. A
maioria dos pesquisadores analíticos examinou mais de perto as
dinâmicas intrapsíquicas da hostilidade ao procurar pela etiologia das
perversões, pois este é o processo analítico tradicional de descoberta.
Descobriu-se que a hostilidade se dividia naquela que é dirigida contra
a própria pessoa\idxhostculp{} (culpa,\idxculpah{} punição) e a que se dirigia para fora (raiva,
vingança). A perversão foi estudada pela análise do perverso.

Considerando que a visão exclusivamente a partir do interior do
neurótico sexual, embora da maior importância, não é suficiente para
contar a história completa relativa às origens da perversão, andei me
interessando também pelas pressões, especialmente na hostilidade, que
eram dirigidas pelos pais à criança,\idxhostrela{} agora perversa. A etiologia da
perversão abre-se, então, à nossa observação: a pessoa perversa é
verdadeiramente incapaz de ver o que seus pais lhe fizeram, quando o
fizeram ou por que. (Será que o estudo da etiologia de todas as
neuroses também não sofreria progressos assim? Esta não é uma sugestão
nova, com certeza, mas é uma pela qual poucos teóricos têm demonstrado
muito entusiasmo.)

Por esta perspectiva, que situa a hostilidade no centro, a perversão
se encontra no significado do ato em que existe ódio e uma necessidade
de machucar, e não de amar, o companheiro. Certamente nós agora nos
complicamos, pois corremos o risco de descobrir que existe muito pouca
coisa, inclusive grande parte do comportamento heterossexual, que
poderia não conter algum toque de perversão. Freud faz alusão a isso em
sua descrição do conflito\idxconfe{} edípico e das armadilhas do desenvolvimento\idxlibid{}
libidinal.

Tais proposições poderão ser testadas com máximo vigor nas
circunstâncias menos perversas: as maiores perversões não são muito
desafiadoras, pois nelas as dinâmicas são demasiadamente visíveis.

Para que possamos ilustrar essas dinâmicas e testar os argumentos
que emergem ao conservarmos o termo
``perversão'' podemos voltar nossa atenção\idxsexuo[|(]
ao olhar sexual,\footnote{ Prefiro usar esse termo preferivelmente a
``voyeurismo'' que já por si só tem uma
conotação perversa.} um dos comportamentos sexuais mais normativos de
nossa sociedade. Nenhum analista discordará do fato de que a ânsia
sexual frenética de olhar, ou seja, o voyeurismo, é uma perversão. Mas
será que podemos chamar o onipresente olhar sexual dos homens, em nossa
sociedade, de perversão? Será que isso não destrói o significado do
termo?

Uma evidência já se estabelece, de cara.\idxpervpredo[|(] Nas sociedades em que a
nudez é comum, ninguém está\idxsadiolha{} vivamente interessado em olhar para a
anatomia que está por ali livremente, à disposição. Por outro lado,
numa sociedade como a nossa, em que certas partes do corpo são
interditas, a curiosidade sexual é atiçada justamente em relação a
essas partes. As sutilezas e as mudanças, em grau, e em termos de
partes proibidas, criam modas na vestimenta, no comportamento, na
fantasia e na pornografia. Em nosso tempo e em nossa cultura, olhar é
um ato bem mais intrincado e estilizado para os homens (o sadismo é o
tema mítico na masculinidade) e ser olhado é mais para as mulheres
(para quem o tema é o masoquismo).

Nosso assunto, portanto, é o mistério,\idxgenetgenimist{} uma qualidade tão importante
para a excitação sexual\idxinfanexcit[|(] que ambos são quase sinônimos. Tal mistério
deriva da infância,\idxsexueinf[|(] e da maneira sinuosa pela qual nossa sociedade
obscurece a descoberta das diferenças anatômicas\idxsexuedif[|(] entre os sexos. Nosso
conhecimento de que a angústia\idxanguinfa[|(] é um elemento essencial do mistério se
confirma, como Freud o demonstrou há muito tempo, no desenvolvimento
edípico\idxconfemeni[|(] e naquelas angústias que estão nele contidas, e que derivam das
diferenças anatômicas.

Mas as crianças de ambos os sexos desenvolvem angústia\idxinfanangu[|(] sobre as
diferenças genitais: por que o diminuído sentido de mistério --- e por
que existe menos perversão --- entre as mulheres? Até certo ponto,
será que este fato pode se explicar pelas restrições impostas aos
meninos, que não podem investigar livremente os corpos das meninas?
Interdição que se funda, reforçando-o, no medo ancestral que os órgãos
genitais femininos suscitam, com sua capacidade reprodutora? Um falo é
perigoso, mas não é misterioso; o perigo do útero advém do silêncio, do
secreto, e do crescimento na escuridão --- que é mistério. Mas, por
detrás desses fatores, podem ser encontradas questões que, surgindo nos
primeiros meses de vida, estarão em atividade ainda anos depois,
enterradas nas profundezas de nossas identidades.

A versão freudiana do conflito edípico\idxfreudedipo[|(] é enigmática em muitos
sentidos (143), um dos quais vem agora. Ele acreditava que a linha
natural de desenvolvimento fosse a do menino que, no inconsciente da
humanidade e nas hierarquias das sociedades, presumivelmente é
heterossexual desde o nascimento, com genitais e status superiores. Se
assim for, por que as perversões são mais frequentes, e muitas vezes de
uma bizarrice feroz, nos homens? Poderemos encontrar uma pista no
próprio mistério.

Examinemos a identificação.\idxidentconf{} Talvez nenhum outro mecanismo mental
resulte no desenvolvimento de uma estrutura de caráter tão
egossintônica, tão inalterável. Portanto, a identificação é posse:
outra pessoa, ou pelo menos algum aspecto dela, torna-se próprio. Este
é o último estado que provoca nossa sensação de mistério. (Meu corpo, e
aqueles iguais ao meu, não são misteriosos; deste modo, por exemplo,
forma-se um dos vínculos em alguns homossexuais, para quem o mistério
--- o da existência de um outro sexo --- é assustador demais para que
o suportem.) O primeiro objeto que se apreende, pelo processo de
identificação, é a própria mãe, a pessoa cuja psique e cujo corpo são
iguais ao da menininha, mas tão diferentes dos do menino. Ele precisa
aprender essas diferenças e, em seu devido tempo, terá que aceitá-las.
Então, para se tornar masculino, precisará se separar do corpo feminino
de sua mãe no mundo exterior, e em seu mundo interno terá que se
separar da sua própria, já formada, primeira identificação com a
\emph{femininidade} e com a feminilidade (61, 135). Essa grande tarefa                   %%B. cf. orig.%%
muitas vezes não é completada --- e isto, eu sugiro muito timidamente,
é a mola propulsora da perversão. (Analisaremos esta hipótese em
maior detalhe no capítulo 8.) Nos homens, a perversão pode estar na
base de um\idxpervtrans{} transtorno de gênero\idxiden{} (ou seja, um transtorno no
desenvolvimento da masculinidade e da feminilidade) e que se constitui
a partir de uma tríade\idxhosttria{} de hostilidade: \textit{raiva} por ter que abrir
mão de sua antiga experiência oceânica e identificação com a mãe,\idxsexuo[|)]
\textit{medo}\idxhomosmedo{} de não ser capaz de escapar à\idxvoy[|)] sua órbita e uma\idxpervpredo[|)]
necessidade de \textit{vingança}, por ela tê-lo colocado nessa\idxfreudedipo[|)]
situação.

Não é novidade que o mistério é excitante, e a maioria dos analistas
provavelmente estão cientes de que ele é um elemento em todas as\idxconfemeni[|)]
perversões. Como é o seu funcionamento?

\begin{enumerate}
	
\item Ao completar um ano de vida, aproximadamente, a criança começa a
acreditar que pertence a um sexo ou ao outro.

\item Então, as diferenças anatômicas entre os sexos são descobertas;
atitudes expressas, no âmbito familiar e na sociedade, informam ao
garoto e à menina, de maneiras diferentes, que este é um assunto da
mais crucial importância (24).

\item O desejo de satisfazer a si mesmo quanto à natureza, e
especialmente quanto à aparência, dessas diferenças, é grande, em
virtude das implicações de perigo, para o próprio senso de
masculinidade ou feminilidade, que elas engendram. Na infância, os
genitais são a única maneira através da qual a anatomia comunica as
diferenças cruciais do sexo a que se pertence. (O comprimento do cabelo
pode comunicar essas diferenças, e até onde o faz, cortá-lo é encarado
como uma ameaça de castração.\idxcastaconf[|(] Os seios também o fazem, porém para outra
categoria de seres: os adultos.) Mas a necessidade de explorar para
descobrir (ou seja, para pôr fim ao temor de que as diferenças de sexo
existem, ou de que são perigosas) é, em nossa sociedade, mais frustrada
nos meninos do que nas meninas. Para ser mais preciso,\idxinfanexib{} podemos dizer
tanto aos meninos quanto às meninas que é feio examinar os genitais do
sexo oposto, ou exibir assim os próprios genitais.\idxexibiinfa{} As mensagens são
sutilmente diferentes, entretanto. O menino aprende que ninguém fica
surpreso quando ele age assim; se ele for ser considerado masculino, em
nossa cultura, espera-se mesmo que ele seja um tanto mau, arrogante, um
tanto sádico. A menina, por outro lado, aprende a já contar com a
tentativa dos meninos, aprendendo também que dela, o que se espera, é
que resista. Essas atitudes, inculcadas em cada sexo, estão refletidas
em alguns automatismos, como cruzar devidamente as pernas, ou o
``hábito'' de puxar as saias para baixo, por
parte das meninas, sempre que homens estiverem por perto.\idxmist[|)] Assim, o
desejo de olhar\idxvoy{} e a promessa de que valerá a pena são elevadas pelo\idxsexueinf[|)]
próprio comportamento usado para impedir isso. Quanto mais impedimento,\idxsexuedif[|)]
maior a supervalorização e a distorção.\idxinfancurio{} A curiosidade\idxgenetgenicuri{} fica muito
atiçada.

\item Neste estágio, a importância fálica,\idxfalo[|(] devida ao aumento da
sensação erótica, tanto no pênis como no clitóris e, simultaneamente,
aos desejos e perigos edípicos, torna essa curiosidade ainda mais
excitante e frustrante.\idxinfanexcit[|)] Deste campo fértil cresce a fantasia do falo
feminino, uma tentativa da criança de explicar um mistério que só faz
com que ele aumente. (``Em todas as perversões, a negação
da castração,\idxcastaconf[|)] dramatizada ou ritualizada, é atuada através de um
reviver regressivo da fantasia do falo materno ou feminino [1, p.\,16]. A angústia da\idxfeticangu{} castração,\idxinfanangu[|)] com sua especificidade ao longo das diferentes fases, até a fase fálica,\idxfasef{} desempenham um papel central na\idxfalo[|)]
perversão''\idxanguinfa[|)] [1, p.\,28].)

\item Uma frustração crônica, intensa --- uma essência de mistério
--- somada às ameaças, já internalizadas, ao se tentar buscar a
autossatisfação, funcionam como um trauma\idxtrauma{} cumulativo. Mas reduzir a
tensão do desejo ``instintivo'' com olhares,
cuja finalidade é sexual, é arriscado. Assim, o mistério aumenta ---
embora, até aqui, sem que haja perversão pois, até este ponto, não
houve gratificação. A perversão, como observamos, é constituída por
ambos os fatores, perigo e gratificação. O problema que se coloca
diante da criança é o de como evitar o perigo (a punição) e de como
conseguir o prazer (a recompensa) que emerge a partir de três
atividades: decréscimo da frustração, conseguir fazer o que é proibido
e ter seu corpo eroticamente estimulado.

\item Uma solução, inadequada, mas pelo menos parcial, pode ser
alcançada pela criação de uma estrutura psíquica neurótica\idxpervneuro[|(] (instável
quando expressa como sintomas\idxsinto{} neuróticos, mais estável na estrutura do
caráter); ``as neuroses são, por assim dizer, o negativo
da perversão'' (24). Algo assim. O que nos parece mais
plausível é que, em vez de ser uma categoria de reação diferente da
neurose, a perversão é uma neurose erótica. Com esse aforismo, Freud
estava observando que, nas neuroses, as fantasias sexualmente perversas
estão disfarçadas, ocultas nos sintomas neuróticos, enquanto que as
perversões expressam seu desejo de forma inequívoca. Entretanto, outros
autores têm, desde então, demonstrado que isso não é exato;\footnote{ Anna 
Freud\idxannaf[|nn] (22, p.\,80) nos dá
a bem conhecida fórmula que\idxfreudperve[|nn] reveste a formação da 
neurose\idxfreudneuro[|nn] em geral --- e ela
certamente reveste também a maioria das teses analíticas sobre a
perversão, incluindo a minha: ``conflito, seguido por
regressão; objetivos regressivos suscitando ansiedade; ansiedade,
repelida por mecanismos defensivos; solução de conflito por intermédio
de acordo de conciliação; formação do sintoma.''\idxsinto[|nn] E, no
entanto, os analistas continuam afirmando que a perversão é uma
categoria diferente da neurose;\idxneuro[|nn] eles o fazem, aparentemente em virtude
do manifesto prazer que marca a perversão e, talvez, em virtude da
antiga crença de Freud, de que a perversão é, simplesmente, a
transferência para a vida adulta de uma parte, fixada, da sexualidade\idxsexui[|nn]
infantil.\idxinfansexua[|nn] Gillespie\idxgille[|nn] (44; 45, pp. 129--131) revê esta questão e demonstra
o quanto Freud, que foi o primeiro a colocar a neurose e a perversão em
polos opostos (24), oportunamente, (30) demonstrou que
``os dois extremos, de fato, encontram-se no complexo de
Édipo.'' Gillespie enfatiza especialmente a contribuição
de H.~Sachs\idxsachs[|nn] (1923), que descreveu a perversão como sendo produto de uma
distorção devida ao conflito edípico, e não somente como um componente
não modificado da sexualidade infantil. Gillespie conclui que a
diferença essencial entre neurose e perversão consiste apenas em que
``na neurose, a fantasia recalcada irrompe para a
expressão consciente somente sob a forma de um sintoma indesejado para
o ego, tipicamente acompanhado por sofrimento neurótico, enquanto, na
perversão, a fantasia permanece consciente, sendo bem-vinda pelo ego e
agradável. A diferença parece situar-se na atitude e no posicionamento
egoico, ou na sinalização emocional negativa, mais do que uma diferença
em termos de conteúdo.'' Se essa for a única diferença e,
especialmente, uma vez que a fantasia está tão recalcada na perversão
quanto na neurose --- sendo que basta o irromper de um fragmento para
formar o cenário consciente da perversão --- não deveríamos abrir mão
da dicotomia artificial, teórica? Tudo quanto perderíamos seria a frase
espirituosa de Freud, de que tanto a neurose quanto a perversão são
negativas. Em um trabalho anterior (43), Gillespie diferenciou a
neurose da perversão pelo fato de a última resultar de clivagem,\idxcliv[|nn] que é
considerada um mecanismo mais primitivo. Em virtude de os kleinianos\idxklein[|nn]
enfatizarem ser a clivagem uma parte do desenvolvimento de todas as
pessoas, e por ser descrita como sendo essencial em todos os pacientes
a quem se referem, será útil utilizar a clivagem como maneira de
diferenciar entre a neurose e a perversão? Não deveríamos nos
agarrar a uma crença clínica que é facilmente refutável pela
observação: simplesmente não é verdade que a perversão adulta seja a
persistência, inalterada, de um fragmento do comportamento sexual
infantil (24; 19 [por exemplo, p.\,358]; 46, p.\,181) uma ideia que Freud
declarou em 1905 mas que já não aceitava em 1919. Glover\idxglove[|nn] se posiciona
de maneira similar: ``Eu sugeriria que ele coloca o
problema dos desvios numa perspectiva mais satisfatória se os
considerarmos como equivalentes das formações de sintomas que, enquanto
tais, podem ser ordenados em encadeamentos de desenvolvimento, de
acordo com a prioridade histórica dos estágios libidinais e sádicos e
com a quantidade de agressão que é liberada pelas frustrações em cada
uma dessas fases'' (47, p.\,156). É interessante notar que
isso contradiz sua posição que diz que ``as perversões
sexuais da adolescência e da vida adulta [\ldots{}] apesar de mais sistematizadas 
do que os componentes infantis da sexualidade, são da mesma natureza. A 
respeito\idxinfanexib[|nn] do [exibicionismo perverso] é preciso que se diga apenas 
que ele não difere, em nenhum aspecto descritivo, do exibicionismo\idxexibi[|nn] que 
é praticado por crianças pequenas''\idxexibiinfa[|nn] (46,
p. 181). O pênis murcho de uma criança é igual ao pênis ereto de um adulto? As 
fantasias infantis são iguais às dos adultos? Anos atrás, Straus\idxstraus[|nn] 
acreditava (como está descrito em 7, p.21) que o que se
acrescentava ao adulto perverso, em relação à sua atividade 
sexual\idxinfansexua[|nn] infantil, era ``decadência'', uma palavra totalmente 
impregnada de hostilidade. Esses últimos autores indicam
também que o hedonismo infantil não pode ser equiparado com a luxúria do adulto. 
Um bebê que se compraz em brincar com as fezes simplesmente não está tendo a mesma 
experiência do que um adulto coprófilo, sobre
cuja boca se acocora uma prostituta que nela defeca.\idxcopro[|nn] Um último 
murmúrio, desesperançado: qual é a importância prática de se classificar as 
perversões como neuroses ou como outra coisa qualquer?} as dinâmicas da 
perversão e da neurose diferem, realmente, apenas
quanto ao aspecto de que a primeira conduz ao prazer e a segunda ao sofrimento, 
ambos conscientes.

Se a solução vai ou não na direção da sintomatologia neurótica, sem
um contorno nítido a acompanhá-la, o prazer erótico subjetivo depende,
creio, da natureza exata do sistema sutil, complicado, de recompensa e
punição que cada família, em geral sobretudo a mãe, desenvolve. Nas
perversões --- as neuroses eróticas --- a sensação de mistério e de
perigo se acentua, pois a criança foi traumatizada ou superestimulada
exatamente no ponto do mistério: os genitais, ou o desejo de
investigá-los. Fenichel\idxfenic{} sugere algo assim ao dizer que
``indivíduos em quem a angústia da castração\idxcasta{} foi provocada
muito subitamente, e de modo intenso, são candidatos a um posterior
desenvolvimento de fetichismo'' (19, p.\,342); ele indica
assim, tal qual Freud\idxfreudperve{} fez antes, que a perversão pode resultar como uma
``cura'' para a angústia provocada pela
conscientização da existência da possibilidade de que se venha a perder\idxpervneuro[|)]
o próprio sexo. (Ver também 1, 53---55, 57, 58; 137, capítulo 19.)

\item Eu disse que a perversão é constituída por perigo, com seus
dolorosos afetos, mais prazer, alguns componentes dos quais são alívio
e sensação erótica. Falta, ainda, entretanto, um fragmento explicativo.
Quando se está angustiado em relação ao mistério, e frustrado e
irritado com as tentativas de adivinhar um fim para ele, o que é que
converte esses dolorosos afetos em prazer?
\end{enumerate}


De algum modo, o perigo precisa ser desfeito. O medo não pode, por
si só, produzir prazer,\idxpraz[|(] nem a raiva. Algo de novo precisa ser
acrescentado, que libere o corpo para a resposta erótica. A
psicofisiologia do medo e da raiva precisa ser transferida para novos
canais, para que a excitação possa alterar sua qualidade e seu curso,
dos músculos e das entranhas, para os genitais.

O final para o mistério\idxmistabo[|(] (com sua angústia de castração somada ao
medo, mais primitivo, de aniquilação da identidade) advém pela criação
do ato perverso totalmente desabrochado, consciente (ou pela fantasia
do ato). Nele,\idxnega[|(] o mistério se resolve por artifícios, tais como a mulher
fálica, a negação, a clivagem, o evitamento, a fetichização, a
idealização, uvas que ``estavam verdes'' ---
como na fábula ---, adoração fálica, teorias sobre a superioridade
masculina\idxmascsu{} e assim por diante --- uma vasta seleção de fantasias e
artifícios, todos servindo para anunciar a inexistência do mistério.
Eles o fazem quer pela negação da diferença entre os sexos,\idxpervnegac{} quer
salientando a superioridade de seu próprio equipamento, o que significa
que a diferença não é ameaça. (Greenacre,\idxgrena{} por exemplo, seguindo Freud,\idxfreud{}
afirma que o fetiche ``serve como uma ponte que ao mesmo
tempo negaria e afirmaria as diferenças sexuais'' [59, p.
150].) Assim, acreditando em mulheres com pênis, nega-se a existência
de todo um conjunto de humanos que são castrados; ou, voltando a
própria fascinação para um fetiche, como uma roupa\idxroup{} de mulher,
consegue-se continuar afirmando, com a equação simbólica fetiche =
pênis, que a mulher não é castrada; ou, acreditando-se que os homens
são o melhor sexo em todas as atividades que importam, um homem pode
dizer que não se importa pelo fato de as mulheres serem destituídas de
pênis uma vez que ele --- felizmente --- não é mulher. Mas certamente
o mistério não foi solucionado; ele permanece ali, inconsciente,
pronto.\idxnega[|)] Cada episódio de excitação sexual alicia o retorno à
superfície da questão e das fantasias\idxfanta[|(] que corporificam o mistério. A
angústia resultante pode agora ser reduzida somente pelo ato perverso
que, entretanto, em seu desempenho, ou em sua condição fantasiada,
levanta outra vez as questões do mistério. E, mais uma vez, o mistério
tem que ser solucionado.\idxpervdeses{} Não é de surpreender que as pessoas perversas
sintam-se tão perseguidas por suas necessidades sexuais.\footnote{ Talvez
aqui entre em jogo também um fator (condicionado)\idxcond[|nn]
intensificador, ligado às principais zonas de prazer do sistema nervoso
central (113), fator esse que se soma ao aspecto desesperado,\idxpervdeses[|nn] coativo,
da busca por prazer\idxpraz[|nn] que caracteriza as perversões e as toxicomanias.
Este fato não contradiz a explicação analítica, que enfatiza a redução
da ansiedade. Os dois aspectos se potencializam mutuamente.} A mais
estável solução possível, face às reais ameaças feitas pelos pais e
pela sociedade, e face à nova forma que esses perigos adquirem quando
incorporados ao superego, é a perversão. E, indissoluvelmente fixada em
seu lugar pela experiência do prazer físico, ela é excessivamente
estável e, portanto, em geral não é passível\idxpraz[|)] de ser alterada,\idxmistabo[|)] seja
pelas experiências de vida do indivíduo, seja pela terapia.

Como primeira linha de defesa, as crianças fantasiam situações que
fazem\idxtraumareve[|(] reverter\idxfantareve[|(] os traumas e as frustrações (é assim com os contos de
fada, com os brinquedos, com os filmes e devaneios gratificantes). Com
o tempo, as ações, que antes eram somente fantasiadas, são postas em
ação, com modificações e disfarces, em situações reais, com pessoas que
não se consideram simplesmente atores que seguem um roteiro. As pessoas
perversas, entretanto, lidam com seus parceiros como se eles não fossem
pessoas reais,\idxdesu{} mas fantoches manipulados no palco em que a perversão é
atuada. No ato perverso,\idxtraumaobje{} a pessoa revive \textit{ad infinitum} a
situação traumática ou frustrante que deflagrou o processo, só que
agora o resultado é maravilhoso, em vez de horrendo, porque não
somente se escapa da ameaça, como finalmente se obtém uma gratificação
sensual imensa, atrelada à consumação do ato. A história completa,
construída com precisão por cada uma das pessoas, para se encaixar
exatamente às suas experiências particulares dolorosas, jaz ali,
oculta, mas acessível,\idxtraumareve[|)] para que possa ser estudada, na fantasia sexual
da perversão.

Existem duas hipóteses que precisarão ser testadas
\textit{a posteriori}, para as quais ainda não se dispõe de dados
comprobatórios e que complementam esta parte da explicação. A
primeira, é que o trauma ou a frustração da infância tenha sido
dirigida especificamente ao aparato sexual anatômico, ou às suas
funções; ou à masculinidade ou feminilidade da pessoa. Caso o alvo
tenha sido outras partes, ou outras funções do corpo ou da psique não
relacionadas à sexualidade, o resultado deveria ser uma das neuroses
não eróticas (por exemplo, a personalidade compulsiva, quando o
controle, especialmente o excretório, é forçado muito cedo à criança,
de maneira demasiadamente dura, ou por tempo excessivo.)

A segunda hipótese é a de que a excitação sexual seja mais passível
de acontecer no momento em que a realidade adulta apresente semelhanças
com o trauma ou com a frustração\idxsexueang{} da infância.\idxangu{} Isto implica que uma
ansiedade maior se faz sentir durante o ato sexual perverso do que
aquela que está presente numa sexualidade menos perversa. Esta
ansiedade --- a antecipação do perigo --- eu acredito que seja
vivenciada como excitação, palavra usada não tanto para descrever
sensações voluptuosas, e sim como uma rápida vibração entre o medo do
trauma e a esperança de triunfar.

A perversão, no entanto (ou seja, a fantasia, que é mais uma vez
criada) faz mais do que solucionar o mistério.\idxfanta[|)] O tema central, que
permite este avanço em direção ao prazer,\idxvinga{} é a vingança.\footnote{ ``A sexualidade\idxsexuateo[|nn] da maioria dos seres humanos
contém um elemento de agressividade\idxagres[|nn] --- um desejo de
subjugar.'' (24, p.\,157). ``O fetiche [\ldots{}]
contém uma raiva congelada, nascida do pânico da
castração'' (59, p.\,162).\idxfreudsexua[|nn] Ao escrever sobre o sadismo,\idxsadiangu[|nn]
embora não sobre a perversão em geral, Fenichel\idxfenic[|nn] (19, p.\,354) diz:
``Qualquer fator que tenda a aumentar o poder ou o
prestígio do sujeito poderá ser usado como uma garantia contra a
angústia.\idxangu[|nn] Aquilo que poderia acontecer, de forma passiva, ao sujeito, é
ativamente executado por ele, como uma antecipação do ataque, sobre os
outros. \ldots{} A ideia
``Antes que eu possa desfrutar da sexualidade, é
preciso que eu convença a mim mesmo de que sou
poderoso'' não é, com certeza, ainda idêntica a:
``Eu obtenho prazer\idxpraz[|nn] sexual torturando as
pessoas''; contudo, este é o ponto de partida para um
desenvolvimento sádico. O tipo
``ameaçador'' de exibicionista,\idxexibi[|nn] o
cortador de tranças e o homem que mostra retratos pornográficos para
seu ``inocente'' parceiro, desfruta da
impotência deste parceiro porque ela significa que ele não precisa
temê-lo, tornando assim possível o prazer que, de outro modo, ficaria
bloqueado pelo medo. Sádicos deste tipo, ao ameaçar seus objetos,
demonstram estarem preocupados com a ideia de que eles podem, eles
mesmos, ser ameaçados.'' Boss\idxboss[|nn] (7, p.\,21) cita Kunz\idxkunz[|nn] (H. Kunz,
``Zur Theorie der Perversion'' Monatsschr. f. Psychiatrie, 105:24, 1942):
``A inclusão de impulsos destrutivos nas atividades sexuais não podem
ser encaradas como sendo específicas apenas do sadismo, é preciso que eles
sejam típicos, também, de todas as outras formas de perversões''; mas
Boss se queixa também, e com razão (p. 22) de que ``nenhuma explicação
é fornecida para explicar como ``o desmembramento e a deformação destrutivos'',
``ações que atentam contra a vida'' ou ``a mais evidente destruição do
significado erótico do amor'' (von Gebstattel) podem ser o conteúdo
sexualmente excitante da ação perversa.'' É a isto que estou tentando responder:
de onde vem o prazer erótico?} Ela reverte as posições dos
atores no drama e, deste modo, reverte também seus afetos. A pessoa
passa de vítima a vencedor, de objeto passivo da hostilidade e do poder
alheios a diretor, comandante; seu algoz passou a ser sua vítima. Com
este mecanismo, a criança imagina a si mesma no lugar de um dos pais, o
impotente que se torna poderoso. Não se teme mais o mistério, ou a
consciência, ou o mundo exterior.\idxfantareve[|)] A perversão é mais uma obra-prima do
intelecto humano.\footnote{ Assim como em outras neuroses,\idxneuro[|nn] ela serve
também a propósitos evolutivos, ao fornecer um mecanismo através do qual
as espécies podem sobreviver e se reproduzir, apesar dos problemas que ele
(nosso desenvolvimento cerebral) nos trouxe ao nos proporcionar a civilização
(cf.\,24, p.\,156). Ver, também, capítulo 12.}
A vida pode prosseguir,\idxtrauma[|(] a criança pode ir adiante em seu desenvolvimento,
a autoestima e a esperança por gratificação são preservadas, e o triunfo
é colocado, oportunamente (quando as ereções e o orgasmo se tornam possíveis)
fora de perigo, desde que o ritual (a vigilância eterna) seja mantido
e se torne autônomo.

Na excitação sexual, que agora se tornou possível, a consciência
subliminar da recompensa e da punição é consequente ao desejo sexual.
É desse modo que, ao ficar excitada, a pessoa se move entre a sensação\idxperigo{}
de perigo\idxpervexpos{} e a expectativa de escapar dele --- o que a conduzirá
diretamente à gratificação sexual. O risco foi assumido e superado. O\idxpervorgas{}
orgasmo,\idxorgas{} então, não é meramente um alívio da tensão, ou até mesmo só
uma ejaculação: é uma explosão de gozo, uma explosão megalomaníaca de
libertação da angústia (como acontece quando se solta uma gargalhada ao
final de uma piada lindamente contada, em que a construção da intenção
hostil é subitamente fraturada, resultando em gargalhadas [25]).
Entretanto, a linha divisória entre a explosão do triunfo e a\idxtrauma[|)]
impotência é muito frágil. O tipo errado de incursão em risco (como por
exemplo, aquela que ameaça revelar suas origens) enfraquece a
excitação. Não admira que qualquer alteração no cerimonial seja capaz
de reduzir a excitação sexual.

Para tornar corriqueira a questão, eis uma prostituta que tece
comentários sobre o mistério\idxprostenf[|(] e o\idxpervenfad[|(] enfado,\idxenfa{} depois de um ano na profissão:

``O que me incomoda no que faço é que ver os caras, a
visão dos corpos dos homens, acaba por se tornar entediante. Parte da
excitação era ver os genitais do cara, ou senti-los. Mas agora, eu já
não tenho a mesma reação que costumava ter antes, quando a coisa toda
ainda estava meio que envolvida em mistério para mim. Não sei se é
somente em função do trabalho --- ver os caras nus, o tempo todo, e me
envolver em relações sexuais com eles. Porque algumas vezes eu gozo,
até mesmo com meus clientes. Porque vários deles são realmente bem
bonitos, e muitos deles me satisfazem. Em certo sentido, eu quase que
uso essa coisa do meu trabalho como uma desculpa para ter relações
sexuais. A hora que fica bom é quando você fica tão completamente
envolvida em sua própria satisfação; então, você se torna completamente
egoísta.

Você tem todos esses homens, e vê todos esses homens; no começo,
você realmente fica excitada, simplesmente pelo fato de que são homens.
Só isso, por si só, te excita e te faz gozar, simplesmente pela
sensação de estar com alguém. Mas, então, eles começam a te sufocar. Eu
experimentei todo tipo de viagem sexual que existe, de masoquistas a
sádicos; tudo, até mulheres. Nem se trata da questão de serem homens.
Mas agora, para conseguir um orgasmo, para mim, é quase como ter que
batalhar por ele; ele realmente já não acontece com facilidade, pela
simples excitação que a coisa toda te dá. Muitas vezes, eu fico mais
excitada quando estou vestida --- e o cliente também --- e a gente
fica só se agarrando, fica só nas preliminares. Aí, eu fico mesmo
excitada. Mas depois, quando ele tira a roupa e a gente vai pra cama, e
de repente a coisa toda se torna\ldots{} --- algumas vezes, chega a ser
engraçado. Algumas vezes me dá vontade de rir, porque é como um jogo,
uma piada\ldots{} eu fico lá e o cara fica naquilo de \textit{bang, bang}
--- sabe né? --- \textit{bang, bang.} Com um monte de caras, talvez
eu tenha até mesmo sido bem injusta: ``Nós não vamos
transformar isto em atletismo, vamos?''

Acredito que não seja um problema só meu. Acho que é dos homens
também. O sexo se tornou tão acessível para os homens, hoje em dia, que
eles começaram a se desinteressar dele. Para eles, atualmente, é só
uma questão de ficar em cima da mulher e gozar.


A excitação causada pelo olhar\idxmistolh{} eivado de sexualidade\idxinfanolhar{} tem,\idxsexuo{} até aqui,
a seguinte explicação:\idxsadivoye[|(] quando a inevitável curiosidade sobre as
diferenças entre os sexos surge na criancinha,\idxvoyrai{} o desejo de olhar se
torna intenso, insaciável e permanente, uma vez que as partes do corpo
para as quais deveríamos olhar são proibidas --- ao mesmo tempo em que
são consideradas partes desejáveis pelos pais;\idxvoysad[|(] com sua proibição, os
pais transmitem aos filhos a noção de que o prazer perigoso é possível.
Portanto, em nossa sociedade, em que a anatomia da fêmea é mais
proibida, mas muito sedutora,\idxpervenfad[|)] os homens tenderão a supervalorizar e a\idxprostenf[|)]
ficar excitados por olhar, e as mulheres, por serem olhadas.

Agora, uma das principais maneiras para que o olhar seja sexualmente
excitante --- caso a tese esteja correta --- para um homem, é ele
acreditar que está agindo de maneira forçada, sádica, impondo-se sobre
uma mulher que não o quer. Ele está fazendo aquilo --- assim prossegue
sua fantasia --- que ela decididamente não quer. Se ele conseguir, ele
a derrota; ele obtém sua vingança pela frustração do passado.
Finalmente, é a vez da mulher sofrer; a excitação, na pornografia,
exige que uma vítima seja retratada --- embora quanto mais normativa a
perversão, menos óbvia a descrição (por exemplo, o retrato de uma
mulher agradavelmente nua oculta essa dinâmica mais do que o retrato de
uma mulher sendo torturada). Inerente ao olhar sexual existe um desejo
de degradar as mulheres, ao qual as mulheres poderão responder com seu
próprio ataque (roupas\idxroup{} ``sedutoras'',
posturas ``provocativas''); as regras do
jogo, hoje em dia, em nossa sociedade, exigem que isso tome a
\textit{forma} (mas não a substância) da passividade.\footnote{ Talvez
aquilo que nós frequentemente consideramos como sexualmente atraente,
em certas mulheres provocativas,\idxroup[|nn] nada mais seja do que uma produção
compacta do sadomasoquismo, do exibicionismo,\idxexibiatra[|nn] uma insidiosa exibição de
vulnerabilidade sexual mesclada a um ataque erótico através da postura,
das expressões faciais e da nudez parcial. Quais são as fantasias da
garota que posa para fotos em revistas que lhe expõem a nudez? Uma
compreensão mais completa da perversão poderá ser obtida a partir da
análise dos parceiros que, voluntariamente, se submetem ao indivíduo
perverso.} Socarides\idxsocar{} (132) observou o quanto ``é
frequente que impulsos sádicos se encontrem vinculados à escopofilia''. O
indivíduo quer ver para poder destruir com o olhar; ou para obter
garantias de que o objeto ainda não foi destruído; ou, ainda, o próprio
olhar é inconscientemente encarado como um substituto para a
destruição. ``Eu não destruí, só\idxsadivoye[|)]
olhei'' (Fenichel,\idxfenic{} 1945).'' (A inveja do
sexo oposto que está encerrada aí --- na pessoa que olha e na que é
olhada --- não será discutida aqui.)

Não precisamos usar aquelas perversões óbvias, tais como o estupro,
o exibicionismo, o sadismo ou a homossexualidade como formas de
confirmação. Podemos, mais uma vez, voltarmo-nos para o trivial. Numa
sala de visitas, uma mulher investe com um imenso\idxvoysad[|)] valor\footnote{ Ou
age para os outros, e talvez para si mesma, como se investisse esse
grande valor. Sob certas\idxsadivoye[|nn] circunstâncias,\idxvoysad[|nn] quando ela intuitivamente
percebe essas dinâmicas de hostilidade num homem que a observa, sua
própria exibição vai excitá-la, pois ela também está empenhada em obter
vingança e triunfo.} a privacidade de cada centímetro de coxa\footnote{ Antigamente
isso acontecia em relação aos tornozelos; as
dinâmicas são mais permanentes do que as fronteiras.} que
possa ser exibido além do que lhe é permitido. Porém, na praia, a visão
anteriormente contestada não passa de pele, simplesmente porque o homem
sabe que, lá, ela não se importa. Do mesmo modo, uma mulher estranha é
excitante, enquanto que, para um grande número de homens, o que é
familiar é entediante. A visão pela qual ele estaria disposto a
sacrificar tanto, rapidamente se torna desinteressante, quando o homem
percebe que, para a mulher,\idxsexuo{} pouco se lhe dá, se ele a olha ou não. Para
consertar esta falha, baseada na psicodinâmica, as mulheres recorrem à
moda; o desenho da moda informa aos homens que ainda existe, realmente,
um mistério, que só poderá ser penetrado se ele conseguir vencer a
resistência. A moda atende à fantasia masculina,\idxinfanolhar{} de que ele poderia
tomar à força aquilo\idxmistolh{} que não lhe seria dado com facilidade.

No indivíduo, esse mecanismo é perverso, ou seja, neurótico; na
sociedade, é normativo, uma vez que a frustração é quase universalmente
aplicada. (Tocaremos em breve no problema que envolve o que o normal e
o que é o normativo.)

Neste material, quero apenas destacar essa perspectiva da frustração
pelos pais, e como ela resulta em raiva. As coisas\idxroup{} supersexualizadas
são, precisamente, determinadas pelos pais; eles o fazem através desse
processo de impor frustração (nervosamente chamando a atenção para seu
próprio prazer secreto), o que, de outro modo, nada mais seria que ---\idxsexuamis[|)]
caso eles se conservassem alheios a isso, como algumas sociedades\idxpervmiste[|)]
fazem --- apenas uma experiência ou atitude levemente erótica.



\section{Perversão e normalidade}

A discussão em torno\idxpervnorma[|(] de se usar ou não os termos
``perversão'' ou ``variação''\idxvaria{} e
``normal''\idxnorma[|(] ou ``normativo'' creio que pode ser abordada
mais uma vez aqui, desde que sejamos cuidadosos.
``Perversão'' depende de uma conotação de
anormalidade; embora eu tenha descrito um mecanismo\idxmecap[|(] de perversão ---
ou será que poderíamos dizer ``\emph{pervértico}'',
um neologismo reminiscente de ``neurótico''?
--- que talvez seja utilizado por todos os humanos. De tal forma que
retornamos aqui com uma questão que tem estado na psicanálise por
décadas: assim como podemos perguntar se existe alguém que não seja
neurótico, alguns de nós perguntam se existe alguém que não seja
perverso. Obviamente, as respostas envolverão graus, em vez de serem
um sim ou um não absolutos.

Deveria tornar aqui manifesto que o termo
``perversão'' foi usado de duas maneiras
diferentes neste capítulo. Uma delas significando um diagnóstico, um
estado da personalidade em que a fantasia sexual motiva a maior parte
do comportamento da pessoa. A outra, rotula um mecanismo. Assim como
uma neurose é diferente de um mecanismo neurótico, do mesmo modo uma
perversão é diferente de um mecanismo perverso. Enquanto ambos servem
para preservar a gratificação sexual contra o trauma de infância, no
primeiro caso (perversão) esse trauma era um ataque, no segundo (o
mecanismo) uma condição civilizatória. De uma maneira ou de outra, dado
que o impulso sexual original necessita ser frustrado, disfarçado e
reinventado, e todo o processo perpetuado, uma vez que a angústia e o
arriscar-se, a violência e a vingança estão ocultos na sintomatologia,
precisamos usar uma palavra que conote essa intensa tensão dinâmica.
``Variação''\idxvaria{} simplesmente não o faz,
especialmente porque é a palavra de preferência daqueles que negam tais
dinâmicas. Ainda assim, para a descrição de um mecanismo onipresente,
``perversão'' é demasiado forte, ela não
consegue se livrar de uma mancha moral. A própria normatividade exige
que se dê importância ao significado das palavras. A este respeito, eis
aqui uma observação de Freud:\idxfreudperve{} ``Nenhuma pessoa saudável,
ao que parece, consegue deixar de somar alguma coisa que pode ser
chamada de `perversa' ao objetivo
sexual normal; e a universalidade dessa descoberta é, em si mesma,
suficiente para mostrar quão inapropriado é o uso da palavra
`perversão' como termo de
reprovação'' (24, p.\,160).

Devereux\idxdever{} nos dá uma importante assistência, ao ressaltar que as
preliminares\idxpreli{} (com seu uso de mecanismos perversos) servem para aumentar
a tensão e elevar o envolvimento entre os parceiros, ao passo que a
própria perversão objetiva o alívio da tensão,\idxdesu{} ignorando a
individualidade do parceiro. Uma crítica à posição de Devereux diz que
``uma relação sexual em que o comportamento é normal, mas
a relação objetal\idxpervrelac{} é defeituosa, é essencialmente pervertida''. Se se
tomar essa definição para abranger a ampla maioria das relações sexuais
humanas, e colocá-las sob a categoria de relações pervertidas, Devereux
sustenta que, então, assim seja. Pode ser lamentável, ele insiste, mas

\begin{quote}
apenas uma fração infinitesimal da humanidade é capaz
de se comportar e vivenciar a sexualidade, até mesmo ocasionalmente, de
maneira madura, em conformidade com suas características
genitais (14).
\end{quote}

O termo ``perversão'', creio, é
necessário --- em que pese o seu significado tradicional pejorativo e
sua definição dicionarizada --- para certos transtornos de caráter em
que as dinâmicas de hostilidade\idxhost{} forçam a pessoa a atos sexuais
aberrantes. Mas é falho, em termos de lógica, e clinicamente inexato
dizer que, independentemente do grau em que um mecanismo é usado, a
pessoa a usá-lo sofre de um transtorno de caráter, ou que aqueles cujas
práticas decididamente não sejam de nosso (dos definidores) gosto são
perversos, sem qualquer vestígio do mecanismo. Talvez possamos pensar
no mecanismo da perversão (como todos os mecanismos neuróticos) como
análogo a mecanismos fisiológicos tais como a febre. O grau importa; a
qualidade e a quantidade de outros sinais e sintomas que acompanham a
febre são importantes, assim como a duração, a evolução clínica, as
variações na etiologia, ou o sucesso ou o fracasso do mecanismo de
restauração da homeostase. \textit{Mas o mecanismo, por si só, nos diz
muito pouco sobre as condições gerais do organismo.}

Não podemos compreender a sexualidade humana (quão tentador é dizer
``sexualidade humana normal'') se não
compreendermos o mecanismo da perversão. Talvez o que o calor faça para
o corpo, o mecanismo da perversão faça para a psique na vida humana.

A velha questão: a repressão, o deslocamento, a simbolização, a
inibição --- e assim por diante --- são fatores normais ou anormais?
A este ponto, deveríamos saber que se trata de mecanismos, não de\idxmecap[|)]
julgamentos. Nós provavelmente não precisamos da palavra
``normal'' no discurso científico; ela só é\idxpervnorma[|)]
útil quando queremos transmitir julgamentos\idxnorma[|)] de valor.



\section{Agressão}


Estão em voga grandes discussões sobre a agressão,\idxagres{} e grandes e
empolados pensadores estão vindo à tona para nos trazer iluminação.
Territorialidade, desejo de morte, pecado original, herança animal, um
cérebro intermediário não exatamente conectado ao córtex, capitalismo,
luta de classes, distorção de moléculas, chauvinismo: cada um
desses fatores pretensamente explica a crueldade humana. E, no entanto,
talvez não fosse o pior a fazer, voltarmos ao estudo analítico de casos
individuais, para obter respostas do tipo: como a agressão (atividade)
se converte em hostilidade (ódio e violência)?

O estudo do mecanismo da perversão, e das perversões, deverá ajudar.


\chapter[\textbf{7}\quad Perversão: perigo \textit{versus} enfado]{{\large\textit{Capítulo 7}}\\ Perversão: perigo \textit{versus} enfado}
\markboth{Dinâmicas: trauma, hostilidade, perigo e vingança}{Perversão: perigo \textit{versus} enfado}

Quero, agora, abordar\idxperigo[|(] a questão do\idxpervexpos[|(] relacionamento essencial entre a
hostilidade e a perversão partindo de uma perspectiva diferente:
examinar a função do arriscar-se consciente e,\idxperigocon{} principalmente, do
arriscar-se inconscientemente como um componente essencial da sensação
de excitação e do prazer sexual na perversão; e fazer uma análise
cruzada da tese, examinando\idxpervenfad[|(] o enfado\idxenfa[|(] sexual. Da mesma maneira como
acontece com a hostilidade, algumas vezes o perigo faz parte do
conteúdo manifesto do ato perverso --- nas perversões francamente
sádicas ou masoquistas --- e, outras vezes, só o faz de maneira
latente, como nos fetichismos.\idxfetic{} O perigo é inerente à dinâmica da
vingança.\idxvinga{} Já vimos que o que acontece é que a pessoa reproduz, na
fantasia, um trauma ou uma frustração do passado com um novo desfecho:
o triunfo. Acrescentemos agora que essa tentativa de reviravolta
depende da sorte; pode acontecer de a pessoa se ver novamente atolada
no trauma. Para que o prazer seja possível, o perigo não pode ser muito
grande; a possibilidade de se vivenciar o mesmo trauma outra vez deve
ser baixa. Entretanto, a perversão deve simular o perigo de outrora. É
este o fator excitante e, desde que se mantenha o controle --- o que é
fácil, quando se trata apenas da fantasia da pessoa --- já se pode
antever, mesmo que não fique claro na história, que o perigo será
sobrepujado.

Não nos deixemos confundir por aquelas perversões nas quais grandes
perigos são efetivamente assumidos. Precisamos ter certeza de que tipo
de perigo estamos falando. O perigo que consiste em vivenciar toda a
experiência do trauma da infância é o primeiro a energizar a formação
da perversão e, para alguns, ele é pior do que arriscar a própria vida,
ou ser preso.


\section{A fantasia sexual}

Assim como todo grupamento humano tem seu mito, talvez para cada
pessoa exista a fantasia sexual\idxfanta[|(] (perversão?). Nela, está resumida a
própria vida sexual da pessoa --- o desenvolvimento de seu erotismo e
de sua masculinidade e feminilidade. No conteúdo manifesto da fantasia,
estão imersas as chaves para os traumas e para as frustrações que foram
infligidas pelo mundo exterior contra os desejos sexuais da infância,
os mecanismos criados para aliviar a tensão daí resultante e a
estrutura de caráter utilizada para obter satisfação, de seu próprio
corpo e do mundo exterior (de onde sairão os objetos da pessoa.)\footnote{ ``É preciso que se
entenda que cada indivíduo,
através da operação conjunta de sua disposição inata e das influências
sofridas durante a infância, acaba por adquirir um método específico,
próprio, para a condução de sua vida erótica\idxfreuddesen[|nn] --- ou seja, os
pressupostos para se apaixonar que ele estabelece, os instintos que ele
satisfaz e os objetivos que impõe a si mesmo no decorrer da
vida'' (28, p.\,99). Essa compactação pode ocorrer em
outros estados, quando os mecanismos de defesa são construídos numa
estrutura complexa, como as neuroses e os transtornos de caráter. Por
exemplo, Khan\idxkhan[|nn] diz (74, p.\,434) a respeito das pessoas com personalidade
esquizoide,\idxesquiz[|nn] ``poderíamos quase dizer que seus mecanismos
de defesa carregam, cristalizadas dentro deles, memórias de experiências
e de traumas reais\idxfantareal{} que o ego infantil, na ocasião, não dispunha de
outros meios para registrar psiquicamente.''} O analista
tem a oportunidade de estudar essa fantasia sexual, e de desvendar suas
origens. As descobertas feitas meramente a partir da análise, conforme
sugeri em outro local, podem ser confirmadas, de forma maciça, pela
pornografia.\idxpornofant{} A pornografia consiste em comunicar a fantasia sexual de
um grupo de pessoas interligadas pela mesma dinâmica. Raramente a
fantasia pode deixar de assumir algum tipo de forma cognitiva, podendo
ser conscientemente manifestada apenas no ritual que se utiliza para a
masturbação (105, p.\,826).

Poderemos encontrar pistas sobre a excitação sexual em seu oposto
--- o enfado sexual. Excetuando o aumento da excitação resultante de
alterações fisiológicas (tais como abstinência prolongada, puberdade ou
outras causas de alterações nos níveis hormonais e na função do sistema
nervoso central), é possível que o aumento da excitação ocorra sempre
que as circunstâncias se aproximem \textit{da} fantasia sexual.
Perguntamo-nos se esta equação é possível: um aumento da excitação ser
equivalente a um aumento no impacto de (nossos próprios) elementos
perversos --- ou seja, crueldade? Uma excitação moderada (salvo pelas
alterações fisiológicas) significaria, então, menos elementos
perversos, e um mínimo de excitação --- ou enfado\idxenfafamil{} --- significaria
menos (ou ausência de) elementos perversos resvalando a consciência
(podendo estar ausentes ou em estado de inibição). Ainda assim, o
ponto-chave não é se o ato sexual que se fantasiou ou vivenciou tem os
elementos perversos presentes, e sim se eles estão realmente presentes,
ou seja, se são capazes de agir.

Com ``realmente'' eu quero dizer algo que
requer algumas palavras explicativas. Tomem o uso da pornografia, com
os elementos perversos que lhe são inerentes. A indústria pornográfica
está construída em torno da questão de proteger seus consumidores do
enfado.\idxpornoenfa{} O material pornográfico tem uma meia-vida curta;\footnote{ Isto
é verdade também para outros estímulos sexuais, como muitos casais
sabem muito bem, assim como estupradores,\idxestup[|nn] masturbadores\idxmastur[|nn] contumazes, os
que têm como objeto de fetichismo\idxfetic[|nn] os sapatos ou a maioria dos outros
tipos de material capaz de causar excitação sexual em humanos: a
variedade dos detalhes, dentro de uma constância temática, conserva a
pessoa potente, ao mesmo tempo que o protege dos rigores da
intimidade.} o material que excita logo se\idxpornoenfa[|nn] torna enfadonho\idxenfafamil[|nn] (121, p.\,28).
A pseudo-explicação é ``familiaridade'' mas
isto pouco mais faz do que lhe conceder um nome. Não explica por que a
familiaridade, na maioria das arenas do comportamento erótico, reduz a
excitação; sem a compreensão das dinâmicas, ou sem ter vivido no mundo,
pode ser que alguém pudesse esperar que a familiaridade produzisse um
prazer maior; e isto às vezes acontece, entre os casais felizes.

Creio que o enfado sexual resulte especialmente da perda da sensação
de perigo. Sendo assim, mesmo os outros elementos adequados estando
presentes na fantasia ou na pornografia, ela não funcionará bem a
menos que possamos estar ainda um pouco temerosos, incertos em relação
a um resultado feliz. (A mesma dinâmica do perigo se aplica em outra
esfera. Eu mencionei as piadas. E deve estar também na base da
avaliação artística, na rapidez com que os estilos de arte saem de
moda; experientes críticos de arte, assim como peritos em pornografia,
são honesta e profundamente incapazes de responder a um diferente
conjunto de expressões dinâmicas. E, dentro de seu gênero preferido,
eles necessitam de um constante fluxo de trabalhos perimetrais que lhes
permita imaginar-se a si próprios em risco, por estarem experimentando
algo novo. Seu inimigo natural, o artista, tem, contudo, uma dinâmica
semelhante: uma necessidade de mistério e perigo simulado.\idxenfa[|)] A essa
altura, eu definiria a arte --- assim como a excitação sexual ---\idxpervenfad[|)]
como a busca pela ambiguidade\idxsexueamb{} controlada, administrada.)

Estou me referindo aqui a dois tipos de perigo. O primeiro, em
geral não o central nas perversões, é o aumento de excitação que
algumas pessoas obtêm ao praticar o ato sexual em locais onde poderiam
ser apanhadas (quebrando um costume, um tabu ou uma regra); trata-se
de um arriscar-se consciente,\idxperigocon{} que é usado para apimentar uma iguaria.\footnote{ Nem sempre;
algumas vezes pode ser o elemento principal no
ato sexual,\idxsadivoye[|nn] como no ritual sexual sadomasoquista, ou no caso das
pessoas que se enforcam,\idxenfor[|nn] ou que se anestesiam, para assim produzir o
orgasmo.} O outro perigo, mais importante para nossa discussão, está
relacionado aos depósitos inconscientes da situação edípica,\idxconfe{} tais como
o mistério da diferença anatômica entre os sexos. O conflito edípico,\idxperigocof{}
não solucionado nos adultos, coloca a possibilidade de fracasso no
centro de seu ato sexual. A excitação sexual (além de suas sensações
meramente físicas) é, então, o produto de uma oscilação entre a
possibilidade de fracasso (pequena) e o prelibar do triunfo (maior). A
perversão é a complexa trilha que costura seu caminho através dos
perigos, até alcançar o triunfo da gratificação sexual.


McDougall\idxmcdoug[|(] (103, p.\,378) já observou a esse respeito:

\begin{quote}
Em todos os casos, a trama [da perversão] é a mesma:\idxsexuecas[|(]
a castração\idxcastr{} não dói e é, na verdade, a própria condição da excitação
erótica e do prazer\ldots{} Há sempre um espectador para essa peça teatral
--- o indivíduo frequentemente fará o papel de si mesmo e ao
mesmo tempo acompanhará, no espelho, a produção de sua cena sexual
especial. [Pense no relacionamento disto com a pornografia, em que o
leitor, ou o espectador, é o diretor, com os participantes retratados
como crianças traumatizadas cujos pais são os perpetradores do trauma.
-- R.\,J.\,S.] Existe uma importante reversão de papeis aqui; a
criança, antes vítima da angústia da castração, é agora seu agente, o
comerciante da castração. [\ldots{}] a criança\idxsexueinf{} excitada,\idxinfanexcit{} então o indefeso
espectador do relacionamento dos pais, ou a vítima de uma estimulação
incomum, com a qual não pôde lidar, está agora no controle e produz a
excitação, seja a sua própria ou a de seu parceiro. Com efeito, muitos
pervertidos estão interessados unicamente na manipulação da resposta
sexual da outra pessoa (assim como os adultos um dia fizeram a eles;\idxsexuecas[|)]
	[conf. cap.\,5].\idxmcdoug[|)]
\end{quote}

Uma vez que uma parte tão grande do desenvolvimento e,
especialmente, da diferenciação, consiste em perigo, especialmente na
primeira infância,\idxinfanperig[|(] eu posso estar passando a impressão de ter tornado
minha tarefa demasiadamente fácil, proclamando que o perigo está no
cerne da perversão; na verdade, ele está no cerne de muitas coisas, nas
estruturas de caráter e na sintomatologia --- só que nós o chamamos de
angústia.\idxangu{} Mas aquilo a que me refiro aqui é algo de mais preciso.
Primeiramente, sabemos que o perigo não é exatamente a mesma coisa que
a angústia. O perigo implica ter dado um passo além da mera experiência
do medo ou da expectativa do perigo, e que as chances de sucesso
versus fracasso estão sendo avaliadas. E assim, um novo e complexo
efeito --- a excitação --- é adicionado à mistura; a excitação
introduz a possibilidade do prazer. Em segundo lugar, na perversão,
descobrimos que a angústia não é um estado generalizado de alerta
edípico. Ao invés disto --- esta é minha hipótese --- na infância,
nossa sexualidade foi efetivamente ameaçada, seja em relação a partes
do corpo capazes de produzir prazer erótico (não apenas prazer
sensual), seja à nossa masculinidade ou feminilidade. A pessoa que
disferiu o golpe visava somente --- pelo menos, a pessoa tem o
sentimento de que ela visava somente --- o sexo ou a identidade
sexual\idxiden{} da criança. O ataque foi diretamente dirigido aos órgãos ou à
função que assegura a diferenciação sexual, ou à liberdade de se usar
aqueles órgãos para chegar a uma clara distinção dos sexos. Este foi um
trauma\idxtrauma[|(] muito sério, e com isto quero dizer (como na neurose traumática)
que ele foi muito prolongado, ou então foi um golpe muito súbito, ou a
pessoa era nova demais para poder se defender adequadamente. A
intensidade, a brusquidão, o caráter incompreensível de um perigo que
ameaça nosso aparato psíquico sexual, tudo isso nos predispõe a algum
tipo de perversão (1,~54). Uma perversão --- eu o repito --- é uma
neurose\idxpervneuro{} sexual, erotizada. Nas outras neuroses --- aquelas em que os
principais sintomas são angústia, depressão, fobias, compulsões e assim
por diante --- o ataque é dirigido a outras partes do corpo ou da
psique, e não àquelas que distinguem os sexos. (Aqui, provavelmente,
pagamos o preço pela insistência de Freud\idxfreudsexua{} em afirmar que, na criança,
sensualidade e sexualidade\idxsexuateo{} são a mesma coisa, em virtude de elas
frequentemente convergirem. Se ele não tivesse insistido tanto, as
perversões poderiam ter sido, já há muito tempo, consideradas como mais
uma entre as categorias de neuroses.) E nas perversões, contrastando
com as outras neuroses, a resolução é sensacionalmente recompensada:
por um grande prazer erótico.

Talvez, se o trauma for total (se é que isto é possível) não resulte
qualquer tipo de perversão; talvez simplesmente ocorra de uma função
ser completamente varrida. (Existem pessoas fisicamente intactas que
nunca sentiram excitação sexual.) Quanto à perversão, podemos supor que
ela é resultado de uma avaria, não de uma destruição; restam
esperanças. O ``perigo'' implica isso.\idxinfanperig[|)] O
perigo indica probabilidades, em prol e contra o sucesso; a ingenuidade
humana pode encontrar um desvio, ou um substituto; ou, eventualmente,
perpetrar um ato verdadeiramente criativo,\idxcria{} elevando a perversão à
categoria de arte.

O trauma original, a batalha que aconteceu fora das vistas por anos,
na infância, e que ainda está à tona de modo manifesto, a perversão
genitalmente descarregada, tudo isso está registrado nos detalhes do
ato (capítulo 5). Da mesma forma está também a raiva que o trauma
produziu na criança, e que precisou ser superada, para que a vida
pudesse prosseguir. Podemos, portanto, pressupor a existência de
fantasias de vingança\idxvinga[|(] contra quem causou o trauma --- em geral a mãe,
algumas vezes o pai. (O indivíduo perverso se constitui em ameaça menor
aos outros, na medida em que diferencia o objeto imediato do desejo de
hoje do objeto original que frustrou os grandes impulsos da
primeira infância e da infância. Obviamente, quanto mais ele igualar o
objeto imediato ao objeto que originalmente o forçou a criar as
dinâmicas perversas, mais perigosa a perversão. O homem que tem um
fetiche\idxfetic{} por roupas apenas mancha seu objeto imediato, uma peça do
vestuário; um estuprador\idxestup{} ou assassino\idxassas{} sexual disfarça muito pouco ---
conserva inconsciente --- a imensidão de seu ódio por seu objeto
original.) Nisto jaz outra fonte de sensação de perigo, pois não se
pode ter certeza de que as representações subsequentes do
traumatizador, que se tornarão os objetos sexuais posteriores da
perversão, não perceberão os motivos do ato (como aconteceu com os
objetos primitivos, que agora estão integrados ao superego) e infligir
punição pelo pecado da vingança. A execução do ato e da fantasia
sexuais para reencenar a situação de perigo --- mas, dessa vez, sem
que o antigo trauma volte a ocorrer, sem que haja punição para a
arrogância desse ato audacioso (a perversão) ou sem que aflore à
consciência a raiva que está ali, oculta, contra os que causaram o
trauma --- desemboca numa explosão de gozo,\idxorgasfet{} que é manifestada pelo\idxfeticorga{}
orgasmo. (Freud [32, p.\,154]: ``O fetiche \ldots{} permanece
sendo um símbolo de triunfo sobre a ameaça da castração,\idxcasta{} e uma proteção
contra ela.'')\idxfreudfetic{} É típico que esses pacientes descrevam seus
orgasmos como imensamente prazerosos, o que se poderia supor um exagero
para justificar a perversão. Contudo, tenho tido a impressão, ao ouvir
tais descrições, de que as experiências que os pacientes descreveram\idxtrauma[|)]
eram, efetivamente, das mais intensas.

Para resumir os perigos pertinentes a esta discussão:\footnote{ Outras
formas de perigo,\idxperigofor[|(] especialmente aquelas com as quais já estamos
familiarizados há tempo em nossos estudos sobre os perigos inerentes à
situação edípica, serão ignorados agora, caso contrário esta
apresentação se tornaria infindável. Analisar um material com o qual
muitos leitores, especialmente os analistas, já estão familiarizados,
iria emaranhar a linha do presente argumento.}

\textit{Primeiramente}, perigo consciente: o que estou fazendo me
coloca em risco com a sociedade (a realidade exterior e minha avaliação
dessa realidade). Se eu for apanhado, haverá problemas.

\textit{Em segundo lugar}, também consciente: o que estou fazendo
vai contra meus princípios (consciência). Se eu for apanhado, vou me
odiar.

\textit{Terceiro} (pode ser consciente ou inconsciente): Isto, que
estou fazendo, meus pais me disseram que era feio, quando eu era
criança. Crianças boazinhas não agem dessa maneira.

Até aqui, os perigos que estão à tona, ou próximos da superfície,
são aqueles que todos conhecem e sempre compreenderam. Não são
específicos à perversão. O perigo é percebido como sendo decorrente do
fato de se estar envolvendo com uma anatomia proibida: parte proibida
da pessoa, ou sexo proibido.

Mas, à medida que vamos nos aproximando das coisas mais ancestralmente
proibidas, abandonamos então o simples erotismo pela hostilidade,\idxperigohos{} pela
raiva, pela vingança,\idxvinga[|)] pela violência e pela destruição. Trata-se,
agora, de perigos que são percebidos como ameaçadores à vida, à própria
e à alheia (inconscientemente em algumas perversões, conscientemente
nas perversões que consideramos bizarras; nosso sentido de grau de
bizarrice funciona, comumente, como uma forma de intuição com relação à
intensidade do ódio).

\textit{Quarto.} Estou repleto de ódio e é preciso que eu não o
saiba. Por eles (os adultos) terem me frustrado (mistificado) a este
ponto, minha liberdade sexual foi tirada de mim. Não apenas houve
restrições que me foram impostas mas, o que é mais avassalador, eu fui
acusado de ser responsável por elas: é preciso que eu sinta a tentação
e evite minhas próprias ações. Por tudo isso, eu devo amá-los e
respeitá-los. Odiar é errado e eu serei punido.

\textit{Quinto.} Meus desejos sexuais são maus; meu ódio é pior
ainda.\footnote{ Ele é, em última instância, assassino.\idxagrespais[|nn] Mas quando não
chega a tanto, ele é ainda fabuloso.\idxperigohos[|nn] Para se libertar daquele primeiro
objeto, mãe, e se constituir a si mesmo, isto requer que uma barreira
seja erigida, para ajudar a evitar que se sucumba ao anseio de se
fundir com ela. Este traço da estrutura de caráter pode ser sustentado
pelas fantasias de causar dano à mãe;\idxmaeshosr[|nn] mais uma vez, um negócio
arriscado. Diz McDougall:\idxmcdoug[|nn] ``A fantasia que se dirige à
castração\idxcastr[|nn] fálica\idxrelpccast[|nn] da imagem paterna oculta outra, que é a castração da
mãe que amamenta. Se, do primeiro desejo, se pode dizer que ele ameaça
o próprio indivíduo com a castração, o segundo produz uma angústia que
se liga à depressão, ao medo da desintegração psíquica e à morte.
Esses desejos agressivos-castrativos, com as angústias que os
acompanham, são mantidos em cheque através de comportamentos sexuais
compulsivos que assumem as características de um jogo ou brincadeira,
com regras rígidas, conduzindo a uma forma de relação objetal dominada
pelos mesmos mecanismos defensivos: repúdio e negação, clivagem e
projeção, regressão instintiva, defesa maníaca. Como na
infância, o jogo se destina a dominar eventos e condições traumáticos,
permitindo que o indivíduo brinque com coisas que ele não levará à ação
(desejos libidinosos e agressivos);\idxagrespais[|nn] ele permite, também, uma reversão
de papeis que, frequentemente, toma a forma de um controle da resposta
orgástica do parceiro, sendo que essa ``perda de
controle'' é encarada como uma castração\idxcastr[|nn] do parceiro, ou
uma redução de sua condição à de uma criança indefesa. Em sua fantasia,
ele se julga o único que desfruta dos seios maternos: consequentemente,
ele pode possuir e punir tais objetos. Assim sendo, esse desesperado
jogo sexual permite a recuperação, na fantasia, de objetos perdidos,
assim como a erotização das defesas dirigidas contra os desejos
proibidos.'' (103, pp. 373--374)} Caso eles soubessem sua
extensão, teriam que me destruir. Mas essa violência é minha, ela faz
parte do eu essencial que é mau e que, entretanto, precisa ser
protegido, preservado. Ela está escondida nos meus desejos eróticos.

\textit{Sexto}. Eu me vingarei\idxvinga{} disso tudo que me fizeram, e essa
vingança acontecerá também no ato sexual.\idxperigohos{} Mas, se pretendo danificar
meu objeto, pode ser que ele o pressinta, e faça comigo pelo menos a
mesma coisa que pretendo fazer com ele.\idxfanta[|)] O que é bem mais arriscado,
efetivamente.

A perversão é ódio, ódio erotizado.\idxperigofor[|)]


\section{Fatores de Segurança}

O triunfo requer que\idxfantafato[|(] as probabilidades estejam a favor; mas se o
perigo,\idxfeticarti{} na vida adulta, fosse tão sério como o foi na infância, o
prazer estaria ausente e, assim, não haveria perversão. Então
artifícios --- fatores de segurança --- precisam ser adicionados à
fantasia, para reduzir a angústia e garantir que as probabilidades
estão, efetivamente, a favor do triunfo. Seus traços estão por todo
lado, nos microelementos que constituem a fantasia ou o comportamento
perverso. Citarei alguns exemplos aleatórios. Peguem o fetiche: ele é
absolutamente passivo e, assim, não pode ameaçar, interferir,
testemunhar ou acusar; ele pode ser atacado, emporcalhado, odiado,
destruído e, ainda assim, ser infinitamente renovável\ldots{} O autor de
telefonemas\idxtelef{} obscenos não tem que confrontar sua vítima para se
conscientizar, assim, de que se trata apenas de um ser humano\ldots{} A
prostituta é paga para ser agradável e tolerar atos que, em outras\idxperigo[|)]
circunstâncias, ela trataria com desprezo\ldots{} O travesti espera\idxpervexpos[|)]
convencer sua mulher\idxtravempape{} a ajudá-lo a se vestir de\idxfantafato[|)] mulher\ldots{}

\section{Material clínico}

O caso a seguir exemplifica como o perigo e a vingança se
transformam em excitação na perversão. Mesmo para esse homem, muitas
das dinâmicas de sua hostilidade são óbvias. (Ele não é um paciente, e
as origens desse comportamento são desconhecidas.)

Trata-se de um\idxprosthom[|(] homossexual que se\idxprost[|(] prostitui\idxhomospros[|(] e que tem modos
nitidamente masculinos. Ele tem uma maneira de operar nas ruas que se
mantém sempre inalterável. Ele só se manifestará quando um cliente
em potencial demonstrar interesse por ele, de dentro de seu carro.\footnote{ Em Los Angeles,
como o automóvel desempenha um papel
fundamental no estilo de vida, em algumas áreas, em certos horários do
dia, o procedimento padrão de operação é os clientes cruzarem as ruas
em seus automóveis, e não a pé.} Então ele vem até o carro, mas não o
toca, nem mesmo para abrir a porta. Ele simplesmente fica ali, de pé,
aguardando que seu cliente dê o próximo passo; ele só entra no carro
quando é convidado. Ele se senta, espera que lhe seja dirigida a
palavra e então, em aparente passividade,\idxprostpas[|(] deixa-se conduzir para onde o
cliente deseja. Ele não sugere preço, duração, lugar ou tipo de atos
sexuais; ele aguarda que o cliente peça; então, responde. Dotado de
muita experiência e muito esperto, ele escolhe, com base nas perguntas
e nas observações que o cliente faz, quais fantasias ele porá em ação.
Quando, por exemplo, espera-se dele que seja um animal ignorante mas
forte, sem educação, o cliente o indicará correspondentemente, ao fazer
observações sobre seu físico, ao lhe perguntar se ele é um trabalhador
braçal; ele concorda e inventa alguns detalhes a respeito, para agradar
 seu cliente. Em outras palavras, ele colabora com seu parceiro nas
fantasias dele, deixando claro que foi contratado e, portanto, não faz
o papel de instigador. Como ele raramente tem orgasmos quando está a
serviço, é capaz, assim, de atender a muitos clientes sem ficar
sexualmente exausto, podendo se dedicar à atividade indefinidamente.

Agora, dizer que ele é perverso só em função de sua
homossexualidade, é perder os detalhes essenciais --- e, com isto, as
dinâmicas. É importante notar que todo o processo citado corporifica a
perversão, e não só o ato homossexual anatômico. Ele revela isso ao
observar que, sempre que tem problemas para conseguir a ereção --- e é
importante que ele a tenha, ou o cliente se sentirá trapaceado ---
tudo quanto tem que fazer é pensar no preâmbulo, para renovar-se.\idxpreli{} Esse
preâmbulo é o ritual em que o cliente demonstra estar atraído por ele,
o prostituto, o serviçal. (A vítima, a pessoa inferior se transforma,
assim, em superior, em vencedor.) Sua maior excitação, entretanto, é ao
lembrar-se de que o dinheiro mudou de mãos; algumas vezes, quando se
sente esgotado, ele pedirá por pagamento adiantado, para poder ter a
visão do dinheiro ali, guardado, durante o ato sexual. Pode ser até
que ele o ponha em local visível, para poder visualizá-lo durante o
ato sexual; o dinheiro que o cliente lhe deu passa a ser o elemento
mais excitante. Seria, contudo, negligência chamarmos esse dinheiro de
fetiche; embora seja inanimado, e embora sua visão faça com que ele se
sinta excitado, não é o dinheiro em si que provoca isso. Ele não fica
excitado com dinheiro a qualquer momento como ficaria, por exemplo, um
travesti, ao ver roupas femininas. É antes o conhecimento consciente
daquilo que aquele dinheiro representa que lhe causa a excitação.

Essa excitação, estimulada pela visão do\idxprostdin[|(] dinheiro,\idxprosthos[|(] vem da
hostilidade.\idxhostpros{} O que o observador desatento poderia chamar de passividade
nesse ritual, não o é, em absoluto. Quando rastreamos o que se passou,
quando entramos em contato com o que esse homem, o profissional do
sexo, estava pensando, descobrimos que ele está usando a aparente
passividade como um ato de hostilidade, especificamente para a
vingança.\idxvingapro[|(] Cada movimento que ele faz, a partir do primeiro momento de
contato em potencial até o final do ato sexual, é um esforço vitorioso
para forçar o outro homem a demonstrar carência, excitação, fraqueza e,
portanto, dependência em relação a ele. Portanto, o outro tem que
perguntar; ele apenas concorda, ou confirma. O cliente está sexualmente
excitado e carente; ele faz de seu cliente um mendigo. O símbolo
supremo da fraqueza do cliente, portanto, é o dinheiro que muda de
mãos.

A necessidade que ele tem dessa vingança é tão grande que ele
precisa continuar desempenhando esse papel; ele nem ao menos ainda
racionaliza que a prostituição é apenas pelo dinheiro. Ele passa a
maior parte de seu dia repetindo esse comportamento e, algumas vezes,
nem mesmo isso é suficiente. Ele tem episódios, que chegam a durar
alguns dias, em que precisa apanhar o maior número de homens que
consegue; nessas ocasiões, ele não desempenha todo o complicado ritual
do acordo financeiro mas, simplesmente, rapidamente, conduz um homem ao
orgasmo em algum beco, ou em outro local escondido, partindo
imediatamente para o próximo,\idxvingapro[|)] fazendo isto a quinze, vinte ou até mais
homens, em poucas horas.

Mesmo nos raros relacionamentos que ele tem, nos quais ele busca
satisfação sexual (nesses dois estilos descritos acima, seu próprio
prazer sexual não exerce nenhum papel em sua motivação), ele precisa
desempenhar cada fragmento de atividade sexual e expressar sua
excitação de modo a demonstrar que ele está menos comprometido, menos
ávido, menos desesperado, menos envolvido do que o seu parceiro.

Ele conta, magoado, os raríssimos episódios de fracasso pelos quais
passou, na prática de sua perversão. Houve, por exemplo, homens que
insistiram em serem reembolsados, dizendo que ele não havia feito um
bom trabalho; ou o homem que o desafiou, apostando como ele seria
incapaz de conseguir uma ereção; sabendo que esse homem era mais forte
do que ele, acabou ficando mesmo impotente;\idxprostpot{} ou o homem que, depois de
fazê-lo passar por um elaborado ritual para chegar ao ato sexual,
inclusive levá-lo a seu suntuoso, lindo apartamento (para, assim,
demonstrar sua superioridade, sua força), disse não considerá-lo
excitante; mas que ficaria contente de contratá-lo, por uma tarifa\idxprosthos[|)]
menor --- dez dólares --- para que ele lhe limpasse a casa.\idxprostdin[|)]

O profissional do sexo sabe que está impotente; reconhece ter uma
vantagem sobre a maioria de seus clientes: ele não está tão excitado
quanto eles. Mas sabe, também, que essa batalha pelo poder acontece dos
dois lados; seus clientes estão fazendo a mesma coisa com ele. Ele
observa que, dos profissionais do sexo, se espera que sejam fortes e
burros; os clientes precisam de homens inferiores para seus parceiros
sexuais. (Uma vez, em seus dias de inocência, sem compreender as
dinâmicas, ele disse a um cliente que estudava no colégio; o homem
imediatamente perdeu sua ereção --- e seu interesse também.) Ele está\idxprostpas[|)]
consciente de que, muito de vez em quando, encontrará homens que o manipularão.

Em outras palavras, a luta pelo poder, na prostituição masculina, é
mais ou menos a mesma que tememos em relação às prostitutas e às
\emph{strippers}.

Vemos, aqui, dinâmicas de hostilidade comparáveis àquelas
encontradas na ninfomania\idxninfo{} e na satiríase,\idxsatiri{} em que também (embora com uma
aparência heterossexual) são necessários inúmeros parceiros para que a
pessoa continue atestando sua superioridade. Em todas essas situações
--- em todas as perversões --- o objeto sexual é vitimizado,\idxhostviti{}
engrandecendo assim o neurótico sexual. Apesar de suas experiências
sexuais traumáticas\idxtraumapros[|(] viverem eternamente dentro dele, suas vitórias
duram apenas um curto espaço de tempo, precisando de repetições
infinitas. Nas situações em que o sentimento de desespero e de
inferioridade se aproxima demais da superfície, a pessoa precisa
repetir \textit{ad infinitum} e rapidamente, como no caso do homem que
descrevemos acima.

Ele diz que o relacionamento entre o cliente e o profissional do
sexo termina instantaneamente, depois que o cliente obtém seu prazer;
pois, seja qual for a hostilidade que tenha pairado durante o ato
sexual, ela é liberada quando ele atinge seu fim. A equação de poder
alterou-se; o cliente não se deixa mais humilhar. O homem rico despacha
o amante num táxi --- devolvendo o servo sexual para as ruas --- e
acende um charuto em sua sala de estar. Ambos guardam, ao final,
reminiscências do episódio maníaco que foi vivenciado, em particular,
por cada um; ambos experimentam a ilusão de que só o outro foi
trapaceado e humilhado. Ambos correram o risco de fracassar e, exceto
nos casos tristes, em que o fracasso realmente ocorre (tais como os que
descrevemos mais acima) cada um se convence de haver triunfado. O que
espera por esse homem é o suicídio, pensa ele, quando sua aparência,
galhardia e potência sexual o abandonarem, e quando o desgaste, causado
pela vida sexual frenética que leva, cobrar seu preço.

(Observação: Este homem não tipifica todos os profissionais do sexo;
ele está lutando, até mais do que a maioria, para refrear sua raiva; a
excitação que o dinheiro lhe transmite, e sua frenética busca por\idxprost[|)]
parceiros, dão testemunho disso. Sua perversão, apesar de apresentar
semelhanças com a de outros, é estritamente a sua --- embora isto\idxprosthom[|)]
possa, igualmente, ser dito da relação entre todas as pessoas perversas
e da categoria de diagnóstico em que elas possam ser\idxhomospros[|)] classificadas.)

Alguns anos atrás, Khan\idxkhan{} apontou para essas questões, com dados
clínicos semelhantes, que descreviam um homem homossexual para quem o
perigo e a vingança\idxhomosperi{} desempenhavam um papel essencial.

``Ele esquadrinhou cada nuance de sentimento e de
tensão em sua fisionomia e em sua postura (de seus parceiros), até
obter deles uma `colossal ereção'. A
essa altura, sua sensação de conquista, de triunfo e de domínio sobre o
objeto fetichista deveria ser completa. Ele iria agora, de maneira
solícita e compassiva, se oferecer para chupá-los e/ou masturbá-los. A
excitação indefesa desses jovens rudes, fortes, agressivos, exerciam um
impacto especialmente prazeroso sobre o paciente. Aqui, um elemento
claramente sádico-agressivo entrou em jogo em sua relação com eles. Ele
iria gozar, secretamente, em cima deles: eles estavam em seu poder.
Quanto mais se excitavam e ficavam frenéticos de tesão, mais
imperturbavelmente quieto e gentil ele se tornava em suas maneiras. Ele
frequentemente os obrigava a observar, enquanto o assistiam
masturbá-los e fazê-los ejacular. [\ldots{}] Ele tinha sempre uma apreensão
culposa, de que esse estado de excitação sexual não fosse prazeroso
para os jovens. (75, p.\,69)''

O perigo prepara o terreno para a promulgação do triunfo,\idxtraumapros[|)] embora os
indivíduos perversos, a cada episódio, carreguem fortemente os dados a
favor do sucesso; nos devaneios e em suas extensões --- a pornografia
--- o perigo é apenas simulado, e assim o enfado interfere com
rapidez. No ato verdadeiramente perverso, contudo, o perigo é parte da
realidade que, acredito, contribui para aumentar a excitação de tais
encontros.

Essas dinâmicas de perigo, vingança e triunfo presumivelmente
alimentam outros tipos de derrotas e de frustrações infantis, que não
envolvem o gênero nem os genitais. As tensões de cada estágio libidinal\idxlibid{}
--- oral, anal e urinário, fálico e, finalmente, o Édipo plenamente
desenvolvido --- com suas exigências de controle biológico (de maneira
sádica, na opinião da criança) feitas pelos pais, são batalhas em que o
triunfo, para a criança, consistiria em estar no controle, enquanto a
outra pessoa o perde.\footnote{ Não creio, entretanto, que a teoria da
libido\idxlibid[|nn] segundo a qual as perversões são fixações ou regressões a pontos
de fixação em certos estágios de desenvolvimento --- oral, anal,
fálico --- seja de muita ajuda para nossa compreensão. Primeiramente
porque, nos perversos, existem fixações em todos os estágios. Em
segundo lugar, não fica explicado por que ocorre a perversão, ao invés
de outra forma de neurose; e o reconhecido fracasso em encontrar essa
explicação atira os teóricos de volta a especulações como algum tipo
de vulnerabilidade orgânica,\idxvulne[|nn] para explicar a especificidade dos atos
perversos, ou a disparates, frases pomposas do tipo
``hipercatexia da libido anal'' enfim,
explicações pseudocientíficas.} Creio ser esta a questão central nas
perversões. Essa luta por controle, e o perigo a ela subordinado,
podem incrementar a excitação --- mas podem, também, ser exaustivos.

Mais uma vez testamos, com maior rigor, as hipóteses, ao examinar
uma condição, o exibicionismo,\idxexibi[|(] em que o mecanismo é menos manifesto;
descobriremos que ele requer ainda menos habilidade clínica --- ou
capacidade de criar teorias --- para que percebamos como o ato
perverso procura, entre os perigos, o tipo adequado para criar
excitação.

Por que é tão comum que os exibicionistas sejam apanhados?
Dispondo, como dispõem, de um adequado grau de inteligência, não sendo
psicóticos, conscientes da possibilidade de que podem ser presos e, em
muitos casos, já tendo sido expostos a graves consequências sociais,
por que insistem em seu comportamento\idxexibiexpo{} perigoso?\footnote{ O grupo de
Kinsey\idxkins[|nn] relata, acerca dos exibicionistas:\idxexibicond[|nn] ``De todos os
que atentam contra a sexualidade, sua maior proporção (72\%) era por
delitos sexuais e, reciprocamente, a menor proporção (28\%) eram
delitos não sexuais. [\ldots{}] Em termos de condenações per capita eles
são, novamente, notórios [\ldots{}] e ficam em primeiro lugar no número de
contravenções que resultam em prisão. [\ldots{}] Nenhum outro grupo se
aproxima deles no número per capita de condenações por delito sexual
(3, 12). Com relação àquilo que denominamos de delitos sexuais
`específicos' --- ou seja, exibicionismo,
no caso dos exibicionistas, estupro de menores, onde temos agressores
versus menores, etc. --- os exibicionistas tiveram, de longe, o maior
percentual per capita de delitos sexuais específicos: 2.13. [\ldots{}] Em
resumo, os exibicionistas cometeram mais delitos sexuais (medindo com
base nas condenações) do que qualquer outro grupo. [\ldots{}] Os
exibicionistas\idxexibicond[|nn] costumam reincidir muito. São relativamente poucos 
(13\%) os que contam com somente uma condenação; cerca de um terço, a
segunda maior proporção registrada, tinha de quatro a seis condenações;
e eles exibem a terceira maior porcentagem entre os que são condenados
sete vezes ou mais (16\%). Um grupo que pode se gabar sete vezes mais
do que os fracotes com apenas uma condenação, pode, com justiça, se
rotular de reincidente''. (41, pp. 393--394)} Obteremos as explicações ao
examinarmos a estrutura da perversão.

Para fazê-lo, entretanto, é preciso estar atento, conduzir nossas
investigações a partir de nossa definição prévia: ela diz que, na
perversão sexual, o indivíduo expressa suas técnicas eróticas
prediletas, costumeiras. Sem esta compreensão, os múltiplos usos do
termo ``exibicionismo'' vão nos confundir,
pois ele tem outros significados: (1) O desejo, não sexualmente
excitante, não genital, de se mostrar, em crianças ou adultos, homens
ou mulheres; (2) o prazer feminino de mostrar partes de seu corpo,
inclusive (menos frequente que outras partes) seus genitais, a fim de
se tornar sexualmente excitante para outrem, sendo que este tipo de
exibicionismo não é um fim em si mesmo; (3) a exibição, por parte de
homens homossexuais,\idxhomosexib{} de seus pênis, para a publicidade. Assim, reduzido
por nossa definição, descobrimos que o exibicionismo como perversão
--- a necessidade de mostrar seus genitais a outrem, com a finalidade
de se excitar --- só existe nos homens.\idxpervpredo{}

Esse homem, casado, francamente heterossexual durante toda a sua
vida, irrepreensivelmente masculino em suas maneiras, com uma profissão
masculina, foi condenado três vezes por exibicionismo. Embora ele
tivesse estado preso anteriormente, e esteja agora em liberdade
condicional, ele arrisca seu casamento, sua profissão e sua reputação
reincidindo no desempenho de seu ato perverso uma ou duas vezes a cada
quinze dias. Isto, em geral, ocorre após alguma humilhação que ele
tenha sofrido, mais frequentemente no trabalho, ou em casa, por parte
de sua mulher. Ele é então levado às ruas por uma tensão que ele não
percebe como erótica, para procurar, num bairro estranho, por alguma
mulher ou garota para quem ele possa exibir seu pênis. Ele escolhe
desconhecidos; ele nunca fez tal demonstração para alguma mulher
conhecida. Em realidade, ele é tímido, sente-se constrangido quando é
avistado nu por sua mulher, que tem a ele --- e a seu pênis --- como
coisas certas. (Ele diz que ela não o respeita; ela concorda.) Ele
espera chocar a desconhecida, não mostra seu pênis como introdução ao
ato sexual; ele não sabe por que age assim: simplesmente, sente-se
compelido a fazê-lo. Em algumas ocasiões, quando as mulheres não estão
aborrecidas com ele, quando brincam com ele, fingindo estarem
interessadas, ele foge. Mas quando a mulher está zangada e chama a
polícia --- que é quando ele sente estar correndo o maior perigo\idxexibiexpo[|(] ---
ele sente relutância em se afastar rapidamente. Apesar de ser envolvido
pelo medo, a sensação que toma conta dele se mescla com um sentimento
de confusão que desliza para a imobilidade. Quando essa excitação
letárgica demora muito, é quando ele é capturado.

O não exibicionista, incapaz de compreender, acredita que esse homem
é idiota. Quão mais estranho então é o humor que invade esse homem, ao
ser fichado: no âmago de seus sentimentos de catástrofe, existe uma
quietude absurda, pacífica, agradável. Acredito que é possível
compreender: o risco foi corrido e superado; o trauma\idxtraumaexib{} converteu-se em
triunfo. O fato de estar sendo derrotado pelo policial significa menos,
para ele, do que a vitória que ele conquistou diante dessa mulher
desconhecida.

Nosso erro seria pensar que o perigo era a polícia; não é. Os
policiais são, quando muito, agentes do triunfo. O perigo verdadeiro,
do ponto de vista da perversão, brota da humilhação que aconteceu
antes, naquele mesmo dia: uma repetição da humilhação infantil que o
deixou com uma linha de fratura, um medo de que ele não seja um homem
independente, potente, formidável.

E assim o perigo --- o perigo de toda uma vida --- não é porque
poderá ser preso, e sim o perigo de que a humilhação persista.\idxexibiexpo[|)] Exibindo
seu pênis, ele demonstra, do modo mais concreto, que ele não foi
humilhado, que ele não é castrado,\idxcastaexib{} que ele não foi derrotado pelas
mulheres; esse é o modo como ele protesta, insistindo em ainda ser
homem. Compreendemos seu comportamento ao percebermos que ele está
preocupado em exibir sua virilidade (seu \textit{self} ideal) mais do
que sua masculinidade (anatomia). Por isso, a mulher que se choca, que
fica zangada e, melhor que tudo, assustada, que fica tão perturbada que
chama a polícia, prova que ele conseguiu reverter a situação da
infância. Ela está colaborando com uma parte necessária de sua
perversão: ela, agora, é a atacada e ele, o atacante. Mesmo sendo
preso, está singularmente tranquilo, pois a prisão indica --- embora
de forma breve --- que, de fato, ele realmente tem um belo pênis,
poderoso o suficiente para criar tal confusão na sociedade. Portanto,
não nos surpreendemos ao descobrir que o índice de detenção, no
exibicionismo, é mais alto do que em todas as outras perversões.

Não deveríamos ficar perplexos com o fato de os exibicionistas
organizarem as probabilidades de tal maneira, que sua detenção é mais
comum do que a dos autores de quaisquer outros atos de perversão.\idxexibicond{} Ele
aspira a se livrar não da polícia, e sim da ameaça interna de ser
inadequado como homem. A detenção ratifica a importância da pessoa;
consiste em sua vitória sobre o medo de ser insignificante --- como o
faz, também, a já esperada reação de choque, por parte das mulheres
para quem ele se expõe.

Repito mais uma vez: ao examinar as minúcias da fantasia sexual,
provavelmente descobriremos que nada, ali, é fortuito. Tudo tem seu
lugar e desempenha a função de assegurar ao perverso que ele, agora,
está a salvo. Desta vez, o ataque que lhe foi perpetrado, e que é
reencenado na fantasia, transformar-se-á numa ofensiva contra seu
antigo vencedor; desta vez, ele obterá uma vingança precisa: o então
agressor terá que sofrer precisamente aquelas sensações que afligiram a
criança, então vítima. Mas a história não pode sair dos trilhos; senão,
como um comediante cuja hostilidade fugiu ao controle,\idxexibi[|)] e que começa a
destruir seu humor, a excitação da perversão se transforma em angústia,
ou em raiva, com a perda do prazer e da potência.

\section{Clivagem, Desumanização, Fetichização, Idealização: Anulação}

Como Bak\idxbak{} (1) sugeriu, o fetichismo\idxfetic[|(] é o modelo\idxdesu[|(] para todas as
perversões.\footnote{ E muito antes disso, Freud disse:\idxfreudfetic[|nn]
``Nenhuma outra\idxdesfa[|(] variação do instinto sexual que beira o
patológico é capaz de reivindicar tanto o nosso interesse como esta [o
fetichismo]'' (24, p.\,153). Até que ponto a fetichização
sexual é sinônimo de perversão?} Aquele que é incapaz de tolerar a
totalidade do outro fragmentará --- por clivagem\idxcliv{} (35) e
desumanização (67) --- esse objeto, em conformidade aos traumas e às
fugas do passado; ele é, então, capaz de isolar um fragmento neutro
--- um aspecto --- daquela pessoa, e deslocar sua resposta sexual em
potencial, de seu todo, para aquela parte que a representa com maior
segurança (fetichização). Quando o processo de fetichização é benigno,
como costuma sê-lo nas preliminares, ou nas variações dos costumes
sexuais, de lugar para lugar e de tempos em tempos, o objeto inteiro é
finalmente restaurado, relativamente intacto. Isto significa um mínimo
de vingança e um mínimo enfrentamento de perigo; infelizmente, a
satisfação sexual plena, sem muito recurso a mecanismos perversos, para
a maioria, parece uma conquista difícil.

Uma vez que a parte do corpo (ou um objeto inanimado equivalente,
como uma peça de vestimenta)\idxfeminsimb{} tenha sido destacada do objeto humano em
sua forma inteira, é necessário um outro processo --- idealização ---\idxideal{}
para reinventar o novo objeto.\footnote{ A transferência\idxtransf[|nn] positiva é
outra instância óbvia dessa reinvenção.} A hostilidade (destruição
potencial do objeto) que paira nas fantasias latentes que energizam a
perversão deve ser neutralizada e introduzida de maneira positiva,
prazerosa, erótica, caso contrário não existirá perversão. A essa
altura do processo, as qualidades destrutivas orais, anais e fálicas,
tão bem conhecidas nas perversões, deverão ser mantidas dentro de seus
limiares. O que é remotamente possível em estados de descontrole, como
acontece na personalidade \textit{borderline} e nas psicoses\idxpsico{}
declaradas, condições que se destacam por seus atos sexuais primitivos,
francamente hostis (e, portanto, bizarros). Descobrimos, então, que
esses objetos precisam verdadeiramente --- e não simbolicamente ---
sofrer contusões, ou até mesmo destruição; ser maculados com
excrementos (ou palavras), ou espancados, feridos e fisicamente
brutalizados.

Examinando a pornografia, encontramos desumanização,\idxpornodesu{} fetichização e
reinvenção. Nela, escolhem-se aspectos da sexualidade nos quais são
focalizados os temas essenciais das dinâmicas perversas; por exemplo:
nas formas mais suaves da pornografia heterossexual masculina, as
fotografias de nus. Ela reduz a mulher verdadeira a uma criatura
bidimensional, congelada, indefesa, incrustada na página, de tal modo
que ela não pode se defender ou revidar, como, no mundo real, poderia.
Mesmo no caso de exibir uma imagem perigosa, o perigo se anula por seu
encarceramento no papel. Ela pode ser insultada, suja, forçada a agir
conforme o desejo do espectador e permanecer ali, sem reclamar,
sorrindo, ou até em atitude fálica --- o que seja necessário --- mas
imóvel. E ela não está ali somente se exibindo, passível de ser
hostilizada por qualquer tipo de fantasia; ela é, também, idealizada.\idxpornoidea{}
Ela não causa mágoas; satisfaz o espectador e é esteticamente perfeita
(caso não o seja, outra fotografia é escolhida); ela é retocada,
restaura-se indefinidamente, não clama por vingança, é absolutamente
cooperativa e guarda os segredos; não custa nada, em termos de tempo ou
dinheiro, não precisa ser compreendida, não tem necessidades próprias:
é ideal (conf. 42). Não é de se admirar que se torne\idxpornoenfa{} enfadonha (121).

Embora seja mais difícil obter-se a mesma submissão quando se está
efetivamente desempenhando uma perversão do que ao imaginá-la, a
perversão, na pornografia ou nos devaneios, quando corretamente
planejada, ainda permite que a pessoa escolha objetos do mundo real,
que poderão ser tratados da mesma forma. Assim, por exemplo, o
fetichismo (a utilização de objetos inanimados), ou o recurso às
prostitutas\idxprost{} (mulheres contratadas para agirem com fantoches), ou a
escolha de pessoas, como a esposa\idxtravempape{} indulgente do travesti, cujas
próprias neuroses complementam --- ou seja, encontram utilidade para
--- o ato perverso.

Khan\idxkhan{} discute como, através da

\begin{quote}
  ``técnica\idxpervtecni{} da intimidade\idxintim{} [\ldots{}] o perverso induz e coage a
  outra pessoa a se tornar cúmplice [criando] uma situação de
  faz de conta que envolve, na maioria dos casos, a cooperação solícita,
  obtida por sedução, de um objeto exterior. [\ldots{}] Existe sempre,
  entretanto, uma condição. O próprio perverso não pode se abandonar à
  experiência e seu ego mantém um controle clivado, dissociado,
  manipulativo da situação.'' (76, pp. 399, 402.)
\end{quote}

(A descrição de Khan é, de fato, uma descrição acurada de todo ato de
sedução,\idxseduc{} que muito frequentemente é uma atividade hostil, fetichista,
que gira em torno da busca do poder.) Aquele que transforma seus
objetos em fetiches reduz sua capacidade de intimidade a ponto de sua
própria dimensão humana acabar por não ser maior do que a do fetiche
que ele cria (escolhe criar).

A clivagem, a desumanização, a fetichização e a idealização são
resultado do fracasso da empatia, e da dificuldade ou da
impossibilidade de se identificar com os demais. Ou será ao contrário?
A condição natural dos humanos poderia, talvez, resumir-se a uma baixa
capacidade para a empatia, sendo que analistas, artistas, santos e
psicóticos sofreriam de uma aberrante hipertrofia desse mecanismo\idxmasoq{}
masoquista.

Para que uma hipótese se confirme não basta teorizar, citar
autoridades e criar jargões clínicos. Acredito ainda assim que,
aplicando a nossos pacientes as sugestões sobre o papel essencial de
autopreservação somada à hostilidade\idxhost{} --- perigo, vingança\idxvinga{} e triunfo
--- quer essas sugestões sejam de nossa autoria ou já publicadas, as
ideias se confirmarão. Isto será aplicado aos comportamentos em que o
perigo faça parte do conteúdo manifesto, como o exibicionismo\idxexibi{} genital,
o masoquismo físico, o travestismo, a promiscuidade\idxpromiscomp{} compulsiva ou o
voyeurismo;\idxvoy{} também naqueles em que a vingança é patente, tais como o\idxdesfa[|)]
estupro,\idxestup{} o sadismo,\idxsadi{} o emporcalhamento de seus objetos, com excrementos\idxcopro{}
ou palavras; ou, ainda, naqueles em que ambos, perigo e vingança,\idxdesu[|)]
estejam ocultos, como ocorre com o paradigma do gênero, o\idxfetic[|)] fetichismo.


\chapter[\textbf{8}\quad A angústia da simbiose e o desenvolvimento\\\hspace*{6mm}
da masculinidade]{{\large\textit{Capítulo 8}}\\ A angústia da simbiose e o\\ desenvolvimento da masculinidade}
\markboth{Dinâmicas: trauma, hostilidade, perigo e vingança}{A angústia da simbiose e o desenvolvimento da\ldots{}}


Mencionei, na Introdução,\idxangusimb[|(] que minha reflexão sobre a perversão foi se
construindo\idxangus[|(] a partir da pesquisa sobre o desenvolvimento da\idxmaessimb[|(]
masculinidade e da feminilidade. O presente capítulo entra em maiores
detalhes sobre a maneira como essas duas áreas de estudo podem estar
entrelaçadas nos primórdios da vida. A partir daí, o capítulo passa a
investigar por que a maioria das perversões é praticada por homens e
não por mulheres. O desenvolvimento das mulheres, desde a primeira
infância, é certamente repleto de trauma, frustração, ansiedade e
conflito. A vanguarda da explicação psicanalítica contemporânea é capaz
de explicar por que as mulheres são tão perversas quantos os homens;
contudo, ela carece de argumentos que expliquem por que não é isso o
que acontece.


Poderíamos apelar para respostas biológicas generalizantes --- os
homens são diferentes das mulheres. Isto sempre poderá servir como
argumento a ser usado, em vez de entrarmos em especificidades. Porém,
acredito que uma explicação mais completa possa ser encontrada no mundo
dos relacionamentos interpessoais, em dinâmicas intrapsíquicas e no
estudo das forças culturais; só quando derrotados é que precisamos
retroceder ao campo do incomprovado --- o biologismo.

Embora o psicanalista raramente tenha a chance de percebê-lo, até
mesmo o bem pode ser excessivo. Investimos a maior parte de nosso
tempo, na teoria e na prática, lutando contra os efeitos do trauma, da
frustração e da privação; sabemos que cuidados maternos negligentes,
desastrados, que se limitam ao mínimo, ou permeados de hostilidade,
causam danos à criança. Contudo, nem mesmo os trabalhos de importantes
analistas teóricos, ao chamarem nossa atenção para os processos mais
benéficos no desenvolvimento da personalidade, serviram como alerta
suficiente contra os poderosos efeitos que a gratificação em excesso
pode causar sobre certos aspectos do desenvolvimento.

O sentimento de bem-estar simbiótico da primeira infância, que tanto
a mãe quanto o bebê vivenciam, pode não apenas dar a base, mas podem
também ameaçar o desenvolvimento psíquico: esta simbiose, quando
excessivamente intensa ou prolongada, pode comprometer o
desenvolvimento da masculinidade. Até mesmo a maternagem mais
competente impõe um fardo sobre o bebê do sexo masculino; uma mãe\idxmaes{} que
queira tentar poupar seu filho desse fardo poderá sufocar completamente
seus potenciais inatos para a masculinidade.


\section{Duas teorias sobre o desenvolvimento masculino}

A masculinidade\idxmascu[|(] nos homens, de acordo com Freud,\idxfreudmascu[|(] advém de três
principais fontes: fatores biológicos, heterossexualidade\idxheteroprim[|(] primária
(desejo pela mãe) --- que começa imediatamente após o nascimento, tão
logo o processo de compreensão se inicie --- e identificação\idxident{} com a
masculinidade do pai, à medida que o conflito edípico é resolvido
(24). Um dos corolários desta teoria é que a masculinidade é a condição
superior na compreensão humana --- sendo o pênis o órgão sexual mais
respeitado e a ambição e as conquistas masculinas as atividades mais
desejadas, para ambos os sexos. Outro corolário é que as mulheres são
inferiores, pois seus genitais são inferiores;\idxcondfinfe{} e que elas, desde o
início, são homossexualmente orientadas, uma vez que seu primeiro amor
tem o mesmo sexo que o seu (33).

No capítulo 2, fiz menção à minha crença de que esta teoria está
parcialmente errada, em virtude de a segunda fonte citada acima --- a
heterossexualidade primária dos homens --- necessitar de correção. A
simbiose entre mãe e bebê, mais do que qualquer outro fator, fornece a
medida desse erro.

Percorramos uma vez mais, brevemente, a teoria, adicionando a ela o
fator da simbiose. Apesar de ser verdade que a mãe é o primeiro objeto
de amor do bebê menino, existe uma fase anterior, durante a qual ele
está amalgamado a ela, antes que ela exista como objeto separado; ou
seja, ele ainda não fez a distinção de seu próprio corpo e psique como
diferentes dos dela --- e ela é uma mulher, com identidade de gênero
feminina. É, então, possível, que o menino não comece heterossexual,\label{cita1}
como Freud presumiu, e sim que ele precise se separar do corpo de
mulher e da feminilidade da mãe, vivenciando um processo de
individuação em direção à masculinidade.\idxmascupap[|(] A heterossexualidade, nos
homens, é uma conquista e, portanto, não, como disse Freud, um
pressuposto; se esta hipótese puder ser confirmada, então a
masculinidade não é a condição de ocorrência natural que Freud dizia
ser. Existe uma forma embrionária de feminilidade. Precisamos
verificar, portanto, se não é isso o que acontece: a primeira, a mais
primitiva das fases, no desenvolvimento da masculinidade,\idxheteroprim[|)] é uma fase
feminina.

Eu não acredito que a sensação de unidade com a mãe encoraje nem ao
menos uma noção primitiva de virilidade, nos primeiros meses de vida;
acredito, antes, que tal sentimento de unidade com uma mãe mulher
precise ser antagonizado. Somente quando a mãe dá suporte ao
desenvolvimento da masculinidade essa união será rapidamente
suplantada, à medida que o desenvolvimento do ego procede.
Ela fará isso, em primeiro lugar, porque deseja e se sente feliz com um
filho homem; ocorrendo esta motivação fundamental, ela encorajará o
desenvolvimento de um comportamento que considere masculino, e
desencorajará os que forem considerados femininos --- um processo que
tem lugar incessantemente, em todos os momentos do dia e da noite. Mas
na medida em que ela se sinta pouco entusiasmada com a perspectiva de
seu filho se masculinizar, ela lhe comunicará sua desaprovação em
relação àqueles comportamentos que ela considera masculinos. (Não
precisamos, aqui, nos preocupar com uma definição de masculinidade que
seja adequada a nós; o que conta é o que essa mãe, como resultado de
sua história de vida e de suas dinâmicas atuais, encara como sendo
masculino na criança.) Os estilos específicos que ela usar para
recompensar e para punir o comportamento do filho moldarão as
perturbações dele em relação à masculinidade, da mesma maneira que
outros estilos de maternagem moldam qualidades, nos bebês, que se\idxfreudmascu[|)]
tornam traços de personalidade nas crianças (por exemplo: 17, 38, 39,
90--93, 95, 96).

Se alguma masculinidade se desenvolve --- e ela começará a ser
detectável a partir de um ano de idade --- então o estágio anterior e
hipotético, com sua capacidade feminizante, ficará encoberto; e, dado
que a aparência comportamental de feminilidade que o bebê sente não é
visível antes de aproximadamente um ano de idade (ou seja, não há como
se distinguir um comportamento masculino de um feminino antes disso),
aquela fase anterior jamais se revelará a quem a observe. Para
confirmar a hipótese da existência de uma fase protofeminina,\idxmascufas{}
precisaríamos encontrar uma evidência na qual uma fase feminina da
primeira infância --- num menino --- tenha se prolongado o suficiente
para poder ser vista e mensurada.\idxmascu[|)] Se encontrarmos uma evidência como
essa, poderemos procurar ver o que poderia ter sido feito para criar
uma feminilidade assim, que persistiu por tempo\idxmascupap[|)] suficiente para que
pudéssemos observá-la. Existe uma: o homem transexual.\idxtranse[|(]

\section{A ``evidência'' transexual}

Hesito em relatar novamente essas descobertas, já o tendo feito
demasiadas vezes (exemplificando, 112, 137, 138, 144, 148) mas isto
ajudará o leitor não familiarizado com os dados e hipóteses a
revisitá-las, pois elas estão relacionadas a fatores que são úteis na
compreensão da masculinidade e da perversão. Muito condensadamente, o
que se segue descreve o desenvolvimento do transexualismo nos homens.
(O transexualismo nas mulheres tem, acredito, uma etiologia diferente
[141], que não se aplica aqui.) O primeiro fator nessa construção é a
necessidade de isolar, dentre as diversas condições em que os homens
vestem roupas de mulher, aquela que eu chamaria de transexualismo. Sua
característica essencial não é, como frequentemente se acredita, o fato
de o paciente reivindicar uma ``mudança de
sexo'' pois outros tipos de pacientes também o fazem.
Trata-se, antes, de não ter havido, em sua vida, nenhuma fase
significativa que ele, um homem cuja anatomia é normal, ou outro
observador qualquer, pudesse reconhecer como masculina. (Há uma noção
rudimentar, construída sobre o conhecimento do transexual, de que ele
é, e será sempre, anatomicamente, um homem; e de que a mãe, que o
batizou como homem, jamais negará sua masculinidade [148]. Assim, saído
daqueles dias da infância, quando um dos dois tipos de comportamento de
gênero faz sua primeira aparição, surgiu esse menino que se
considerava, a si mesmo, menina --- não uma mulher, mas uma menina. Seu
comportamento sempre foi\idxhomosafem[|(] feminino;\idxidenefe[|(] e não se detecta nele uma maior
qualidade imitativa, ou de estar representando um papel, do que em
meninas verdadeiramente femininas. O comportamento, nesses meninos,
tratados a partir da idade de quatro ou cinco anos\footnote{ Eu nunca
vi nenhum mais jovem, mas meu colega Green,\idxgreen[|nn] recentemente, esteve
pesquisando um número maior de garotos femininos, não necessariamente
transexuais (52); em meio a quarenta e cinco, quatro estavam entre os
três e os quatro anos de idade.} até a fase adulta, brota de um
sentimento de feminilidade que se expressa sob a forma de uma convicção
de que eles deveriam ser mulher (apesar de os transexuais não se
proclamarem efetivamente mulheres; eles reconhecem sua anatomia como
masculina). Não há efeminação em seu comportamento, nem na mais
remota infância (efeminação, aqui, implica mimetismo ou caricatura; em
outras palavras: hostilidade e inveja em relação às mulheres,
sentimentos que precisam ser minimizados ou negados; feminilidade,
aqui, implica naturalidade, sem nada de caricatural). Os            %%%cf. essa passagem no original [vírgula solta no tex]%%%
transexuais, ao contrário, têm uma inveja consciente, declarada, não
defendida, comparável àquela que sente uma pessoa que nasceu sem os
membros em relação às mais afortunadas. A naturalidade dessa
feminilidade se faz notar por todos que a observam: familiares,
parentes, companheiros, vizinhos, professores, estranhos, e por nós,
que observamos a criança e o adulto transexual em nossa pesquisa. Esses
meninos, ao atingirem a idade de três ou quatro anos, já estão
sendo confundidos com meninas pelos adultos, independentemente da
roupa que estiverem usando. Ao brincarem, esses meninos querem fazer de
conta que são meninas; eles só assumem papéis de meninas, e são quase
que imediatamente aceitos por elas quando estão fazendo brincadeiras de
meninas, das quais os outros meninos são excluídos. À medida que a infância
progride e eles entram na adolescência e na idade adulta, a
feminilidade não diminui, o desejo de ter um corpo de mulher persiste
e, mesmo sob ameaça, nada faz com que um transexual consiga, nem mesmo
por alguns momentos, imitar\idxidenefe[|)] uma pessoa do sexo masculino (137).

A condição é rara, muito mais rara do que o número de pacientes que
solicitam ``mudança de sexo''. Talvez por
isso, os teóricos afirmem que os transexuais\idxgenetgenitran{} não são os que descrevemos
acima. Com exceção do meu, em quase todos os trabalhos escritos a
respeito do transexualismo masculino, os dados clínicos relatados
revelam episódios, ou longos períodos de tempo, durante os quais o
paciente parecia homem, comportava-se de maneira masculina e teve
experiências heterossexuais ou perversões sexuais assumidas --- além
de outros indícios de que a feminilidade não era do mesmo tipo que a
relatada acima. Isso deveria ser enfatizado: quase todos os homens têm
 sentimentos profundos por seus genitais:\idxtransegeni{} preocupam-se com eles e
extraem deles um grande prazer. Tais órgãos são, ao mesmo tempo, uma
fonte direta de sensações e uma confirmação de que seu papel sexual
está correto, sua identidade de gênero é inevitável, e sua
masculinidade, preciosa. Se essas posições forem ameaçadas, quase todos
os homens erigirão defesas --- mas não os verdadeiros transexuais.
Eles, simplesmente, não querem, não precisam, não estão ligados a seus
genitais masculinos, e não fazem nenhum esforço para preservar esses
órgãos, seja em termos de realidade ou simbolicamente. A perversão, por
outro lado, é intensamente perseguida como forma predileta de dar
satisfação a esses mesmos genitais, e não como maneira de rejeitá-los.

Os fatores etiológicos que se seguem\idxtranseetio[|(] somente estarão presentes
quando existir o quadro acima descrito e, inversamente, somente com
esses fatores etiológicos encontraremos esse tipo de feminilidade.\footnote{ Existe,
entretanto, outro tipo de feminilidade marcante em
meninos --- até mais raro, acredito, do que esse descrito acima ---
no qual a mãe,\idxmaesdese[|nn] conscientemente, se programa para feminizar seu filho,
pois ela havia desejado uma menina durante toda a gravidez; por isso,
deu ao menino um nome bissexual, para marcar seu desejo de que ele
tivesse nascido menina (147).} Encontrei as características descritas
a seguir em todos os casos que preenchem os critérios clínicos
fornecidos acima para o transexualismo masculino; sempre que esses
fatores aparecerem de maneira mais fraca, ou quando alguns estiverem
ausentes, o grau de feminilidade é menor, e o paciente já não terá a
aparência do transexual clássico.

Primeiramente,\idxmaesbiss{} a mãe:\idxpaisfilh[|(] durante sua própria infância,\idxfeminmaes{} pouco valor foi
concedido à sua condição de mulher\idxtransesimb[|(] e à sua feminilidade.\idxmaestran[|(] Quando menina,
a mãe dela a tratou\idxmaesfilh[|(] como se ela pertencesse a um gênero neutro; o pai,
que a apreciava mais, encorajou-a a se identificar com seus interesses
masculinos. Entre a primeira infância e a puberdade, a menina acentuou
de tal modo as qualidades masculinas que veio a desejar ser homem, e
por muitos anos usou apenas roupas de menino; cortava seu cabelo como o
dos meninos, só brincava com meninos e conseguiu competir com eles como
igual, especialmente no atletismo.

Quando começaram as mudanças físicas da adolescência, a garota ---
diferentemente das transexuais do sexo feminino, a quem ela se havia
assemelhado até então --- abandonou todas as esperanças de que um dia
pudesse se tornar homem. Ao invés disso, ela adotou uma fachada
feminina e, finalmente, se casou. O homem com quem se casou --- o pai
do menino transexual --- é um homem passivo, distante; e, embora no
geral não seja afeminado,\idxhomosafem[|)] ele não está destinado a ser uma figura
forte, ou significativa, no casamento. Dele se espera que sustente a
família e que, além disso, simplesmente se entregue à esposa, como
objeto de desprezo.

Dentro deste casamento infeliz nasce o menino --- o futuro
transexual. No entanto, e apesar de as dinâmicas da família serem como
as esboçadas acima, o transexualismo não ocorre em nenhuma das crianças
--- exceto quando nasce um menino que é considerado pela mãe como belo
e gracioso. Esse bebê é a melhor coisa que já aconteceu na vida dessa
mãe. Finalmente, após anos de uma desesperança muda, sem qualquer
sentimento de valor em relação a seu sexo ou a sua identidade de
gênero, cheia de ódio e de inveja pelos homens --- que têm o que ela
queria, mas de que teve que abrir mão --- ela criou um pedaço de si
mesma, saído de seu próprio corpo, como que por partenogênese; sem
necessidade de um marido, ela criou o melhor de si, seu próprio ideal
--- o falo perfeito.\idxmaesfalo{} Este menino não terá o invejado e odiado
empecilho à masculinidade; ela sente que, nele, isso é um direito, já
adquirido de nascença, por sua beleza física; tal sensação é aumentada
por uma exuberante experiência de amamentação, em que ele é um lactente
encantador, que aprecia o corpo da mãe.\idxmaestran[|)] O sentimento de plenitude
simbiótico se estabelece a partir do nascimento, e é ferozmente mantido
pela mãe; pois ela agora abrigou dentro de si a cura para a sua
perpétua e desesperançada tristeza. Uma felicidade transbordante é a
energia que a autoriza, que quase a obriga a manter um contato
corpóreo e psíquico excessivamente próximo com esse bebê --- por
demasiadas horas por dia, e por anos. Ao criar tal simbiose, ela amarra
--- incorpora --- o filho a si mesma, tanto quanto é fisicamente
possível. Ao identificar-se com ele, ela tenta desfazer\idxdesfa{} sua própria
infância traumática, substituir a mãe má; a mãe de agora, e seu bebê,
devem ser totalmente bons. A sensação de plenitude criada na simbiose
se torna, assim, a consagração de uma nova mãe para ela, idealizada e
perfeita.

Quando essas famílias são vistas pela primeira vez, com seus meninos
de quatro anos ou mais, mãe e filho compartilham ainda uma intimidade
excessiva, tocam-se demais, apreciam demais a companhia um do outro,
compreendem-se sem falar. (Isto não tem nenhuma qualidade erótica, seja
para a mãe ou para o filho [60, 61]). Não é tão absoluta quanto na fase
de bebê porque, ao mesmo tempo em que essas mães desejam estar próximas
demais dos filhos, elas dão permissão para que as outras funções
egoicas se desenvolvam neles --- como mover-se, conversar, ler e assim
por diante. A proximidade parece ser mantida num determinado setor,
aquele relacionado com a passagem à feminilidade.\footnote{ Em outras
palavras,\idxgrena[|nn] uma\idxmaesfoca[|nn] simbiose focal.\idxsimbf[|nn] ``Por simbiose focal eu me
refiro à condição na qual um relacionamento simbiótico existe no que
diz respeito ao funcionamento de um órgão especial, ou área do corpo.
[A isto eu acrescentaria: ou função psíquica ou tema de identidade.] Em
geral, as pessoas que participam desse relacionamento simbiótico têm um
nível de desenvolvimento desigual: são mãe e filho, irmãos mais velho e
irmão mais novo, ou até mesmo gêmeos, mais fortes e mais fracos. A
simbiose focal representa o local especial do distúrbio emocional em
ambos os membros do par simbiótico. Mas ela, em geral, se manifesta no
parceiro mais fraco, ou no menor, que permanece funcionalmente
dependente, nessa área específica, da resposta ativa do outro parceiro,
muito além do período de amadurecimento no qual essa função especial,
em geral, se tornaria autônoma'' (56).} Notamos também o
poder da simbiose quando tentamos tratar mãe e filho, e quando a\idxtransesimb[|)]
exigência do tratamento é que eles terminem com a simbiose para que
possa sobrevir a masculinidade. Ambos resistem ferozmente (112,
148). O pai, como era de se prever, não participa: permanece,
obscuramente, em segundo plano.

Qual é o papel do pai quando a fusão da mãe com o filho persiste, e
quando o comportamento feminino vem à tona? Ele deve ser ausente e,
portanto, desprezado. Ele mal é visto --- literalmente --- pelo
filho, durante os primeiros poucos anos de vida. O pai sai de casa para
trabalhar antes que a criança tenha acordado, e volta quando ela já
está dormindo. Nos fins de semana, o pai não fica com a família porque,
encorajado pela mulher, tem permissão para passá-los solitariamente,
dedicando-se a seus passatempos, ou assistindo televisão.

A situação edipiana\idxmascucon{} que se desenvolve vem ratificar a estranheza da
simbiose. Sua mais notável característica é a ausência de conflito. O
garoto jamais desenvolve um relacionamento heterossexual com a mãe (sem
tratamento) e, como resultado, nunca desenvolve o conflito edípico.
Ambos, mãe e filho, são de tal modo um só, têm tamanha liberdade sobre
os corpos um do outro, que não se desenvolve aí nenhuma tensão sexual.
O garoto não deseja a mãe como um objeto separado, do sexo oposto; e
ela não tem nenhum desejo sexual por ele. (Sua falta de interesse em\idxtranseetio[|)]
que ele se masculinize\idxconfemasc{} exemplifica isto.) Só com tratamento, e com o
início da masculinidade, é possível entrever o conflito\idxmaesfilh[|)] edípico com a
sintomatologia neurótica da infância com os quais estamos
familiarizados\idxpaisfilh[|)] quando pensamos em conflito edípico (60, 61, 112).\idxtranse[|)]


\section{A patogenia da ``homossexualidade latente''}

Esta revisão da situação\idxhomoslate[|(] transexual serve para demonstrar que
existem forças que produzem feminilidade num homem. Creio ser correto
(apesar de precisarmos ser cautelosos, pois não foi estudado um número
suficiente de famílias para comprová-lo) que, quando todos esses
fatores estão presentes, e são fortes,\idxconfetran{} a feminilidade será a maior
possível. Se reduzimos a intensidade dos fatores, ou se eliminamos
os fatores, a feminilidade é menos pura. Daí, eu
extrapolo para uma crença: a de que exista, pelo menos, uma tendência
mínima em direção ao transexualismo, no estado de masculinidade comum.
O que nos traz de volta aos princípios enunciados por Freud\idxbisseteor{} já em 1905,
e que nunca foram repudiados,\idxfreudhomos{} em sua teoria ou em suas observações
clínicas, e que se resumem no seguinte:\idxfreudbisse{} a bissexualidade\idxbisse{}
(homossexualidade, protesto masculino,\idxmascupro{} medo das mulheres) faz parte da
constituição dos homens. A única diferença é que, ao que hoje em dia
chamamos ``transexual'' ele chamava ``homossexual.'' (Este não se constitui, de
modo algum, o único significado dado por ele à palavra ``homossexual''.)
Desenvolveremos essas ideias adiante.

Em sua última afirmação com relação à sexualidade\idxsexuateo{} dos homens e das
mulheres, Freud\idxfreudsexua{} afirmou jamais ter conseguido solucionar, em nenhum dos
dois sexos, seu ``protesto masculino''\idxmascupro{} ou
seja, a necessidade que os homens têm de insistir em sua masculinidade,
e de temer um ataque contra ela, e a necessidade que as mulheres têm de
reagir, com a inveja\idxinvej{} do pênis e suas permutações, ao efeito da
castração imaginada (34). Ele atribuiu tais ideias à
``homossexualidade latente'' da qual outra
das manifestações era o desejo proibido --- inconsciente ou consciente
--- por prazer sexual com pessoas do mesmo sexo. Ele achava que o
temor à homossexualidade fosse patogênico em muitas das principais
condições diagnosticadas; e seus seguidores mais próximos estenderam a
lista a tal ponto, que esse fator passou a ser escalado como agente
causador em todos os transtornos psíquicos. À época, essa explicação
foi cuidadosamente analisada, e tanto clínicos quando teóricos a
consideraram uma explicação demasiadamente inclusiva. Alguns sugeriram
que a homossexualidade era, em si mesma, uma defesa, mais do que uma
causa fundamental\idxmaeshoms[|(] (82, 123); outros enfatizaram que a homossexualidade
masculina --- que para Freud parecia originar-se principalmente da
relação problemática do filho com o pai --- poderia ser rastreada até
chegar a distúrbios pré-edípicos nos relacionamentos\idxhomosexpe[|(] entre mãe\idxmaeshomo[|(] e filho
(153, 3, 130).

Em sua talvez mais notável exposição sobre o papel da
homossexualidade latente no desencadeamento de doenças --- o caso
Schreber\idxschreb{} --- Freud acreditou ter demonstrado a etiologia dos estados
paranoides e,\idxparan{} inclusive, da psicose, como motivados pelo temor à
homossexualidade; ele encarava a homossexualidade, nos homens
especialmente, como uma patologia da resolução do conflito edípico do
menino com o pai (27). Essa ideia, posteriormente, foi bastante
criticada por aqueles que enfatizavam o papel da frustração, do trauma
e do conflito nos primórdios da vida. Na visão desses teóricos, a
relação mãe-bebê toma a dianteira na explicação. Alguns (125, 153)
sugeriram que, dentro dos potenciais violentos, hostis, do estágio
oral,\idxforal{} há também, incorporado em Schreber --- e, por extensão, em
outros psicóticos, assim como nas pessoas abertamente homossexuais ---\idxtranseetio[|(]
um desejo de fundir-se\idxtransesimb[|(] outra vez\idxmaestran[|(] com a mãe.\idxmascupap[|(] Para nosso propósito
presente, podemos observar, como o fizeram MacAlpine\idxmacal{} e Hunter\idxhunt{} anos
atrás (88), que aquilo que Freud definiu como sendo a homossexualidade
de Schreber é, na verdade, uma erupção de impulsos\idxtranse[|(] transexuais:\idxmaestran{} o corpo
de Schreber está se tornando feminino. E esse impulso é uma das fontes
do temor à homossexualidade,\idxmaeshoms[|)] que deveria ser melhor definido como\idxmaeshomo[|)]
``temor ao transexualismo''.

Essas últimas modificações, com as quais eu concordo, fazem com que
os conflitos patológicos retrocedam aos primórdios da infância. Todos
eles enfatizam que uma maternagem mal desempenhada, imperfeições inatas
no bebê, ou ambos, arruínam, de modo traumático, o que deveria ser uma
feliz simbiose. Ainda assim, é bom que nos lembremos também de que,
para muitos bebês, existem aspectos não conflitantes, não defensivos,
dessa fusão: para alguns, a experiência foi, muitas vezes, simplesmente
maravilhosa. Nem todos os bebês têm a mesma experiência simbiótica com
as mães. Para alguns, ela é amarga --- e, por isso, eles estão em
perigo. Mas, para outros, ela é extremamente prazerosa. Entretanto,
mesmo em se tratando do último caso, ainda assim, esses bebês
privilegiados podem estar correndo um risco, pois um remanescente
perigoso é deixado para trás, à medida que o menino se esforça por
abandonar a simbiose e seguir em direção à masculinidade. Em outras
palavras: para os meninos, não é só uma simbiose imperfeita que ameaça
seu desenvolvimento; num outro sentido, o cuidado materno
suficientemente bom é ameaçador também e, com mais razão, o são os
cuidados que gratificam em excesso. Ao destacar esse aspecto não
conflitante, talvez eu possa delinear o conceito da fusão, partindo do
caso especial da psicopatologia para o caso geral da psicologia
normativa. O transexual é nossa ponte.

Vimos como uma mãe perturbada,\idxmaesfilh[|(] infeliz, que tem necessidade de
preservar a única experiência de felicidade que ela jamais teve, fará
esforços sobre-humanos para impedir que a dor, a frustração, o trauma e
o conflito se desenvolvam em seu bebê. Ela o cerca com os prazeres
todo-dadivosos de seu corpo, jamais se separando dele, para assim
preservar os sentimentos bons que sua presença produz nela, e também
por seu desejo de protegê-lo da experiência de mau aleitamento, da má
infância da qual foi vítima. Seguimos com o argumento de que uma
maternagem normal, sob as mais favoráveis circunstâncias, segue um
padrão parecido --- embora com menor intensidade e duração (conf.
155). Sabemos que, quando a mãe é suficientemente boa, episódios de
plenitude acontecem. Portanto, é possível que permaneça, embutido em
todo homem, mesmo com a passagem dos anos, pelo menos algum vestígio
daquela plenitude primordial, uma ``identificação
primária'' com sua mãe-mulher\idxmaesiden{} e, portanto, com a condição
de mulher e sua feminilidade.\idxmaesdese{} (As aspas indicam minha crença de que
existe mais nesse processo, especialmente em seus primórdios, do que
aquilo a que geralmente chamamos de identificação; vejam abaixo: o
``biopsíquico.'' ``A dependência
que o bebê tem da mãe [\ldots{}] não envolve identificação; sendo identificação\idxtranseetio[|)]
um estado de coisas complexo, não aplicável aos primórdios da
infância'' [155, p.\,301].) A mesma tendência para se\idxtransesimb[|)]
fundir, nas mulheres, não precisa ser uma ameaça à identidade de\idxhomosexpe[|)]
gênero; ela só aumenta,\idxmaestran[|)] ou pelo\idxmascupap[|)] menos a\idxhomoslate[|)] auxilia a desenvolver a
feminilidade\idxmaesfilh[|)] dentro de si.

\section{A simbiose com relação à identidade de gênero}

A simbiose\idxidengen[|(] no que concerne à identidade\idxident{} de gênero é o aspecto da
simbiose\idxfeminsimb{} que transmite, do bebê para a mãe\idxidentsimb{} e da mãe para o bebê,
atitudes e informação sobre a masculinidade e a feminilidade\idxmascufas{} de ambos
os parceiros. Infelizmente, os mecanismos que conservam o bebê assim,
tão preso à mãe, ainda não foram compreendidos. Acredito que as
teorias psicanalíticas precisem de algumas das descobertas e dos
conceitos dos teóricos da aprendizagem;\idxsocia[|(] estes, embora ainda sejam
rudimentares em termos de pesquisa envolvendo humanos, podem, pelo
menos, nos ajudar a especular --- conforme se segue.

Esses mecanismos, nos primeiros meses de vida, são
``biopsíquicos'';\idxmecan[|(] com isso quero dizer que
estímulos do ambiente (e outros, provavelmente sentidos de forma menos
aguda, provenientes do ambiente interno --- tais como dor, ou auto
percepção) criam alterações no sistema nervoso que funcionam (mais ou
menos) permanentemente como fontes neurofisiológicas de motivação ---
a mudança servindo agora como uma ``memória''\idxnaome{}
não mental. (As aspas indicam que esta é uma experiência
psicologicamente diferente daquilo a que comumente chamamos de memória;
ainda não se descobriu de que maneira esse tipo de memória poderia
estar fisiologicamente relacionado à memória psíquica). Como exemplos,
podemos citar o \emph{imprinting},\idxestam{} o condicionamento clássico, o
condicionamento\idxcond{} visceral e, talvez, algumas formas de condicionamento
operante.

Por não mentais quero dizer que os estímulos e as alterações que
eles causam não têm representação psíquica, nem jamais a tiveram. Esses
novos focos comportamentais, portanto, não são lembrados, no sentido
comum do termo, nem são sentidos por nenhum de nossos sentidos. Eles
não podem ser lembrados, pois nunca existiram como parte da vida
mental. Eles são mais silenciosos do que o que comumente chamamos de
``inconsciente'' e são uma categoria
diferente, a ser acrescentada àquilo que Freud\idxfreudinsti{} se referiu como sendo
fontes de ``pulsões'' (``instintos'') (29). Eles são tão
silenciosos quanto, digamos, os efeitos dos hormônios.\footnote{ De
forma alguma estou sugerindo que tais forças não mentais, que são
domínio especialmente dos teóricos da aprendizagem,\idxsocia[|)] são tudo quanto
existe em matéria dos primórdios do desenvolvimento psíquico.\idxmecan[|)] À medida que
o tempo passa e as representações objetais, sob a influência das
pulsões, se reúnem em memórias e fantasias, a influência da mãe ajuda a
fazer com que o aprendizado do bebê se torne enriquecido, cognitivo,
mental.} (Ver essa discussão no capítulo 6, acerca do livre-arbítrio e
do determinismo.)

Se tais noções se aplicam à pesquisa acerca do desenvolvimento do
bebê, então o desenvolvimento da personalidade não pode ser totalmente
compreendido através da técnica usada numa psicanálise. Como diz Racker\idxracker{}
(119, p.\,79), ``O estudo da transferência\idxtransf{} tem sido uma
das mais importantes fontes de conhecimento com relação aos processos
psicológicos da criança.'' Exatamente: processos, mas não
experiências reais. Freud\idxfreudestru{} faz alusão a isso ao advertir que é frequente
 a estrutura de caráter egossintônica\idxinfanego{} não ser passível de modificar-se
pela psicanálise (34). Precisamos de uma observação minuciosa,
sistematizada, do comportamento infantil em seu estado natural --- o
que inclui, especialmente, a mãe,\idxmaestran[|(] o ambiente que\idxtransesimb[|(] ela cria, e quais das
atitudes e comportamentos que ela adota afetam o bebê --- para,
assim, nos munirmos de mais informações acerca do desenvolvimento da
personalidade.

Através desses meandros, estou sugerindo um modelo incipiente, mas\idxtranseetio[|(]
talvez algum dia utilizável, sobre o qual possamos acumular dados
experimentais, observações e alguma teoria a respeito dos estágios
primordiais do desenvolvimento psíquico. De momento, este modelo me
serve como racionalização, uma espécie de conforto, neste período de
poucos dados, para ``explicar'' no
transexual, a transmissão da feminilidade da mãe\idxpaisfilh{} para seu bebê menino,
de tal forma que,\idxmaesfilh[|(] por volta de um ano de idade (aproximadamente) ele se
comporta de maneira abertamente feminina. Na realidade, com certeza,
os únicos dados de que dispomos no momento são: que uma mãe, dotada de
uma determinada forma de bissexualidade,\idxmaesbiss{} e um pai que é intensamente
passivo e incapaz de estar próximo de seu filho, têm um lindo e
gracioso bebê; este bebê estimula a mãe a constituir com ele uma
simbiose, excessivamente próxima e arrebatadora, da qual o restante do
mundo é excluído, e por tempo demasiado. Então, à época em que o
primeiro comportamento de gênero passa a poder ser avaliado, ele se
revela feminino. Ainda não dispomos de outros dados. O que acontece
dentro dessa simbiose ainda não foi examinado, nem grosseira nem
microscopicamente, e por isso eu introduzi, nesse vácuo, este modelo
teórico.

Mas esta teoria não é crucial para nosso principal argumento, que
diz respeito ao papel da angústia\idxangusimb{} da simbiose\idxangus{} na formação da
masculinidade. Por isso, é suficiente dizer que o anseio de voltar a
constituir uma unidade com a mãe, há muito conhecido pelos analistas,
permanece como fundamento permanente da estrutura do caráter e,
dependendo da experiência de vida de cada um, passada a primeira
infância, pode servir como local de fixação, mais forte ou mais fraco,
para a regressão.\idxhomosregr{} (Ele é provavelmente latente em todo
``ato'' de regressão.) Estou apenas
enfatizando agora --- uma vez mais, pois o tema é bem conhecido desde
os trabalhos iniciais de Freud\idxfreudhomos{} --- que essa regressão é
frequentemente acompanhada por aquilo a que Freud chamou de
``homossexualidade'' e que eu, de minha
parte, encaro como ``uma tendência
transexual.'' Recordemo-nos da descoberta de que o medo
da mudança de sexo é onipresente (alguns dizem universal) em psicóticos\idxpsico{}
homens, mas pouco frequente nas mulheres (cujos sistemas
delirante-alucinatórios, quando sexuais, são mais frequentemente
heterossexuais [62, 79, 81, 107]). Observamos também que, na população
em geral --- na maioria das culturas e na maior parte das eras sobre\idxtranseetio[|)]
as quais dispomos de informação\idxmaesfilh[|)] --- os homens parecem mais preocupados\idxtransesimb[|)]
em defender sua masculinidade\idxmaestran[|)] contra ataques,\idxidengen[|)] reais ou imaginários, do
que as mulheres em perder sua feminilidade.

\section{A angústia da simbiose}

A angústia\idxangus[|(] da simbiose,\idxmaesproc[|(] portanto,\idxangusimb[|(] é o medo\idxhomosneur[|(] que alguém sente de não
ser capaz de permanecer separado da mãe. Vamos, agora, observar mais
atentamente esse medo, e ver de que forma ele contribui para o
desenvolvimento da masculinidade.\idxmascumud{} O argumento começa com a observação
de que a sempre presente memória de unidade com a mãe\idxmaesiden{} atua como um ímã,
puxando a pessoa de volta, no sentido de repetir a experiência oceânica
de que desfrutou quando em contato com o corpo da mãe. Isto,
entretanto, é um negócio arriscado para quem batalhou --- e conseguiu
--- ficar independente dela. É especialmente arriscado quando um dos
aspectos dessa independência é formado por aqueles comportamentos a que
 chamamos de\idxmascag{} masculinidade.\idxagresmasc[|(] Uma parte vital do processo de separação da
mãe, portanto, é a de libertar-se de seu corpo de mulher e de sua
psique feminina.

Chamo de angústia de simbiose ao medo universal que faz com que
todos os homens sintam que sua virilidade, e sua masculinidade, estão
ameaçadas, obrigando-os a construir, em sua estrutura de caráter,
defesas sempre em guarda para impedir que eles esmoreçam e sejam
puxados de volta ao estado de fusão com a mãe.\footnote{ Tanto nas
mulheres quanto nos homens, embora, nas mulheres,\idxangusmul[|nn] o medo de ser como a
mãe em corpo e identidade de gênero, em geral, não se constitua em
perigo.} Embora ostensivamente montado para nos proteger de ameaças
e agressões externas, esse medo precisa finalmente instalar-se, contra
nosso próprio anseio interno, primitivo, de nos conservar em união com
a mãe. Se isso for assim, então vamos entrever o esboço de um dos
principais fatores constitutivos da masculinidade, e que está tão
entrelaçado a outros que, à época em que o assim chamado comportamento
masculino começa a se evidenciar (com a idade de um ano
aproximadamente), a masculinidade já está inextricavelmente permeada
pelos efeitos da angústia da simbiose. Esta última, potencializada pelo
vigor biológico da virilidade (nos peixes, lagartos, ratos, macacos e
homens) produz, nos machos, um índice de agressividade\idxmascag{} e de
competitividade maior,\idxagresmasc[|)] em comparação às fêmeas. Isso sugere que a
masculinidade, como a observamos em meninos e homens, não existe sem o
componente do esforço contínuo para se afastar da mãe, quer
literalmente, no primeiro ano de vida, quer psicologicamente, no
desenvolvimento da estrutura de caráter, que obriga a mãe\idxmaesmald{} interior a ir
para o fundo e para fora da consciência. Mencionarei a ideia de que a
mãe, em sua representação como criatura má, odiada, pode também cumprir
a tarefa de permitir que a mãe simbiótica seja recalcada; dificilmente
gostaríamos de nos fundir a uma bruxa. Podemos conjecturar se, em seu
nível mais primitivo, a perversão não é o que há de mais definitivo em
termos de separação,\idxpervproce{} que é o assassinato da mãe (ainda mais do que,
como Freud pode ter sentido, o assassinato do pai). Seria irônico se
algumas das formas que a masculinidade assume, se alguma de sua força,
insistência, ferocidade --- o machismo --- tiverem como \textit{requisito}     %%require (it.) anlagen of femininity
formas incipientes de feminilidade;\idxmascufem{} o potencial de ser feminino é uma
tentação inaceitável, que precisa ser combatida com comportamentos e
atitudes que a sociedade rotula como
``masculinos''. Talvez fique mais claro,
então, por que a maioria dos homens parece tão sensível em relação à
sua masculinidade.

Greenson,\idxgrens[|(] ao tratar de um garoto transexual cuja mãe eu analisei,
chegou a conclusões similares. Ele fala em
``desidentificar-se'' da mãe. Não é adequado
citar o trabalho de Greenson como confirmação objetiva, uma vez que
trabalhamos juntos por anos, mas sua clara exposição, já publicada
alguns anos atrás, é digna de nossa atenção. (Conservo suas citações
bibliográficas para que o leitor possa fazer uso delas.)

\begin{quote}
É minha impressão clínica que o temor à homossexualidade, no
neurótico, que constitui a base do medo de perder a própria identidade
de gênero,\idxiden{} é mais forte e mais persistente nos homens do que nas
mulheres (Greenson, 1964). Acredito que nós todos concordaríamos
com o fato de que, nos primórdios da infância, tanto meninas quanto
meninos formam uma identificação simbiótica primitiva\idxpaisiden{} com a pessoa que
lhes dispensa os cuidados maternos, com base na fusão das primeiras
percepções visuais e tácteis, atividade motora, introjeção e imitação
(Freud,\idxfreud{} 1914, 1921, 1923, 1925; Fenichel,\idxfenic{} 1945; Jacobson,\idxjacob[|(] 1964). Isso
resulta na formação de um relacionamento simbiótico com a mãe\idxidentsimb{} (Mahler,\idxmahle{}
1963). O próximo passo, no desenvolvimento das funções\idxegoi{} egoicas\idxinfanego{} e das
relações objetais, é a diferenciação da autorrepresentação das
representações objetais. Mahler\idxmahle{} (1957), Greenacre\idxgrena{} (1958), Jacobson
(1964), e outros elucidaram como as diferentes formas de identificação
desempenham um papel central nessa transição, à medida que o
amadurecimento torna possível progredir da incorporação total às
identificações seletivas. A capacidade de diferenciar entre semelhanças
e contrastes propicia a capacidade de discriminar entre o dentro e o
fora e, em última instância, entre o \textit{self} e o não \textit{self}. Nesse
processo, a criança aprende que ela é uma entidade separada, diferente
da mãe, do cachorro, da mesa etc. Contudo, ela aprende também
gradualmente, por identificação, a se comportar e a desempenhar certas
atividades da mesma maneira como a pessoa que exerce a função de mãe o
faz, tais como falar, andar, comer com uma colher etc. Tais atividades
não são cópias, mas são modificadas de acordo com a constituição da
criança e com seus dotes mentais e físicos. O estilo de seu
comportamento e de suas atividades é adicionalmente alterado por sua
posterior identificação com outras pessoas que a rodeiam. O que
chamamos de identidade parece ser o resultado da síntese e da
integração de diferentes autorrepresentações isoladas (Jacobson,\idxjacob[|)] 1964;
Spiegel, 1959). (61, pp.~371--372)
\end{quote}

Finalmente, Greenson observa:

\begin{quote}
O menino precisa tentar renunciar ao prazer e à proximidade
reconfortante que a identificação com a pessoa que exerce os cuidados
de mãe propicia, e formar uma identificação com o pai, figura menos
acessível. O resultado será determinado por diversos elementos.\idxhomosneur[|)] A mãe
deve estar disposta a permitir que o garoto se identifique com a figura
paterna. Ela pode facilitar esse acontecimento mostrando sua apreciação
e admirando genuinamente os traços e\idxangusimb[|)] habilidades masculinos do menino,
e desejando que ele continue a se desenvolver ao longo dessa linha (A.\idxangus[|)]
Freud,\idxannaf{} 1965).\idxgrens[|)] (61, pp.~372--373)
\end{quote}

\section{A perversão}

Não nos devemos deixar enganar por aqueles homens que parecem não
proteger sua\idxtravemmasc{} masculinidade,\idxmascutra{} retrocedendo, através dela, à identificação
primordial com a mãe para criar perversão. Penso em especial nos
homossexuais afeminados\idxhomosafem[|(] e nos travestis fetichistas. Embora tais homens
queiram excessivamente ser como (fundir-se com) suas mães ---
``rituais perversos desempenham a função de desfazer\idxdesfa{} a
separação''\idxpervproce[|(] (1, p.\,29) --- creio que os rituais sirvam,
ao mesmo tempo, para promover separação; a característica essencial, à
base dessas perversões, é que a masculinidade está sendo preservada.
Esses homens, por meio de suas perversões, retêm a potência\idxtravempote[|(] de seus
pênis,\idxpenisfant[|(] seu senso de virilidade: o cerne de masculinidade. Eles têm, ao
menos, alguma masculinidade, a ser preservada a qualquer custo. Foi por
isso que considerei o transexualismo não como uma perversão mas, mais
simplesmente, como uma variante sexual. O transexual jamais teve um
episódio de masculinidade em sua infância, nem se pode encontrá-lo no
transexual adulto; ao passo que nos homossexuais afeminados, em
travestis fetichistas e em outros homens com transtornos de gênero, a
masculinidade será facilmente encontrada, em sua infância e na vida
adulta. Talvez as perversões sejam linhas de fratura resultantes desse
processo de oscilar entre o desejo de fusão e o desejo de separação; e,
embora eles possam estar trancafiados e concretados nos não perversos,
encobertos no neurótico, e conservados abertos, como canais, nas
perversões, essas falhas, não obstante, correm para as profundezas da
identidade nos homens, exigindo um trabalho de reparação e uma
vigilância maiores do que nas mulheres. (Isto não quer dizer que as
perversões sejam, simplesmente, o produto de uma perturbação no
processo de separação executado por ambos, mãe e bebê. Antes, estou
sugerindo que a derrota em efetuar uma boa separação pode ser uma
matriz que encoraje a perversão, quando ocorrem eventos tardios, na
infância, que requeiram esse desvio no desenvolvimento sexual.)

Lembrem-se de o quanto essas ideias se coadunam com as condições nas
quais um homem, detentor de alguma masculinidade, identifica-se tanto
com as mulheres ao ponto de adotar alguns de seus comportamentos, e até
mesmo suas roupas. Repetindo: tal aberração é uma perversão quando a
feminilidade (ou a efeminação) foi determinada por um trauma\idxtrauma[|(] ou por uma
frustração da infância, que é recordada de modo inconsciente, e que
está sempre ativa, resultando num conflito que necessita de contínua
resolução; sua resolução é a perversão. O homossexual afeminado
valoriza seu pênis; dele obtém prazer; está focado nele. Ele não é
feminino (comportamento que deveria ser indiferenciável da feminilidade
numa mulher), ele é, antes, uma caricatura de feminilidade.\idxfemincari{} Sua
identificação com as mulheres está encoberta pela hostilidade.\idxhost[|(] Ele tem
bons motivos para sua raiva secreta: quando criança, a mãe lhe ofereceu
os prazeres de uma proximidade excessiva, mas somente quando ele, em
função do abuso materno, renunciou às suas tendências em direção ao que
a mãe considerava comportamentos masculinos (144). Sua masculinidade
está ali, preservada, disfarçada no mimetismo --- hostil ---
efeminado\idxhomosafem[|)] (144).

Vimos masculinidade\idxtravemmasc{} no travestismo também.\idxmascutra{} O adulto travesti comum é
um homem que, por mais que sua identidade de gênero esteja danificada,
vive a maior parte do tempo, de modo suficientemente confortável, como
alguém do sexo masculino. Intermitentemente, contudo, ele é impelido a
usar seu disfarce de travesti. Ele age assim, precisamente, em prol de
seu pênis: quando está excitado, e para poder ter uma experiência
peniana gratificante. É quando ele disfarça sua masculinidade que ele
obtém o ápice da virilidade, a saber, com uma\idxpenisfant[|)] potente ereção.\idxtravempote[|)]

A hostilidade da perversão (e sua versão mais atenuada, encontrada
na pessoa ``normal'') é uma reação a um
trauma, um voltar-se para fora para encontrar uma vítima que se encaixe
em sua vingança. Mas se a pessoa não se sentiu vitimada, então não terá
suficiente motivação. A confirmação para esta tese advém daqueles a
quem chamo de transexuais. Existe, neles, uma estranha ausência de
hostilidade, uma suavidade em nosso relacionamento, descrito com mais
detalhe em outra parte (147), e que tem persistido durante anos,
infindável, até o momento; e que é diferente do que eu jamais senti em
outros pacientes.\idxhost[|)] Eles me tratam da mesma maneira como as mães os
trataram: como coisas, como apêndices --- em vez de uma pessoa
separada. Eu jamais me senti em perigo com um transexual --- mas um
número dolorosamente grande de outros pacientes meus, com distúrbios
de identidade de gênero, só escaparam de serem assassinos (de outrem ou\idxtrauma[|)]
meus) através de um trabalho terapêutico extenuante e assustador.\idxpervproce[|)] Um
desses casos será relatado em alguma parte,\idxmaesproc[|)] no decorrer deste livro.
(146)


\section{Discussão}

Como foi\idxangus[|(] descrito aqui, a masculinidade,\idxmascsu[|(] nos homens, começa como um
movimento para se distanciar da arrebatadora e perigosa, para sempre
recordada e ansiada, simbiose mãe-bebê.\idxmascupap[|(] O bebê menino, que deve se
tornar masculino,\idxmaesmasc[|(] precisa ser abençoado com uma mãe que o encoraje a se
separar dela, a se individuar apropriadamente. Se ela não conseguir lhe
dar permissão para que o faça, prolongará e, deste modo, aumentará seu
estágio primário de feminilidade; se, por outro lado, ela for brusca
demais no combate a tudo quanto considerar feminino, poderá produzir o
caráter fálico congelado, brutal, que resulta quando as possibilidades
de até mesmo um retorno momentâneo a ela ficam barradas (120, 136).

Certamente estamos familiarizados com os efeitos traumáticos da
angústia,\idxangusimb[|(] e de como ela se torna um fator central na motivação. No
caso da angústia da simbiose, nós temos um problema: precisamos
justificar um desejo quase universal, por parte dos bebês, de se
separarem de um estado de plenitude, arriscando, assim, a angústia.
Podemos mensurar a força desse desejo, baseados não só no fato de quase
todos os homens desenvolverem algum grau de masculinidade, apesar da
simbiose inicial; é também sabido (137) quão monumental é o esforço que
se requer, por parte da mãe\idxmaesfilh{} do menino transexual, para manter o estado
de simbiose, que ela considera tão precioso. Levando em conta a
necessidade de se separar da mãe, alguns observaram o quanto uma mãe,
respeitadora da virilidade e da masculinidade, recompensará os
comportamentos que ela considera masculinos, e desencorajará os que ela
não considerar assim; monitorará seus cuidados, fazendo com que sejam
adequados ao humor, à capacidade e ao estágio de desenvolvimento da
criança; e aproximará o marido, para que forneça os necessários
reservatórios de masculinidade para o filho. Adicionalmente, talvez o
ódio à maldade materna\idxmaesmald{} ajude o menino na separação. Se, além disso,
postularmos também mecanismos inatos, que empurram o bebê em direção
à separação favorecendo comportamentos que se encaixam no que a mãe
considera masculino, poderemos ter uma explicação suficientemente boa
para o prazer --- o sentimento de autoconfiança --- que o motivará a se
separar dela.

Uma vez que o menino tenha sido reconhecido como homem, e tenha
começado a fixar este sentimento de virilidade e de orgulho da
masculinidade em sua estrutura de caráter, torna-se crucial que ele
erga uma barreira --- a angústia\idxangushom[|(] da simbiose\idxhomossimb[|(] --- contra sua tendência
de regressar para o aconchego materno.\idxconfeangu{} Neste processo, a angústia da
simbiose exerce uma função normalizadora essencial, possibilitando o
processo de desidentificação e permitindo que a individuação prossiga.
Sem essa barreira, a feminilidade persistirá, a situação edípica\idxconfe[|(] não
será percebida como conflito, o conhecimento do corpo da mãe como
objeto separado e como objeto de desejo (uma raiz para a posterior
heterossexualidade) não se desenvolverá, e o resultado feliz --- a
masculinidade --- não se estabelecerá.

Talvez uma parte do mal-estar que os homens sentem em relação às
mulheres --- o mistério a respeito do qual poetas (homens)
sentimentalizam --- reflita a necessidade de que se erga essa
barreira, contra o desejo de se fundir com a mãe.\idxmascupap[|)] Esta, então, poderia
ser mais uma contribuição às múltiplas causas da homossexualidade;
paralelamente a ela, e talvez ainda pior, penetrar num corpo feminino,
aí seria arriscar demais. O menino teme perder sua masculinidade e seu
senso de virilidade não somente através da perda de seu precioso e
frágil pênis, mas também porque ele poderia ser subjugado pelo desejo
de se tornar, uma vez mais, uno com a infinitude negra da feminilidade
interior.\footnote{ ``O fracasso geral no reconhecimento
da absoluta dependência\idxangushom[|nn] inicial contribui para o medo da \textsc{mulher}, que é
o quinhão de ambos, homens e mulheres''\idxhomossimb[|nn] (155, p.\,304).}
Isto poderia explicar, em parte, por que tantos homens não conseguem
viver amorosamente com as mulheres, exceto por curtos períodos, e por
que alguns, após o\idxmascsu[|)] coito,\idxangusimb[|)] precisam\idxangushom[|)] se\idxmaesmasc[|)] levantar\idxhomossimb[|)] e partir, o mais
rapidamente possível.\idxangus[|)]

Muito do que digo aqui não é novidade. Por exemplo, numa carta a
Freud, datada de 06 de novembro de 1927, Lou Andreas-Salomé\idxsalome[|(] escreve:

\begin{quote}
Pois as mulheres nunca vivenciaram o grande choque de descobrir
a ausência de seu próprio pênis na mãe. No homem, é essa descoberta que
origina, em primeiro lugar, a situação do incesto;\idxinces[|(] é o que o confirma
como homem na relação com a mulher que o gerou. Assim, mesmo antes de a
situação de incesto ter surgido, ele foi confrontado com uma
experiência avassaladora, que é totalmente reprimida, e à qual ele não
retornará jamais no decorrer da vida. Enquanto a menina se preocupa
com coisas reais e experiências sensuais, o homem é assombrado, nos
mais recônditos recessos de sua mente, por um romantismo velado e
estranho, um emocionante fragmento de irrealidade que,
inevitavelmente, continua a exercer uma influência secreta sobre sua
vida amorosa. Enquanto, ajudado por seus temores de castração, ele
tenta resolver a situação de incesto dentro de si mesmo, ele desvia o
desejo secreto que sente pela mãe procurando degradá-la, juntamente
com todas as representantes de seu sexo, mais ou menos à categoria de
uma prostituta --- não obstante permaneça nele um relacionamento
primordial com a ``figura da mãe dotada de
pênis'' que era seu igual sexual e, no entanto, ainda
muito superior a ele, pois, além de protegê-lo, ela o ultrapassava. É
preciso que ele encontre alguma solução para esta situação; será que
não é isso que ele faz, naquele amor do ``tipo anaclítico
masculino'' que você descreveu para nós (in
``The Ego\idxfreudegoid{} and the Id'')? Isso é algo
que se pode compreender por si só --- como resultado de uma luta
contra o incesto, e por conta da ternura reverencial exagerada que a
substitui. Se o processo é vitorioso isto é, em parte, resultado da
experiência primordial que, nesse sentido, parece restabelecer seu\idxinces[|)]
lugar no mundo real.

Talvez o fetichista\idxfetic{} seja precisamente uma pessoa para quem esse
processo fracassou; então, ele o condensa num fragmento absurdo da
realidade --- uma bota, um feixe de cabelo, ou o que seja --- que
passa, então, a se revestir de fantástico esplendor. Mas é exatamente
esse absurdo que explica a significação total do êxito do
desenvolvimento libidinal no indivíduo normal. Sempre me pareceu que o
homem, apesar de seu ajuste mais consciente e mais firme à realidade,
possuía, entretanto, a centelha de uma capacidade de romantismo,
``idealismo'' ou algo de profundamente
imaginativo --- chame como quiser --- maior do que a mulher; e, por
essa razão, é ele o mais criativo.\idxcriavs{} Ele teve que renunciar de forma mais
profunda diante daquela decepção primordial; isto fez com que
conservasse suas faculdades mais imaginativas intatas, sem máculas
produzidas pela realidade, daí elas poderem irromper em atividade
criativa; ao passo que a mulher, apesar de todas as tendências
sentimentalistas de sua parte, nunca abandou inteiramente a realidade
podendo, assim, adotar um relacionamento sóbrio e harmonioso com ela.\idxconfe[|)]
(116, pp.~168--169)\idxsalome[|)]
\end{quote}

Da mesma forma, anos atrás, Boehm,\idxboehm{} discutindo ``o
complexo de feminilidade\idxcompl[|(] nos homens'' (ou seja, aspectos
da masculinidade), descreveu o papel do medo que surge nos meninos e
nos homens em virtude da inveja\idxfemininve[|(] que sentem em relação à condição
feminina das mulheres:

\begin{quote}
O ódio às mulheres se origina da [\ldots{}] angústia da
castração.\idxcasta{} Pelo fato de os meninos imaginarem a concepção e o parto
como coisas tão complicadas e estranhas, e porque esses processos são
tão misteriosos para eles, eles têm um desejo ardente de lidar, dentro
de si mesmos ou em outra parte, com uma intensa inveja dessa capacidade
das mulheres\ldots{} A inveja da capacidade que as mulheres têm de ter
filhos (que eu chamo, para simplificar, de `inveja do
parto') é um considerável incentivo à capacidade
produtiva dos homens.
\end{quote}

Existe, contudo, uma outra forma que a inveja masculina dedicada aos
atributos femininos pode assumir: a saber, a inveja dos seios\idxseios{} das
mulheres. Acredito que, quando crianças, costumamos invejar as
pessoas quando elas têm alguma coisa a mais do que nós. É inevitável
que os seios femininos despertem inveja nos meninos, e que suscitem o
desejo de possuir tais órgãos; especialmente porque os seios, como
mencionei acima, representam, no inconsciente dos meninos, um
formidável pênis. Mesmo sem considerar esse ponto, contudo, eles têm
uma função diferente do que qualquer outra coisa que os meninos
possuam\ldots{}

Acabei de dizer como nossa inveja é despertada sempre que alguém tem
alguma coisa a mais do que nós mesmos. Podemos, além disso, dizer que,
quando alguém tem alguma coisa diferente, alguma coisa que nós nunca
poderemos ter, experimentamos um sentimento de inferioridade. O tipo
de coisa ``diferente'' não importa muito. Já
nos disseram tantas vezes, e toda análise de paciente mulher o
confirma, que as garotinhas têm inveja\idxinvej{} dos meninos porque eles detêm o
poder de expelir sua urina em jatos contínuos, a uma distância maior e
mais alto do que elas jamais poderiam ser capazes. Mas muitos homens
conseguem se lembrar da seguinte experiência, de seus dias de criança:
a de como suas irmãzinhas conseguiam despejar um jato mais volumoso de
urina do que eles, e de como isso fazia um ruído diferente, mais denso,
no quarto. Um de meus pacientes se lembrava perfeitamente de como o
oprimia e envergonhava o fato de ser incapaz de produzir o mesmo som ao
urinar. Mais tarde, na vida, seu grande hobby era brincar com uma
mangueira, que lhe possibilitava escolher entre jatos volumosos ou
finíssimas pulverizações de\idxfemininve[|)] água.

Todos os fenômenos que descrevi brevemente até aqui podem ser
resumidos na expressão: ``o complexo de feminilidade\idxcompl[|)] nos
homens.'' (6, pp.~456--457)


Pode-se\idxtransesimb[|(] argumentar\idxmaestran[|(] que a mãe\idxmaesfilh[|(] 
do transexual,\idxhostmaes[|(] com seu intenso e
patente ódio pelos homens em geral, não conseguirá deixar de transmitir
a seu filho um sentimento de raiva. Ela, certamente, transmite esse
sentimento em relação aos homens em geral; mas o que observei foi que,
nesse filho, ela encontra uma exceção à regra; ele sabe que a
depreciação feita ao pai não inclui a ele, o transexual. Este menino em
particular --- esse falo bonito\idxmaesfalo{} --- é a tal ponto o orgulho de sua
mãe, o final de sua desesperança, o arremate feliz de seu corpo, antes
inadequado, a felicidade de sua vida, que não existe razão para esperar
que ele vá sofrer --- desde que permaneça imerso na simbiose. Pode-se
especular que ele sofra, que ele seja perpassado pela\idxangushos[|(] angústia,\idxanguinfa{} que ele
seja psicótico e que mal consiga disfarçá-lo através da sintomatologia
transexual; mas isto é uma ``explicação'',
que ganha sua força com base na expectativa, e não na observação. O que
não quer dizer que não existam pessoas que usam uma forte identificação
com as mulheres para defenderem a si mesmos contra essa angústia
avassaladora; há, até mesmo, os que resvalam para a psicose; eu vejo
muito mais pacientes desse tipo do que transexuais. Porém, eu discordo
daqueles que dizem que essa angústia primitiva está presente também
nesses casos raros, nos transexuais do sexo masculino. Pois, para que
essa angústia aparecesse também nesses últimos, eu teria que ou
acreditar, como alguns kleinianos\idxklein{} o fazem, na existência de um estado
de terror inerente a todos os bebês, independentemente do tipo de
maternagem\idxhomosexpe{} que lhes foi administrada (o que não explica a razão de não
serem todos transexuais) ou, então, que todas essas mães estão
infligindo traumas monumentais, terríveis (embora ocultos) em seus
bebês do sexo masculino, suficientes para produzir essa colossal
defesa --- porém, por demais sutis para que possam ser observados.

Deveríamos suspeitar que a\idxmaeshost{} hostilidade\idxinfanangu{} infligida sobre o transexual
pela mãe\idxmaeshomo{} pudesse ter ajudado a causar sua feminilidade; essa
hostilidade causa isso, nos homossexuais efeminados e nos travestis (3,
144). Uma vez que ela expressa seu ódio e sua inveja de outros homens,
não parece plausível que essa mãe possa se abster em relação ao filho,
não importa o quanto ela pudesse, conscientemente, tentar. Embora isso
seja provável (e essa foi a posição que assumi anos atrás, quando
comecei a estudar a simbiose) o fato é que a hostilidade, simplesmente,
não apareceu. Talvez em virtude da sua sutileza, ou de minha inépcia,
ou de sua ausência --- e tenho me debatido em torno dessas
possibilidades.

Presentemente, tudo quanto posso afirmar é que não encontrei essa
hostilidade dentro da simbiose. Se, ainda assim, ela estiver lá, um dia
ela haverá de marcar sua presença. A reversão de gênero\idxidenrev{} é uma mudança
dramática de identidade; não é plausível que ela seja causada por
alguns débeis sussurros da vontade da mãe. Se o ódio ou suas
permutações são fortes, os efeitos se refletirão também no bebê, sob as
formas com as quais aqueles que estudam as crianças já nos
familiarizaram há tempos:\idxtransepres[|(] desenvolvimento anormal\idxegoi[|(] de um ego\idxinfanego[|(] desprovido
de gênero, como atraso ou precocidade nas funções intelectuais, na
motilidade ou na fala; transtornos em funções fisiológicas como o sono
e a alimentação; turgidez muscular e choro; desenvolvimento fora de
fase, desintegrado; afetos turbulentos --- como raiva, terror,
depressão, apatia, ansiedade, retração --- inapropriados, excessivos,
bizarros ou inoportunos; desenvolvimento das relações objetais
distorcido ou atrasado, com a família ou com estranhos, com humanos ou
com animais, com objetos animados ou inanimados; curiosidade reduzida
ou ausente, como em relação a jogos ou fantasias; transtorno do
pensamento envolvendo outras áreas que não as relacionadas ao gênero; e
assim por diante. Posso afirmar que nenhum desses efeitos, praticamente
nunca, está presente nos garotinhos a quem chamo de transexuais.

De forma alguma esse sentimento de fusão arrebatadora deve ser
confundido com a ``fusão, encontro, e ausência de
diferenciação entre o \textit{self} e o não \textit{self}'' (90, p.\,309) que
se nota nas crianças que são produto da simbiose psicótica. As mães dos
transexuais conservam seus bebês no colo por tempo demasiado e com
proximidade excessiva, mas elas não restringem sua motilidade (o que
seria um outro sinal de hostilidade materna), o que poderia
desencorajar a descoberta do mundo do não \textit{self}. Essas mães ajudam seus
filhos a definir os limites entre o \textit{self} e o mundo exterior em todos os
aspectos --- exceto quanto à sua condição de mulher e à sua
feminilidade como mãe. Elas também encorajam a criatividade\idxcriatran{} do menino e
o crescimento de outras funções egoicas, de tal modo que esses garotos
são tipicamente vivazes, atentos e artísticos (60, 137) --- sugerindo,\idxhostmaes[|)]
mais uma vez, que a hostilidade materna é\idxmaesfilh[|)] fraca ou ausente na simbiose.\idxangushos[|)]

À medida que os garotos crescem, eles não são isolados mas, ao
contrário, se entrosam facilmente com os colegas em jogos e estudos. É
só quando entram na fase em que são vítimas de um assédio impiedoso na
escola, por serem femininos, é que eles se afastam dos outros.

Esta explicação pode ser difícil de digerir, pois sugere que um
desvio importante na estrutura do caráter pode ser criada sem trauma.
E, no entanto, pressões\idxtransepres{} que não envolvem trauma estão entre os mais
importantes fatores no desenvolvimento da estrutura do caráter, tanto
da ``normal'' quanto da\idxtransepres[|)]
``anormal''.

Este capítulo é sobre a masculinidade nos homens. Apesar disto, a
tese sobre o papel da simbiose primitiva deveria ser testada também
nas mulheres: a feminilidade aumenta, como a tese levaria a prever,
quando existe uma simbiose saudável da mãe com seu bebê do sexo
feminino? Será que a masculinidade é encorajada nas meninas, quando a
intimidade na simbiose é diminuída? Há indícios de confirmação: as
mulheres mais masculinas de que se tem notícia, as transexuais do sexo\idxtransetran{}
feminino,\idxmascumul{} parecem se desenvolver a partir do seguinte: elas não são
consideradas bonitas ou graciosas ao nascer; elas não são bebês
fofinhos; existe uma simbiose flagrantemente incompleta, com mães que
não estão acessíveis psicológica ou fisicamente nos primeiros meses,
ou por um tempo ainda maior de vida --- e não há ninguém com aptidão
para substituí-las; a menina é encorajada, especialmente pelo pai, a
ser forte e masculina, ou seja, a prescindir da simbiose (141).
Lembrem-se também do relacionamento da mãe do garoto transexual com
sua própria mãe como caso semelhante de simbiose inadequada, e que
contribuiu para a masculinidade dela. Estes fatores sugerem que, como
ocorre com os homens, o fato de abreviar a simbiose e de valorizar seu
enfraquecimento cria, nas mulheres, aqueles comportamentos e aquela
identidade que designamos como masculinidade. (Um aspecto do próximo
livro que fará o relato de minha pesquisa\idxmaestran[|)] será o desenvolvimento da
feminilidade em meninas e mulheres.)\idxtransesimb[|)]

Façamos um resumo. As mães, como vimos, têm uma tarefa adicional na
criação dos filhos que, com as filhas, é dispensável. Elas devem
encorajar a\idxpervproce{} separação\idxmaesproc[|(] (1) com maior intensidade, firmeza e vigilância;
2) no(s) momento(s) certo(s); 3) com as quantidades adequadas de
frustração, combinadas com (4) quantidades adequadas de amor, cuidado e
compreensão; (5) apreciar seu marido o bastante para oferecer esse pai
à criança como objeto valioso para identificação.

Além de encorajar a separação, elas também precisam encorajar o
desenvolvimento de um sentimento de autoconfiança.\idxegoiauto[|(] Isto foi estudado em
relação a muitas funções egoicas, mas talvez menos sistematicamente no
que diz respeito àquelas funções que são percebidas pelos outros, e
pela própria pessoa, como masculinidade. Requer-se de uma mãe (1) que
sua própria inveja da condição masculina seja subjugada; (2) que ela
seja feminina\idxmascufem{} ou, caso nem tanto assim, que ela o seja pelo menos em
alguns aspectos, quando estiver com seus filhos (146); e (3) que ela
goste de bebês. É uma grande vantagem (4) quando ela é genuinamente
heterossexual e especialmente útil quando ela é casada, de tal forma
que um homem, masculino, que ela ama, possa estar permanentemente
presente na família.\footnote{ Essas qualidades femininas estão
resumidas, quando ausentes, no comentário de McDougall:\idxmcdoug[|nn]
``Na criança destinada a uma solução perversa do desejo
sexual, o inconsciente da mãe desempenha um papel fundamental. Somos
tentados a especular que a mãe do futuro perverso, ela mesma, nega a
realidade sexual e denigre a função fálica do pai.\idxmaesfalo[|nn] É possível que ela
passe ao bebê, adicionalmente, o sentimento de ele ou ela ser um
substituto fálico''\idxmaesfalo{} (103, p.\,381).}

Em biologia --- animal e humana --- a condição de macho é uma
qualidade que se evidencia a partir de um substrato de fêmea. De forma
análoga, e aqui lançada de maneira hipotética, a masculinidade é uma
qualidade que se evidencia a partir de um substrato feminino. Converter
esta hipótese em descoberta exigirá que tragamos um
``microscópio'' clínico para aplicar ao
relacionamento entre mãe e bebê,\footnote{ Podemos ter expectativas de
que poderemos fazer isso graças,\idxmaessimb[|nn] principalmente --- mais do a
qualquer outro --- a Mahler.\idxmahle[|nn] Ela apontou o caminho, com sua
metodologia e suas conceitualizações, para que nos tornemos capazes de
focar de forma precisa as dinâmicas dos relacionamentos entre mãe e
bebê.} para nele examinarmos a maneira pela qual a simbiose é
dissolvida em duas pessoas que reconhecem mutuamente o que as
diferencia.\idxinfanego[|)] Esta ação de dissolução\idxegoi[|)] trará o bebê menino para o mundo
--- e para sua\idxegoiauto[|)] masculinidade.

Eis aqui uma proposição que poderá,\idxmaesproc[|)] em seu devido tempo, ser testada
empiricamente: nossa cultura, como a maioria das demais, define a
masculinidade --- para o melhor ou para o pior\idxtranse[|)] --- em termos de quão\idxangus[|)]
completamente\idxangusimb[|)] alguém demonstra\idxmaessimb[|)] estar livre da necessidade de simbiose
com a mãe.



\chapter[\textbf{9}\quad Um delito assume a forma de um ato sexual]{{\large\textit{Capítulo 9}}\\ Um delito assume a forma de um ato sexual}
\markboth{Dinâmicas: trauma, hostilidade, perigo e vingança}{Um delito assume a forma de um ato sexual}


O leitor encontrará, neste capítulo, informação suficiente para tornar
mais complexo o conceito de perversão. O ato delinquencial\idxcrime[|(] a ser
estudado aqui é parte habitual do comportamento da pessoa,
enquadrando-se em algum ponto entre a perversão --- neurose\idxneuro{} erótica
 --- e um tipo de neurose\idxpervneuro{} cujos sintomas não são abertamente eróticos.
Para essa paciente, o ato francamente sexual --- o planejamento de
um ``estupro''\idxestup{} --- é desprovido de prazer
erótico, ao passo que certas partes não eróticas de seu ritual ---
como a invasão de uma casa --- simulam, quase que literalmente, um ato
sexual, sem que a paciente se conscientize disso --- algo que faz
lembrar as convulsões histéricas dos dias vitorianos. Sua perversão
ilustra, assim, a opinião de Freud\idxfreud{} de que os sintomas neuróticos e
psicóticos são (eu diria \textit{poderiam ser}) atividade sexual
disfarçada, inconsciente.\footnote{ Eu não creio, entretanto, como
Schmideberg,\idxschmid[|nn] que ``quase todos os atos de delinquência
patológica podem ser, inteira ou parcialmente, classificados como
perversões ou fetiches'' (124, p.\,45), uma opinião
estranha que ela abraça simplesmente por ambos serem
``atos repetitivos'' com ``um
padrão claramente marcado e rígido.'' (p. 45) Isto
dificilmente seria razão suficiente para igualar os dois. Será que ela
chamaria todos os rituais sexuais de perversões?} Este caso é
apresentado, portanto, para dar visibilidade a um tipo de impulsos
sexuais primitivos que permanecem inconscientes até se tornarem
manifestos.

A paciente, que foi minuciosamente descrita em outro local (146), é
uma mulher de seus trinta anos que, até o final de seu tratamento (não
por análise) era, em geral, extremamente masculina; ela era
intermitentemente psicótica --- com alucinações\idxaluci{} e delírios --- e
apresentava estados de transe e múltipla personalidade.\idxmultp[|(] Ela estava
também firmemente convencida de que possuía um pênis; de que ele era da
mesma qualidade, e tinha as mesmas dimensões anatômicas, do pênis de um
homem normal --- embora se situasse dentro de sua pélvis. A parte mais
importante do tratamento era, para ela, encontrar o seu \textit{self},
tarefa impossível até então. Houve ocasiões em que ela se sentiu
claramente como a mulher que desejava ser, e outras, em que se sentiu,
de maneira igualmente nítida, como o homem que desejava ser --- tendo
havido épocas, também, em que ela quis se sentir como sendo mulher mas
se sentiu como um homem, e vice-versa.

Durante a adolescência e até seus vinte e poucos anos, ela se deixou
levar por inúmeras formas de comportamentos criminosos --- desde
passar cheques sem fundos até tentativas de assassinato --- pelos
quais, algumas vezes, ela foi detida e presa. No material que se segue,
de vários anos atrás e após muitos anos de tratamento, ela revela pela
primeira vez o ritual (o \textit{modus operandi}, que a
polícia sabe ser um detalhe que se encaixa, em muitos criminosos, como
uma impressão digital) que ela usava quando invadia residências, com o
propósito de roubar. Ao ler este material, o leitor poderá pensar em
cleptomania,\idxclept{} com sua conhecida dinâmica --- o desejo que as
mulheres têm de roubar um pênis\idxpenisclep{} (19, pp. 370--371), um pênis que a
satisfará como somente um seio que amamenta poderia fazer. Contudo, a
experiência dessa paciente não é exatamente a de uma cleptomaníaca ---
inclusive porque as dinâmicas estão muito pobremente disfarçadas, em
comparação àquelas do cleptomaníaco médio. O ato delinquente dessa
mulher se assemelha à perversão por ser repetitivo,\idxpervrepet{} gratificante,
compulsivo, construído a partir de dinâmicas de hostilidade --- e por
converter a vítima em vencedor. A diferença é que ele não é erótico
 --- apesar de os genitais serem a arena e de uma das etapas do ritual
exigir o coito, sendo a outra um ato em que o corpo é usado como órgão
genital.

Finalmente, em virtude de envolver a execução de um crime como parte
desse complicado ritual sexual, somos remetidos a questões\idxpervrespo{} morais\idxrespo{} que,
por dizerem respeito à responsabilidade do indivíduo por suas próprias
ações, podem ter relação com o problema do pecado,\idxpecad{} que será abordado na
última seção deste livro.

A um certo ponto, a transcrição evidencia o resvalar da paciente de
um estado de plena consciência para um estado de transe. Após ter
atingido esse estágio, e até o final do tratamento, ela foi capaz de
entrar em transe sempre que quis, evidenciando um dos aspectos de sua
capacidade de cindir seu \textit{self} em partes; outros aspectos dessa
capacidade são suas personalidades múltiplas\idxmultp[|)] e seus estados delirantes.\idxaluci{}
Durante esses transes ela reviveria --- em vez de simplesmente
recordar --- experiências do passado, inclusive eventos de sua mais
tenra infância, dos quais era incapaz de se lembrar quando totalmente
consciente.

Nessas conversas, a paciente e eu estamos muito à vontade um com o
outro, confiantes, sem tentar ser educados, usar de tato ou de mesuras.
A maioria de minhas observações extensas e explicativas, como ambos
sabíamos, eram perguntas e formas de pesquisa, destinadas a fomentar
tanto a minha compreensão quanto a dela. Elas não são --- como podem
aparentar assim, impressas --- uma sequência de interpretações
taxativas. Apesar dos riscos envolvidos, o material deste caso será
apresentado aqui através das próprias palavras que a paciente e eu
trocamos. Escolhi este método para que o leitor possa ter um vislumbre
de como, em minha pesquisa, a teoria emerge dos fatos.\idxcrime[|)] Existe uma
grande distância entre o que foi dito durante a sessão (ou, digamos,
seu pálido reflexo, que é o que permanece ao se imprimir uma conversa)
e a teoria e as hipóteses que se acumulam neste livro. E alguém poderá
se perguntar: como é que qualquer espécie de teoria pode ser levada a
sério, em um trabalho científico, se os dados que a sustentam
permanecem, como acontece com a psicanálise, inacessíveis?


\bigskip

\begin{quote}

\noindent\hskip0mm\textsc{s.} Por que você quer fazer isso?

\noindent\hskip0mm\textsc{g.} Porque isso me dá alguma coisa. Você não imagina como é
maravilhoso\ldots{} É melhor do que sexo\ldots{} dirigir-se para algum lugar,
entrar e roubar. É melhor do que conseguir uma mulher. Quando eu
comecei a roubar, eu ainda era criança, eu roubava comida, sabe? Eu
consigo me lembrar de mim, entrando\ldots{} eles cultivavam hortas
comunitárias, eram tempos de guerra, quando eu era garota; eu me lembro
de mim, roubando comida\ldots{} não é algo de que eu me lembre como sendo
tremendamente excitante; eu me lembro que era gratificante, porque
minha barriga estava vazia. Quando eu tinha por volta de treze anos:
encher o saco deles; a vontade de ir testando até onde você consegue
aborrecê-los\ldots{} e mais você consegue roubar\ldots{} eu nem ao menos sei o
que eu fazia com o que\ldots{} provavelmente eu dava a alguém, de presente;
é isto que faço com a maioria das coisas que eu roubo, eu as dou para
alguém. Eu não as quero. Quando eu estava no colégio, nós costumávamos
ir em grupos. Um grupo vigiava, outro cuidava de dar cobertura, e era
sempre eu quem roubava. Sou eu mesma quem escolhe fazer isso. E não sou
só eu não, eles também me escolhem, porque eu faço isso tão bem! Eu
sempre sinto medo quando estou roubando, mas não de ser pega. Não sei
de que tenho medo.

Depois que termina, eu fico tão excitada\ldots{} fico tão excitada. Eu
não tremo nem nada enquanto estou agindo, nem um pouco\ldots{} não é nada
demais. Mas quando eu termino, eu tremo até que minhas mãos tremam
também; eu tremo toda. Aí, eu dou a volta no quarteirão e transo com
alguém\ldots{} geralmente eu como alguma coisa. Eu sempre tomo um sorvete
com calda quente de chocolate. Não sei por que, mas é o que me atiça a
vontade\ldots{} Sabe, tem um restaurante que, quando eles me veem entrar,
eles logo me preparam uma tigela de sopa de \textit{sundae} com calda
quente. Eles não me conhecem, sou uma estranha pra eles. Eu sempre vou
lá, toda vez, depois.

\end{quote}

\bigskip

Uma outra parte de seu modo de agir, que nunca sofria variações, era
escolher um estranho, uma espécie de máquina, escolhida por parecer
capaz de desempenhar com vigor um ato sexual gélido, desprovido de
qualquer sentimento ou carinho, com um pênis rijo --- um ato sexual
que ela jamais suportaria, muito menos iria em busca dele em outro
momento qualquer.

\bigskip

\begin{quote}

\noindent\hskip0mm\textsc{g.} Eu simplesmente me deito ali\idxestup[|(] e deixo que me comam (para ela, essa
palavra nunca significa somente o ato sexual; ela é usada aqui por sua
evocação precisa de ataque forçado contra uma fêmea). Eles (os homens)
podem fazer comigo tudo o que quiserem. Eu nem sei se chego a gozar (em
nenhum outro momento ela deixa de saber). Eu não estou falando de sexo;
estou falando sobre quando me comem. É importante, ser comida\ldots{} é como
ser rasgada. Se o primeiro homem não conseguir fazer isso, eu saio e
arranjo outro. Depende de ter sorte ou não. Se já consigo o cara certo
da primeira vez, então fica nisso. Senão, eu vou de um em um, até
sentir que fui comida. Eu lhe digo: ``me
come!'' É isso que eu quero. Se tenho orgasmo, não é nos
meus genitais, é na minha cabeça. Uma explosão. Então eu fico aliviada;
aí, eu paro de tremer.

Eu faço isso pelo menos uma vez por mês; pelo menos. No máximo,
uma vez por semana. Eu nunca pratiquei um roubo sem fazer tudo isso
depois. Começou na minha adolescência, quando os garotos --- meus
amigos --- e eu roubamos alguns carros e fomos para o Arizona (eu
tinha uns catorze anos). Fui eu que os roubei. Então eles me comeram,
no caminho para o Arizona, naquela mesma noite. Chegamos ao Arizona na
manhã seguinte.

\end{quote}

\bigskip

À noite, depois de ter me contado todas essas coisas, o sentimento ---
um desejo --- de roubar, tomar sorvete e ser estuprada voltou.\idxestup[|)]

No decorrer da sessão seguinte, ela forneceu mais detalhes sobre o
ritual.

\bigskip

\begin{quote}

\noindent\hskip0mm\textsc{s.} O que você costumava fazer com as coisas que roubava?

\noindent\hskip0mm\textsc{g.} Depende do que era. A maioria eu dava. Quase tudo de valor, exceto\ldots{}
utilidades domésticas, nada que fosse grande, as coisas grandes seriam
difíceis de carregar. Depende, também\ldots{} bem, se eu estivesse fazendo
aquilo pra mim mesma, eu roubava coisas pequenas. Se eu estiver
roubando porque quero satisfazer alguma coisa dentro de mim, eu roubo
um objeto apenas. Se estiver roubando para meu parceiro, então eu roubo
mais alguma coisa. Eu quase nunca conservei nada. Teve uma vez que eu
conservei uma caixinha de música por muito, muito tempo.

\end{quote}

\bigskip

Seguiu-se, então, uma discussão sobre técnicas e conhecimentos, o
que revelou o profissionalismo da paciente. À medida que vai falando,
vai ficando envergonhada.

\bigskip

\begin{quote}

\noindent\hskip0mm\textsc{g.} Ontem à noite eu não saí (apesar de ter sentido o impulso de
fazê-lo).

 Eu sonhei com meus tempos de criança, quando não tínhamos geladeira;
tínhamos uma caixa de gelo, e o homem do gelo vinha\ldots{} e eu me lembrei
de que ele também dizia isso. Não sei por que sonhei com isso, mas
consigo me lembrar de ter dito isso\ldots{} Quando nós éramos crianças,
quando o homem vinha trazer gelo, nós engatinhávamos na carroceria do
caminhão, enquanto ele permanecia em casa; e eu sonhei sobre isso a
noite passada, sonhei que ele estava em casa e que eu montei na
carroceria do caminhão, apanhei um pouco de gelo\ldots{} e ele veio e me
disse que, se eu roubasse seu gelo, ele iria me furar com o picador de
gelo. E eu pensei que isso realmente seria muito legal, ser apunhalada
por um picador de gelo.\footnote{ Quando criança, uma vez, num ataque
de fúria, ela havia esfaqueado sua mãe na coxa; foi o mais alto que ela
conseguiu alcançar.} Eu não senti medo.

Eu não sou má, quando eu roubo. Eu não roubo porque eu sou má. Eu
roubo porque preciso, e não porque sou má. Se eu fosse sentar e pensar:
``Eu vou roubar dessa velhinha em Pasadena, e essas são
todas as economias que ela guardou durante toda sua vida, é toda sua
herança e tudo mais''\ldots{} então, eu suponho que me sentiria
má; mas eu não poderia fazer isso. Eu não roubo de alguém. Eu não penso
em ninguém. Eles simplesmente não existem\ldots{} eu não sou má, pois não
estou fazendo isso para prejudicar alguém\ldots{} eu quero chorar mas não
vou. Eu não sei\ldots{} porque eu me sinto como quando eu era uma
guriazinha, e era punida por alguma coisa que eu não tinha feito.
Talvez, se eu fosse punida, eu não fizesse mais isso.

Eu vou lhe dizer o que acabei de pensar. Começando pelo médico, e
ele dizendo: ``Por que você tem que estar sempre tendo
bebês?'' (ela, repetidamente, teve filhos ilegítimos) e eu
dizendo: ``Bom, o senhor sabe, quando alguma coisa se vai,
você precisa de outra coisa que a substitua.'' Quando
levam embora alguma coisa sua, você tem que recuperar, ou então você se
sente vazia lá dentro. O propósito de roubar, sempre leva uns dias, até
se definir. A primeira coisa que percebo é quando eu acordo. Faminta.
Em meu estômago. Eu não como; não se trata desse tipo de fome. Eu
costumava comer, mas aí eu vomitava. Eu teria a sensação\ldots{} eu estava
pensando naquela velha\ldots{} Quando eu era criança, talvez com uns oito
anos, tinha essa velha que eu costumava visitar. Ela morava num porão.
Seu filho e sua nora, com os filhos deles, moravam em cima; ele tinha
um apartamento no andar de baixo, e me dizia: ``Você só
poderá conseguir isso (um presente que ele tinha para ela) se roubar.
Eu não darei a você\ldots{} mas, se você roubar, poderá ter.''
E um dia minha irmã e eu fomos vê-lo, e ele tinha morrido\ldots{} Então, não
pode ser considerado propriamente um roubo.

Na manhã em que a coisa começa, eu acordo; eu sempre acordo cedo,
sabe, ainda está escuro e algo faz com que eu acorde, e eu sinto fome.
Talvez seja algum pensamento mau durante o sono, ou algo assim. Mas eu
não me lembro. Porém, estou excitada. Não sinto vontade de sair da
cama. Estou\ldots{} alguma coisa. Não quero sair da cama\ldots{}

Eu sempre me visto do mesmo modo. Faço sempre as mesmas coisas e uso
as mesmas roupas. Mas eu sempre usei o mesmo estilo de roupas. Um par
de jeans, tênis e uma camisa.

\noindent\hskip0mm\textsc{s.} Que tipo de camisa?

\noindent\hskip0mm\textsc{g.} Uma camisa comum. Não uma camisa de mulher\ldots{} camisa de homem, das
minhas. Minhas camisas são de homem.

\noindent\hskip0mm\textsc{s.} De que tipo?

\noindent\hskip0mm\textsc{g.} De cor forte. Mangas compridas.

\noindent\hskip0mm\textsc{s.} Sempre?

\noindent\hskip0mm\textsc{g.} Sim, mas eu as dobro até aqui. Azul. Gosto de camisas azuis. Não sei
por quê. Sempre uma camisa azul. Sei que não poderia ser vermelha, nem
verde.

\noindent\hskip0mm\textsc{s.} Você se lembra de alguém que usava camisas arregaçadas, azuis?

\noindent\hskip0mm\textsc{g.} Não sei\ldots{} meu avô usava.

\noindent\hskip0mm\textsc{s.} E por que tem que arregaçar as mangas?

\noindent\hskip0mm\textsc{g.} Fica mais confortável assim.

\noindent\hskip0mm\textsc{s.} Então por que não usar camisas de manga curta?

\noindent\hskip0mm\textsc{g.} Elas são curtas demais.

\noindent\hskip0mm\textsc{s.} Veja bem, você pensa sim que é mau. O coito parece uma punição: você
não pode ter nada disso (as coisas boas) sem que antes tenha tido
aquilo (a punição). Uma vez que você tenha sido devidamente punida,
então sua consciência lhe permite ter o pacote inteiro, o que
significa: paz. E é alguma coisa relacionada com o homem do gelo.

\noindent\hskip0mm\textsc{g.} O homem do gelo usava uma coisa de couro em cima dos ombros, assim
quando ele\ldots{}

\noindent\hskip0mm\textsc{s.} Quem era ele?

\noindent\hskip0mm\textsc{g.} Está na hora de ir (não a deixo). Nós tínhamos um homem do gelo, um
padeiro, um leiteiro\ldots{} e todos detestavam a minha mãe. O padeiro a
obrigava a pagar a conta e depois me dava o dinheiro. Nem sempre
tivemos o entregador de gelo. Isso foi só quando eu era muito
pequena\ldots{} não sei\ldots{} não sei.

\noindent\hskip0mm\textsc{s.} Seu pai está de algum modo envolvido nisso?

\noindent\hskip0mm\textsc{g.} Não.

\noindent\hskip0mm\textsc{s.} E ele tinha qualquer coisa que se relacionasse à camisa azul, às
mangas arregaçadas, ou alguma coisa relacionada ao entregador de gelo?

\noindent\hskip0mm\textsc{g.} Sabe, quando você me perguntou sobre a camisa azul\ldots{} eu tinha um
retrato do meu pai vindo para casa, com suas mangas arregaçadas e seu
suéter\ldots{} ele tinha um suéter, disso eu me lembro muito bem; e ele
sempre o dependurava no trinco da porta; e ele sempre trazia no bolso
alguma coisa pra mim. Alguma coisa só pra mim. Era estranho ele fazer
isso, pois eu tinha irmãos e irmãs, sabe. Eu tinha que surrupiar o
presente, para que os outros não o vissem. Pois ele só trazia algo para
mim; nada para as outras crianças.

\noindent\hskip0mm\textsc{s.} Então você o roubava!

\noindent\hskip0mm\textsc{g.} Sim. E o que eu roubo, é meu. Primeiro eu o seguro entre as mãos,
depois levo para casa.

\noindent\hskip0mm\textsc{s.} E depois você toma o sorvete: você comemora. E depois você deixa que
te comam, a grande punição. Agora, e a sua mãe, como ela procedia em
relação a esse joguinho?

\noindent\hskip0mm\textsc{g.} Ela sempre ficava brava por ele não trazer nada para os outros\ldots{}

\noindent\hskip0mm\textsc{s.} Seu pai nunca colocou você numa camisa como essa?

\noindent\hskip0mm\textsc{g.} Ele costumava me vestir com as roupas dele. A camisa e as calças,
traje completo. Ele achava muito engraçado me vestir com as roupas
dele.

\noindent\hskip0mm\textsc{s.} E ele lhe arregaçava as mangas?

\noindent\hskip0mm\textsc{g.} Claro, pois elas eram supercompridas.

\end{quote}

\bigskip

Vem, então, um material sobre sua mãe, que seria muito difícil para
o leitor acompanhar, pois as associações e interpretações referem-se a
acontecimentos e ideias que abrangem todos os anos de tratamento
anteriores a este momento. Basta dizer que numa sessão, logo antes do
sonho com o entregador de gelo, ela havia dito que sua mãe tinha gelo
nas veias, de tal forma que a sessão que acabo de descrever terminou
com minha sugestão de que o entregador de gelo seria a irredutível mãe
da paciente, e que para ela representara a morte, durante seus anos de
bebê e sua infância (é possível obter-se um retrato mais completo de
seu relacionamento com a mãe à p.\,146).

Sessão seguinte:

\bigskip

\begin{quote}

\noindent\hskip0mm\textsc{g.} As coisas que pego só têm valor pra mim por um certo tempo; então eu
preciso me livrar delas. A única coisa que roubei e conservei por um
bom tempo foi uma caixa de música. Era uma caixa de música de criança.
Tinha imagens esculpidas nela. Era como um carrossel, e tinha crianças
que passavam girando, num círculo, quando a música tocava\ldots{} Eu nunca
roubei nada ``valioso'', a menos que eu
estivesse roubando para meu parceiro. Digamos que tivesse uma joia
sobre a mesa, e que tivesse uma pedra. É mais plausível que eu
apanhasse a pedra em vez da joia\ldots{} conservo-a por alguns poucos dias
e então, jogo fora. Eu jogo tudo fora, seja o que for. As coisas que eu
pego não são triviais, mas você pensará que são. Quando entro numa
casa, eu olho em volta: não sei o valor das coisas\ldots{} mas se elas me
derem a impressão de ter \textit{aquele} tipo de valor, então eu as
apanho. Não é um valor monetário.

\noindent\hskip0mm\textsc{s.} E por que você se livra delas? Por que elas não permanecem valiosas?

\noindent\hskip0mm\textsc{g.} Pela mesma razão que eu jogo o lixo fora; elas ficam destituídas de
todo valor. Não preciso delas. Agora eu tenho o sentimento dentro, o
sentimento que obtenho quando apanho o objeto.

\end{quote}

\bigskip

Sessão seguinte. Entre esta e a anterior, o paciente me enviou uma
carta e telefonou. Ela se refere agora a essas comunicações.

\bigskip

\begin{quote}

\noindent\hskip0mm\textsc{g.} Isso (o assunto ``roubar'') não tem nada a
ver com eu ser um bebê.

\noindent\hskip0mm\textsc{s.} Claro que não. Você me escreve uma carta falando sobre seios, e
sonhos com seios, e me conta que bebe\ldots{} quanto? Quase dois litros de
leite por dia\ldots{} e me conta que, na véspera, você andou comendo como
uma verdadeira porca. E que na noite passada, por duas vezes,
deliciosamente, você molhou as calças.

\noindent\hskip0mm\textsc{g.} Estava pensando no momento em que minha mãe me telefonou, ontem à
noite.

\noindent\hskip0mm\textsc{s.} (Ela havia me contado por telefone): Você caiu no sono de tardezinha
e molhou a cama. E acordou com o sentimento: ``Nossa, isso
não é o máximo?'' Então sua mãe ligou e você ainda estava
na cama?

\noindent\hskip0mm\textsc{g.} Sim. Eu quase me lembrei de uma coisa, e agorinha mesmo eu pensei:
``Meu Deus, agora eu me lembro!''\ldots{} mas não
sei o que é. (Em retrospecto, percebo ter sido este o primeiro
movimento em direção a um transe; em geral, essas quase lembranças não
se relacionam com um transe).

O que vejo em minha cabeça é minha mãe dando de mamar a um bebê. Não
sei qual deles (dos irmãos mais jovens de G.). Sinto fome. Minha mãe
sempre teve um cheiro bom. Uma coisa quente e boa. Provavelmente, se
tudo não tivesse sido tão frio\ldots{} Quando minha avó me segurava, sabe,
minha avó era gorda, tinha seios grandes, macios, você meio que podia
mergulhar neles; não era a mesma coisa com a minha mãe. Não sei, eu
simplesmente não quero\ldots{} não quero pensar em coisas ruins. (Ela
resvala para um leve transe).

Lembrar do\ldots{} você se lembra de eriçar o mamilo do bebê para que se
parecesse com um pênis? (Ela fizera isso quando criança, de tanto que
queria ter um pênis). ``Isso não é para entrar aí, é para
entrar em sua boca'' (ela parece estar citando a
observação de sua mãe). Mas você não pode ter ambos, sabe, você precisa
decidir qual dos dois é mais importante (o mamilo na boca ou o pênis no
corpo). Eu não sei qual dos dois é mais importante. Naquele tempo
tudo\ldots{} tudo entrava na boca, tudo. E nada entra direito. Sabe, quando
você põe o dedo na sua boca\ldots{} há um buraco ali, tem um espaço vazio
ali. Porque o buraco nunca fica cheio. E isso causa uma dor bem aqui
(nos lábios). Faz com que fique apertado, frustrado.

\noindent\hskip0mm\textsc{s.} Me conte então sobre quando você se satisfaz.

\noindent\hskip0mm\textsc{g.} Quando você tem aquilo (o objeto roubado) nas mãos\ldots{}

\noindent\hskip0mm\textsc{s.} Você o coloca na boca?

\noindent\hskip0mm\textsc{g.} Sim. Eu o ponho contra minha boca. É legal. Eu não preciso chorar.
Simplesmente dá uma sensação boa\ldots{} Você não se lembra dessas coisas?
(Em transe). Tinha patinhos, sabe. Muito difícil de enfiar na boca.

\noindent\hskip0mm\textsc{s.} O que é o melhor de tudo?

\noindent\hskip0mm\textsc{g.} Minha mãe. Ela cheira bem. Quando ela me coloca na banheira, sabe,
quando nós duas estamos na banheira, eu o roubo.

Eu não me lembro\ldots{} estou cansada\ldots{} eu não sei\ldots{} foi a única vez
em que fiquei aquecida\ldots{} não quero chorar\ldots{} vamos para outro lugar\ldots{}
eu preciso ir para algum outro lugar. É preciso saber o que vai onde.
Se você é menino\ldots{} eu não sei o que fazer quando sou um menino. Não
sei o que fazer. Não me lembro como fazer isso. Sabia que tentei
demais? Mas simplesmente jamais conseguia fazer isso direito. Eu
preciso ir para algum outro lugar. Você quer ir? Por que você está
sempre aqui? Você está sempre aqui. Existem tantos lugares terríveis
lá. Você está conseguindo ouvir? (Alucinação\idxaluci{} no transe, vivenciando-se
a si mesma, agora como criança). Se eu fosse alta o bastante\ldots{} eu
simplesmente não entendo. Não sei como eles podem esperar esses tipos
de coisas. Você sabe, eles me dizem tipos de coisas tão diversos.
Primeiro eles dizem que está bem, depois eles me dão um tapa\ldots{} nunca
sei o que esperam de mim. E aquele garoto, ele simplesmente\ldots{} ele
simplesmente é tão mau! Sabe, não teria sido tão mau, se não fosse frio
o tempo todo.

\noindent\hskip0mm\textsc{s.} Não é sempre frio. Não fica quente quanto você se molha?

\noindent\hskip0mm\textsc{g.} Sim.

\noindent\hskip0mm\textsc{s.} Não é por isso que você se molha?

\noindent\hskip0mm\textsc{g.} Sim. É quase tão bom quanto estar na banheira com alguém que é bom.
Sabe, uma vez eu estava lá, e eu estava com frio e então eu tive a
ideia (de urinar) e foi tão bom! Você se lembra de quando eu estava
numa gaiola (uma caixa que se assemelhava a uma gaiola e que foi usada
como berço durante seu primeiro ano de vida), eu fiz isso estando
acordada. Existem duas coisas boas: estar aquecida e estar com a boca
cheia\ldots{} se ela estiver com você.

\noindent\hskip0mm\textsc{s.} Quando você se urinava, o que você fazia para que sua boca ficasse
cheia?

\noindent\hskip0mm\textsc{g.} Eu punha meu polegar na minha boca, ou as coisas boas. Como o
patinho. O cobertor; mas do cobertor, você só consegue por uma parte na
boca. Onde fica a ourela. Quando eu tiver minhas próprias crianças, eu
as colocarei na boca. Meus amigos também. Sabe, não os homens, exceto
talvez o peito deles, alguma coisa assim. Ou morder; tenho vontade de
morder. Todas as mulheres, eu as ponho em minha boca. Mas se você fizer
isso, vai se machucar.

Está sentindo isso? (Uma sensação dentro dela; ela toca o abdômen).
Foi só uma pequena excitação para conseguir algo. A sensação é bem aqui
(lábios) e desce bem pra cá (estômago)\ldots{} Essa noite foi a necessidade
de estar aquecida e molhada. Eu me lembro. Me lembro de como foi bom.

\noindent\hskip0mm\textsc{s.} Você conversou comigo no telefone ontem à noite. Eu lhe disse para
que não saísse, que não fosse à rua (para obedecer ao impulso de
roubar). Eu lhe disse que alguma outra coisa poderia fazer as vezes
disso. E você se urinou. Por que esperou adormecer, para se urinar?

\noindent\hskip0mm\textsc{g.} Eu não conseguiria me molhar acordada\ldots{} eu estava sonhando que
estava na banheira. Sonhei que eu estava molhada e aquecida. Se você se
mijar, você apanha. Eu me lembrei de ambas as coisas ontem à noite. Que
é bom e que é ruim.

Sonhei com a mulher dos seios. Sabe, eu ir lá, e pegar neles. Mas
não vou até lá\ldots{} eu só fui até\ldots{} até o lugar com as fontes quentes,
que era quente e que cheirava bem; isso está errado, porque lá, nunca
cheira bem, mas estava cheirando bem e fui até lá, e ela estava lá
tomando um banho e eu me aproximei dela por detrás, mas ela não se
zangou. Eu peguei em seus seios. Eu simplesmente os apanhei e os
coloquei em minha boca. E então eu me molhei. Eu fiquei molhada e
aquecida.

(Fora do transe:) Estou ficando muito cansada. Estou cansada de
muitas coisas. Estou cansada de fazer o que é errado; estou cansada de
me sentir mal; estou cansada daquele tipo de pensamentos que me vêm
quando eu estou dormindo\ldots{} de querer aquelas coisas que não poderia
ter. Cansada de não me lembrar quais são as coisas certas, sabe\ldots{}
Ninguém deveria chorar. Se você chorar, acaba recebendo alguma coisa, e
chorará por ela.

\noindent\hskip0mm\textsc{s.} É nisso que você acredita, ou é isso que ela diz?

\noindent\hskip0mm\textsc{g.} Não sei. E não descobrirei.

\end{quote}

\bigskip

Sessão seguinte:

\bigskip

\begin{quote}

\noindent\hskip0mm\textsc{s.} Então, como está a comilança? Parou?

\noindent\hskip0mm\textsc{g.} Que comilança?

\noindent\hskip0mm\textsc{s.} A compulsão por comida, passou?

\noindent\hskip0mm\textsc{g.} Sim. Estou me sentindo muito bem hoje. Estou me sentindo muito bem.
E comprei um pato pra mim, um pato muito legal, um pato de verdade.
(Eu havia dado à paciente alguns dólares, instruindo-a para que não
usasse esse dinheiro para comida ou roupas, mas que comprasse alguma
coisa para si mesma, pois ela jamais tinha feito isso antes. Quando era
criança, sua família vivia em extrema pobreza e sua mãe era gélida;
tudo isso era demais, para que fosse permitido dar a ela alguns
centavos, para serem gastos com ``alguma coisa
boa'' para ela mesma. Nos últimos anos, a paciente jamais
roubara coisas de valor que tenha conservado consigo. Por isso, ela
estava profundamente em dívida consigo mesma. Mesmo caso tivesse tido o
dinheiro --- e algumas vezes ela até teve esse dinheiro nas mãos ---
poderia ter sido indulgente e gasto com alguma coisa para si própria;
``nunca me passou pela cabeça comprar alguma coisa boa
para mim'').

\noindent\hskip0mm\textsc{s.} Isso é alguma coisa que você jamais fez no passado? Sim, nunca tinha
feito antes?

\noindent\hskip0mm\textsc{g.} Sim, nunca fiz. Eu agora estava justamente pensando\ldots{} esta noite
estou indo jantar na casa de D. (uma amiga), e eu estava pensando: eu
fico cismando se aquela prateleira lá de cima fica mesmo empoeirada, e
se meus patos estão lá; eu estava pensando em meus patos. Eu tenho um
punhado deles. As pessoas dizem: ``O que você
quer?'', eu direi: ``Me dê um
pato.'' Eu tenho alguns patos de quando eu era\ldots{} eu tenho
um pato de borracha que é meu desde que eu era ainda bebê.

\noindent\hskip0mm\textsc{s.} Foi nele que você ficou pensando ontem, durante a sessão?

\noindent\hskip0mm\textsc{g.} Não sei; não sei de que você está falando.

\noindent\hskip0mm\textsc{s.} Você falou sobre quando você era bebê, disse que tinha um pato e que
o esfregava na boca.

\noindent\hskip0mm\textsc{g.} Não sei. Não me lembro disso\ldots{} Espera um pouco. Nós deveríamos estar
falando sobre\ldots{} alguma coisa. Eu realmente não acho que tudo se
resuma a seios.\idxseios{} Não me sinto confortável com isto. Eu tenho sentimentos
em relação a seios, você sabe; eu cavo seios, você sabe (risos). O que
me deixa mais confusa é: por que roubar? Por que não comprar alguma
coisa? Então tem a parte sobre penetrar (uma casa). Quando eu digo
``penetrar'' eu penso em
``penetrar'' isso e aquilo, como\ldots{} penetrar
em uma mulher ou\ldots{} e o que me passa pela cabeça é meu filho nascendo,
ou minha filha nascendo e eu digo: ``Coloque de
volta.'' Todas essas coisas passam pela minha cabeça.
Ontem, quando eu fui à loja para comprar o pato, tinha uma mulher
saindo bem na hora em que eu estava entrando, e eu lhe dei um empurrão,
assim\ldots{} muito sem educação\ldots{} eu me sentia bem por estar entrando.
Entende o que quero dizer? Eu nunca roubei em lojas, só nas casas. Mas
para satisfazer o impulso, tinha que ser uma casa particular, o lar de
uma família; escura e quieta. E que não fosse fria. E não algum lugar
onde fosse fácil entrar. Eu consigo entrar em qualquer prédio em minha
vizinhança. Eu invadi meu próprio apartamento dezenas de vezes, quando
eu esquecia minha chave. É muito fácil entrar num apartamento. Nunca me
ocorreria entrar num lugar como \textit{aquele} e roubar alguma coisa.
(Sabe, você está me deixando nervosa --- é isso que você está
fazendo.) Um apartamento não é um lar. A maioria dos apartamentos são
simplesmente lugares em que uma mulher mora, ou em que mora um homem\ldots{}
Em uma lar existe uma mãe, um pai, e as crianças.

Sabe que mesmo depois que me tiraram da caixa (ler acima) eu
costumava engatinhar para entrar nela. Eu acabei de pensar nisso. Eu
consigo me ver, engatinhando e entrando na caixa. Era lá que minha
coberta ficava, de qualquer modo. Era difícil de entrar. Era uma caixa
grande. Eu tinha que incliná-la\ldots{} acho que é por isso que eu me senti
tão bem quando estive na cadeia. Quando eu invadia uma casa, estava
sempre escuro. Não por segurança; eu simplesmente nunca tenho o desejo
de invadir uma casa, digamos, às duas da tarde. É bom quando está frio
lá fora. Eu sempre tenho a sensação de ter estado lá antes, como se eu
soubesse onde estava, e faz com que eu me sinta bem, estar entrando lá,
onde é confortável. Não é uma boa ideia ir até a porta. E existem
coisas que você precisa fazer para sua própria segurança\ldots{} tipo, pode
ser que um cão morda sua perna. Então é sempre bom certificar-se de que
não há cães por ali. Legal mesmo é se esgueirar pela janela. As janelas
são difíceis de abrir, exceto essas que estão pondo nas casas
atualmente --- essas são fáceis\ldots{} eu não quero fazer isso (quer
dizer, me contar sobre o ritual).

O que eu quero é me esgueirar.\idxidenhis[|(] É legal quando tem uma cortina na
janela. Pra roçar nela. Acho que eu passo pelo menor lugar que eu
puder. Senão, por que é que eu não passo por uma porta de vidro,
daquelas de correr, ou qualquer outra coisa que seja relativamente
fácil de abrir? Antes disso\ldots{} estou tensa. Não sei exatamente. É como
se você estivesse toda preparada para alguma coisa, ansiosa para\ldots{}
como se você estivesse transando e você tivesse atingindo aquele ponto,
em que está pronta\ldots{} Então, quando eu entro, é bom. Está\ldots{} está
quente ali dentro; é\ldots{} é onde você deveria estar. Faz algum sentido?
Então bate aquele alívio, é tão bom! Aí, quando seguro o treco na mão,
eu posso sair, está feito; acabou. Aí já não faz diferença, de que
maneira eu saio; posso sair pela porta da frente. Não tenho necessidade
de me esgueirar para fora. Então vem\ldots{} vem a parte ruim\ldots{} não é ruim
de verdade, mas talvez você ache que sim. Não sei o que você acha;
alguns acham que é ruim. É a parte que envolve trepar; alguém tem que
comer. Isso é realmente necessário; você realmente precisa ser punida
pelo que fez; aquilo [invasão e roubo] realmente foi mau. Mas trepar
não elimina a coisa, você sabe; simplesmente não faz diferença com que
brutalidade eu for comida, a coisa torna a voltar. Trepar \textit{não}
faz passar. Mas se apoderar da \textit{coisa}, isso faz com que passe,
a parte da fome passa. Às vezes eu penso, quando estou vagando pelas
ruas, ou seja lá o que eu estiver fazendo\ldots{} procurando por um lugar
para invadir\ldots{} isso não vai fazer sentido algum, mas eu penso\ldots{} eu
não acho que sou uma mulher. Entende o que quero dizer? Não sinto que
sou mulher. Não, o que acontece\ldots{} isso me faz sentir\ldots{} como uma
pessoa do sexo masculino, um homem, funcional, completo, que pensa, que
sente, que quer.\idxiden[|(] É um pouco diferente de quando eu sentia que era
mulher mas ainda acreditava ter um pênis. Quando eu era pequena e
pensava: ``Bem, eu serei um bom menino'', eu
sabia que eu era menina; eu tinha conhecimento disso, sabe? Ou quando
eu tinha um pênis, tudo que eu tinha que fazer era abrir minhas pernas
e aí, eu era mulher, sabe; mas quando eu estou caminhando [para ir
assaltar] e estou usando minhas roupas masculinas\ldots{}

\noindent\hskip0mm\textsc{s.} Você tem um pênis?\idxpenisimag{}

\noindent\hskip0mm\textsc{g.} Não sei. Nem sei se tenho uma língua.

\noindent\hskip0mm\textsc{s.} É isso que eu quero dizer; não tem nada a ver com o pênis.

\noindent\hskip0mm\textsc{g.} Não. Não, não, não.

\noindent\hskip0mm\textsc{s.} É diferente de toda a coisa do pênis. Não tem nada a ver com a
anatomia, ou com as roupas que você estava usando.

\noindent\hskip0mm\textsc{g.} Certo. O que mais eu usaria, caso fosse homem? As roupas são
imateriais\ldots{} Depois que conversamos ontem\ldots{} ontem eu não me senti
[mais] como homem. Não me senti como isso ou como aquilo. Eu me senti
\textit{bem}, sabe como? Não precisei pensar sobre ser homem, ou
mulher, ou no que eu tinha ou\ldots{} Eu me pergunto: como isso aconteceu?
\textit{Isso} sim é estranho.

\noindent\hskip0mm\textsc{s.} Já tinha acontecido antes?

\noindent\hskip0mm\textsc{g.} Não.

\noindent\hskip0mm\textsc{s.} Foi a primeira vez em sua vida?

\noindent\hskip0mm\textsc{g.} Foi.

\noindent\hskip0mm\textsc{s.} E tudo porque eu lhe dei dinheiro e lhe disse que comprasse alguma
coisa boa. Como foi que isso aconteceu, ontem?

\noindent\hskip0mm\textsc{g.} Ah, ontem foi iupiiii, uau!!! ``Uau, eu posso sair; sei
quem sou e estou caminhando com minha irmã e minha irmã sabe quem eu
sou''; e o homem da loja, que me vendeu o pato, disse:
``Obrigado, senhora''\ldots{} Você sabia quem eu
era e me deu\ldots{} você \textit{me} deu o dinheiro para gastar porque sabe
quem eu sou. E quando eu cheguei aqui com a minha irmã [nesse mesmo
dia, mais tarde, para uma visita social] com o pato, eu senti vontade
de abraçá-lo e dizer: ``Oh, obrigada!'' E
isso eu disse, eu lhe disse: ``Obrigada!''.

\noindent\hskip0mm\textsc{s.} Então, quão além o sentimento de ontem ia, do que você sente quando
você rouba um objeto?

\noindent\hskip0mm\textsc{g.} Não tem comparação! Quando eu pego aquilo na minha mão, sabe, eu
ainda\ldots{} eu preciso pensar sobre a coisa toda. Eu sei o que eu sou
quando estou invadindo a casa --- é óbvio, você sabe --- eu sou
apenas um homem assaltando. Mas há algo sobre ser homem que eu sei que
é errado, sabe --- eu não sei como descrever isso pra você, não tenho
como\ldots{} eu simplesmente quero estar aquecida e\ldots{} quero que alguém
saiba quem eu sou\ldots{} O que é que quero ser? Não acho que queira ser um
homem. Você olhou ontem pra mim, e sabia quem eu era. Eu sou capaz de
estar lá fora, na rua, olhar pra mim e não saber quem eu sou\ldots{} Como é
que eu vou saber quem eu sou, se ninguém mais sabe? Ninguém nunca me
disse quem eu sou. Eu preciso ser alguém definido\ldots{} você sabe. Ontem
eu era real pra você. Depois de invadir e ter o objeto nas mãos, eu me
sinto bem em relação a mim mesma, mas há algo de ruim\ldots{} é como se eu
não tivesse o direito de ser o que sou naquela hora; eu preciso ser
punida pelo que sou naquele preciso momento. Mas eu nunca vou poder ser
realmente ``eu'' mesma até que eu possa ter a
coisa sem ter que roubar.\idxiden[|)] Ontem, porém, você me disse:
``Vou dar isto a \textit{você}; estou \textit{dando} isto
a \textit{você} --- não era um empréstimo ou coisa\idxidenhis[|)]
parecida\ldots{}''

\end{quote}

\bigskip

Aqui ela nos deu uma ideia da razão pela qual rouba. Quando era
bebê, ela desejava ardentemente estar próxima da mãe, que era
completamente fria, e de quem ela não conseguia obter quase nada, fosse
leite ou calor; ela ansiava por fazer parte de uma família íntegra,
capaz de amar. Em algumas partes do ritual, podemos ver como ela
prepara o cenário de sua história, que vai se construindo, verbalmente,
sobre perigo e mistério; e fazendo uso da dinâmica que transforma a
vítima em vencedor --- do mesmo modo como o conto erótico objetiva
provocar a excitação genital.

Conservando em mente que, caso soubéssemos o bastante, todos os
elementos da história precisariam ser levados em conta, para assim
encontrar seu fio condutor, voltemo-nos para a sessão seguinte à que
acabamos de narrar acima. O assunto é a masculinidade\idxmasc[|(] dela, uma
qualidade inventada para aliviar sua devastadora vulnerabilidade. Tendo
sido rejeitada pela mãe ao nascer, à medida que foi crescendo notou o
quanto seu irmão --- alguns anos mais velho --- era admirado por
ela; concluiu que era pelo fato de ser homem que a mãe o desejara,
tendo-a contemplado, em troca, com uma frieza extrema. Portanto, a
partir de seus quatro anos, ela decidiu que tinha um pênis,\idxpenisimag{} e que ele a
dotava dos atributos masculinos de força e poder, capazes de evitar
que ela sentisse frio, fome, abandono e humilhação. (Ver [146] para
mais detalhes sobre seu pênis). Mas essa forma de delírio não foi
suficiente --- nem mesmo o uso que fez do corpo de mulheres,
especialmente seios,\idxseios{} para aquietar sua fome. Ela precisava também de um
pedaço de realidade que pudesse, literalmente, segurar em sua mão e
esfregar ou enfiar na boca. O roubo dos objetos gratificadores lhe
concedia conforto por algum tempo e, ao roubá-los, ela se vingava, em
fantasia, da falta de generosidade da mãe.

Porém, esse ato mau exigia punição\idxmasoqpun{} (ou antes, como em todos os atos
masoquistas, somente aquele tipo de punição que a própria pessoa
escolhe para si; não importa quão doloroso ele seja de fato, pois
trata-se, no fundo, de uma punição controlada, parcial, gratificante,
libidinosa --- falsa). Para satisfazer sua imensa culpa por roubar
aquilo que a mãe (e agora o mundo) não lhe daria de graça, ela, por
alguns instantes, abria mão de seu pênis: numa forma indefesa de
feminilidade, ela providenciava, com equanimidade, seu
``estupro'' --- o pênis, e sua
invulnerabilidade, só retornavam a ela depois que a punição tivesse
sido consumada.

Podemos observar a forma que sua masculinidade assume no ritual da
invasão. Não é preciso muita imaginação para perceber que ela sente a
si mesma como um pênis se introduzindo em um corpo de mulher --- sua
mãe --- um retorno à sensação de completude primordial. Este
simbolismo\idxpervsimbo{} faz paralelo com aquele dos atos perversos, embora o
propósito --- e seu resultado --- não seja o prazer erótico;
chamá-lo de perversão obscurece o significado da palavra.


\bigskip

\begin{quote}

\noindent\hskip0mm\textsc{s.} Em que altura\idxidenhis[|(] você sente, pela primeira vez, que você é um homem?

\noindent\hskip0mm\textsc{g.} Acho que quando me dirijo para a casa. Eu preciso estar caminhando.
Eu não me sinto um homem quando estou dentro do carro. Por que eu não
me sentiria assim dentro do meu carro? Suponho que seja por seu meu
carro.

\noindent\hskip0mm\textsc{s.} Você quer dizer: quando você desce do carro, você se desligou de você
como mulher --- desligou-se fisicamente. Mas você já está
propositadamente vestida de forma adequada para o que acontecerá.
Você vestiu camisa de homem, calças masculinas, calçou tênis [para
poder caminhar em silêncio e com segurança] --- roupas que costumavam
ser exclusivamente masculinas quando você era criança. Que tipo de
roupa de baixo você costuma estar usando nessas ocasiões?\idxmasc[|)] [A paciente
parece ficar atordoada]. Você esqueceu de mencionar isso.

\noindent\hskip0mm\textsc{g.} É verdade. Nem sei se uso roupa de baixo. Eu preciso usar roupa de
baixo. Acho que simplesmente uso minha roupa de baixo de sempre, meu
sutiã e minha calcinha\ldots{} Aquele ``homem''
está usando sutiã e calcinha!

\noindent\hskip0mm\textsc{s.} Está bem. Agora você desceu do carro, é um homem e está procurando na
vizinhança pela casa certa. Como você faz a escolha?

\noindent\hskip0mm\textsc{g.} Tem que ser um lar, uma casa aconchegante. Tem que ter uma mãe, um
pai e crianças. É possível saber, talvez porque tenha uma bicicleta no
jardim\ldots{}

\noindent\hskip0mm\textsc{s.} A pessoa, então, que escala através da janela, é o homem?

\noindent\hskip0mm\textsc{g.} Hmm\ldots{} não; não tenho certeza. Eu consigo pensar em subir até a casa,
e posso pensar em ter a coisa em minha mão\ldots{} é preciso ser\ldots{} tem que
ser\ldots{} Tem que ser um homem que passa pela janela. Quem mais passaria
por aquele tipo de abertura? Trata-se de um tipo de abertura feita para
um homem, você sabe.

\noindent\hskip0mm\textsc{s.} Como você passa por ela? Mostre-me, com seu corpo, como você faz para
passar.

\noindent\hskip0mm\textsc{g.} Não, é muito embaraçoso\ldots{} Eu passo primeiro a cabeça. Simplesmente
empurro meu corpo para dentro. Faço a mesma coisa toda vez. Primeiro
passo a cabeça. Eu me empurro com meus braços, com minhas mãos. Não sei
como faço. Não me pergunte. Eu poderia muito bem entrar primeiro com os
pés\ldots{} algumas vezes isso seria mais fácil. Mas é sempre a cabeça
primeiro. Não dá pra entrar muito depressa, poderia fazer barulho\ldots{}
Não quero mais conversar com\idxidenhis[|)] você.

\noindent\hskip0mm\textsc{s.} Quando você passa pela janela, você é gorda ou magra? [Ela era, em
realidade, gorda].

\noindent\hskip0mm\textsc{g.} Sou como sou de fato. Magra\ldots{} Quando eu passo pela janela, estou
apoiada em meu estômago. Minhas pernas ficam completamente penduradas.
Acho que assim: [seus pés, calcanhares e pernas se tocam; os braços
estão tensos, completamente estendidos em linha reta, com seu corpo e
a cabeça rígidos]. Como um girino. Um só pedaço. Não tinha curvas nem
ângulos agudos. Se fosse uma coisa, eu diria: eu entraria reta como uma
flecha. Eu estou aquecida, sou real e sou essa coisa que é real\ldots{} você
sabe. Sou um homem que é real. E, no entanto, você ficaria confuso.
Haveria algo de errado. Veja bem, está escuro e estou ali, de cabelo
curto e usando calças, mas tenho aquela coisa em minha mão. Então, como
é que você iria saber se eu sou menino ou menina? \textit{Você tem que
saber}. Como é que eu vou saber se \textit{você} não sabe?

\noindent\hskip0mm\textsc{s.} Isso fica faltando toda vez que você está na casa, que alguém não
tenha lhe dito?

\noindent\hskip0mm\textsc{g.} Para que você acha que eu tenho que ser comida? [Sua esperança
primordial, no tratamento, era que pudesse um dia restaurar
completamente seu senso de feminilidade e aceitar a si mesma,
completamente, como mulher. ``Ser comida''
servia para forçá-la a ter plena consciência de que seu corpo era um
corpo de mulher].

\noindent\hskip0mm\textsc{s.} Não sei. É você quem está me dizendo. Está bem. Eu pensei que fosse
apenas uma forma de se punir.\idxmasoqpun{} E com relação ao sorvete?

\noindent\hskip0mm\textsc{g.} É tudo a mesma coisa.

\noindent\hskip0mm\textsc{s.} Está bem, estou entendendo melhor. Quando você passa pela janela,
quando seu corpo inteiro, até mesmo os dedos de seus pés tenham
penetrado, isto significa o fim de sua condição de homem. E quando você
sai andando, o que você é?\ldots{} algo incerto?\ldots{} até que seja comida?

\noindent\hskip0mm\textsc{g.} Então é todo tipo de coisa, você sabe. Eu preciso ser comida por eu
ser má e por ser o que sou, e não por não ser o que sou\ldots{}

\noindent\hskip0mm\textsc{s.} Quero repassar. Você acabou de entrar. O que você é então? Você não
tem certeza. Você apanha o objeto. Você ganha um pouco mais de certeza,
ou fica na mesma?

\noindent\hskip0mm\textsc{g.} Sim, então, veja, eu\ldots{} se você me visse, ficaria confuso. Não sei\ldots{}

\noindent\hskip0mm\textsc{s.} Mas eu fico menos confuso, ou eu fico igualmente confuso, a partir do
momento em que você cai lá dentro, e até arrumar alguém que te coma?
Não existe uma alteração no grau de confusão? Você não vai se tornando
mais mulher, com o passar do tempo?

\noindent\hskip0mm\textsc{g.} Sim.

\noindent\hskip0mm\textsc{s.} Quando você entra na sorveteria\ldots{}

\noindent\hskip0mm\textsc{g.} Mas veja, na verdade, isso não é justo.

\noindent\hskip0mm\textsc{s.} Por quê?

\noindent\hskip0mm\textsc{g.} Porque eu não sei se é isso que eu devo ser: uma mulher.

\noindent\hskip0mm\textsc{s.} Você faz isso para se sentir mulher?

\noindent\hskip0mm\textsc{g.} Claro.

\noindent\hskip0mm\textsc{s.} É em parte o anseio de se tornar mulher, quando você é acometida
pelo desejo de roubar?

\noindent\hskip0mm\textsc{g.} Saber\ldots{} o que eu deveria saber sobre\ldots{} Veja, quando eu tenho o
objeto em minha mão, eu quase me lembro\ldots{} não consigo me lembrar. Eu
tentei tanto. Eu o sinto em meus seios e em meu\ldots{} em todo lugar.
Você\ldots{} quando você era criança você nunca apanhou girinos e colocou-os
numa jarra, e então as pernas deles nasceram?\idxidenhis{} Um girino se parece
comigo. Era isso que eu era realmente, eu era um girino. Primeiro, eu
perdi meu rabo. Quando eu tinha 4 anos e meio, eu já era velha demais;
eu era velha demais para continuar sendo o que quer que eu fosse; já
era tarde demais, então.

\noindent\hskip0mm\textsc{s.} Você deixou uma coisa de fora. Quando você escorrega pela janela, o
que acontece imediatamente depois que está lá dentro? O que você sente
em seu corpo?

\noindent\hskip0mm\textsc{g.} Na verdade é muito confuso\ldots{} se eu não usasse a roupa de baixo, eu
não seria uma mulher quando eu segurasse o objeto em minha mão. Eu não
tenho nada de seios\idxseios[|(] quando eu passo pela janela. Como é que um homem
pode ter seios?

\noindent\hskip0mm\textsc{s.} O que acontece com aquelas coisas que estão dentro de seu sutiã, à
medida que você vai passando pelo parapeito da janela? Você diz a si
mesma que elas estão lá, ou não? O que acontece a elas? Como você
consegue ficar sem saber que elas estão lá?\footnote{ Aqui estou
tentando compreender o momento --- o estado de consciência, o desejo
de regredir --- quando a conversão histérica\idxidenhis[|nn] acontece.}

\noindent\hskip0mm\textsc{g.} Não se trata de saber que elas estão lá. Homens não têm seios. Acho
que não os tocando ou\ldots{} acho que eu simplesmente os ignoro [negação?
clivagem? repressão? recalque? Qual é a experiência subjetiva, que
esses termos técnicos não conseguem capturar adequadamente?] Não me
lembro de saber que eles estão lá. Simplesmente é ilógico ter seios, se
você é homem. Por que teria que fazer alguma coisa, se sabe que eles
não estão lá?

\noindent\hskip0mm\textsc{s.} Você sabe o que você é quando passa pela janela? Como é essa
sensação?

\noindent\hskip0mm\textsc{g.} É como entrar em um lugar quente, apertado.

\noindent\hskip0mm\textsc{s.} Quando você passa pela janela, você é um homem, ou você é uma coisa?

\noindent\hskip0mm\textsc{g.} Eu era um homem quando comecei a passar pela janela. Eu devo ser uma
coisa, porque eu não sou ninguém. Então vem aquele sentimento forte e
bom e\ldots{} eu realmente desejo, simplesmente, ser eu. Você sabe\ldots{} quando
eu era uma paciente, e estava me tratando aqui nessa ala, e eles diziam
``você parece confusa'' ou ``o
que você diz é confuso'', ou mesmo quando eu respondia
``Ah, bom, estou confusa sobre isso'' ---
tudo isso era tão irrelevante! \textit{Não há confusão como aquela que
diz respeito a não saber se você é real ou não, ou o que você é ou}\ldots{}
[os itálicos são meus; ela falou com suavidade].

\noindent\hskip0mm\textsc{s.} Deixe-me voltar por um momento. Quando você sai e usa roupa íntima de
mulher, o que você faz em relação a isso, quando você vira homem?

\noindent\hskip0mm\textsc{g.} A mesma coisa que faço com meus seios. Mas deve ter uma parte de mim
que tem conhecimento disso, pois eu sei, quando eu entro em uma
determinada casa, que eu \textit{preciso} daquela roupa íntima, e que
eu \textit{preciso} daqueles seios.

\noindent\hskip0mm\textsc{s.} Sim. Mas antes disso, quando você está atravessando a janela, você é
um pênis em ereção. Que tal lhe parece isso? Apesar de não fazer nada,
depois de cruzar a janela. Você conhece o pênis de um modo que eu
desconheço. Para você, um pênis pode ir embora e, ainda assim, você não
fica com a sensação de alguma coisa de completo, ou de incompleto. É
suficiente ser, simplesmente, uma ereção. Acho que é isso que você é:
você é um falo, o que é diferente de um pênis. Um pênis é uma coisa que
efetivamente existe; um falo é um símbolo.\idxpervsimbo{} E, à medida que você vai andando
em direção à casa, você tem seios\idxseios[|)] e barriga e vagina, e um útero, de
onde já saíram bebês. Você --- preste atenção agora, porque é
realmente grotesco, quer dizer, você é grotesca ao fazer isto: uma
mulher, uma mulher biologicamente completa, em suas roupas de baixo,
caminha em direção a uma casa dizendo ser uma pica, um pênis,\idxpenisimag{} com mais
propriedade, um falo. É grotesco. Quer dizer, se você realmente quer
ser \textit{você}, então tem que se sentir envergonhada [p. 182]\ldots{} por
ter me mostrado que você, que é uma mulher, era um falo. Quando você
põe o pé na entrada da casa, você nega a si mesma. Como pode fazer isso
consigo mesma? A resposta é: você precisa. Mas como você pode
destituir-se assim a si mesma, ainda que somente durante alguns
minutos? Você não é um falo.\idxidenhis{} Por que quer chorar?

\noindent\hskip0mm\textsc{g.} Não sei. Só porque estou aliviada, ou algo assim. [Silêncio].

\end{quote}

\bigskip

Sessão seguinte:

\bigskip

\begin{quote}

\noindent\hskip0mm\textsc{s.} A coisa que você rouba é a coisa mais real do mundo, não é assim?
Nada poderia ser mais sólido e substancial. E é do tamanho justo! Não é
uma coisa grande\ldots{} é pequena. É uma coisa feminina? [Sacode a cabeça
afirmativamente]. É sempre um objeto feminino, alguma coisa que uma
mulher tem? [Sacode a cabeça que não].

\noindent\hskip0mm\textsc{g.} Eu não sei. Eu pensava em coisas que as crianças têm. A caixa de
música, uma boneca em um estojo de vidro, pequena\ldots{} o retrato de uma
mãe com uma criança.

\noindent\hskip0mm\textsc{s.} Está bem. Então o objeto tem algo a ver com seu relacionamento com
sua mãe.

\noindent\hskip0mm\textsc{g.} Ora, vamos lá, não ponha minha mãe nisso, pode ser? Deixe que eu e
minha mãe fiquemos fora disso. Estou cansada de minha mãe. Não quero
mais ter nada a ver com ela. Por que tem que ser sempre minha mãe? Por
que não poderia ser minha tia, ou minha vó, ou\ldots{} Você sabe que minha
mãe também rouba coisas. [Isto nunca havia sido ventilado antes]. Acho
que todo mundo faz isso. Não, eu acho que ninguém faz isso. Ela nunca
vai a um restaurante, ou a um motel, ou a qualquer lugar, sem pegar
alguma coisa\ldots{} cinzeiros, ou coisa parecida. A casa dela é cheia de
bugigangas que ela roubou. Nós todos roubamos. A família toda rouba.
Meu pai costumava roubar\ldots{} coisas divertidas, você sabe. Uma vez ele
roubou um caminhão de laranjas --- jamais me esquecerei disto. Eu não
sei\ldots{} [risos].

\noindent\hskip0mm\textsc{s.} O que você roubou de sua mãe?

\noindent\hskip0mm\textsc{g.} Nada. Dinheiro. Eu roubei dinheiro dela. Um monte de vezes\ldots{} (Vou
embora. Vou sim. Não quero ouvir essa conversa mole. Deixe-me colocar
minhas botas\ldots{} e sair daqui). Eu nunca roubei nada da minha mãe,
exceto dinheiro; e isso vinha a propósito, já que dinheiro era a única
coisa que ela valorizava. (Fiquei de novo com câimbra no dedo do pé, e
por sua culpa!) Eu roubava dinheiro. Ela o escondia de mim. Ela fazia
todo aquele ritual medonho, pra me conservar longe do seu dinheiro\ldots{}
mas eu sempre o encontrava e pegava. Ela sabia que era eu. Eu me lembro
que eu tinha seis ou sete anos --- essa foi a primeira vez --- ela me
mandou ir ao mercado comprar pão e eu perdi o dinheiro. Então ela
disse: ``Você o roubou!'' Eu respondi que
não, e ela disse: ``Você roubou sim e, se não me
confessar, eu vou te bater!''; então eu disse:
``Eu roubei''.

\noindent\hskip0mm\textsc{s.} E assim, você ganhou.

\noindent\hskip0mm\textsc{g.} Certo.

\noindent\hskip0mm\textsc{s.} E depois disso você tentou roubar dela.

\noindent\hskip0mm\textsc{g.} Certo.

\noindent\hskip0mm\textsc{s.} Pois se ela, de qualquer modo, dizia que você roubava, melhor então
aproveitar e tirar alguma vantagem disso.

\noindent\hskip0mm\textsc{g.} Minha mãe tinha sempre razão. Eu roubava e tornava a roubar. E o
dinheiro era tão precioso para ela! ``Oh meu
Deus!'', dez centavos --- você chegaria a pensar que a
vida dela dependia daqueles dez centavos --- talvez dependesse, não
sei. Ela tinha cinco filhos que estava tentando alimentar.

\noindent\hskip0mm\textsc{s.} E o que você rouba [quando invade as casas] é o tipo de coisas que
valem alguns poucos centavos --- não é assim?

\noindent\hskip0mm\textsc{g.} Isso mesmo. Sabe o que eu costumava fazer com o dinheiro?

\noindent\hskip0mm\textsc{s.} Tem que estar sempre com o dinheiro seguro em sua mão [calculo]. O
que foi que você fez com aquele dinheiro que lhe dei outro dia?

\noindent\hskip0mm\textsc{g.} Eu fiquei segurando, até o momento de gastar.

\noindent\hskip0mm\textsc{s.} Mas você não\ldots{} eu nem saberia como descrever isso, quando aconteceu
 --- você não apanhou o dinheiro como se fosse uma nota, colocando no
bolso, ou dobrando\ldots{} nada disso; você o apanhou e o amassou no mesmo
instante, de tal modo que ele se transformou num pedaço de seu punho.

\noindent\hskip0mm\textsc{g.} Quando eu o entreguei para a moça da loja, chegava a estar molhado.

\noindent\hskip0mm\textsc{s.} Sim. Agora, então, eu entendo. Quando você invade uma casa e pega
alguma coisa, você a segura em sua mão e não larga até --- quando?\ldots{}
Depois do sorvete? Não tem como ir à sorveteria, tomar o sorvete e
continuar segurando\ldots{} o objeto em sua mão.

\noindent\hskip0mm\textsc{g.} Quer valer uma aposta?

\noindent\hskip0mm\textsc{s.} Mesmo? Não, você não fica segurando. Você o coloca a seu lado e fica
olhando pra ele enquanto come.

\noindent\hskip0mm\textsc{g.} Eu fico com ele na mão.

\noindent\hskip0mm\textsc{s.} O tempo todo, enquanto toma o sorvete?

\noindent\hskip0mm\textsc{g.} Isso mesmo, bem assim [mostra o punho].

\noindent\hskip0mm\textsc{s.} Estamos conseguindo extrair mais coisas sobre o ritual; \textit{nada}
é fortuito.

\noindent\hskip0mm\textsc{g.} Certo.

\noindent\hskip0mm\textsc{s.} Então você se levanta, paga pelo sorvete, sai e faz com que alguém te
coma. O que acontece então com o objeto?

\noindent\hskip0mm\textsc{g.} Eu o solto.

\noindent\hskip0mm\textsc{s.} Não tem como segurar o objeto enquanto estão te comendo\ldots{}

\noindent\hskip0mm\textsc{g.} Eu poderia.

\noindent\hskip0mm\textsc{s.} Em que altura você o solta?

\noindent\hskip0mm\textsc{g.} Depois que saio para que alguém me coma. Quando estou no carro, eu o
largo. Ponho no assento traseiro; não quero olhar para ele. Não quero
ter nada a ver com ele.

\noindent\hskip0mm\textsc{s.} Quando você está no bar, digamos que com o homem [escolhendo o
estranho fálico], você está ali sentada, com uma bebida em uma das mãos
e o objeto na outra. O que você fez com ele?

\noindent\hskip0mm\textsc{g.} Quando eu vou para o bar\ldots{}

\noindent\hskip0mm\textsc{s.} Ah, já sei. Quando a parte masculina da coisa começa [a caminho de
encontrar um homem], então você joga o objeto para o banco de trás. E
a coisa termina aí, tão depressa? Na mesma noite já abre mão dele\ldots{}
exceto nas raras ocasiões em que você o conservou por mais tempo.
Nessas ocasiões raras, deve ter sido realmente algo muito no estilo
mãe-criança. Por quanto tempo você conservou o retrato [da mãe com a
criança]?

\noindent\hskip0mm\textsc{g.} Por algum tempo.

\noindent\hskip0mm\textsc{s.} Quanto tempo?

\noindent\hskip0mm\textsc{g.} Três semanas.

\noindent\hskip0mm\textsc{s.} Qual foi o maior tempo durante o qual você já conservou alguma coisa?

\noindent\hskip0mm\textsc{g.} Três semanas.

\noindent\hskip0mm\textsc{s.} O segundo mais longo?

\noindent\hskip0mm\textsc{g.} Não sei.

\noindent\hskip0mm\textsc{s.} A caixinha de música?

\noindent\hskip0mm\textsc{g.} Umas duas semanas. Você não gostaria de entrar para o departamento de
polícia?

\noindent\hskip0mm\textsc{s.} É parecido. Me conte mais sobre roubar dinheiro de sua mãe.

\noindent\hskip0mm\textsc{g.} Estou tentando atinar do que é que você está falando. Eu pegava o
dinheiro e saía, e o gastava --- não comigo, com meus amigos. Eu não
estava nem aí pro dinheiro. Dinheiro não significa nada pra mim. Ela
ficava maluca, tentando imaginar esconderijos onde eu não conseguisse
encontrar o dinheiro [risadas].

\noindent\hskip0mm\textsc{s.} Não é esse tipo de sentimento que você tem quando invade as casas\ldots{}
que você encontrará o que quer, não importa em que lugar o tenham
posto?

\noindent\hskip0mm\textsc{g.} Certo. Costumava ser todo um ritual, que envolvia ela a esconder o
dinheiro, e eu a encontrá-lo. E eu nem sempre pegava tudo. Se tivesse
quinze dólares ali, eu pegava dez.

\noindent\hskip0mm\textsc{s.} Para mostrar que ela tinha sido enganada.

\noindent\hskip0mm\textsc{g.} Ah, era uma boa dica, porque eu pegava três quartos do total. Eu
pegava para ludibriar minha mãe e trazê-la bem ali, para onde ela
vivia\ldots{} e ela vivia naqueles quinze dólares, ou quantos fossem.
``O que vamos fazer? Estou tão cansada, trabalho tanto! O
que faremos sem dinheiro?'' Ela comentava isso com todos,
em geral. Mas ela nunca discutiu esse assunto comigo.

\noindent\hskip0mm\textsc{s.} Acho que você se sentirá aliviada, se puder se lembrar que a coisa
mais terrível que existe, pra você, é ser trapaceada por sua mãe, sua
mãe a enganar você --- a partir de quando você era bebê\ldots{} e dali em
diante.

\noindent\hskip0mm\textsc{g.} Eu não sei. Não me lembro dela me trapaceando, quando eu era bebê.
Você conhece parte dessa história. Sei que ela me trapaceia, em
qualquer oportunidade que tenha. Ela faz isso de uma forma tão
silenciosa\ldots{} Basta uma palavra, um olhar. Acho que é algo um pouco
diferente, mas parecido. E, seja o que for, é o que faz com que eu
tenha que roubar, quando alguém faz com que eu me sinta como se eu não
fosse uma mulher. Em precedência aos meus roubos, alguém fará alguma
observação, ou comentário\ldots{} ou me lançará um olhar, ou fará alguma
coisa indicando que eles podem estar confusos a respeito\ldots{} a mim, ao
menos, isso confunde: sou masculina ou feminina, sou homem ou mulher?
Dar um telefonema e me tomarem por homem. Alguns comentários que minha
mãe poderia fazer. Mas se isso acontecesse todos os dias, eu não teria
que roubar todos os dias. Parece ser alguma coisa que vai num
crescendo\ldots{}

\end{quote}

\bigskip

O impulso de roubar desapareceu a esse ponto, e permanece ausente,
vários anos depois. Entretanto, a sensação de privação é recorrente;
ela, finalmente, tem consciência do que é a privação; e o ritual do
roubo tendo sido exposto para ela, para sua própria inspeção, ela pôde
começar a procurar por gratificações mais diretas --- e menos
perigosas e hostis. Hoje em dia, ela tem se voltado para as pessoas,
extraindo delas as emoções amorosas que a mãe não foi capaz de lhe dar
no passado. Além disso, outras coisas também estão no lugar: a sra.~G.~sabe, agora, que é uma pessoa, e não um pênis; e que não há
necessidade de desempenhar seus atos sexuais nas janelas apertadas dos
lares suburbanos, e sim em companhia dos corpos vivos dos amantes. Que
grande infortúnio para as pessoas, quando o óbvio precisa ser
disfarçado, e uma verdade que se pode suportar permanece sendo
considerada como insuportável.

Uma vez que já demonstrei, em outra parte deste trabalho, (146) de
que modo a sra.~G.~criou o seu pênis,\idxpenisimag{} sua masculinidade e sua
homossexualidade, talvez umas poucas palavras bastem, aqui. Como parece
ser verdadeiro em relação a outras mulheres que sentem fortes desejos
de ser homens --- as mães\idxmaesfilh{} dos transexuais masculinos (capítulo 8) e as
mulheres transexuais (147) --- uma terrível ruptura ocorre na
infância, separando a menina de sua mãe, que se torna inatingível.
Essas mães ensinaram para essas meninas --- para a sra.~G.~e às demais
citadas --- que a feminilidade não tem valor; então, por exemplo,
quando os irmãos são favorecidos, uma insuportável inveja dos homens se
produz. Essas meninas se tornam adultas: as mães dos transexuais
satisfazem essa inveja do pênis ao criarem seus próprios, lindos,
falos:\idxmaesfalo{} o futuro transexual; os transexuais femininos o fazem pela
``mudança de sexo'', hormonal e cirúrgica,
que lhes acrescenta um falo que é, literalmente, costurado a seus
corpos. A sra.~G.~tanto fez crescer um pênis para si (via alucinação)
como transformou seu próprio corpo em um.

Foi sua grande sorte, perder a necessidade de ter e de ser um pênis;
ao fazê-lo, ela perdeu, também, aquele desejo intenso que se
manifestava através do ato de roubar.


\part[Parte \textsc{iii}: Aspectos sociais da questão]{Parte \textsc{iii}:\\ Aspectos sociais da questão}


\chapter[\textbf{10}\quad A homossexualidade é um diagnóstico?]{{\large\textit{Capítulo 10}}\\ A homossexualidade é um diagnóstico?}
\markboth{Aspectos sociais da questão}{A homossexualidade é um diagnóstico?}

Já que tantas\idxhomos[|(] batalhas sociais\idxhomosaspe[|(] são travadas tendo\idxdiag[|(] a homossexualidade
como tema, ela não poderia se constituir\idxaspec[|(] em ponto focal para um estudo
da perversão --- palavra cujas próprias\idxpervaspec[|(] conotações recendem a questões
morais, ou seja, sociais?\idxpervpecad[|(] Certamente. Mas em função de sua complexidade
e de sua obscuridade como objeto de estudo (e, especialmente, por serem
muito diversas as condições em que o comportamento homossexual
acontece), eu conservei a homossexualidade à parte, nesta busca pelo
sentido da perversão. Contudo, o mesmo não pode ser feito no que tange
aos problemas sociais que ela suscita, em virtude de a homossexualidade
estar tão em voga hoje em dia. E por sua importância, me incomoda a
maneira pela qual essa espinhosa questão é discutida; ela é importante
demais para ser solucionada pela habilidade do raciocínio, por dados
incorretos, pelo autoritarismo ou pela sofística. Vitórias efêmeras
obtidas com espalhafato, de maneira jocosa ou apelando à astúcia, com o
tempo, desmerecem a mais valiosa das causas.

Duvido que alguém já tenha \textit{expertise} suficiente para nos
dizer o que fazer em relação aos problemas sociais que são gerados pelos
diagnósticos psiquiátricos, pois ninguém é capaz de saber o que
aconteceria com o passar dos anos, ainda que fossem postas em prática
medidas de engenharia social. Embora possamos influenciar os problemas
sociais com esses arremedos, como tem sido feito com os diagnósticos
psiquiátricos, com o tempo, o preço a se pagar é alto demais, além de
desnecessário. A mesma postura se aplica a todas as aberrações sexuais
 --- variações e perversões --- em que a pessoa perversa não causa
dano físico ao parceiro ou, como no caso da sedução de crianças,\idxpedof{}
deficientes mentais ou psicóticos, a pessoa não obtém seu prazer pela
força ou por algum outro meio hediondo.

Um diagnóstico é uma palavra, ou uma frase, que rotula uma condição.
``Diagnóstico'' possui também um segundo
sentido, quando se refere ao processo que justamente o precede, de
coletar dados e abstraí-los. Em virtude de a validez de muitos dos
diagnósticos psiquiátricos (diferentemente dos demais diagnósticos
médicos) ter sido questionada através dos anos, deveríamos tentar
avaliar todo nosso sistema classificatório. Esta, contudo, é uma
questão delicada. Os homossexuais, que têm sido vítimas do uso desse
diagnóstico como forma de opressão --- indo desde o insulto até a
negação de seus direitos civis --- não haverão de estar minimamente
interessados nos motivos secretos dos critérios utilizados para os
diagnosticar; ao contrário, a maioria deles desejará que o termo
``homossexual'' seja simplesmente removido,
não por ser inadequado como diagnóstico, mas porque pode ser usado de
maneira maldosa (e já que o mero uso do termo poderia sugerir um desejo
meu de diagnosticar, observo que o termo
``homossexual'', aqui, foi e continuará sendo
usado unicamente para indicar a pessoa que prefere manter relações
sexuais com alguém do mesmo sexo).

Mas nós realmente deveríamos fazer uma distinção nítida entre essas
duas correntes de pensamento --- precisão diagnóstica, por um lado, e
o uso dos diagnósticos como forças sociais\idxdiagforc{} --- ou estaremos sujeitos
àquele tipo de discussão turbulenta, que rende um bom espetáculo, mas
que termina por eclipsar os objetivos mais sérios; porque, mesmo ambas
as linhas de argumentação tendo seu valor, cada uma delas é uma questão
diferente, que requer dados de outro tipo, e outro tipo de lógica. Se
misturarmos as duas, acabaremos dizendo asneiras --- como em geral
fazemos quando problemas sociais, que acirram paixões e demandam ação
 --- são disfarçados como questões científicas ou procedimentais. Já
que tenho preparo para pensar sobre a fenomenologia e o processo de
fazer diagnósticos, mas sem qualquer tipo de qualificação ou\idxpervaspec{}
treinamento para deslindar problemas sociais, eu me concentrarei,\idxpervpecad[|)]
sobretudo, na primeira forma de procedimento,\idxpervaspec[|)] deixando para os mais bem\idxhomosaspe[|)]
informados as excitantes revelações da Verdade Social.\idxaspec[|)]

\section{Critérios para diagnosticar}

Caso não\idxmetod[|(] tivéssemos sido conduzidos a isto,\idxdiagcrit[|(] não teríamos outros
motivos para escolher a homossexualidade --- não mais do que qualquer
outro dos diagnósticos sugeridos --- como assunto sobre o qual travar
nosso debate a respeito da validade dos diagnósticos psiquiátricos. Na
resolução das questões gerais, poderíamos, portanto, facilmente julgar
a pertinência da maioria das rubricas inseridas dentro de cada uma das
categorias da nomenclatura:\idxdiagnome[|(] psicoses,\idxpsico{} neuroses,\idxneuro{} distúrbios de caráter,
ou a balbúrdia de peças soltas, incluindo, entre elas, os distúrbios
sexuais, que não poderiam estar contidos em nenhuma delas. Em sua
maioria, são apenas rótulos.\idxdiagrotu{}

Por exemplo as neuroses, os desvios sexuais, o alcoolismo ou a
toxicomania. De um modo dolorosamente simplista, cada um desses
diagnósticos é concebido para uma característica distinta; para rotular
alguém, acabamos nos decidindo pelo que é mais evidente, incapazes de
lidar com grande parte dos processos que ocorrem no íntimo do paciente.
Naturalmente que este sistema está condenado a uma trajetória infeliz;
um polimento a cada década, aproximadamente, não tem ajudado muito.
Seria comparável a uma classificação, destinada à medicina em geral, e
que fornecesse ``diagnósticos'' tais como:
tosse, febre, dor de cabeça, indigestão crônica, fraqueza geral, humor
deprimido, ou dispepsia. Do mesmo modo, a maior parte dos
``diagnósticos'' psiquiátricos deveriam
também ser eliminados. Mas se os eliminarmos, ficaremos sem
nomenclatura.\idxnomen[|(] Sem nomenclatura, não teremos por onde começar a fazer
nossas comunicações a respeito do tratamento, ou da pesquisa.

Permitam-me recapitular o que considero ser um diagnóstico e, ao
fazê-lo, demonstrar por que, presentemente, não apenas nenhum sistema
de classificação funciona na psiquiatria, mas também por que existe um,
na medicina como um todo, que funciona perfeitamente. Um diagnóstico
deve ser uma explicação extremamente compacta. Para fazer um
diagnóstico adequado, em qualquer ramo da medicina, deveria haver: (1)
uma síndrome\idxdiagsind{} --- uma constelação de indícios e sintomas,
compartilhados por um grupo de pessoas, e visíveis a um observador; (2)
dinâmicas subjacentes (patogenia;\idxdiagpato{} fisiopatologia, na medicina em geral,
e neurofisiopatologia --- ou psicodinâmicas --- na psiquiatria; (3)
etiologia\idxdiagetio{} --- aqueles fatores dos quais as dinâmicas se originam.
Quando essas três condições existem, poderemos, então, poupar nosso
tempo usando abreviações, sabendo que uma ou duas palavras --- um
rótulo, um diagnóstico --- comunicará aos demais estudiosos aquilo que
sabemos. Infelizmente para os psiquiatras em geral, nós não nos
confrontamos com pessoas cujos pensamentos, sentimentos e comportamento
possam ser categorizados desse modo. Exceto pelos distúrbios que são
propriamente ``doenças'', no mesmo sentido em
que o termo é usado na medicina como um todo --- tais como as
síndromes cerebrais orgânicas,\idxcerea{} que podem incluir alguns dos tipos de
esquizofrenia\idxesquiz{} e de psicoses\idxpsico{} afetivas --- as condições para as quais
nossa especialidade foi desenvolvida, no geral, não satisfaz àqueles
três critérios. E assim, se o que acabei de expor for um modo adequado
de encarar a estrutura de um diagnóstico (e alguns pensam que sim
[49]), o atual sistema de classificação é extremamente imperfeito.

Poderíamos, até mesmo, discutir se o nosso atual sistema de
diagnóstico deveria ser totalmente descartado, como alguns chegaram a
sugerir. O preço poderia ser demasiadamente alto; mas devo admitir que
há, em mim, uma leve tentação de ver isso acontecer; parece lógico que
se uma abreviatura, que deveria transmitir um diagnóstico, não é uma
informação passível de ser compartilhada entre aqueles que a usam,
então ela servirá apenas para confundir; e poderia ser melhor
substituída, por algum tempo, por frases descritivas. Contudo, nós não
eliminaremos o sistema classificatório; e, assim, a psiquiatria
persistirá percorrendo, um a um, a lista dos diagnósticos, ano após
ano, testando a popularidade dos itens --- as prioridades sendo
determinadas por questões tanto sociais\idxdiagforc{} quanto científicas. Nossos
problemas com a nomenclatura\idxdiagnome[|)] demonstram quão longo é ainda o caminho,
até que se possa constituir uma psiquiatria que esteja enraizada na
metodologia científica.\idxdiagmeto[|(] Mas é mau sinal já darmos início à tarefa de
forma fragmentada; isolar a homossexualidade do restante do cambaleante
sistema --- a menos que todos compreendam que a particularidade do\idxmetod[|)]
exemplo deverá servir apenas para iluminar as questões gerais --- é
ignorar a paralisia de que padece a estrutura como um todo,\idxnomen[|)] que é de
uma incoerência mortal.

Assim, nós não deveríamos excluir o termo
``homossexualidade'' pelo fato desse
diagnóstico trazer angústia aos que com ele são diagnosticados. Esse
efeito indica problemas sociais importantes;\idxhomosaspe[|(] mas nosso argumento ficará
confuso, caso insistamos em falar de diagnóstico quando, de fato,
estaremos falando sobre a maneira pela qual os diagnósticos podem ser
usados de maneira perversa. Muitos homossexuais, hoje, sentem que o
simples diagnóstico de ``homossexualidade''
serve, nas mãos dos psiquiatras (que deveriam ser mais precavidos) e do
público (que não se importa em se precaver) como ferramenta para
oprimir pessoas cujo único crime é seu estilo sexual. Eu concordo: já
que a sociedade faz isso, e a psiquiatria se permite ser subserviente a
tal ponto, aí se comete uma injustiça, ofensiva para os homossexuais e
degradante para os psiquiatras. Mas um diagnóstico não deveria ser
invalidado por esse tipo de motivo.

Se usarmos os três critérios antes aventados para que se considere
uma condição como sendo um diagnóstico, então a homossexualidade não é
um diagnóstico: (1) existe apenas uma preferência sexual (e que é tão
digna de atenção porque assusta tanta gente em nossa sociedade), não
uma constelação uniforme de indícios e sintomas; (2) pessoas
diferentes, e que têm essa mesma preferência sexual, têm outros tipos
de psicodinâmicas subjacentes a seu comportamento sexual; e (3)
experiências de vida muito diferentes podem causar esse comportamento e
essas dinâmicas. \textit{Existe} um comportamento homossexual; ele é
variado. Pessoas com todo tipo de personalidade preferem a
homossexualidade como sua prática sexual: pessoas sem nenhuma
sintomatologia abertamente neurótica --- assim como esquizofrênicos,
obsessivos-compulsivos, alcoólatras, pessoas com outras perversões ---
praticamente todas as categorias da nomenclatura. Porém, não existe uma
\textit{coisa} que corporifique a homossexualidade. Neste sentido, a
palavra deveria ser removida da nomenclatura.

Com relação à patogenia, provavelmente ninguém, hoje em dia --- nem
mesmo entre aqueles que são a favor do diagnóstico\idxdiag[|)] --- acredita numa
causa única para o comportamento homossexual, o que faria com que ele
fosse uma \textit{coisa}. Os excelentes estudos da literatura sobre a
etiologia,\idxhomosetio{} feitos por Bieber\idxbieb{} et al. e Socarides,\idxsocar{} somados às suas
próprias descobertas (3, 130) reforçam a impressão de que muitos
caminhos conduzem à predileção de alguém por indivíduos de seu próprio
sexo. Isto é verdadeiro mesmo, e especialmente, em relação às teorias
analíticas sobre a etiologia.

Será que devemos abrir mão de um diagnóstico por ele causar dor? Há
algo de desabonador no uso de nosso frágil método de\idxdiagcrit[|)] diagnóstico?
Também quando os psiquiatras em massa são considerados como bodes
expiatórios pela maneira cruel com que os homossexuais têm sido, e
continuam sendo, tratados? Estes não são os verdadeiros motivos pelos
quais os homossexuais são maltratados (apesar de poderem ser
emprestados, para fazerem deles tal uso). Na melhor das hipóteses, nós,
a começar por Freud,\idxfreud{} somos parcialmente responsáveis pelo fato de os
homossexuais poderem, agora, começar a lançar um revide contra a
sociedade. Mesmo sendo imprecisos ao considerarmos a homossexualidade
como um diagnóstico, fazer isto significou que o homossexual é parte do
reino natural, e não um membro da espécie dos pecadores malditos.

Os oprimidos,\idxhosthomo[|(] à medida que vão encontrando sua força, podem ver que
existem duas verdades --- em geral não relacionadas --- que sua causa
incorpora. A primeira é a moralidade de seu estado oprimido, e as
efetivas falsificações contadas por um lado (ou pelo outro) a fim de
remover (ou manter) uma opressão; a primeira verdade será que a
opressão é má (ou boa). Os oprimidos são sempre vítimas das definições.

A segunda verdade --- a busca da realidade (método científico)\idxdiagmeto[|)] ---
podemos colocá-la de lado por enquanto, enquanto tratamos da primeira
 --- nossa convicção de que nossa causa é justa. A posição do
homossexual se torna mais honrada, ou ao menos mais pungente, face à
crueldade pública. Em realidade, não existiria nenhum diagnóstico de
homossexualidade --- somente seria reconhecida a miríade de formas de
comportamento homossexual --- se o fanatismo em relação àquilo que é
certo não forçasse a crença (partilhada até mesmo entre os
homossexuais) de que uma determinada essência --- a homossexualidade
 --- existe.

 Parte do ódio que a sociedade\idxdiagforc{} dirige contra a sexualidade é %%cf. original&&
indevido, porque não foi provocado, como sua preocupação em relação a
quais orifícios de qual sexo são usados. Esse ódio é, sobretudo, nossa
herança cultural. Mas outra parte, que tem pouco a ver com o ato
homossexual do coito, o homossexual (em geral homem) provoca de forma
proposital, consciente; ele contribui --- e até mesmo sente prazer
nisso --- para sua própria opressão. Por muitas e complicadas razões,
muitos homossexuais estão mancomunados com o histriônico, a mímica, a
caricatura, sempre que uma audiência --- seja ela homo ou
heterossexual --- estiver presente. O sarcasmo --- a hostilidade ---
entra como ingrediente dessas encenações, onde existe uma piada:
``Quando eu pareço estar ridicularizando a mim mesmo,
estou, na verdade, ridicularizando o mundo certinho, careta --- com o
bônus adicional de eles serem estúpidos demais até mesmo para
perceberem o que estou fazendo com eles.''

Embora seja verdade que o público não percebe exatamente o que está
sendo feito ali, não há como não intuir a presença de uma gozação;
então ele se zanga e arremete contra seu algoz, como um touro
enraivecido. Este ataque pode ferir o homossexual; mas, mesmo sendo
ferido, ele ainda se sente superior, pois \textit{ele} não é um touro
 --- um animal cego, estúpido. Antes, ele é um esteta --- um
provocador, não um cavalo de batalha.

Muitos homossexuais aprenderam esses e outros métodos de lidar com
as situações com base nas escaramuças que, na infância, perderam para
seus pais.\idxrelpchomo{} Alguns homossexuais, a maioria deles, é derrotada por suas
mães\idxmaeshomo{} chantagistas,\idxhomosexpe{} e tinham pais que forneciam apenas uma inépcia
passiva; isto dificilmente encorajaria um filho a rivalizar, tendo um
pai desse tipo. Outros, brutalizados por pais dominados por uma raiva
cega, fogem, refugiam-se nos maneirismos de suas pobres mães (esses
dois exemplos não pretendem explicar as origens do comportamento
homossexual --- embora eu realmente acredite que fatores como este,
somados a muitos outros, possam contribuir para isso). Em todo caso,
há motivos de sobra para a vingança que,\idxvingahom{} acredito, energiza muitos
aspectos do comportamento homossexual --- e não só de seu
comportamento erótico. Assim, para salvaguardar um sentimento de
autoestima das fossas do desespero, eles precisam lançar um
contra-ataque a todos os que têm as mesmas qualidades dos antigos
inimigos de sua infância. Esses mecanismos, embora em diferentes formas
e gradações, podem ser encontrados também em pessoas não homossexuais;
o masoquismo não é domínio exclusivo deles. Realço,\idxhosthomo[|)] uma vez mais, que
não ofereço estas ideias como explicações completas.

Três mecanismos utilizados pelos homossexuais, e que provocam
rajadas de ódio que lhes são disparadas pelos heterossexuais, são:

\begin{enumerate}
\item Os homossexuais transferem o ódio originariamente dirigido contra
seus pais aos substitutos da figura paterna na sociedade --- e esses
substitutos contra-atacam;

\item Os homossexuais, que aprenderam na infância a odiarem a si
mesmos, continuam a atrair punição, em parte porque eles concordam com
a crueldade da sociedade estabelecida; eles provocam ataques a fim de
serem humilhados;

\item Os homossexuais podem ameaçar a posição
``espada'' do heterossexual militante com
provocações que procuram dirigir a atenção deles para seus próprios
potenciais homossexuais ou afeminados.\idxhomosafem{} Para se autoafirmar, pode ser
que o heterossexual retalie.
\end{enumerate}

Essas dinâmicas de hostilidade\idxhomoshost{} são, creio, características da
homossexualidade masculina mais do que da feminina.\idxlesb{} Por exemplo, não se
vê muita caricatura no comportamento\idxhomoslesb{} masculino das mulheres,\footnote{
Será que isto tem relação com o fato de a mulher\idxlesb[|nn] saber que ela é
mulher,\idxhomoslesb[|nn] como a mãe?} mas ela é parte essencial da efeminação nos
homens. Em geral se acredita que a homossexualidade feminina tenha sido
 ignorada por milênios, pelas culturas, em virtude de as mulheres
ocuparem uma linha que fica abaixo do desprezo --- e da preocupação
 --- em todas as sociedades. Isto pode não ser tudo: as mulheres
homossexuais, menos abertamente ásperas e hostis em relação a seus
opressores do que a maioria dos homens homossexuais, atraem menos
ataques sobre si. Ou, pelo menos, vinha sendo assim, até recentemente.

Retornando de nossa digressão: a primeira verdade, então, é a
imoralidade da opressão. A segunda verdade, --- menos importante nas
crises do oprimido --- que acredito ser também uma causa social que, a
longo prazo, tem a sua devida importância: ela diz respeito ao método\idxdiagmeto[|(]
científico,\idxmetod{} este conjunto de regras magnificamente construído --- uma
consciência confiável para cada pesquisador que é dono, por sua vez, de
sua própria consciência corruptível --- e que é a norma condutora do
esforço para localizar os fatos. Essa terceira verdade (embora não fale
necessariamente de algum cientista em particular) representa uma
verdade maior: que a honestidade tem um valor social a longo prazo para
a humanidade, e que precisa ser protegida, encorajada e ensinada; e que
sua metodologia precisa ser indefinidamente refinada. O processo de
diagnosticar, na medicina --- um processo de detecção --- é um
instrumento desse método científico.\idxdiagmeto[|)] De tal forma que somos\idxhomosaspe[|)] remetidos
para a\idxhomosetio{} etiologia.

Nessa busca pelas múltiplas causas do comportamento homossexual,\idxheterohomo[|(]
podemos encontrar dados que demonstram que, para muitos homossexuais,
as preferências na escolha do objeto, assim como alguns de seus
comportamentos essenciais, habituais e não eróticos (tais como a
efeminação\idxhomosafem{} dos homossexuais homens) desenvolveram-se como resultado de
traumas e frustrações durante a infância. Estas observações valem
também para a maioria dos heterossexuais,\idxhomoshete[|(] apesar de os traumas e as
frustrações serem de tipos e intensidades diferentes.

Se dividirmos os humanos em dois tipos, os heterossexuais e os
outros, como é o costume, poderemos distinguir entre os dois pelas
seguintes formas idiossincráticas (aliás, como Freud\idxfreud{} já o fez): os
hábitos sexuais da maioria dos humanos, incluindo a maioria dos que
preferem relações homossexuais, são heterossexuais. (A
heterossexualidade também pode, obviamente, conter a homossexualidade).
As neuroses eróticas --- as perversões óbvias e, até mesmo, a maioria
das variações da heterossexualidade evidente, tais como a promiscuidade\idxpromiscomp{}
compulsiva, a utilização de pornografia,\idxporno{} a preferência por\idxprost{} prostitutas\idxprost{}
e a masturbação\idxmastur{} adulta --- são distorções heterossexuais, acordos de
conciliação; não obstante, eles são repletos de excitação,
possibilitando que alguém abra mão de certos desejos, desde que outros
possam ser salvaguardados. Se isto fizer com que as minorias oprimidas
se sintam melhor, poderemos conceder, a cada um de nós, um diagnóstico;
tal afirmação raramente seria uma distorção dos fatos. Todos têm seu
próprio estilo ou fantasia específica, de que se utilizam em seus
devaneios, ou que encenam com objetos; todos têm direito a uma
categorização.

Mas por que alegar que a heterossexualidade é a preferência da
humanidade? Muitos insistem em dizer que a heterossexualidade\idxheterocrit[|(] é o
estado biologicamente natural do homem: primeiramente porque é assim
que acontece em todas as demais espécies; em segundo lugar, porque
somente ela é capaz de evitar a morte de todas as espécies. No entanto,
não existe evidência direta para essa propensão biológica no homem, a
não ser pelo fato, aparentemente surpreendente, de a maioria das
pessoas ser mais ou menos heterossexual.\idxheterohomo[|)] Apesar de inexistir razão para
negar que possa haver alguma dessa tendência biológica, sabemos que
acontecimentos psicológicos podem, com tanta frequência, sobrepujar
essa heterossexualidade latente, que a consideração de que isto seja
biologicamente fixo constitui uma fundação demasiadamente frágil para
que, sobre ela, se possa construir uma teoria --- ou uma sociedade.
Talvez uma força até mesmo mais poderosa, que pressiona em direção à
heterossexualidade nos humanos, seja a constituição da família,\idxheteroinfl[|(] que
pode ter sido inventada não por causa da heterossexualidade biológica,
mas por realidades que envolviam questões de vida e morte, e que veio
pautando nossa existência até os dia de hoje. À medida que a segurança
e o conforto aumentam, alguns acreditam que a próxima vítima, depois de
Deus, possa vir a ser, justamente, a família.

Ao dizer que a família\idxinflu{} é mais efetiva do que qualquer estímulo
biológico na promoção da heterossexualidade,\idxheterocrit{} quero dizer o seguinte:
toda criança sabe ser o produto de um inevitável ato sexual
heterossexual, ato que é íntimo, excitante, misterioso, surpreendente,
profundo, perigoso, proibido --- e terrivelmente desejável; e toda
família --- mesmo aquelas cujo fracasso produz graves transtornos à
criança que abrigam --- transmite incessantemente à sua prole
mensagens de que a família ideal é heterossexual.\idxheteroinfl[|)] Por mais restrito que
o mito da heterossexualidade possa ser, por mais que existam militantes
sexuais que o abominem, e por mais amargas que as pessoas tenham se
tornado por sentirem o quanto a realidade de sua vida sexual caiu
abaixo da perfeição, a heterossexualidade amorosa é o critério --- e
ela envolve afeição, respeito, honestidade, diminuição do egoísmo,
interesse erótico duradouro, gratificação prazerosa, fidelidade,
sentimento de regozijo pelas crianças, e a criação de uma unidade maior
e mais original do que as duas pessoas que a constituíram. E isto é
assim não porque tenha sido ordenado pelos céus, pela biologia ou pela
teoria econômica, mas porque quase todos os membros de nossa sociedade
o aceitam, em algum ponto dentro de si, como o ideal que os persegue.

Assim, as perversões (porém não todas as aberrações sexuais) são
modificações que precisamos inventar a fim de preservar alguma de nossa
própria heterossexualidade. A forma que a perversão assume pode estar
distante dessa forma extrema, qual seja,\idxhomoshete[|)] a preferência do homem pela
mulher e vice-versa, com ambos desfrutando, com todo seu ser, dos
aspectos sexuais e amorosos de seu relacionamento.\idxheterocrit[|)] Contudo, embora
invisível, esse ideal está lá, encerrado no âmago da maioria de nós,
ainda que se manifeste apenas em alguns poucos.

Agora, retornando à ideia de que os diagnósticos podem ser usados
para manipular as pessoas.\idxhomosaspe{} Eu sugeriria ainda que só fosse removido de
nossa classificação o diagnóstico --- e não apenas
``homossexualidade'' --- se for possível
provar que a condição vinculada a ele não existe --- e não pelo fato
de esse diagnóstico poder trazer sofrimento. Sendo assim, é nossa
responsabilidade definir com precisão cada um dos diagnósticos
sugeridos em nossa classificação, pois, com uma definição adequada,
poderemos determinar se aquilo que foi definido existe de fato. Talvez,
com o tempo, isto ocorrerá com relação a um certo número de condições
que são, hoje, indiscriminadamente definidas como
``homossexualidade'' --- apenas porque o
objeto sexual preferido pela pessoa é de seu mesmo sexo. Vamos contar,
então, com uma quantidade de subdiagnósticos dentro de uma categoria
principal, ``as homossexualidades''.

Um dia, talvez, nosso conhecimento seja suficiente, e nos tornemos
capazes de diagnosticar da mesma forma como as demais áreas da medicina
o fazem. Por outro lado, poderemos chegar à conclusão de que
diagnósticos são um tipo de ocupação utópica, quando envolvem a
identidade humana. Nesse meio tempo, eu sugeriria que abríssemos mão
dos diagnósticos utilizados apenas para dar realce quer ao
comportamento sexual mais extravagante (que pode ter sido apenas um ato
momentâneo) ou mesmo ao comportamento sexual predileto, simplesmente
porque a pessoa não tem por hábito relacionar-se sexualmente com alguém
do sexo oposto.

Podemos ser tão espertos: encobrimos nossos insultos fazendo uso da
gramática. Dizer: ``Ele é \textit{um}
neurótico'' é mais incondicional do que dizer
``ele é neurótico''; a primeira versão torna
a pessoa sinônimo de sua neurose. Se, seja por gentileza ou por
acurácia, dissermos ``ele \textit{tem} uma
neurose'', diminuímos nossas chances de causar desagrado.
Suponha agora que, ao querer comunicar, de modo preciso e sucinto, nós
disséssemos: ``Ele tem uma
homossexualidade''; o poder de insulto das palavras se
enfraquece. Não estamos mais dizendo nem que seus hábitos sexuais são a
totalidade de seu ser, nem que nós --- os grandes árbitros ---
simplesmente não temos interesse no resto de sua personalidade. Uma
pitada das regras de ouro poderia aprimorar nossos hábitos, ao
diagnosticar.

Talvez pudéssemos tentar, como regra à qual atermo-nos --- isto até
o dia em que saibamos o que estamos fazendo, e se as circunstâncias
exigirem que um rótulo\idxdiagrotu{} psiquiátrico seja fornecido --- proceder da
seguinte forma:

\begin{itemize}
  \item[A.] \textit{O tipo de personalidade (caráter) habitual desde a infância
  ou adolescência,} por exemplo, obsessivo-compulsivo, esquizofrênico,
  histérico, depressivo.

  \begin{itemize}
    \item[1.] \textit{A síndrome\idxsindr{} presente,}\idxdiagsind{} por exemplo: toxicomania, neurose
    de angústia, psicose esquizoide.

    \begin{itemize}
      \item[a.] \textit{Síndromes} \textit{subsidiárias igualmente presentes},
      por exemplo: alcoolismo, síndrome cerebral orgânica não psicótica
      acompanhada por senilidade; distúrbio respiratório psicossomático
      (asma).

      \begin{itemize}
        \item[(1)] \textit{Preferência sexual}, por exemplo: heterossexual,
        monogamista; acompanham fantasias de ser estuprada por um garanhão;
        homossexual com fetichismo por prepúcios; heterossexual, dando
        preferência a cadáveres; homossexual que pratica sexo anônimo com
        pênis desincorporados\footnote{ \textit{Disembodied penises}, no original:
        	o autor se refere aqui --- no contexto do sexo casual, homossexual e masculino,
        anônimo e em lugares públicos --- ao fato de que o pênis, como objeto de fetiche, é \textit{independente} do sujeito. [Nota da Edição]}
        (promiscuidade de banheiros públicos); heterossexual,
        voyeurismo; homossexualidade que só se expressa em fantasias, durante
        as relações sexuais com a mulher.
      \end{itemize}
    \end{itemize}
  \end{itemize}
\end{itemize}

A vantagem desse sistema classificatório,\idxclass{} pelas síndromes,\idxdiagsind{} é que ele
não pretende ser um sistema diagnóstico, ou seja, explicativo. Ele
admite sua ignorância: é descritivo.

Esta posição vale para ambos os litigantes, os oprimidos e os
opressores (vale, inclusive, certamente, para alguns psiquiatras): os
diagnósticos não deveriam ser descartados por poderem aborrecer a
pessoa rotulada, nem deveriam ser mantidos porque podem ser usados para
oprimir. Nenhum desses extremos faz jus à função dos diagnósticos.
Somente deveriam ser removidos aqueles diagnósticos que não conseguirem
descrever a patologia de modo sucinto e preciso. Este é o caso em
relação à homossexualidade, que ainda não é capaz de funcionar como um
diagnóstico verdadeiro: devemos removê-lo. E, uma vez que isto vale
também para a maioria dos demais\idxhomos[|)]
``diagnósticos'' da psiquiatria, rejeitemos o
sistema (embora ainda não a totalidade dos rótulos) e comecemos de
novo.




\chapter[\textbf{11}\quad Sexo e pecado]{{\large\textit{Capítulo 11}}\\ Sexo e pecado}
\markboth{Aspectos sociais da questão}{Sexo e pecado}


Para nosso propósito de agora, estando preocupados somente com o
erotismo, permitam-me definir pecado\idxpecad[|(] como um termo exagerado para o
desejo de fazer mal a alguém. Sendo assim, a ética e a\idxpervrespo{} moralidade\idxrespo{} são
escalas de que a sociedade se utiliza para mensurar o pecado; elas\idxsexuehos[|(]
existem para justificar ou para atenuar a hostilidade. Ao demonstrar
que a hostilidade desempenha um papel essencial na geração e na\idxpecaddel{}
manutenção da\idxsexue[|(] excitação sexual\idxhostexci[|(] humana, eu estive também,
subliminarmente, estudando algumas das dinâmicas da ética e da
moralidade.

Colocar a hostilidade no centro dessas definições me põe um tanto em
desacordo com os que explicam a sensação de pecado que experimentamos
com o prazer sexual apenas como um fenômeno histórico-cultural. A
justificativa mais recente descreve esse sentimento de pecado como
sendo o produto de uma herança judaico-cristã, fortalecida, a cada
nova geração e em cada localidade, pelas estruturas ali existentes, a
serviço dos fanáticos. Se esta explicação for usada, a solução para o
sofrimento causado por essas forças repressivas\idxaspecrep[|(] é simples de se
conceber (embora difícil de se obter): ao mudarem as crenças da
sociedade, o sentimento de pecado se dissipará.

Talvez. Mas enquanto esperamos por esse dia feliz, lembrando-nos de
que a vida interior não é somente resultado, mas também causa, da
cultura, examinemos também as dinâmicas\idxsexuedin{} do prazer sexual \textit{no
íntimo} de uma pessoa. Ao fazê-lo, poderemos descobrir que algumas das
forças repressivas sociais --- vivenciadas, dentro do indivíduo, como
um sentimento de pecado --- têm sua origem em ataques disferidos por
uma parte da pessoa contra outra (a ferroada da consciência); as forças
sociais não existem apenas lá fora, ao vento; com efeito, na trilha
final, compartilhada por cada membro da sociedade, elas estão presentes
sob a forma de dinâmicas intrapsíquicas.

Em círculos esclarecidos, hoje em dia, é difícil defender a ideia de
que o sexo e o pecado estão ligados. Quer dizer que não existe nenhuma
base lógica para as pessoas, quando sexualmente excitadas,
experimentarem --- infelizmente --- um sentimento de desconforto e
estranhamento? E para que partam numa busca voluntária de depravação,
insalubridade e devassidão?

Ao responder, poderemos encontrar nossa primeira pista no fato, há
muito tempo conhecido, de que a consciência de estar fazendo alguma
coisa proibida, em geral, aumenta a excitação sexual. Isto se torna
mais evidente pelos dados reunidos neste próprio livro, que indicam
que, quando um certo grau de hostilidade e\idxsexuedes[|(] desumanização\idxdesuexci[|(] do objeto
desejado não faz parte do cenário, para a maioria das pessoas, o
sentimento de lascívia apaixonada é substituído por um prazer insosso.
Efeitos tão superficiais como os que são afirmados pelos estudiosos da
cultura não bastam como explicação principal para o sentimento de
pecaminosidade --- ou seja, ele não é, simplesmente, uma imposição
cruelmente lançada ao cidadão pela sociedade inconsequente; em
verdade, ele se origina\idxpecadori{} também do fato de estarmos, pelo menos
levemente, conscientes de que uma parte da excitação sexual está
subordinada ao desejo de se fazer mal a alguém. O estudo da perversão
demonstra esse mecanismo em funcionamento, tendo me conduzido a uma
maior compreensão das aberrações de menor porte, às quais, comumente,
nos referimos como\idxpervnorma{} ``sexualidade\idxnorma{} normal''.

Em ambas, tanto na perversão quanto na ``sexualidade
normal'', temos encontrado diversos temas: à medida que o
ato sexual se desenvolve, a pessoa\idxperigo{} enfrenta perigos\idxpervexpos{} imaginários, e
sente que os superou; dentro da excitação sexual existem desejos ---
conscientes e inconscientes --- de machucar alguém, como forma de
vingança\idxvinga{} por traumas\idxtrauma{} e frustrações do passado; o ato sexual serve para
transformar o trauma da infância em triunfo do adulto; trauma, perigo e
vingança criam um clima de excitação que se intensifica, quando envolto
em mistério.

As ideias que acabamos de examinar servem apenas para complementar o
conhecimento que temos sobre as origens do pecado; estas origens
encontram-se em conflitos que são gerados nos primórdios do
desenvolvimento, na primeira infância e na infância --- primórdios
esses que são, conceitualmente, ordenados como as fases oral, anal,
fálica e edípica.\idxlibid{} A possessividade cruel e os impulsos destrutivos dos
primórdios da vida, que ficam mais ou menos inseridos naquelas
experiências psíquicas a que denominamos superego, é o que nos
proporciona tanto os dados quanto uma moldura, ambos essenciais para a
compreensão do sentimento de pecado. Ao observar a raiva e a crueldade
que emergem das frustrações e dos traumas primitivos, conseguimos
rastrear como esses sentimentos, e o sentimento de opressão que o
acompanha, são convertidos em prazer sexual.

Este conhecimento das dinâmicas hostis\idxhost[|(] na excitação sexual podem
conduzir os mais propensos a preocupações relacionadas à ética e à
moralidade, uma vez que essas duas instituições lidam com a modulação
da hostilidade, entre indivíduos ou no interior de cada um. Se a ética
e a moralidade são úteis à sociedade por definirem e lidarem com o
pecado, então esta exploração\idxsexul[|(] da excitação sexual sugere que a ética e
a moralidade do comportamento sexual tentam, intuitivamente, subjugar
tais dinâmicas hostis. Talvez, se essa tentativa pudesse ser trazida à
consciência, fôssemos capazes de diminuir a hostilidade --- que, em
suas formas extremas, atinge níveis de perversão --- que os sistemas
éticos e morais reformistas encaram como sendo um contrapeso à noção de
pecado. Como estratégia de ação social, talvez aqueles que desejem
incrementar sua liberdade sexual devessem não se apoiar demais no
argumento de que o sentimento de pecado existe apenas como consequência
de nossa escravidão aos processos históricos repressivos.\idxaspecrep[|)] Pode ser que
o sentimento de pecado não desapareça simplesmente por resolvermos
decretá-lo como fora de moda e, caso excluamos o pecado de nossos
estudos, a complexa riqueza da excitação sexual humana sairá
perdendo.

Se essas ideias sobre a perversão fossem aceitas,\idxcrime{} a função dos
tribunais seria simplificada. Se fosse do conhecimento de todos que
quase todo o comportamento sexual traz em si traços do mecanismo
perverso e, portanto, que impulsos e atos perversos são universais
(isto, certamente, já é sabido, embora ainda não tenha sido reconhecido
pela legislação),\idxleis[|(] o critério para se decidir sobre a ocorrência de
crime não precisa mais ser: ``A perversão estava
presente?'' Ao invés disto --- uma maneira mais sensata,
mais justa, de definir um crime --- um juiz ou um júri precisaria
apenas decidir se foi perpetrado um ato hostil causador de danos, a
pessoas ou a propriedades, a um grau que o código penal considere como
significativo para crimes não sexuais; assim, deixaria que o mesmo
critério prevalecesse.

Uma decisão deste tipo dispensaria a opinião de
``especialistas'' --- psiquiatras.

Esta discussão certamente cai no absurdo se nos esquecermos que nem
todos os pecados são iguais. A fantasia de estupro não é estupro; a
fantasia inconsciente de vingança do travesti conduz a uma violência
que é, no máximo, fazer com que ele se masturbe num chapéu de mulher. A
presença do sentimento de pecado, portanto, não deve ser relacionada à
violência real, efetiva; felizmente, as leis que governam o
comportamento sexual, apesar de geralmente estúpidas, algumas vezes
levam isso em conta.

Os psicanalistas se apegam às discussões sobre moral e ética assim
como os bêbados à bebida. Não desejo fazer aqui o papel de mais um
grande mestre do comportamento sexual, julgar se a liberdade sexual
prejudica ou enriquece a sociedade, ou declarar quais leis deveriam ser
criadas, e com que grau de severidade, para que elas refletissem,
assim, a nossa moralidade. Existe, porém, uma preocupação que acredito
ser digna de ênfase: se negarmos a hostilidade e a desumanização\idxdesu[|(] nas
fantasias que colaboram para a excitação sexual\idxdesuexci[|)] --- se dissermos que o
pecado não está ali --- estaremos negando o óbvio, o que é uma
idiotice.

Existem aqueles para quem, em sua ortodoxia, pecado e
responsabilidade pessoal constituem a pedra fundamental da estrutura da
sociedade: cada um deverá saber, pesar e colher as consequências de seu
próprio comportamento. A tese, então, de que a excitação sexual e a
necessidade de machucar seus objetos são assuntos que se relacionam de
perto, faz com que o controle do sexo seja domínio não apenas de
dinâmicas pessoais mas, ao mesmo tempo, com que seja uma questão
política.\idxsexue{}

Aqueles que, sem cair no fanatismo, pensam no sexo em termos de
pecado, partem de uma posição semelhante à minha: quando, por angústia,
nós desumanizamos nossos objetos sexuais, nós minimizamos a nós mesmos,
renunciando ao que há de melhor no ser humano --- a capacidade de
amar.\idxsexulres[|(] Além disso, para os mais veementes partidários desse
posicionamento, a sociedade também não tem qualquer interesse em
encorajar sua própria dissolução, seja pela propaganda
pró-libertinagem, pela pornografia, pelo afrouxamento das leis ou
através de pesquisas laboratoriais sobre o comportamento sexual humano
(67).\idxleis[|)] Por outro lado, se nós desencorajarmos a desumanização,\idxaspeccon[|(]
restringindo a sexualidade infantil ilimitada das crianças,
adolescentes e adultos, nossa recompensa será o poder do amor. Essa
posição é corajosa, se não quase suicida pois, ao fazer oposição ao
direito à perversão, ela tenta impedir um poderoso impulso que hoje em
dia move a nossa sociedade. Aqueles que articulam esse conservadorismo,
não apenas fomentam o ataque dos novos intelectuais e das maiorias
morais da esquerda, mas têm também que tolerar os ataques políticos da
direita, como colegas daqueles que, há muitas gerações, defenderam esse\idxsexuedes[|)]
mesmo terreno.\idxsexue[|)]

Um apoio teórico para tal argumento vem de dois grupos que, em
geral, não costumam estar de acordo: os psicólogos inspiracionistas
(como May,\idxmay{} Polanyi\idxpolan{} e Frankl)\idxfrank{} e os psicanalistas (como Freud\idxfreud{} e Khan).\idxkhan{} A
partir da primeira convicção, cuja premissa básica é a de que o homem é
bom quando não corrompido, retira-se a força moral para reivindicar
``sanções'' que preservem essa bondade, com o
argumento de que o amor necessita de restrições sexuais, para que uma
relação duradoura com outra pessoa possa vingar. Sendo assim, se
quisermos preservar o que há de mais valioso nas relações humanas, é
necessário que façamos oposição\idxhostexci[|)] ao inimigo do amor --- a
licenciosidade sexual (``fascista'',
``esquizoide'',
``ilusória'' [67]).

A segunda ideologia, a psicanálise,\idxpsica[|(] demonstra que a hostilidade jaz
no âmago da perversão --- podendo, assim, ajudar a fortalecer a\idxsexuehos[|)]
convicção dos que acreditam que o atual aumento da liberdade sexual é
mau (ver [76] para recapitulação).\idxhost[|)] A descoberta de Freud,\idxfreudsexua{} de que as
aberrações sexuais são resultado de uma ruptura traumática do
desenvolvimento infantil, fornece um bom pano de fundo. As descobertas
de Khan\idxkhan{} ampliaram nossa compreensão sobre o significado e a função das
perversões: o uso das pessoas como coisas (desumanização) ou como
objetos de inveja e cobiça, em vez de objetos de amor; o uso de
técnicas manipulativas\idxpervtecni{} de intimidade\idxintim{} para explorar os parceiros; a
falsificação do próprio \textit{self}; e a perversão como encenação,
mais do que um verdadeiro relacionamento entre pessoas (75, 76). Não
podemos duvidar, depois de examinar tais descobertas, que a perversão
não é --- como uma variação --- apenas ``um modo
alternativo de vida'', como alguns apologistas a
classificariam hoje.\idxpervdirei[|(] O argumento está claramente enunciado, aliás,
desde Freud: a capacidade para o prazer sexual, na perversão, só pode
ser conservada às custas de algum sacrifício feito à humanidade de seus
objetos, e da deformação do próprio \textit{self}. Quando alguém
precisa reduzir pessoas a coisas, o amor --- combinado a seu
adversário, o ódio --- é incapaz de persistir.\idxpsica[|)]

Essas, portanto, são as duas peças usadas para construir a tese
conservadora: primeiramente, que a sexualidade\idxsexuateo{} sem controle desumaniza
a vida erótica, impedindo, portanto, o amor; e, em segundo lugar, que a
necessidade de desumanizar, que tem sua origem em experiências infantis
traumáticas, repletas de conflito, é construída a partir\idxhostdesu{} da\idxdesu[|)]
hostilidade.

Concordo, embora desaprove a solução a que os conservadores
chegaram: a punição. Eles são homens da lei e da ordem. Seu clamor por
ação, na ausência de uma população de pessoas maduras, capazes de um
amor sem perversão, reivindica atos de repressão por parte da sociedade
sobre seus cidadãos, para forçar a contenção da perversidade. Mas a
maturidade --- a despeito das esperanças dos filósofos políticos ---
muito dificilmente costuma ser produto de ação política, e ao
psicanalista resta apenas imaginar as formas que a perversão, e o ódio
a ela subordinado, tomarão, ao serem reprimidos, relegados à
clandestinidade. Como seria bom se o amor pudesse ser criado num povo a
partir da contenção sexual! Mas quando foi que existiu uma civilização
cuja robustez tenha resultado da supressão de uma sexualidade
irrestrita? (Aliás, quando foi que existiu --- e o que seria --- uma
civilização saudável?) Como seria bom se o amor não envolvesse também
as dinâmicas do recém-nascido e da infância; que bom seria caso o amor
pudesse ser criado nos adultos, \textit{ex novo}, pela exortação e
pelas leis.\idxcria{} Como seria bom se fosse verdade que o amor é tão inerente
às pessoas, que poderíamos contar com sua ocorrência mediante alguns
poucos ajustes na lei. Se a neurose fosse uma aberração e fosse um
ingrediente menos presente no homem\ldots{} Esse clamor por contenção é
apenas modestamente utópico; não é difícil acreditar que, quanto menos
ódio houver na intimidade, mais feliz será o resultado para os
participantes. Mas o programa de transformar o ódio em amor via ação
policial não alcançou êxito no passado. Além disso --- como outros
terão notado --- as utopias são calmas, porém enfadonhas (e
perigosas); sem o componente perverso --- para aqueles que não toleram
uma intimidade prolongada --- pode ser que sejamos privados da maioria
de nossos artistas, das descobertas científicas, dos gênios da política
e dos grandes filósofos.\idxaspeccon[|)]

Mesmo assim, eu concordo que existe menos perversão num
relacionamento onde há mais amor; é minha crença particular; eu não
poderia provar, como ninguém até hoje o fez. Ainda assim, se alguém for
declarar tais crenças publicamente, precisará ser muito convincente.
São questões importantes demais. Antes de reduzirmos a liberdade de
discurso, ou as liberdades de adultos responsáveis em seu ambiente
privado, incluindo aí a privacidade de se envolver com comportamentos
sexuais aberrantes, é preciso que o argumento esteja à altura dos
perigos que teremos que correr. Eis aquilo de que necessito, a fim de
que esteja convencido: alguma comprovação de que a humanidade como um
todo é inerentemente boa, e de que a sua capacidade para o amor, e não
para o ódio, pode ser posta em ação \textit{agora}, e não em algum
ponto indefinido do futuro; uma descrição fidedigna do que é o amor
entre duas pessoas, de tal forma que eu possa julgar se ele é mais
valioso para a nossa sociedade --- precisamente agora, nestes tempos
perigosos --- do que a liberdade de imprensa e de opinião, e o direito
à perversão privada, cujos limites estão querendo nos impor; uma
comprovação razoável de que este amor está acessível --- agora, por
alguma trilha que pode ser revelada --- para a maioria das pessoas, de
tal forma que esse meio de salvar a sociedade possa ser instituído;
alguma comprovação de que, se a perversão e a hostilidade não se
afastarem, leis punitivas vão ou dissipar tais condições, ou
conduzi-las para a clandestinidade, sem que elas continuem a\idxsexulrep[|(]
representar perigo; é preciso saber quais normas estabelecerão o modo
de conclamar essas forças repressivas e, em seguida, tratar de
suavizá-las, antes que elas --- e, especialmente, as pessoas que
tomarão em suas mãos o poder de repressão --- acabem ultrapassando os
limites considerados desejáveis pelos aprendizes de feiticeiros. Não
me sinto confortável diante da ideia de que a melhor solução para o
problema da corrupção moral é que a capacidade inerente de amar do
homem dará conta do recado, mas que, até que possamos canalizar esse
amor, é preciso furtar à humanidade uma parcela de sua liberdade.

Eu até concordo (embora sem tanta veemência) que a pornografia\idxporno{} é
aviltante, que as pessoas estariam em melhor situação na ausência da
perversão, que se lançar avidamente aos prazeres pré-genitais tornará
as pessoas frenéticas (ou serão justamente as pessoas frenéticas que se
lançam a tais prazeres?) Eu poderia até mesmo concordar que a
licenciosidade prejudica a estrutura da sociedade (apesar de que eu, na
realidade, acredito que a licenciosidade seja mais o resultado de uma
alteração nessa estrutura do que o contrário). Mas, talvez por eu viver
nos Estados Unidos de hoje, eu esteja ainda mais preocupado com a
repressão da liberdade do que com o preço que teríamos que pagar, caso
permitíssemos a corrupção. Nossa civilização sofreu, neste século, o
trauma do totalitarismo; neste momento, os Estados Unidos estão ainda
tão ameaçados por aqueles que desejariam tornar ainda mais severas as
leis, que sou partidário de deixar a liberdade à solta por mais algum
tempo, antes de entrarmos em pânico.

Existem dois tipos de liberdade, entre outros. Um deles, que é
relativo, é a liberdade que se experimenta quando estamos livres de
nossas necessidades neuróticas inconscientes; na perversão, esse tipo
de liberdade se perde. O outro tipo, também relativo, é a liberdade que
a sociedade pode garantir a todos os seus cidadãos. Ambas essas\idxsexulrep[|)]
liberdades são preciosas; porém, neste período crítico em que vivemos,
eu tentaria salvaguardar, primeiramente, o segundo tipo.\idxsexul[|)]

Comecemos por uma velha tese:\idxsexulres[|)] a experiência sexual de todos está sempre
imbuída de moralidade. E podemos compreender a razão disso, depois de\idxpervdirei[|)]
termos examinado como a hostilidade, o mistério, o perigo e a vingança\idxpecad[|)]
são capazes de aumentar a excitação.



\chapter[\textbf{12}\quad A necessidade da perversão]{{\large\textit{Capítulo 12}}\\ A necessidade da perversão}
\markboth{Aspectos sociais da questão}{A necessidade da perversão}


Uma vez que\idxpervneces[|(] a família\idxinflu[|(] já não funciona como unidade essencial à
manutenção da sociedade, a perversão satisfaz a quatro necessidades: à
preservação do prazer individual, à preservação da família, à\idxrelpc[|(]
preservação da sociedade, e à preservação das espécies. Ao reivindicar
isto, estou indo além da descoberta fundamental de Freud,\idxfreudperve{} de que o
indivíduo perverso é uma \textit{vítima} dessa necessidade social que é
a família; estou indo para a posição de que a perversão é uma\textit{
necessidade,} criada tanto pela sociedade quanto pela família, com a
finalidade de que tanto uma quanto a outra não se tornem, elas mesmas,
ainda mais combalidas.

A primeira necessidade, a preservação do prazer, foi suficientemente
discutida neste livro; ela foi também o cerne tanto da teoria quanto
dos dados obtidos pela psicanálise, desde seus primórdios. De maneira
que posso deixar que o conhecimento que compartilhamos sobre essa
realidade fale por si só, caso este livro não tenha cumprido de forma
totalmente satisfatória sua função.

Como sabemos, com base no estudo do conflito\idxconfe[|(] edípico,\idxheteroconf[|(] a intimidade\idxintim{}
gera tensões\idxpervtecni{} eróticas tão graves que a estabilidade familiar se
encontra cronicamente em perigo. De onde uma segunda necessidade: a
perversão precisa agir como um repositório de conservadorismo, para
estabilizar forças que, de outro modo, explodiriam. Ela permite que a
crueldade e o ódio familiares sejam amenizados, para que não se tornem
excessivamente destrutivos --- e é um mecanismo tão eficiente, que
garante a segurança dos pais e da família, graças à existência da
criança perversa. Por exemplo, a futura mãe de um homem homossexual, em
inúmeras pequenas doses, poderá ir transferindo para o seu garotinho a
amargura que nutre, em relação aos homens, em geral, e a seu
lamentável marido, em particular; desde que permaneça distante, e que
aceite seu desdém sem reagir, a mãe permitirá que esse marido continue
passivo; e a criança, desenvolvendo uma efeminação caricatural, pode,
secretamente, adotar uma atitude de desprezo em relação à mãe.

Além disso, bodes\idxbodex{} expiatórios\idxpervaspec{} prestam auxílio a muitas famílias, que
escolhem um de seus membros para fazer o papel do
``doente'', ou do
``mau'', permitindo, assim, as projeções;\idxpervpecad{}
isto faz com que os outros membros da família permaneçam protegidos, e
com que a própria família sobreviva, unida. Uma vez isto feito, os pais
poderão vivenciar alguns de seus desejos perversos na criança escolhida
(conf. 70). Por outro lado, a perversão faz, também, com que os pais
possam desempenhar os papeis a eles atribuídos na representação
edípica, preservando, assim, seu prazer sexual, e consolidando-os em
sua difícil tarefa de parentagem. Para tanto, a perversão de seu filho
é um sacrifício que eles estão dispostos a fazer. Em resumo, a
perversão pode ser não apenas o preço, em termos de neurose, pago pela
instituição que a família é, mas também, ao olharmos o reverso da
moeda, descobrimos que seu outro lado permitiu --- como uma força\idxrelpc[|)]
contrarrevolucionária --- que a família sobrevivesse.

Em terceiro lugar, ao preservar a família, a perversão tem sido a
salvaguarda da sociedade sob todas as formas que ela assumiu no
decorrer de sua evolução, milênios afora. E, tendo em vista as
terríveis tensões lançadas sobre ela, especialmente no século 19, pelos
êxitos materiais da Revolução Industrial, é de se esperar vermos uma
demanda-reflexo, e que a perversão amplie sua função
contrarrevolucionária, que é a de salvaguardar as diferentes formas de
sociedades do presente da dissolução, com que é ameaçada pelo bem-estar
físico. Mas parece existir, agora, uma tendência para que a produção de
bens, em alguns países, atinja, em breve, um ponto, em que tornará
obsoletas algumas funções desempenhadas pela família, e que
anteriormente eram necessárias; o progresso provê, mais e mais,
proteção para muitos que, no passado, só poderiam subsistir mediante
meios que apenas a família\idxinflu[|)] poderia lhes proporcionar, de maneira
consistente: alimento, abrigo, proteção para as crianças, uma pitadinha
de luxúria, e alguns momentos de tranquilidade. Não menos importante, o
controle da natalidade\idxcontr{} reduz, em grande medida, a tremenda carga de
trabalho que a procriação maciça requeria. Tais mudanças poderão
libertar as forças da perversão para tarefas mais lúdicas, como as
artes.\idxcria{} E então, como em geral acontece com os principais fenômenos
sociais, alguma coisa --- como a perversão, ou a metodologia
científica --- que começou como uma defensora do \textit{status quo},
será gradualmente transformada, quase sem necessidade de esforços, num
poderoso agente de mudança.

A perversão, portanto, está a serviço da permanente imutabilidade da
sociedade e da espécie. Porém, uma ameaça constante ao funcionamento
regular da \textit{perversão} é o \textit{indivíduo perverso} e sua
paranoia. Aquele que vai contra as regras, ao se recusar a desempenhar
o papel de perverso, tal como está escrito nos costumes e nas sanções
sociais --- aquele que se rebela contra o papel que lhe foi atribuído,
e que não fará ao vizinho o favor de ser palhaço e vítima --- poderá,
com o passar do tempo, forçar a\idxheteroconf[|)] mudança social, se não, simplesmente,
causar uma revolução na sociedade.

Uma quarta necessidade à qual a perversão satisfaz é a da\idxpervsobre[|(]
sobrevivência das espécies. Paradoxalmente, enquanto suas dinâmicas
conduzem, em algumas poucas famílias, à eventual destruição da
capacidade de as crianças de se reproduzirem, a perversão, como Freud\idxfreudperve{} a
definiu, e como a discutimos no capítulo 10, é uma tentativa de
preservar a\idxhetero[|(] heterossexualidade.

Enquanto a situação edípica, aquele produto da família
heterossexual, é danosa à heterossexualidade emergente, com suas
ameaças e angústias, ela também atormenta a criança durante a fase de
seu crescimento, com suas possibilidades de segurança, afeto e prazer
físico. E assim eu sugiro, juntamente com outros, que nem a
heterossexualidade, nem a família, sejam inevitáveis ou eternas. Antes,
que ambas --- essencialmente criações sociais ---\idxconfe[|)] trabalhem para se
fortalecer mutuamente; e ambas criaram seu mito de serem uma verdade
fundamental eterna (estabelecida por Deus, segundo a religião, e pelos
genes, segundo os cientistas).

Em muitos casos, o desejo de preservar a espécie é conservado apenas
no inconsciente do perverso --- esta vítima heterossexual
\textit{manquée;} contudo, mais frequentemente, o trauma\idxtrauma{} e a frustração
da infância se resolvem de modo menos dramático. Alguma
heterossexualidade permanece, de tal forma que o sobrevivente das
dinâmicas familiares é capaz de se administrar genitalmente e, em raros
casos, poderá até mesmo ser vitorioso ao ponto de constituir aquilo que
a sociedade atual ainda exige: uma família que, um dia, se reproduzirá
a si\idxhetero[|)] mesma.\idxpervsobre[|)]

Além das necessidades que preenche, a perversão traz ainda
vantagens. Por exemplo, uma vez que sua dinâmica central é a
hostilidade,\idxhost{} ela é útil para direcionar o ódio assassino para correntes
mais calmas da imaginação, tais como a religião, as artes, a
pornografia\idxporno{} e os devaneios.\idxcria{} Tais deflexões são quase sempre preferíveis
à expressão direta das forças que elas contêm, e que ficam contidas no
inconsciente. Essa dispersão da raiva presta auxílio às nossas quatro
necessidades essenciais: proporcionando um prazer erótico mais intenso
e com menos culpa, diminuindo o índice de assassinatos nas famílias
(tanto na família à qual a criança pertenceu quanto àquela que formará
quando adulta), atando ao prazer e à exaustão eróticos energias que,
caso não fossem assim extravasadas, poderiam fazer explodir a sociedade,\footnote{ A preferência,
por parte dos revolucionários fanáticos,\idxsexulres[|nn] pela
abstinência sexual é uma perversão que subjuga a hostilidade presente
no interior da família (partido),\idxinflu{} permitindo que ela seja liberada no
exterior, com finalidades destrutivas.} e desviando o ódio passível de
ser gerado entre os sexos, possibilitando assim, pelo menos por alguns
poucos momentos, que homens e mulheres sejam capazes de suportar a
presença um do outro --- coisa que, com demasiada frequência, lhes
parece totalmente impossível.

Em outras palavras, como todas as condições produzidas por
mecanismos neuróticos e que estão tão estabilizadas e são tão
eficientes que costumamos chamá-las de estrutura do caráter, a
perversão acaba sendo o único conjunto manejável de acordos de
conciliação; ele drena uma quantidade suficiente de raiva e desespero,
fazendo com que o indivíduo e a sociedade não sejam completamente
solapados pelas tendências destrutivas que brotam da frustração
infantil e dos traumas familiares.

Enquanto existirem crianças, a sociedade inventará maneiras de
educá-las e, ao fazê-lo, dará forma ao desejo sexual delas. Posto que
não sabemos o que poderá acontecer caso a família desapareça, não temos
como prever de que maneira a sexualidade humana, ao adaptar-se, será
modificada. Meu palpite é que, se tudo transcorrer bem para a nossa
raça, a perversão perderá força, e as variações aumentarão. Talvez\idxpervneces[|)]
chegue o dia em que a perversão não será mais uma necessidade.



\chapter[Referências bibliográficas]{Referências bibliográficas\footnoteInSection{Quando há edição brasileira 
de uma obra citada ela é indicada por uma seta. [N.~da~E.]}}

\markboth{Referências bibliográficas}{Referências bibliográficas}

\begin{bibliohedra}\labelsep0ex\parsep0ex\footnotesize{}
\renewcommand{\tit}[2]{\item[\textbf{#1}~\textnormal{\textsc{\MakeTextLowercase{#2}}}]}
\renewcommand{\titidem}[1]{\item[\textbf{#1}~\line(1,0){25}]}

\tit{1.}{Bak}, R. C., ``The Phallic Woman: The Ubiquitous Fantasy in
Perversions''. In: \textit{Psychoanal. Study Child} 23 (1968):
15--36, New York, International Universities Press.

\tit{2.}{Bandura}, A. e \versal{Walters}, R. H. \textit{Social Learning and Personality
Development}. New York, Holt, Rinehart \& Winston, 1963.

\tit{3.}{Bieber}, I.; \versal{Dain}, H. J.; \versal{Dince}, P. R.; \versal{Drellich}, M. 
G.; \versal{Grand}, H. G.; \versal{Gundlach},
R. H.; \versal{Kremer}, M. W.; \versal{Rifkin}, A. H.; \versal{Wilbur}, C. B. e \versal{Bieber}, T. B.,
\textit{Homosexuality}. New York, Basic Books, 1962.

\tit{4.}{Biller}, A.\,B., ``Father Absence and the Personality Development of
the Male Child''. In: \textit{Developmental Psychology} 2 (1970),
181--270.

\tit{5.}{Blumer}, D., ``Trans sexualism, Sexual Dysfunction and Temporal
Lobe Disorder''. In: \versal{Green}, R. and \versal{Money}, J. (ed.). 
\textit{Transsexualism and Sex Reassignment}. 
Baltimore, Johns Hopkins Press, 1969, pp.\,213--219.

\tit{6.}{Boehm}, R., ``The Femininity-Complex in Men''. In:
\textit{Int. J. Psycho-Anal. }11 (1930), 444--469.

\tit{7.}{Boss}, M., \textit{Meaning and Content of Sexual Perversions. }New York, Grune \&
Stratton, 1949.

\tit{8.}{Bowlby}, J., \textit{Attachment}, New York, Basic Books, 1969 {$\bm{\rightarrow}$} [\versal{BOWLBY}, J.,
\textit{Apego}. São Paulo, Martins Fontes, 1984. Trad. Álvaro Cabral].

\tit{9.}{} ``Brain Surgery for Sexual Disorders''. In: Br. Med. J. 2
(1969), 250.

\tit{10.}{Brodie}, H.\,K.; \versal{Gartrell}, N.; \versal{Doering} C. e \versal{Rhue}, T., ``Plasma
Testosterone Levels in Heterosexual and Homosexual Men''. In: \textit{Am.
J. Psychiat.} 131 (1974), pp.\,82--83.

\tit{11.}{Chodoff}, P., ``A Critique of Freud's Theory of
Infantile Sexuality''. In: \textit{Am. J. Psychiat. }123 (1966), pp.
507--508.

\tit{12.}{Cooper}, A.\,J.; \versal{Ismail}, A.\,A.\,A.; \versal{Phanjoo}, A.\,L. et al.
``Antiandrogen (Cyproterone Acetate), Therapy in Deviant
Hypersexuality''. In: \textit{Br. J. Psychiat. }120 (1972), pp.\,59--63.

\tit{13.}{Cramer}, B., ``Sex Differences in Early Childhood''. In: 
\textit{Child Psychiat. and Human Develop. }1 (1971), pp.\,133-151.

\tit{14.}{Devereux}, G., ``Panel Report: Perversion'' (J. A.
Arlow, reporter). In: \textit{J. Am. Psychoanal. Assoc. } 2 (1954), pp.\,336--345.

\tit{15.}{Doerr}, P; \versal{Kockott}, G.; \versal{Vogt}, H.\,J.; \versal{Pirke}, K.\,M. e \versal{Dittmar},\,F.,
``Plasma Testosterone, Estradiol and Semen Analysis in Male
Homosexuals''. In: \textit{Arch. Gen. Psychiat. }29 (1973), pp.\,829--833.

\tit{16.}{Epstein}, A.\,W., ``The Relationship of Altered Brain States to
Sexual Psychopathology''. In: \versal{Zubin}, J. e \versal{Money}, J. (ed.). 
\textit{Contemporary Sexual Behavior: Critical Issues in the 1970s}. Baltimore,
Johns Hopkins Press, 1973, pp.\,297--310.

\tit{17.}{Escalona}, S.\,K. e \versal{Gorman}, H.\,H. ``Early Life Experience and the
Development of Competence''. In: \textit{Int. Rev. Psychoanal. }1 (1974),
pp.\,151--168.

\tit{18.}{Fenichel}, O. (1930). ``The Psychology of
Transvestitism''. In: \textit{Collected Papers}, New York, W.
W. Norton \& Co., 1953.

\titidem{19.}. \textit{ The Psychoanalytic Theory of Neurosis}, New York, W.
W. Norton \& Co., 1945 {$\bm{\rightarrow}$} [\textit{Teoria psicanalítica das neuroses}, Rio de
Janeiro/São Paulo; Atheneu, 1981. Trad.: Samuel Penna Reis].

\tit{20.}{Fox}, C.; \versal{Ismail}, A.; \versal{Love}, D.; \versal{Kirkham}, K. e \versal{Loraine}, J.
``Studies on the Relationship between Plasma, Testosterone Levels and
Human Sexual Activity''. In: \textit{J. Endocrin.} 52 (1972), pp.\,51--58.

\tit{21.}{Freud}, A. \textit{The Ego and the Mechanisms of Defense}. London, Hogarth
Press, 1937 {$\bm{\rightarrow}$} [\textit{``O ego e os mecanismos de
defesa''}. Rio de Janeiro, Civilização Brasileira, 1982; Trad.:
Álvaro Cabral].

\titidem{22.}. ``The Infantile Neurosis: Genetic and Dynamic
Considerations''. In: \textit{Psychoanal. Study Child }26 (1971). 
New York, International Universities Press, pp.\,79--90.

\tit{23.}{Freud}, S. (1900). \textit{The Interpretation of Dreams, Standard Edition }4,
London, Hogarth Press, 1953 (\textit{Standard Edition hereafter noted as
SE)}.{$\bm{\rightarrow}$} [``A Interpretação dos Sonhos''. Edição
Standard Brasileira das Obras Psicológicas Completas de Sigmund Freud. Rio de
Janeiro. Imago, 1980, vols.\,IV--V. Trad.: Jayme Salomão].

\titidem{24.}. (1905), \textit{Three Essays on the Theory of Sexuality, SE
}7\textit{ } (1953), pp.\,135--243 {$\bm{\rightarrow}$} [``Três Ensaios sobre a Teoria da
Sexualidade'', op. cit., vol.\,VII].

\titidem{25.}. (1905), \textit{Jokes and their Relation to the Unconscious,
SE 8}, (1960). {$\bm{\rightarrow}$} [``Os Chistes e sua Relação com o Inconsciente''.
Op.cit., vol. VIII].

\titidem{26.}. (1909), ``Analysis of a Phobia in a Five-Year-Old
Boy'', SE \textit{10} (1955), pp.\,5--149. {$\bm{\rightarrow}$} [``Análise de
uma Fobia em um Menino de Cinco Anos''. Op. cit., vol.~X].

\titidem{27.}. (1911), ``Psycho-analytic Notes on na
Autobiographical Account of a Case of Paranoia (Dementia
Paranoides)'', \textit{SE 12 }1958, pp.\,9--32 {$\bm{\rightarrow}$} [``Notas
Psicanalíticas sobre um Relato Autobiográfico de um Caso de Paranoia (dementia
paranoides)''. Op. cit., vol.~XII].

\titidem{28.}. (1912), ``The Dynamics of
Tranference'', \textit{SE 12 }(1958), pp.\,99--108 {$\bm{\rightarrow}$} [``A
Dinâmica da Transferência''. Op. cit., vol.\,XII].

\titidem{29.}. (1915), ``Instincts and Their
Vicissitudes'', \textit{SE 14 }(1957), pp.\,117--140.
{$\bm{\rightarrow}$} [``Os Instintos e suas Vicissitudes'', Op. cit., vol.\,XIV].

\titidem{30.}. (1919), ``A Child is Being Beaten'',
\textit{SE 17 }(1955), pp.\,179--204. {$\bm{\rightarrow}$} [``Uma Criança
é Espancada: uma Contribuição ao Estudo da Origem das Perversões
Sexuais''. Op. cit., vol.\,XVII].

\titidem{31.}. (1923), \textit{The Ego and the Id, SE 19 }(1961), pp.\,12--66.
{$\bm{\rightarrow}$} [``O Ego e o Id'', Op. cit., vol.\,XIX].

\titidem{32.}. (1927), ``Fetishism'', \textit{SE 21
}(1961), pp.\,152--157. {$\bm{\rightarrow}$} [``Fetichismo''. Op. cit., vol.\,XXI].

\titidem{33.}. (1932), ``Femininity'', \textit{SE
22 }(1964), pp.\,112--135. {$\bm{\rightarrow}$} [``Conferência~XXXIII: Feminilidade, Novas
Conferências Introdutórias sobre Psicanálise''. Op. cit., vol.\,XXII].

\titidem{34.}. (1937), ``Analysis Terminable and
Interminable'', \textit{SE }23 (1964), pp.\,216--253.
{$\bm{\rightarrow}$} [``Análise Terminável e Interminável'', Op. cit., vol.~XXIII].

\titidem{35.}. (1940), ``The Splitting of the Ego in the Process
of Defense'', \textit{SE }23 \textit{ }(1964), pp.\,275--278.
{$\bm{\rightarrow}$} [``A Divisão do Ego no Processo de Defesa''. Op. cit., vol.~XXIII].

\tit{36.}{Friday}, N., \textit{My Secret Garden: Women's Sexual
Fantasies}, New York, Trident Press, 1973.

\tit{37.}{Gadpaille}, W. J., ``Research into the Physiology of Maleness and
Femaleness'', \textit{Arch. Gen. Psychol.} 26 (1972), pp.\,193--206.

\tit{38.}{Galenson}, E., ``A Consideration of the Nature of Thought in
Childhood Play''. In: \versal{McDevill}, J. B. e \versal{Settlage}, C. F. (ed.). 
\textit{Separation-Individualism, Essays in
Honor of Margaret S. Mahler}. New York, International Universities Press, 1971, pp.\,41--59.

\titidem{39.}. e \versal{Roiphe}, H., ``The Impact of Early Sexual
Discovery on Mood, Defensive Organization and Symbolization'',
\textit{Psychoanal. Study Child }26 (1972). New York, Quadrangle
Books, 1972, pp.\,195--216.

\tit{40.}{Galle}, O. R.; \versal{Gove}, W. R. e \versal{McPherson}, J. M. ``Population Density
and Pathology: What Are the Relations for Man?'', \textit{Science
}176 (1972), pp. 23--30.

\tit{41.}{Gebhard}, P. H.; \versal{Gagnon}, J. H.; \versal{Pomeroy}, W. B. et al. \textit{Sex Offenders.
}New York, Harper \& Row Publishers, 1965.

\tit{42.}{Gillespie}, W. H., ``A Contribution of the Study of
Fetishism'', \textit{Int. J. Psycho-Anal. } 21 (1940), pp.\,401--415.

\titidem{43.}. ``Notes on the Analysis of Sexual
Perversions'', \textit{Int. J. Psycho-Anal. }33 (1956), pp.\,396--403.

\titidem{44.}. ``The General Theory of Sexual
Perversion'', \textit{Int. J. Psycho-Anal. }37 (1956), pp.\,396--403.

\titidem{45.}. ``The Psycho-analytic Theory of Sexual
Deviation with Special Reference to Fetishism''. In: \versal{Rosen}, I. (ed.). \textit{The
Pathology and Treatment of Sexual Deviation: A Methodological Approach.} 
New York, Oxford University Press, 1964, pp.\,123--145.

\tit{46.}{Glover}, E., \textit{The Roots of Crime}, New York, Hillary House Publishers,
1960.

\titidem{47.}. ``Aggression and Sado-Masochism''. In: \versal{Rosen}, I. (ed.).,
\textit{The Pathology and Treatment of Sexual Deviation: A Methodological
Approach}. New York, Oxford University Press, 1964, pp.\,146--162.

\tit{48.}{Goldberg}, S. e \versal{Lewis}, S. ``Play Behavior in the Year-Old Infant:
Early Sex Differences'', \textit{Child Devel., }40 (1969), pp.\,21--33.

\tit{49.}{Goldman}, R. \textit{Principles of Medical Science.} New York, McGraw-Hill Book
Co., 1973.

\tit{50.}{Gray}, P. H., ``Theory and Evidence of Imprinting in Human
Infants''. In: \textit{J. Psychol.} 46 (1958), pp.\,155--166.

\tit{51.}{Green}, R., ``Homosexuality as a Mental Illness''. In:
\textit{Int. J. Psychiat. }10 (1972), pp.\,77--98.

\titidem{52.}. \textit{Sexual Identity Conflict in Children and Adults.} New
York, Basic Books, 1973.

\tit{53.}{Greenacre}, P., ``Respiratory Incorporation and the Phallic
Phase''. \textit{Psychoanal. Study Child } 6 (1951), New
York, International Universities Press, pp.\,180--205.

\titidem{54.}. ``Certain Relationships Between Fetishism and the
Faulty Development of the Body Image''. \textit{Psychoanal. Study
Child }8 (1953), New York, International Universities Press, pp.\,79--98.

\titidem{55.}. ``Further Considerations Regarding
Fetishism'', \textit{Psychoanal. Study Child }10 (1955), New York, 
International Universities Press, pp.\,187--194.

\titidem{56.}. ``On Focal Symbiosis''. In: \versal{Jessner}, L. e \versal{Pavenstedt}, E. 
(ed.). \textit{Dynamic Psychopathology in Childhood}. New York, 
Grune \& Stratton, 1959, pp.\,243--256.

\titidem{57.}. ``Further Notes on Fetishism'',
\textit{Psychoanal. Study Child } 15 (1960). New York, International
Universities Press, pp.\,191--207.

\titidem{58.}. ``Perversions: General Considerations Regarding
Their Genetic and Dynamic Background'', \textit{Psychonanal. Study
Child }23 (1968). New York, International Universities Press. pp.\,47--62.

\titidem{59.}. ``The Fetish and the Transitional
Object'', \textit{Psychoanal. Study Child } 24 (1969). New York, International 
Universities Press, pp.\,144--164.

\tit{60.}{Greenson}, R. R., ``A Transvestite Boy and a
Hypothesis'', \textit{Int. J. Psycho-Anal. } 47 (1966), pp.\,396--403.

\titidem{61.}. ``Dis-identifying from Mother'',
\textit{ Int. J. Psycho-anal. } 49 (1968), pp.\,370--374.

\tit{62.}{Greenspan}, J. e \versal{Myers Jr.}, J. M. ``A Review of the Theoretical
Concepts of Paranoid Delusions with Special Reference to Women''.
\textit{ Penn. Psychiat. Quart. } 1 (1961), pp.\,11--28.

\tit{63.}{Harlow}, H. F. e \versal{Harlow}, M. K., ``Social Deprivation in
Monkeys'', \textit{Sci. Am. } 207 (1962), pp.\,136--146.

\titidem{64.}. ``The Effect of Rearing Conditions on
Behavior''. In: \versal{Money} J. (ed.). \textit{Sex Research, 
New Developments}. New York, Holt, Rinehart \& Winston, 1965, pp.\,161--175.

\tit{65.}{Hartmann}, H. (1939). \textit{Ego Psychology and the Problem of Adaptation}. New
York, International Universities Press, 1958 {$\bm{\rightarrow}$} [\textit{Psicologia do Ego e o
Problema da Adaptação. }Rio de Janeiro, Biblioteca Universal Popular, 1968. Trad.:
Álvaro Cabral].

\tit{66.}{Heath}, R. G., ``Pleasure and Brain Activity in Man''. In: 
\textit{J. Nerv. Ment. Dis., } 154 (1972), pp.\,3--18.

\tit{67.}{Holdbrook}, D., \textit{Sex and Dehumanization}. London, Pitman, 1972.

\tit{68.}{Hooker}, E., ``The Adjustment of the Male Overt
Homosexual'', In: \textit{ J. Proj. Tech. }21 (1957), pp.\,18--31.

\tit{69.}{Iverson}, W., \textit{Venus USA, }New York, Pocket Books, 1970.

\tit{70.}{Johnson}, A. M. e \versal{Szurek}, S. A., ``The Genesis of Antisocial Acting
Out in Children and Adults'', \textit{Psychoanal. Q. }21 (1952), pp.\,323--343.

\tit{71.}{Jost}, A., ``A New Look at the Mechanisms Controlling Sex
Differentiation in Mammals''. In: \textit{Johns Hopkins Med. J., }130
(1972), pp.\,38-53.

\tit{72.}{Kallmann}, F. J., ``A Comparative Twin Study on the Genetic Aspects
of Male Homosexuality''. In: \textit{J. Nerv. Ment. Dis., }115 (1952),
pp.\,282--298.

\tit{73.}{Karlen}, A., \textit{Sexuality and Homosexuality}. New York, W. W. Norton \&
Co., 1971.

\tit{74.}{Khan}, M. M. R., ``Clinical Aspects of the Schizoid
Technique''. In: \textit{Int. J. Psycho-Anal., } 41 (1960), pp.\,430-437.

\titidem{75.}. ``Foreskin Fetishism and Its Relation to Ego
Pathology in a Male Homosexual''. \textit{Int. J. Psycho-Anal. } 46
(1965), pp.\,64--80.

\titidem{76.}. ``The Function of Intimacy and Acting Out in the
Perversions''. In: \versal{Slovenko}, R. (ed.). \textit{Sexual Behavior and the 
Law}. Springfield, \versal{III}, Charles C. Thomas, 1965, pp.\,397--412.

\tit{77.}{Kinsey}, A. C.; \versal{Pomeroy}, W. B. e \versal{Martin}, C. E. 
\textit{Sexual Behavior in the Human Male. }Philadelphia, W. B. Saunders Co., 1948.

\titidem{78.}. et al, \textit{Sexual Behavior in the Human Female}. 
Philadelphia, W.\,B. Saunders Co., 1948.

\tit{79.}{Klaf}, F.\,S., ``Female Sexuality and Paranoid
Schizophrenia''. \textit{Arch. Gen. Psychiat. } (1961), pp.\,84--86.

\tit{80.}{Kleeman}, J.\,A., ``The Establishment of Core Gender Identity in
Normal Girls''. \textit{ Arch. Sex. Behavior, } 1 (1971), pp.\,102--129.

\tit{81.}{Klein}, H.\,R. e Horwitz, W.\,A., ``Psycho-sexual Factors in the
Paranoid Phenomena'', \textit{Am. J. Psychiat., } 105 (1949), pp.\,697--701.

\tit{82.}{Klein}, M., \textit{The Psychoanalysis of Children}. London, Hogarth Press, 1932
{$\bm{\rightarrow}$} [\textit{Psicanálise da Criança}. São Paulo, Mestre Jou, 1981. Trad.: Pola
Civelli].

\tit{83.}{Kolodny}, R.\,C.; \versal{Masters}, W.\,H.; \versal{Hendryx}, J. et al. ``Plasma
Testosterone and Serum Analysis in Male Homosexuals''. In: \textit{N.
Engl. J. Med. } 285 (1971), pp.\,1170--1174.

\tit{84.}{Kreuz}, L.\,E.; \versal{Rose}, R.\,M. e \versal{Jennings}, J.\,R. ``Suppression of
Plasma Testosterone Levels and Psychological Stress'', \textit{Arch.
Gen. Psychiat.} 26 (1972), pp.\,479--482.

\tit{85.}{Laschet}, U., ``Antiandrogen in the Treatment of Sex Offenders:
Mode of Action and Therapeutic Outcome''. In: \versal{Zubin}, J. e 
\versal{Money}, J. (ed.). \textit{Contemporary
Sexual Behavior: Critical Issues in the 1970s}. Baltimore, Johns Hopkins Press, 
1973, pp.\,311--319.

\tit{86.}{Leites}, N., \textit{The New Ego: Psychoanalytic Concepts. }New York, Science
House, 1971.

\tit{87.}{Lorenz}, K., \textit{King Solomon's Ring}. New York, Thomas Y.
Crowell co., 1952.

\tit{88.}{MacAlpine}, I. e \versal{Hunter}, R.\,A., \textit{David Paul Schreber: Memoirs of My
Mental Illness. }London, Dawson \& Sons, 1955.

\tit{89.}{Maclean}, P.\,D., ``Studies on the Cerebral Representation of
Certain Basic Sexual Functions''. In: \versal{Gorski} R. A. e 
\versal{Whalen} R. E. (ed.). \textit{Brain and Behavior},
vol.\,3, Los Angeles, University of California Press, 1966, pp.\,35--79.

\tit{90.}{Mahler}, M. S., ``On Child Psychosis and Schizophrenia: Autistic
and Symbiotic Infantile Psychoses''. In: \textit{Psycho-anal. Study
Child }7 (1952). New York, International Universities Press, pp.\,286--305.

\titidem{91.}. ``Autism and Symbiosis, Two Extreme Disturbances
of Identity''. \textit{Int. J. Psycho-anal., }39 (1958), pp.\,77--83.

\titidem{92.}. ``Thoughts About Development and
Individualization''. In: \textit{Psychoanal. Study Child, }18 (1963). 
New York, International Universities Press, pp.\,307--324.

\titidem{93.}. ``On the Significance of the Normal
Separation"-Individuation Phase: With Reference to Research in Symbiotic Child
Psychosis''. In: \versal{Schur}, M. (ed.). \textit{Drives, Affects, Behavior, }vol.\,2. New York, 
International Universities Press, 1965.

\titidem{94.}. \textit{On Human Symbiosis and the Vicissitudes of
Individuation, }vol. 1, New York, International Universities Press, 1968.

\titidem{95.}. ``Rapprochement Subphase of the
Separation-Individuation Process''. \textit{Psychoanal. Q. }41
(1972), pp.\,487--506.

\titidem{96.}. e \versal{Furer}, M., ``Certain Aspects of the
Separation-Individuation Phase'', \textit{Psychoanal. Q. }32 (1963),
pp.\,1--14.

\tit{97.}{Malinowski}, B., \textit{Sex and Repression in Savage Society, }New York,
Humanities Press, 1927.

\tit{98.}{Margolese}, M.\,S., ``Homosexuality: A New Endocrine
Correlate''. In: \textit{Hormones and Behavior }1 (1970), pp.\,151--155.

\tit{99.}{Marmor}, J., ``Orality in the Hysterical
Personality''. In: \textit{ J. Am. Psychoanal. Assoc. }1 (1953), pp.\,656--671.

\titidem{100.}. ``\,`Normal' and `Deviant' Sexual Behavior''.
\textit{JAMA. }217 (1971), pp.\,165--170.

\titidem{101.}. ed. \textit{Sexual Inversion, }New York, Basic Books, 1965.

\tit{102.}{Masters}, W.\,H. e Johnson, V.\,E. \textit{Human Sexual Response. }Boston,
Little, Brown \& Co., 1966 {$\bm{\rightarrow}$} [\textit{A conduta sexual humana}, Rio de Janeiro,
Civilização Brasileira, 1968. Trad.: Dante Costa].

\tit{103.}{McDougall}, J. ``Primal Scene and Sexual
Perversion''. In: \textit{Int. J. Psycho-Anal. }53 (1972), pp.\,371--384.

\tit{104.}{Michael}, R.\,P., ``Biological Factors in the Organization and
Expression of Sexual Behavior''. In: \versal{Rosen}, I (ed.). \textit{The Pathology and
Treatment of Sexual Deviation}. New York, Oxford University Press, 1964, pp.\,24--54.

\tit{105.}{Miller}, I. ``Unconscious Fantasy and Masturbatory
Technique''. \textit{J. Am. Psychoanal. Assoc. } 17 (1969), pp.\,826--847.

\tit{106.}{Millett}, K., \textit{ Sexual Politics, }New York, Doubleday \& Co., 1970.

\tit{107.}{Modlin}, H.\,C. ``Psychodynamics and Management of Paranoid States
in Women''. In: \textit{ Arch. Gen. Psychiat. }8 (1963), pp.\,263--268.

\tit{108.}{Money}, J., ``Sex Reassignment as Related to Hermaphroditism and
Transsexualism''. In: \versal{Green}, R. e \versal{Money}, J. (ed.). \textit{Transsexualism and 
Sex Reassignment.} Baltimore, Johns Hopkins Press, 1969, pp.\,91--115.

\titidem{109.}. e \versal{Ehrhardt}, A. \textit{Man and Woman, Boy and Girl,}
Baltimore, Johns Hopkins Press, 1972.

\titidem{110.}. e \versal{Pollitt}, E., ``Psychogenetic and Psychosexual
Ambiguities: Klinefelter's Syndrome and Transvestism
Compared'', \textit{Arch. Gen. Psychiat.} 11 (1964), pp.\,589--595.

\tit{111.}{Moore}, B., ``Frigidity: A Review of the Psychoanalytic
Literature''. \textit{Psychoanal. Q. }33 (1964), pp.\,323--349.

\tit{112.}{Newman}, L.\,E. e \versal{Stoller}, R.\,J. ``The Oedipal Situation in Male
Transsexualism''. In: \textit{Br. J. Med. Psychol., }44 (1971), pp.\,295--303.

\tit{113.}{Olds}, J. ``Self-stimulation Experiments and Differential Reward
Systems''. In: \versal{Jasper}, H.\,H. \versal{Proctor}, L.\,D. \versal{Knighton}, 
R.\,S. \versal{Noshay}, W.\,C. e \versal{Costello}, R.\,T. (ed.). \textit{Reticular Formation 
of the Brain.} Boston, Little, Brown \& Co., 1958, pp.\,671--687.

\tit{114.}{Ostow}, M. (ed.). \textit{ Sexual Deviation: Psychoanalytic Insights. }New York,
Quadrangle Books, 1974.

\tit{115.}{Pare}, C.\,M.\,B. ``Etiology of Homosexuality: Genetic and
Chromosomal Aspects''. In: \versal{Marmor} J. (ed.). \textit{Sexual Inversion.} New York, 
Basic Books, 1965, pp.\,70--80.

\tit{116.}{Pfeiffer}, E. (ed.). \textit{Sigmund Freud and Lou Andreas-Salomé: Letters}. New
York, Harcourt Brace Jovanovich, 1972.

\tit{117.}{Pillard}, R.\,C.; \versal{Rose}, R.\,M. e Sherwood, M. ``Plasma Testosterone
Levels in Homosexual Men''. In: \textit{Arch. Sex. Behav. }3 (1974), pp.\,453--457.

\tit{118.}{Rachman}, S. ``Sexual Fetishism: An Experimental
Analogue''. \textit{Psychol. Record } 16 (1966), pp.\,293--296.

\tit{119.}{Racker}, H. \textit{Transference and Countertransference}. New York,
International Universities Press, 1968.

\tit{120.}{Reich}, W. ``The Phallic-Narcissistic Character''. In: \textit{Character 
Analysis}. New York, Orgone Institute Press, 1949, pp.\,200--207. {$\bm{\rightarrow}$} 
[``O Caráter Fálico"-Narcisista''. In: \textit{Análise do caráter}. São Paulo:
Martins Fontes, s/d. Trad.: Maria Lizette Branco e Maria Manuela Pecegueiro].

\tit{121.}{} \textit{The Report of the Commission on Obscenity and Pornography}. New York,
Bantam Books, 1970.

\tit{122.}{Roeder}, R.\,C. ``Homosexuality `Burned
Out': German Surgeon Claims Hypothalamotomy Normalizes Sex
Drive''. \textit{Medical World News}. September 25, 1970, pp.\,20--21.

\tit{123.}{Rosenfeld}, H., ``Remarks on the Relation of Male Homosexuality to
Paranoia, Paranoid Anxiety and Narcissism''. In: \textit{Int. J.
Psycho-Anal.} 30 (1949), pp.\,36--47.

\tit{124.}{Shmideberg}, M., ``Delinquent Acts as Perversions and
Fetiches''. \textit{Br. J. Delinq. }7 (1956), pp.\,44--49.

\tit{125.}{Searles}, H.\,F., ``Sexual Processes in
Schizophrenia''. \textit{Psychiatry }24 (1961), pp.\,87--95.

\tit{126.}{Sears}, R.\,R.; \versal{Maccoby}, E.\,E. e \versal{Levin}, H., \textit{Patterns 
of Child Rearing, Evanston, \versal{III}, }Row, Peterson \& Co., 1957.

\tit{127.}{Sherfey}, M.\,J., ``The Evolution and Nature of Female Sexuality in
Relation to Psychoanalytic Theory''. \textit{J. Am. Psychoanal.
Assoc. } 14 (1966), pp.\,28--128.

\tit{128.}{Slater}, E., ``Birth Order and Maternal Age of
Homosexuals''. \textit{Lancet } 1--1 (1962), pp.\,69--71.

\tit{129.}{Smirnoff}, V.\,N., ``The Masochistic Contract'',
\textit{Int. J. Psycho-Anal. }50 (1969), pp.\,665--671.

\tit{130.}{Socarides}, C.\,W., \textit{The Overt Homosexual. }New York, Grune \& Stratton,
1968.

\titidem{131.}. ``Homosexuality and Medicine''.
\textit{JAMA }212 (1970), pp.\,1199--1202.

\titidem{132.}. ``The Demonified Mother: A Study of Voyeurism
and Sexual Sadism''. \textit{Int. Ver. Psycho-Anal. }1 (1974), pp.\,187--195.

\tit{133.}{Spitz}, R.\,A., \textit{The First Year of Life. }New York, International
Universities Press, 1965. {$\bm{\rightarrow}$} [\textit{O Primeiro Ano de Vida}. São Paulo\textit{, }
Martins Fontes, 1979. Trad:. Erothildes Millan Barros da Rocha].

\tit{134.}{Stoller}, R.\,J., ``The Hermaphroditic Identity of
Hermaphrodites''. \textit{J. Nerv. Ment. Dis. }139 (1964), pp.\,453--457.

\titidem{135.}. ``The Mother's Contribution to
Infantile Transvestic Behavior''. \textit{Int. J. Psycho-Anal } 47
(1966), pp.\,384--395.

\titidem{136.}. ``Shakespearean Tragedy:
Coriolanus'', \textit{Psycho-anal. Q. }35 (1966), pp.\,263--274.

\titidem{137.}. \textit{Sex and Gender, }vol.\,1, New York, Science House,
1968.

\titidem{138.}. ``The Transsexual Boy:
Mother's Feminized Phallus''. \textit{Br. J. Med.
Psychol. }43 (1970), pp.\,117--128.

\titidem{139.}. ``The Term `Transvestism'\,'', \textit{Arch.
Gen. Psychiat. }24 (1971), pp.\,230--237.

\titidem{140.}. ``The `Bedrock' of Masculinity and Femininity:
Bisexuality'', \textit{Arch. Gen. Psychiat. }26 (1972), pp.\,207--212.

\titidem{141.}. ``Etiological Factors in Female Transsexualism:
A First Approximation''. \textit{Arch. Sex. Behav. }2 (1972), pp.\,47--67.

\titidem{142.}. ``Transsexualism and
Transvestism'', \textit{Psychiatric Annals }1 (1972), pp.\,6--72.

\titidem{143.}. ``The Impact of New Advances in Sex Research on
Psychoanalytic Theory''. \textit{Am. J. Psychiat. }130 (1973), pp.\,241--251, 1207--1216.

\titidem{144.}. ``The Male Transsexual as `Experiment'\,'', \textit{Int. J.
Psycho-Anal. }54 (1973), pp.\,215-226.

\titidem{145.}. ``Psychoanalysis and Physical Intervention in
the Brain''. In: \versal{Zubin} J. e \versal{Money} J. (ed.). \textit{ Contemporary Sexual Behavior:
Critical Issues in the 1970s}, pp.\,339--350.

\titidem{146.}. \textit{Splitting: A Case of Female Masculinity}. New York,
Quadrangle Books, 1973.

\titidem{147.}. \textit{Sex and Gender, }vol.\,2, New York, Jason Aronson, 1975.

\titidem{148.}. e \versal{Newman}, L.\,E. ``The Bisexual Identity of
Transsexuals: Two Case Examples''. In: \textit{Arch. Sex. Behav. }1
(1971), pp.\,17--28.

\tit{149.}{Tourney}, G. e \versal{Hatfield}, L. ``Androgen Metabolism in
Schizophrenics, Homosexuals, and Normal Controls''. In: \textit{Biol.
Psychiat. }6 (1973), pp.\,23--36.

\tit{150.}{Vanggaard}, T. \textit{Phallós: A Symbol and Its History in the Male World,
}New York, International Universities Press, 1972.

\tit{151.}{Walinder}, J., \textit{Transsexualism: A Study of Forty-three Cases. }Göteborg,
Scandinavian University Books, 1967.

\tit{152.}{Wermer}, H. e \versal{Leving}, S. ``Masturbation Fantasies''. In:
\textit{Psychoanal. Study Child }22 (1967). New York, International
Universities Press, pp.\,315--328.

\tit{153.}{White}, R.\,B. ``The Mother-Conflict in
Schreber's Psychosis''. In: \textit{Int. J. Psycho-Anal.
} 42 (1961), pp.\,55--73.

\tit{154.}{Williams}, A.\,H. ``The Psychopathology and Treatment of Sexual
Murderers''. In: \versal{Rosen}, I. (ed.). \textit{The Pathology and Treatment of Sexual
Deviation: A Methodological Approach}. New York, Oxford
University Press, 1964, pp.\,351--377.

\tit{155.}{Winnicott}, D.\,W. (1960). ``Ego Distortion in Terms of True or
False Self''. In: \textit{The Maturational Processes and the
Facilitating Environment: Studies in the Theory of Emotional Development}. New
York, International Universities Press, 1972. {$\bm{\rightarrow}$} [“Distorção do Ego em Termos de
Falso e Verdadeiro \textit{Self}”. In: \textit{O ambiente e os processos
de maturação}: \textit{estudos sobre a teoria do desenvolvimento emocional}. Porto
Alegre, Artes Médicas, 1983. Trad.: Irineo Constantino Schuch Ortiz].

\end{bibliohedra}

\cleardoublepage


